\documentclass{standalone}

%!Tex Root = ../main.tex

% ┌────────────┐
% │ Formatting │
% └────────────┘
\usepackage[english]{babel}
\usepackage[top=0cm,bottom=0cm,left=0cm,right=0cm]{geometry}
\usepackage[export]{adjustbox} % use c, l, r for images
\usepackage{csquotes}
\usepackage[parfill]{parskip}
\usepackage{fontspec}
% \usepackage{anyfontsize}
% \usepackage[]{enumitem}

% ┌──────┐
% │ Math │
% └──────┘
\usepackage{amssymb} % for black triangleright, https://tex.stackexchange.com/questions/570303/use-blacktriangleright-as-itemize-label
\usepackage{amsmath}
\usepackage{mathtools} % for \mathclap and 
\usepackage{breqn}

% ┌────────┐
% │ Tables │
% └────────┘
\usepackage{tabularray}
 % \UseTblrLibrary{diagbox}

% ┌────────┐
% │ Images │
% └────────┘
\usepackage{graphicx}
% \usepackage{float} % for the letter H
% \graphicspath{figures/}
\usepackage{subcaption}

% ┌────────┐
% │ Graphs │
% └────────┘
\usepackage{tikzit}
\usepackage{tikz}
\usetikzlibrary{backgrounds}
\usetikzlibrary{arrows}
\usetikzlibrary{shapes,shapes.geometric,shapes.misc}

% this style is applied by default to any tikzpicture included via \tikzfig
\tikzstyle{tikzfig}=[baseline=-0.25em,scale=0.5]

% these are dummy properties used by TikZiT, but ignored by LaTex
\pgfkeys{/tikz/tikzit fill/.initial=0}
\pgfkeys{/tikz/tikzit draw/.initial=0}
\pgfkeys{/tikz/tikzit shape/.initial=0}
\pgfkeys{/tikz/tikzit category/.initial=0}

% standard layers used in .tikz files
\pgfdeclarelayer{edgelayer}
\pgfdeclarelayer{nodelayer}
\pgfsetlayers{background,edgelayer,nodelayer,main}

% style for blank nodes
\tikzstyle{none}=[inner sep=0mm]

% include a .tikz file
\newcommand{\tikzfig}[1]{%
{\tikzstyle{every picture}=[tikzfig]
\IfFileExists{#1.tikz}
  {\input{#1.tikz}}
  {%
    \IfFileExists{./figures/#1.tikz}
      {\input{./figures/#1.tikz}}
      {\tikz[baseline=-0.5em]{\node[draw=red,font=\color{red},fill=red!10!white] {\textit{#1}};}}%
  }}%
}

% the same as \tikzfig, but in a {center} environment
\newcommand{\ctikzfig}[1]{%
\begin{center}\rm
  \tikzfig{#1}
\end{center}}

% fix strange self-loops, which are PGF/TikZ default
\tikzstyle{every loop}=[]


% ┌────────┐
% │ Citing │
% └────────┘
% \usepackage[style=authortitle]{biblatex}
% \addbibresource{./Graph_Theory.bib}
% \usepackage{cleveref}

% ┌──────────┐
% │ Diagrams │
% └──────────┘
% \usepackage{tikz}
% \usetikzlibrary{shadows, backgrounds} % , calc

% ┌──────────────────┐
% │ Multiple columns │
% └──────────────────┘
% \usepackage{multicol}

% ┌────────────────────┐
% │ Code hightligthing │
% └────────────────────┘
% \usepackage{minted}

% ┌────────────────────────┐
% │ Latex Programming Help │
% └────────────────────────┘
\usepackage{etoolbox}
\usepackage{xparse}
% https://tex.stackexchange.com/questions/358292/creating-a-subcounter-to-a-counter-i-created
\usepackage{chngcntr}

% ┌───────────────┐
% │ Pretty Boxes  │
% └───────────────┘
\usepackage{xcolor}
\usepackage{tcolorbox}
\tcbuselibrary{skins,theorems}

% ┌──────────────┐
% │ Pseudo Code  │
% └──────────────┘
\usepackage{pseudo}

% \usepackage{background}
\usepackage{gradient-text}
\usepackage{rotating}

% \newlength\mylen
% \setlength\mylen{\dimexpr\paperwidth/80\relax}
%
% \SetBgScale{1}
% \SetBgAngle{0}
% \SetBgColor{blue!30}
% \SetBgContents{\tikz{\draw[step=\mylen] (-.5\paperwidth,-.5\paperheight) grid (.5\paperwidth,.5\paperheight);}}

% ┌────────────┐
% │ Misc Tools │
% └────────────┘
\usepackage{lipsum}

%!Tex Root = ../main.tex

% ┌────────────┐
% │ Formatting │
% └────────────┘
% \setlength{\parskip}{0.4cm} % space between paragraphs, https://latexref.xyz/bs-par.html

% ┌───────┐
% │ Fonts │
% └───────┘
\usepackage{fontspec}
\newfontfamily\gyre{DejaVu Math TeX Gyre}
% colored bold
% \newcommand\alert[1]{\textcolor{SwitchColor}{\textbf{#1}}}
\newcommand\alert[1]{\textcolor{SwitchColor}{#1}}

% ┌──────────────┐
% │ Pseudo Code  │
% └──────────────┘
\newcounter{algorithm}
\setcounter{algorithm}{0}
\newtcbtheorem[use counter=algorithm]{algorithm}{\color{SecondaryColor}Algorithm}{pseudo/ruled}{alg}
% \newcommand{\ma}[1]{$\mathcal{#1}$}
% \renewcommand{\tt}[1]{{\small\texttt{#1}}}

% ┌────────┐
% │ Colors │
% └────────┘
\definecolor{PrimaryColor}{HTML}{800080}
\definecolor{PrimaryColorDimmed}{HTML}{D6D6F0}
\definecolor{SecondaryColor}{HTML}{006BB6}
\definecolor{SecondaryColorDimmed}{HTML}{E5F0F8}
\definecolor{SwitchColor}{named}{PrimaryColor}
\colorlet{BoxColor}{gray!10!white}

% ┌───────┐
% │ Links │
% └───────┘
\usepackage[allbordercolors=PrimaryColor, pdfborder={0 0 .2}]{hyperref}

% ┌─────────┐
% │ Mindmap │
% └─────────┘
\renewcommand{\labelitemi}{$\textcolor{SwitchColor}{\bullet}$}
\renewcommand{\labelitemii}{$\textcolor{SwitchColor}{\blacktriangleright}$}
\renewcommand{\labelitemiii}{$\textcolor{SwitchColor}{\blacksquare}$}

%!Tex Root = ../main.tex

% ┌─────────┐
% │ Mindmap │
% └─────────┘
\newlength{\leveldistance}
\setlength{\leveldistance}{25cm}

\newenvironment{edges}{\begin{pgfonlayer}{background}\draw [concept connection]}{;\end{pgfonlayer}}
\newcommand{\edge}[2]{(#1) edge (#2)}
\newcommand{\annotation}[2]{\path (#1) -- node[annotation, above, align=center, pos=0.03] {#2} (middle);}

\newenvironment{resettikz}{\pgfsetlayers{nodelayer,edgelayer}\tikzset{every node/.style={fill opacity=1.0, draw opacity=1.0, minimum size=0cm, inner sep=0pt}}}{}

\newenvironment{mindmap}{
	\begin{tikzpicture}[
			auto,
			huge mindmap,
			fill opacity=0.6,
			draw opacity=0.8,
			concept color = PrimaryColorDimmed,
			every annotation/.style={fill=BoxColor, draw=none, align=center, fill = BoxColor, text width = 2cm},
			grow cyclic,
			level 1/.append style = {
					concept color=SecondaryColorDimmed,
					level distance=\leveldistance,
					sibling angle=360/\the\tikznumberofchildren,
					% https://tex.stackexchange.com/questions/501240/trying-to-use-the-array-environment-inside-a-tikz-node-with-execute-at-begin-no
					execute at begin node=\definecolor{SwitchColor}{named}{SecondaryColor}\definecolor{SwitchColorDimmed}{named}{PrimaryColorDimmed},
				},
			level 2/.append style = {
					concept color=PrimaryColorDimmed,
					level distance=\leveldistance / 2,
					sibling angle=35,
					execute at begin node=\definecolor{SwitchColor}{named}{PrimaryColor}\definecolor{SwitchColorDimmed}{named}{SecondaryColorDimmed},
				},
			level 3/.append style = {
					concept color=SecondaryColorDimmed,
					level distance=\leveldistance / 3,
					execute at begin node=\definecolor{SwitchColor}{named}{SecondaryColor}\definecolor{SwitchColorDimmed}{named}{PrimaryColorDimmed},
				},
			level 4/.append style = {
					concept color=PrimaryColorDimmed,
					level distance=\leveldistance / 4,
					execute at begin node=\definecolor{SwitchColor}{named}{PrimaryColor}\definecolor{SwitchColorDimmed}{named}{SecondaryColorDimmed},
				},
			level 5/.append style = {
					concept color=SecondaryColorDimmed,
					level distance=\leveldistance / 5,
					execute at begin node=\definecolor{SwitchColor}{named}{SecondaryColor}\definecolor{SwitchColorDimmed}{named}{PrimaryColorDimmed},
				},
			level 6/.append style = {
					concept color=PrimaryColorDimmed,
					level distance=\leveldistance / 6,
					execute at begin node=\definecolor{SwitchColor}{named}{PrimaryColor}\definecolor{SwitchColorDimmed}{named}{SecondaryColorDimmed},
				},
			level 7/.append style = {
					concept color=SecondaryColorDimmed,
					level distance=\leveldistance / 7,
					execute at begin node=\definecolor{SwitchColor}{named}{SecondaryColor}\definecolor{SwitchColorDimmed}{named}{PrimaryColorDimmed},
				},
			level 8/.append style = {
					concept color=PrimaryColorDimmed,
					level distance=\leveldistance / 8,
					execute at begin node=\definecolor{SwitchColor}{named}{PrimaryColor}\definecolor{SwitchColorDimmed}{named}{SecondaryColorDimmed},
				},
			level 9/.append style = {
					concept color=SecondaryColorDimmed,
					level distance=\leveldistance / 9,
					execute at begin node=\definecolor{SwitchColor}{named}{SecondaryColor}\definecolor{SwitchColorDimmed}{named}{PrimaryColorDimmed},
				},
			concept connection/.append style = {
					color = BoxColor,
				},
		]
		}{
	\end{tikzpicture}
}

\newenvironment{mindmapcontent}{
	\begin{scope}[
			every node/.style = {concept, circular drop shadow}, % draw=none
			every child/.style={concept},
		]
		}{
		;\end{scope}
}

% ┌───────┐
% │ Boxes │
% └───────┘
\DeclareTotalTCBox{\inlinebox}{ s m }
{standard jigsaw,opacityback=0,colframe=SwitchColor,nobeforeafter,tcbox raise base,top=0mm,bottom=0mm,
	right=0mm,left=0mm,arc=0.1cm,boxsep=0.1cm}
{\IfBooleanTF{#1}%
	{\textcolor{PrimaryColor}{\setBold >\enspace\ignorespaces}#2}%
	{#2}}

\DeclareTotalTCBox{\inlineboxtwo}{ s m }
{standard jigsaw,opacityback=0,colframe=SwitchColorDimmed,nobeforeafter,tcbox raise base,top=0mm,bottom=0mm,
	right=0mm,left=0mm,arc=0.1cm,boxsep=0.1cm}
{\IfBooleanTF{#1}%
	{\textcolor{SwitchColorDimmed}{\setBold >\enspace\ignorespaces}#2}%
	{#2}}

% ┌──────────────────┐
% │ Case distinction │
% └──────────────────┘
% \newtoggle{absolute}
% % \toggletrue{absolute}
% \togglefalse{absolute}
% \newcommand{\lpathgraph}[1]{\iftoggle{absolute}{/home/areo/Documents/Studium/Summaries/x/}{./}#1}

% ┌───────┐
% │ Fixes │
% └───────┘
% https://tex.stackexchange.com/questions/89467/why-does-pdftex-hang-on-this-file
% \newcommand{\colon}{\mathrel{\mathop:}}

% ┌───────┐
% │ Paths │
% └───────┘
% \newcommand{\script}[2]{\href[page=#1]{}{\inlinebox{#2}}}
\newcommand{\script}[2]{\href{openpdf:/home/areo/Documents/Studium/Semester_1_Master/Hardware_Security_and_Trust/slides/Slides annotated/Hardware_Security_and_Trust_all_in_one.pdf:#1}{\inlinebox{#2}}}
\newcommand{\scripttwo}[2]{\href{openpdf:///home/areo/Documents/Studium/Semester_1_Master/Hardware_Security_and_Trust/slides/Slides annotated/bonus/12_Lecture_06Dec.pdf:#1}{\inlinebox{#2}}}
\newcommand{\videoeight}[2]{\href{https://youtu.be/YcHSlFjcndU?feature=shared&t=#1}{\inlineboxtwo{#2}}}
\newcommand{\videonine}[2]{\href{https://youtu.be/3dL-3EOIfJ8?si=l3OakqHOeCpnNayw&t=#1}{\inlineboxtwo{#2}}}
\newcommand{\videoten}[2]{\href{https://youtu.be/6oF737pa510?feature=shared&t=#1}{\inlineboxtwo{#2}}}
\newcommand{\videoeleven}[2]{\href{https://youtu.be/PJTqfzTIYJs?feature=shared&t=#1}{\inlineboxtwo{#2}}}
\newcommand{\videotwelve}[2]{\href{https://youtu.be/oDxAH7aO-Tk?feature=shared&t=#1}{\inlineboxtwo{#2}}}
\newcommand{\videothirteen}[2]{\href{https://youtu.be/3TkSXxe_Ty8?feature=shared&t=#1}{\inlineboxtwo{#2}}}
\newcommand{\videofourteen}[2]{\href{https://youtu.be/1Y3dZuJ0MHg?feature=shared&t=#1}{\inlineboxtwo{#2}}}


\begin{document}
\begin{mindmap}
	\begin{mindmapcontent}
		\node (middle) at (current page.center) {Machine Learning (Supervised Learning)}
		child [level distance=100cm] {
		node (prediction models) {Prediction Models}
		child {
				node {Non-parametric Models
						% \resizebox{\textwidth}{!}{
						% 	\begin{minipage}[t]{12cm}
						% 		\begin{itemize}
						% 			\item non-parametric models in contrast to the models that um we covered in the previous lecture content um we are not going to have prediction models that have parameters Theta instead the prediction models that we are going to see next are models that do not train Theta instead that they look directly at the data that we have collected in the past in order to make prediction so they are going to exploit the data and the logic is going to be based directly by Computing some information from the data instead of accumulating the knowledge into the parameters
						% 			\item such type of models that do not have their own parameters but instead they look up the data they are called otherwise as lazy models in the sense that they are lazy they don't train their own parameters but they look up the data directly
						% 			\item should understand that both nearest neighbor and naive bayes just did computations on the data
						% 			\item so that is different to for instance neural networks that have weights and that try to memorize things such type of models that we saw such as nearest neighbors and naive bayes they don't have memory they just look look into the values they do computations on the fly that's it that's how those nonparametric models work
						% 		\end{itemize}
						% 	\end{minipage}
						% }
					}
				child {
						node {Nearest Neighbors
								% \resizebox{\textwidth}{!}{
								% 	\begin{minipage}[t]{16cm}
								% 		\begin{itemize}
								% 			\item the logic of the nearest neighbor classifier is that it looks up all the training examples in your data set and finds the ones that are the most similar to the instance for which you want for which you want to make a prediction
								% 			\item so U you find the K nearest neighbors of X and you aggregate the target variables of those instances and when we say we want to find the nearest neighbors it means we want to find the neighbors that have the smallest distance and the distance is measured by a function D that takes as an input two feature vectors and computes their actual inverse of the similarities so the smaller is the D the more the more similar data points are
								% 			\item so basically we take the most likely option here which is one
								% 			\item we say that that the prediction is one because we computed the similarity of this new value nine to all the previous values in our recording data data set and that we found out that the most similar ones have a target of one so our prediction is one
								% 			\item compute the distance between this point and all the points
								% 			\item in this case it can be for example be the euclidean distance or square of the euclidean distance or even the absolute value can be used
								% 			\item that shows you the sensitivity actually of the nearest neighbor algorithm to the setting of how many neighbors you use
								% 			\item k for example is a hyper parameter based on the way you select K you are going to have different results in terms of your prediction and then based on that you are going to have different accuracies of your prediction
								% 			\item has motivations into way how humans reason, if don't know much about domain try to remember similar situations in the past
								% 			\item nearest neighbor is not always accurate and we saw that it is very sensitive to the choice of K and it is not immune to outliers
								% 			\item D is actually a set of tuples and the tuples have two vectors the feature and the target
								% 			\item so X is in the dimensionality of M many features y if it is a multi categorical can be a vector of one hot encoded Target values or it can be a single binary value or a single continuous value if it is a continuous variable
								% 			\item but in essence our data is composed of features and targets for n many instances that we have collected in the past
								% 			\item there is no other combination of three elements which are going to have a smaller sum than the ones that have the smallest distance so the sum of the three smallest values is always the subset of the data where the total sum is the smallest possible
								% 			\item so that's just a math definition of saying give me the instances X that achieve X actually the smallest possible distance to an item for which we want to make a prediction
								% 			\item hausing good example, most similiar hauses based on features of the house, e.g. compute euclidean distance between this feature vector
								%                   \item y nearest of x is just projection y into the y elements of the tuples of the nearest neighbors s of x
								% 			\item short summary: finds similar examples and predicts based on them
								% 		\end{itemize}
								% 		\begin{center}
								% 			\color{SwitchColor}\rule{14cm}{0.25mm}
								% 		\end{center}
								% 		\begin{itemize}
								% 			\item nearest neighbor iterated over training examples and count the most nearest neighbors
								% 		\end{itemize}
								% 	\end{minipage}
								% }
							}
						child {
								node {Aggregation for continuous target
										% \resizebox{\textwidth}{!}{
										% 	\begin{minipage}[t]{12cm}
										% 		\begin{itemize}
										%                     \item for continous target aggregation is for every element $\hat y$ in the set of the target values for the nearest neighbor of x that is basically the average of those values
										% 			\item if those three targets are in fact continuous values then the aggregation will take the average of them
										% 		\end{itemize}
										% 	\end{minipage}
										% }
									}
							}
						child {
								node {Aggregation for nominal targets (class c)
										% \resizebox{\textwidth}{!}{
										% 	\begin{minipage}[t]{12cm}
										% 		\begin{itemize}
										% 			\item nominal / categorical variable
										% 			\item want to make prediction for every class
										% 			\item for evey class prediction is basically: we say how many of the nearest neighbors y' belong to this category (cardinality of set of items that match the class) and we device it by the number of nearest neighbors
										% 			\item der mittlere | ist ein fuer die gilt ':'
										% 		\end{itemize}
										% 	\end{minipage}
										% }
									}
							}
					}
				child {
						node {Naive Bayes
								% \resizebox{\textwidth}{!}{
								% 	\begin{minipage}[t]{14cm}
								% 		\begin{itemize}
								% 			\item problem, this exact example never appears in our data, neither for positive nor negative class of covid
								% 			\item if this term is 0, then this probability becomes 0
								% 			\item so problem is: if have lots of variables then the chance of having coverage of all possible combinations becomes even smaller that have coverage of all possible combinations
								% 			\item example, 10 binary variables, then $2^{10}$ combinations of values, chance of having $2^{10}$ many patients and where all these combiantions of symptoms occur is very low, so don't have sufficient data to cover all those scenarios
								% 			\item symptoms help to get knowledge about the data
								% 			\item is there a way to make us of they (symptoms) even tough such combination does not occur directly in the data, naive bayes tells way how to do it
								% 			\item don't try to find occurrences where they all occur together but tread them as occurring alone and independently
								% 			\item multiply because... Multiplying is that they are all occurring individually in separation
								% 			\item replacing joint conditional probability with independent probabilities
								% 			\item if want to make prediction on y, then denominator is not needed as term in denominator does not depend on y
								% 			\item can compute just nominator part above for covid positive and covid negative and then check which one has the highest value
								% 			\item can drop denominator because which of the value of y is more likely instead of computing the exact probability of values
								% 			\item formula tells probabilty of y given of some additional factors / symptoms x1 to xn is initial probablity of y times a correction factor, how much it helps to know that x1 occured, if x1 occurs often with y it is a high probability, it helps it increases belief to knowing that y occurs
								% 			\item maybe some of the factors will increase probability, some other of them will decrease, in the end total product will tell some information whether those additional factors / variables they will actually increase of decrease initial believe about y happening
								% 		\end{itemize}
								% 		\begin{center}
								% 			\color{SwitchColor}\rule{12cm}{0.25mm}
								% 		\end{center}
								% 		\begin{itemize}
								% 			\item while naive bayes iterated over all the data to compute probabilities conditional variables subject to the target and the probability of the Target or the probability of the of the conditional variable occurring however it did not store any logic it just looked up at the table
								% 		\end{itemize}
								% 	\end{minipage}
								% }
							}
						child {
								node {Bayes Rule
										% \resizebox{\textwidth}{!}{
										% 	% \begin{minipage}[t]{12cm}
										% 	% 	\begin{itemize}
										% 	% 		\item probability of y happening, have additional information x, know a little bit more what factors contribute to y, probability of y occuring will either increase / decrease based on how x will contribute to y
										% 	% 		\item how does ratio work, nominator says what is likelyhood for x occuring among the occurencies of y
										% 	% 		\item in other words, if x is more likely to occur with y then it is likely to occur alone, then it means x is a contributing factor to y because it occurs more often when y occurs
										% 	% 		\item mit x kommt weitere Variable dazu, und wenn man was über diese Variable weiß kann man diese Information nutzen als wenn man normal einfach ne weitere Variable hinzufügt und Faktor 1 (unabhängigkeit) annimmt bei den neuen Paaren
										% 	% 	\end{itemize}
										% 	% 	\begin{center}
										% 	% 		\color{SwitchColor}\rule{12cm}{0.25mm}
										% 	% 	\end{center}
										% 	% 	\begin{itemize}
										% 	% 		\item are those factors if they occur alone more frequent or less frequent when they occur together with a value of y
										% 	% 		\item man tut von der conditional variable alle entsprechenden cases nehmen und schauen wie oft die eigentliche Variable darin ihren entsprechenden Wert hat
										% 	% 	\end{itemize}
										% 	% \end{minipage}
										% }
									}
							}
					}
			}
		child [level distance=20cm] {
				node {Linear Model
						% \resizebox{\textwidth}{!}{
						% 	\begin{minipage}[t]{12cm}
						% 		\begin{itemize}
						% 			\item cannot capture complex interactions between different features
						% 		\end{itemize}
						% 	\end{minipage}
						% }
					}
				child {
						node {Properties}
						child {
								node {High Bias}
							}
						child {
								node {Low Variance}
							}
					}
				child {
						node {Parameters}
						child {
								node {Coefficients of hyperplane}
							}
					}
				child {
						node {Derive weight parameters}
						child {
								node {Analytical solution}
								child {
										node {3 assumptions}
										child {
												node {Expected value of the residual errors is zero}
											}
										child {
												node {Residual errors are uncorrelated and share same variance}
											}
										child {
												node {Residual errors follow a normal distribution}
											}
									}
							}
						child {
								node {Iterative Optimization}
								child {
										node {Gradient Descent}
										child {
												node {Stochastic Gradient Descent}
											}
									}
								child {
										node {Gradients}
										child {
												node {Linear Regression}
											}
										child {
												node {Logistic Regression}
											}
									}
							}
					}
				child [level distance=15cm] {
						node {Regression}
						child {
								node {Fit data}
							}
						child {
								node {Basis function (for nonlinear mapping)}
								child {
										node {Characteristics of enhanced linear regression model}
										child {
												node {Still linear function of weights / parameters $w_i$}
											}
										child {
												node {Nonlinear function of input dimensions / variables $x_i$}
											}
									}
								child {
										node {Corresponds to Feature Extraction / encoding step} % in ML pipeline
									}
							}
						child {
								node {Characteristics}
								child {
										node {Linear function of weights / parameters $w\in \mathbb{R}^{D+1}$}
									}
								child {
										node {Linear function of input dimensions / variables $x\in \mathbb{R}^{D+1}$}
									}
							}
						child {
								node {Quality of model}
								child {
										node {L1 loss}
										child {
												node {Might be preferable if have many outliers}
											}
										child {
												node {Bad: Not differentiable at 0}
											}
									}
								child {
										node {L2 loss}
										child {
												node {Bad: Far away from prediction will be punished a lot}
												child {
														node {Single outlier change a lot prediction}
													}
											}
									}
								child {
										node {Huber loss}
										child {
												node {L2 between -1 and 1}
											}
									}
							}
					}
				child {
						node {Classification}
						child {
								node {Divide into 2 classes}
							}
						child {
								node {Logistic regression}
								child {
										node {Sigmoid / logistic function}
									}
								child {
										node {Loss function} % (for logistic regression)
										child {
												node {Formulation avoiding case distinction}
											}
										child {
												node {Good property: Is convex}
												child {
														node {Gradient descent works well}
													}
											}
										child {
												node {Bad property: No analytic solution}
											}
									}
							}
					}
			}
		child {
				node {Polynomial Regression
						% \resizebox{\textwidth}{!}{
						% 	\begin{minipage}[t]{12cm}
						% 		\begin{itemize}
						% 			\item all these terms called monimials
						% 			\item can capture relation between features
						% 			\item high cofficient downplay inportance of combnation of features
						% 		\end{itemize}
						% 	\end{minipage}
						% }
					}
			}
		child [level distance=30cm] {
				node {Support Vector Machines (SVM)}
				% child {
				% 		node {Pros and Cons}
				% 	}
				% child {
				% 		node {Loss/Response}
				% 		child {
				% 				node {Misclassification Rate}
				% 			}
				% 		child {
				% 				node {Accuracy}
				% 			}
				% 	}
				child [level distance=10cm] {
						node {Hyperparameters}
						child {
								node {Tradeoff $C$}
							}
						child {
								node {Choice of Kernel}
							}
						child {
								node {Kernel width $\gamma$}
							}
						child {
								node {Kernel degree $d$}
							}
					}
				child [level distance=15cm] {
						node {Linear SVM}
						child {
								node {Perceptron: Linear Classification Model}
								child {
										node {Linear Hyperplane}
										child {
												node {$w$ orthogonal to the Hyperplane}
											}
										child {
												node {Infinitely-many scaled $w, w_0$ yield same hyperplane}
											}
										child {
												node {Optimizing perceptron}
												child {
														node {Loss over miss-classified instances}
														child {
																node {Formalisation of Classification errors}
																child {
																		node {Linear classifier}
																		child {
																				node {Sign function}
																				child {
																						node {Produces class labels and restricts the range of the output}
																					}
																				child {
																						node {Weights of model define decision hyperplane}
																					}
																			}
																	}
															}
													}
												child {
														node {Gradient, Update by step for $w$, $w_0$}
													}
												child {
														node {Algorithm: Learning the Perceptron Model}
													}
												child {
														node {Problem: Which one is optimal hyperplane?}
													}
											}
									}
								child [level distance=10cm] {
										node {Maximum margin hyperplane}
										child {
												node {Objective function}
												child {
														node {Failed attempt: Minimum distance of all points}
														child {
																node {Margin will be increased to infinity}
															}
														child {
																node {Need to keep hyperplane between two classes}
															}
													}
												child {
														node {Ensure all data points correctly classified as constraints}
														child {
																node {$\lvert\lvert w\rvert\rvert_2$ out of inner minization}
															}
														child {
																node {Restrict infinite space of parameters $w, w_0$ to subset with margin of 1 unit}
																child {
																		node {Unifying 2 constraints into 1}
																	}
																child {
																		node {Converting objective to minimization}
																	}
															}
													}
												child {
														node {Problem: Linear Separability Assumption}
													}
											}
									}
								child {
										node {Slack margin, Tolerate mistakes}
										child {
												node {Objective function}
												child {
														node {Add amount of vialation $\epsilon_i$ to constraints}
													}
												child {
														node {Total amount of violations should be minimized}
														child {
																node {Increase C}
																child {
																		node {Avoid more to misclassify a point}
																	}
															}
														child {
																node {Decrease C}
																child {
																		node {More tolerance small misclassification errors around margin}
																	}
															}
													}
												child {
														node {Replacing $\epsilon_i$ by $max$-term}
													}
												child {
														node {Multiplying by $\frac{1}{C}$ and defining $\lambda = \frac{1}{C}$}
														child {
																node {$\lambda$-term becomes Regularization}
															}
														child {
																node {$max$-term becomes Loss}
															}
														child {
																node {Model inserted into $max$-term}
																child {
																		node {Classifier is $sign$-fuction around model}
																	}
															}
													}
												child {
														node {Can be solved with Gradient Descent}
													}
												child {
														node {Dual Optimization}
														child {
																node {Primal and Dual Form}
															}
														child {
																node {Simplify Dual Form by solving for $w$}
																child {
																		node {Also: Dual Prediction Model}
																	}
															}
														child {
																node {Inference on new point}
																child {
																		node {Only compute the dot product of point with support vectors}
																	}
															}
														child {
																node {Special algorithms for solving the dual formulation}
																child {
																		node {Platt}
																	}
															}
														% child {
														% 		node {Predict the label of new instance}
														% 		child {
														% 				node {Only compute similarity (dot product) with support vectors}
														% 			}
														% 	}
													}
											}
									}
								% child {
								% 		node {Regularized Hinge Loss Optimization}
								% 	}
								% child {
								% 		node {Linear SVM as a Regularized Loss}
								% 		child {
								% 				node {SGD (like Logistic Regression) but using the sub-gradient of loss}
								% 			}
								% 	}
							}
					}
				child {
						node {Non-Linear SVM}
						child {
								node {Applying nonlinear mapping $\varphi(x)$ to data}
								child {
										node {Make data linearly separable in higher dimensional space}
									}
								child {
										node {Replace dot product $x_i x_j$ by $\varphi(x_i) \varphi(x_j)$}
									}
							}
						child [level distance=10cm] {
								node {Kernel Trick}
								child {
										node {Replace dot product $\varphi(x_i) \varphi(x_j)$ by $K(x_i,x_j)$}
									}
								child {
										node {Computing (mapped) dot product expensive in high dimensional spaces}
										child {
												node {Kernel function computes dot product in higher dimensional space}
												child {
														node {Without computing the mapping $\varphi(\cdot)$}
													}
											}
									}
								child {
										node {Kernel functions}
										child {
												node {Linear kernel}
											}
										child {
												node {Polynomial kernel}
												child {
														node {Have to decide until which degree without overfitting data}
													}
											}
										child {
												node {Radial Basis Function (RBF)}
											}
									}
							}
					}
			}
		child {
		node {Decision Trees}
		child {
				node {Pros and Cons}
				child {
						node {Pros}
						child {
								node {Flexible framework with exchangeable components}
								child {
										node {Splitting criterion}
									}
								child {
										node {Leaf model}
									}
								child {
										node {Type of split}
									}
							}
						child {
								node {Interpretability}
								child {
										node {Can check decisions, understand better how works}
									}
							}
						child {
								node {Handle categorical input values natively}
							}
						child {
								node {Handle unimportant features well}
								child {
										node {Feature that does not reduce purity not used in decision tree}
									}
							}
						child {
								node {Scalable for large datasets}
							}
					}
				child {
						node {Cons}
						child {
								node {Tend to overfit}
							}
						child {
								node {Deterministic, i.e. not suitable for some ensemble methods}
							}
					}
			}
		child [level distance=10cm] {
				node {Properties}
				child {
						node {Change data $\rightarrow$ get very different tree}
					}
				child {
						node {Very expressive models}
					}
				child {
						node {High variance model}
					}
				child {
						node {Low bias model}
					}
			}
		child [level distance=20cm] {
				node (hyperparameters) {Hyperparameters}
				child {
						node {Minimum number of samples per split}
					}
				child {
						node {Minimum number of samples in a leaf (min\_leaf)}
					}
				child {
						node {Maximal depth of the tree (max\_depth)}
					}
				child {
						node {Total number of nodes}
					}
				child {
						node {Leaf model (weak learner; here constant)}
					}
				child {
						node {Split criterion}
					}
			}
		% child {
		% 		node {Pros / Advantages}
		% 		child {
		% 				node {Simple and interpretable}
		% 			}
		% 		child {
		% 				node {Can directly handle categorical features}
		% 			}
		% 		child {
		% 				node {Very flexible framework that can scale to large training data}
		% 			}
		% 	}
		% child {
		% 		node {Cons / Disadvantages}
		% 		child {
		% 				node {Are deterministic}
		% 			}
		% 		child {
		% 				node {Are prone to overfitting}
		% 			}
		% 	}
		child {
				node {Parameteres}
				child {
						node {Regions $R_1, \ldots, R_J$}
						child {
								node {One comes for free}
							}
					}
				child {
						node {Constant prediction models $\hat y^{(R_1)}, \ldots, \hat y^{(R_j)}$}
					}
				child {
						node {Step Function written as tree}
						child {
								node {Feature space partitioned into regions $R_j$}
								child {
										node {Each region $R_j$ own prediction value}
										child {
												node {Subset of data that belongs to that region}
												child {
														node {Region could also be points that are not part of dataset}
														child {
																node {Make computations on our dataset, so subset}
															}
													}
											}
										child {
												node {Formula}
											}
										child {
												node {Leaf model / Aggregation operation} % Regression or classification differernt aggregation operation
												child {
														node {Mean for Regression}
													}
												child {
														node {Majority vote or probability of data in the leaf (full distribution) for Classification}
													}
											}
									}
								child {
										node {$>1$ dimension feature) space}
										child {
												node {In decision tree, split after $x_1$ and then $x_2$}
											}
										child {
												node {In 2D both axis are features, color dots target axis}
											}
									}
							}
						child {
								node {Components of tree}
								child {
										node {Decision node}
										child {
												node {Every step in function has a decision node}
											}
									}
								child {
										node {Prediction leaf}
										child {
												node {Every flat horizonntal region has leaf that makes prediction}
											}
									}
							}
						child {
								node {K split just as series of binary splits}
								child {
										node {Binary splits easier computed and stored}
									}
							}
						child {
								node {Depth and height of tree}
							}
					}
			}
		child {
				node {Tree prediction}
				child {
						node {Formula}
						child {
								node {Indicator function}
							}
					}
				child {
						node {Computational complexity}
						child {
								node {Best case: $O(log(N))$}
							}
						child {
								node {Worst case: $O(N)$}
							}
					}
			}
		child [level distance=30cm] {
				node {Choosing splits that maximize purity}
				child {
						node {CART Algorithm}
						child {
								node {Build trees with more than 1 decision node}
								child {
										node {For training decision trees}
									}
							}
						child {
								node {Exhaustive Search accross all Decision Splits: $SearchSplit(D)$}
								child {
										node {Argmax Formula}
									}
								child {
										node {Runtime complexity is $O(N^2M)$ if naively implemented}
										child {
												node {$N$ many midpoints, num obervations}
												child {
														node {For several features one adds them up}
													}
											}
										child {
												node {$M$ many features}
											}
										child {
												node {Check condition for all $N$ points}
											}
										child {
												node {In CARD algoritm improvement used}
											}
									}
							}
						% child {
						%   node {Optimal split for Regression Tree}
						% }
						% child {
						%   node {Optimal split for Decision Tree}
						% }
						child {
								node {Computational complexity of Fitting}
								child {
										node {Best case: $O(M\cdot N\cdot log(N))$}
										child {
												node {Pay $O(M\cdot N)$ at each of $log(N)$ levels}
											}
									}
								child {
										node {Worst case: $O(M\cdot N^2)$}
									}
								child {
										node {Shift point from one subtree to other one}
										child {
												node {Don't need recompute full impurity of new created left and right subtrees}
											}
										child {
												node {Classification: Remove 1 point from histogram on right and add it to histogram on the left}
											}
										child {
												node {Regression: Substract this residual from left subtree and add to hight subtree}
											}
									}
							}
					}
				child [level distance=20cm] {
						node {Gain in purity $I(x \le v)$}
						child {
								node {$\lvert D\rvert\cdot H(D) - (\lvert D^{(L)}\rvert\cdot H(D^{(L)}) + \lvert D^{(R)}\rvert\cdot H(D^{(R)}))$}
								child {
										node {Maximize difference impurity before and after split}
									}
								child {
										node {Impurity after split should be as small as possible}
									}
								child {
										node {Balance between measuring purity split and sizes of subsets of data that belong to each of the splits}
										child {
												node {Capture effect more importance to subsets that have more elements}
												child {
														node {Multiply withh size of the sets}
													}
											}
									}
							}
						child {
								node {Impurity $H(D)$}
							}
						child {
								node {Split criterion}
								child {
										node {Variance reduction (minimum variance) for Regression Tree}
										child {
												node {Variance: $\lvert D\rvert \cdot H(D) = \sum_{(x, y)\in D} (y - \hat y(x))^2$}
														child {
																node {Impurity of subtree is residual between mean and observations (variance)}
																child {
																		node {Multiplied by size is just counting residuals}
																	}
															}
													}
											}
										child [level distance=10cm] {
												node {(Maximum) information gain for Decision (Classification) Tree}
												child {
														node {Entropy: $H(V) = -\sum_{k=1}^{K} p(v_k) \cdot log_2(p(v_k))$}
														child {
																node {$\lvert D\rvert \cdot H(D) = -N \sum_{v_k\in V} p(v_k)\cdot log_2(p(v_k))$}
															}
														child {
																node {K possible outcomes $v_k$}
															}
														child {
																node {Discrete random variable $V$}
															}
														child {
																node {Worst case: All probabilities for target variable outcomes is same}
															}
														child {
																node {Best case: All poins belong to 1 outcome}
																child {
																		node {Can't compute logarithm of 0, lim going to 0}
																	}
															}
														child [level distance=10cm] {
																node {Difference of a distribution to worst case scenario where all outcomes same probability}
															}
														child {
																node {Derivation over Kullback–Leibler (KL) divergence}
																child {
																		node {Measures how different 2 distributions are}
																	}
															}
													}
											}
										child {
												node {Gini index}
											}
									}
							}
					}
				child {
						node {Regularization (for Decision Trees)}
						child {
								node {Penalize number of leaves}
								child {
										node {Too many steps $\approx$ Too many leaves}
									}
							}
						child {
								node {Reduce maximum depth}
								child {
										node {Too large step jumps $\approx$ Too large leaves’ output values ($w$)}
									}
							}
						child [level distance=10cm] {
								node {Single decision tree}
								child {
										node {Minimum instances per leaf}
									}
								child {
										node {Maximum number of leaves}
									}
								child {
										node {Maximum tree depth}
									}
								child {
										node {Minimum gain for a split}
									}
								child {
										node {Penalty term on leaf outputs}
									}
							}
						child {
								node {Ensemble of decision trees}
								child {
										node {Number of trees (steps in boosting)}
									}
								child {
										node {Boosting additive weight}
									}
							}
					}
			}
		child [level distance=100cm] {
		node {Deeplearning}
		child {
				node {Deep means have lots of these connections between neurons}
			}
		child {
		node {Neural Network Architectures}
		% node {Neural Networks}
		child [level distance=30cm] {
		node {Multi-Layer Perceptrons (MLP) \tiny (Deep Forward Network (DFN), feedforward networks)}
		child {
				node {Feedforward, fully-connected, Network}
				child {
						node {Composite Functions (where every function is parameterized)}
					}
				child {
						node {Feedforward}
						child {
								node {\enquote{Input passes through chain of steps, every step is a neuron before outputting y}}
							}
						child {
								node {No backward connections or cycles}
								child {
										node {Cycles present $\rightarrow$ Recurrent Neural Networks}
									}
							}
						child {
								node {No intra-layer connections}
							}
					}
				child {
						node {Fully-connected}
						child {
								node {Each neuron in layer is connected to all neurons in next layer}
							}
					}
				child {
						node {Acyclic directed graph}
					}
			}
		child [level distance=20cm] {
		node {Single Neuron}
		child {
				node {Aggreegation: Operation defined over input values}
				child {
						node {Summation of linear model + bias is aggregation}
						child {
								node {Passed to activation function}
							}
						child {
								node {Linear model for every neuron}
							}
					}
				child {
						node {Num parameters on every neuron are num inputs +1 cuz of bias}
					}
			}
		child {
		node {Activation}
		child {
				node {Linear model with activation can become non-linear}
				child {
						node {If not use g after aggregation will be linear model}
						child {
								node {If multiply chain of matrices result still matrix}
							}
					}
				% hidden layers do it in a way by parametrizing this representations, projection to a new dimensionality is controlled with parameters of those hidden layers theta
				% with hidden layers learn a mapping, learn how to project original input to a new dimensionality (previous slide)
				% so the actual extraction of the features can be thought as a process that can be trained as a model
				% 3rd approach: don't know which features should use, but feature generation generating function can also be parametrized with theta
				% take pixels and then create polynomial expansions of pixels or rbf projection and those are the generic kernel expansion that you use for the SVM
				% hand-craft with heuristics
				% called representation learning: project features into some new features where linear model can then capturer target variable
				% if I create a new representation then in this new representation a linear function with some linear coefficients can actually approximate a function which originally was nonlinear with respect to the to its input
			}
		child {
				node {Activations per layer can be different}
			}
		child {
		node {Types}
		child [level distance=10cm] {
		node {Output (Layer) Activation fuctions}
		child {
				node (linear) {Linear (Transformation) / Identity}
			}
		child [level distance=10cm] {
		node (sigmoid) {Logistic Sigmoid (Binary Classification)}
		child {
				node {If z goes to 0, e to the power of 0 becomes 1, then it becomes $\frac{1}{1+1}$}
			}
		child [level distance=5cm] {
				node {If z goes to $\infty$, e to the power of $-\infty$ becomes 0, then it becomes $\frac{1}{1+0}$}
			}
		child {
				node {If z goes to $-\infty$, e to the power of $\infty$ becomes $\infty$, then it becomes $\frac{1}{1+\infty} = 0$}
			}
		child [level distance=5cm] {
		node {$\frac{1}{1+e^{-z}}$}
		}
		child {
				node {$\hat y$ between 0 and 1, probability}
			}
		child [level distance=5cm] {
				node {Naive option doesn't work, must be differentiable with respect to aggregation, derivative 0}
			}
		child {
				node {Gradient: $\frac{d\sigma(x)}{dx} = \sigma(x)(1-\sigma(x))$}
			}
		}
		child {
				node (softmax) {Softmax (Multi-Class Classification)}
				child {
						node {Num neurons output layer = Num classes}
					}
				child {
						node {Normalized probability estimation}
					}
				child {
						node {$e^z$ makes it positive}
						child {
								node {Dividing by sum of all total predictions makes it a distribution}
							}
					}
			}
		}
		child {
				node {Rectified linear unit (ReLU)}
				child {
						node {Instead of being a linear function, truncate negative values to 0}
						child {
								node {Keeps only non-negative values}
							}
					}
				child [level distance=10cm] {
						node {Benefit: Sparse out some of neurons in network, learn how to spezialize them}
					}
				child {
						node {$g(z) = max(0, z)$}
					}
			}
		child {
				node {Hyperbolic Tangent}
				child {
						node {Derivation of sigmoid}
						child {
								node {Range between -1 and 1}
								child {
										node {If range is between 0 and 1 than the negative can get is 2*0-1=-1}
									}
								child {
										node {If it has a maximum is 2*1-1=1}
									}
							}
					}
			}
		}
		}
		}
		child {
				node {Project N(l-1) to N(l) many values}
				% die Sache mit Dimensionen, rauscanceln
				% bedeutung eines Eintrage in Matrix wenn transpose und nicht transposed
				% column oder row ist linear model je nachdem ob transposed
				% die Sache mit oben und unten bei den indexes und Matrizen
				child {
						node {$w^{T}v$: (Left edge wT,  Upper edge wT) x (Left edge v, Upper edge v)}
						child {
								node {Inner dimensionalities cancel: (Left edge wT, Upper edge v)}
							}
						child {
								node {Weight matrix}
								child {
										node {Dimensionality: Left edge x Upper edge}
										child {
												node {Num Neurons last layer x Num Neurons current layer}
											}
										child {
												node {Transposed matrix: Num Neurons current layer x Num Neurons last layer}
											}
									}
								child [level distance=10cm] {
										node {$w_{i,j}$, move i down and j right, i-th row, j-th column}
										child {
												node {$w_{i,j}$ connects neuron $i$ (last layer) and $j$ (current layer))}
											}
										child {
												node {Transposed matrix: $w_{j,i}$ connects neuron $i$ (last layer) to $j$ (current layer))}
											}
									}
								child {
										node {Each column prediction models of linear function of a neuron in current layer}
										child {
												node {Transposed matrix: Each row}
											}
									}
							}
						child {
								node {Dim Bias: Num Neurons particular layer}
								child {
										node {For every neuron in particular layer have a bias}
									}
								child {
										node {If take $h^{(i)}\in \mathbb{R}^{d\times a}$ of particular n just $a$ remains}
										child {
												node {Can represent as $a\times1$}
											}
									}
							}
					}
			}
		child {
				node {Parametrize and learn Nonlinear Mapping}
				child {
						node {Hidden layers learn how project original input to new dimensionality}
						child {
								node {Representation Learning: Project features into new features where linear model can capture target variable}
								child {
										node {In new representation a linear function can approximate function which was nonlinear} % with some linear coefficients , respect to the to its input
									}
							}
					}
			}
		}
		child {
				node {Loss Function}
				child [level distance=10cm] {
						node {Continous Target, Regression}
						child {
								node {Interpretation: NN gives mean of the distribution of target y given input X}
								child {
										node {Most likely Target value y given X that is being trained as max likelihood estimation}
									}
							}
						child {
								node (no activation) {No Activation for output layer}
							}
						child {
								node {Least Square Error Loss}
							}
					}
				child {
						node (binary) {Binary Classification  / Binary Classification Target}
						child {
								node {Binary target variables follow Bernoulli distribution}
							}
						child {
								node {Logistic Loss or the Cross-entropy}
								child {
										node {Only of term non-zero: $-0\cdot log(\hat y) - (1-0)\cdot log(1-\hat y)$}
									}
								child {
										node {left $-1\cdot log(1) = 0$, right $-(1-0)\cdot log(1-0) = 0$}
									}
								child {
										node {left $-1\cdot log(0) = \infty$, right $-(1-0)\cdot log(1-1) = \infty$}
									}
							}
					}
				child [level distance=20cm] {
						node (multi-class) {Multi-Class Classification / Multi-category Target}
						child {
								node {Log-likelihood Loss}
								child {
										node {Function only becomes 1 for true class and 0 for all other classes}
										child {
												node {Only for case where $y_n=1$ wants probability 1, cuz $log(1)=0$}
											}
									}
							}
					}
			}
		}
		child [level distance=50cm] {
				node {Optimizing Neural Networks / Optimization Techniques for Neural Networks}
				child {
						node {Gradient is derivative of function j of w with respect to each of the w's, tells tangent line}
						child {
								node {Joined space of all w's}
							}
						child {
								node {Derivative tells direction in which the function goes up}
								child {
										node {If change w one step to left then positive gain (increase) in J(w)}
									}
								child {
										node {Gives direction and magnitude of how big the slope is}
									}
								child {
										node {If gradient small then probably very close to minimum}
									}
								child {
										node {For Minimize have to go in opposite direction of slope}
									}
							}
					}
				child [level distance=20cm] {
						node {Algorithms to find optimal parameters}
						child [level distance=20cm] {
								node {Gradient Descent}
								child {
										node {Move all weights one step towards negative direction of derivative of J with respect to W}
									}
								child {
										node {Hyperparamater: step size / learning rate}
									}
								child {
										node {Most often just x number of epochs}
										child {
												node {Other convergence criteria: Difference between the next - previous less epsilon}
											}
									}
								child {
										node {Problem: Very expensive operation}
										child {
												node {Derivative with respect to all parameters and summation over all points in training set}
											}
									}
							}
						child {
								node {Stochastic Gradient Descent}
								child {
										node {Union of those mini-batches gives training set (but not disjoint)}
										child {
												node {Mini-batches can have repeating instances maybe located in several mini-batches}
											}
									}
								child {
										node {Random order: There's more chance one avoids local optima}
										child {
												node {Because gradients are not going to always follow same sequence}
											}
										child {
												node {Not the same sequence of subsets at every iteration}
											}
									}
								child {
										node {Optimize parameters w to minimize the errors on this mini-batches}
									}
								child {
										node {Algorithm: SGD for Neural Networks}
										child {
												node {Forward pass}
												child {
														node {Compute the loss given paramters x, input features to NN}
													}
											}
                      child [level distance=10cm] {
												node {Backpropagation}
												child {
														node {Finds gradient of loss function with respect to all weights (and biases)}
													}
												child {
														node {Chain rule of calculus}
														child {
																node {Going over other function}
															}
													}
												child {
														node {Partial derivative}
														child {
																node {If h depends on 2 functions}
															}
														child {
																node {Symbol of partial derivative identifies that now take derivative of h with respect to another function f}
															}
														% child {
														% 		node {Partial derivative denoted with gradient of h with respect to g and f}
														% 	}
														child {
																node {Total derivative rule}
																child {
																		% node {Total derivative is summation of paths that contribute to how J depends on X}
																		node {Total derivative is summation of paths through which X is contributing to J}
																		% child {
																		% 		node {Shortcut: Need accumulated gradients of the final function with respect to gradient of X and immediate next function f}
																		% 	}
																		child {
																				% node {Shortcut: Parent node or function of x accumulated all paths}
																				node {Shortcut: Accumulate paths up to an intermediate parent level of variable x}
																				child {
																						node {Accumulated gradients already capture all paths}
																					}
																				child {
																						node {Reshare terms if go one layer at a time backwards}
																					}
																				child {
																						node {Left Gradients in chain rule have higher layer in NN}
																					}
																				% child {
																				%   node {Accumulate paths up to an intermediate parent level of variable x with respect to which want to compute the derivatives}
																				% }
																			}
																	}
															}
													}
												child [level distance=15cm] {
														node {Real Neural Network}
														child {
																node {Chain rule decomposition}
																child {
																		node {Gradient of loss with respect to this neuron in which my weight belongs}
																		child {
																				node {Have to account all paths that go from that neuron to final function}
																				child {
																						node {Go through contributions of the neurons at layer l+1}
																						child {
																								node {If I go from last layer to one layer before then can reshare computations}
																							}
																					}
																			}
																	}
																child {
																		node {Gradient of neuron to that particular weight}
																	}
															}
                              child [level distance=10cm] {
																node {Gradient of final function with respect to parameters of a particular function}
																child {
																		node {Final function is loss of the estimated targets (final layer)}
																	}
															}
														child {
																node {Functions that depend on x but also have a variable or parameter}
																child {
																		node {Theta first influences it's own function and then this function influences next function}
																	}
																% and all those paths from this next function j to n they are just the gradient of n with respect to j how can theta go to j, 
															}
													}
												child {
														node {Algorithm}
														child {
																node {G is within sum, G' is sum itself (sum if >1 successor neuron)}
																child {
																		node {G is always with respect to current / previous neuron h}
																	}
																child {
																		node {G' becomes G for predecessor layer}
																	}
															}
														child {
																node {Line 12 special case}
																child {
																		node {Bias not connected to any neuron from the previous layer}
																		child {
																				node {Derivative is just $g' \cdot 1$}
																				child {
																						node {as inside activation $g$, $w_0$ not multiplied with anything}
																					}
																			}
																	}
															}
														child {
																node {One can calculate G's of all layers first}
																child {
																		node {and then multiply all G' with gradient of neuron with respect to weight}
																	}
															}
													}
											}
										child {
												node {1 forward and backward pass for every mini-batch}
											}
									}
							}
					}
			}
		child {
				node {Regularization approaches for Neural Networks}
				child [level distance=80cm] {
						node {Families of Regularization Techniques for DFN}
						child {
								node {Capacity}
								child {
										node {Number of parameters}
										child {
												node {Smaller capacity is more regularization}
												child {
														node {Worse fit training set, higher bias, smaller variance}
													}
											}
									}
							}
						child {
								node {Structural}
								child {
										node {Different connectivity patterns}
									}
								child {
										node {(NAS)}
									}
								child {
										node {Skip connections}
									}
								child {
										node {(ResNet)}
										child {
												node {2 parallel layers}
											}
									}
								% etc.
							}
						child {
								node (weight decay) {Weight decay}
								child {
										node {Have derivative of empirical risk and of regularization term}
									}
								child {
										node {L1}
										child {
												node {Derivative: Case destinction 1 and -1}
												child {
														node {Include 0 into negative to have applicable case for all w}
													}
											}
									}
								child {
										node {L2}
										child {
												node {When multiply alpha/2 by derivative, becomes just alpha}
											}
									}
							}
						child {
								node {Ensembling}
								child {
										node {Dropout}
									}
							}
						child {
								node {(Weight Aggregation)}
								% child {
								%   node {Aggregate weights from candidate nn prediction models}
								% }
								child {
										node {(Stochastic Weight Averaging)}
									}
								child {
										node {(Snapshot Ensembles)}
									}
							}
						child {
								node {Learning dynamics}
                child [level distance=10cm] {
										node {Early stopping}
										child {
												node {Controlling dynamics of optimization}
											}
										child {
												node {Overfit to noise present in training set}
												child {
														node {Performance new validation set gets worse}
														child {
																node {Validation set not used to update model parameters}
															}
														child {
																node {In beginning both decrease}
															}
													}
											}
										child {
												node {Stop early optimization before starts to overfit}
												child {
														node {Done by checking performance on validation set}
													}
											}
                      child [level distance=10cm] {
												node {Algoritm}
												child {
														node {Stop if p many steps where no reduction in validation set error}
														child {
																node {p many steps = tolerance}
															}
														child {
																node {Steps called epochs}
															}
														child {
																node {Tolerance is needed cuz performance oscillates}
															}
													}
												child {
														node {j counts steps seen without improvement}
													}
												child {
														node {Check validation set error of current w}
														child {
																node {If it does not improve ones previous best v, then increase counter}
															}
														child {
																node {Otherwise one resets counter, stores the best parameters}
															}
													}
												child {
														node {If counter exceeds tolerance number steps one returns best found}
													}
											}
										child {
												node {Acts as a regularizer}
												child {
														node {Stop in between before going to optimum}
													}
											}
									}
								child {
										node {(Learning rate schedulers)}
										child {
												node {Learning rate can be dynamic}
												child {
														node {Can be adapted on every paramete}
													}
											}
										child {
												node {Specify scheduler how initial learening rate change after every epoch}
											}
									}
							}
						child {
								node {Implicit}
								child {
										node {Batch normalization}
									}
								% etc.
							}
					}
				child {
						node {Hyperparameters}
						child {
								node {Number of hidden layers}
								child {
										node {Depth network = number hidden layers}
									}
								child {
										node {The deeper network the higher complexity}
										child {
												node {The higher it's complexity, the more will be low bias and high variance model}
											}
									}
								child {
										node {Number of neurons per layer}
										child {
												node {Width network}
												child {
														node {Also num neurons in every layer defines complexity, but not same as depth}
													}
											}
									}
								child {
										node {Choice of activation function}
									}
								child {
										node {Dropout rate}
									}
								child {
										node {Regularization penalty on weights}
									}
								child {
										node {Learning rate}
									}
								child {
										node {Choice of optimization algorithms}
									}
							}
					}
			}
		% child  {
		% 		node {Neural Networks}
		% 		% change connections between neurons in a way that output matches as close as possible to the true target y
		% 		child {
		% 				node {}
		% 			}
		% 	}
		% child [level distance=50cm] {
		% 		node {Neural Networks}
		% 		child {
		% 				node {Loss/Response}
		% 				child {
		% 						node {Misclassification Rate / Accuracy}
		% 					}
		% 				child {
		% 						node {MSE}
		% 					}
		% 			}
		% 		child {
		% 				node {Regularization Techniques}
		% 				child {
		% 						node {Specific to Neural Networks}
		% 						child {
		% 								node {Structural and Linearization}
		% 								child {
		% 										node {Skip connections (Across layers)}
		% 									}
		% 								child {
		% 										node {Residual blocks}
		% 										child {
		% 												node {Skip connections from all preceding layers to all succeeding layers}
		% 											}
		% 										child {
		% 												node {Make the gradient flow backwards}
		% 												child {
		% 														node {Avoiding vanishing / exploding gradient}
		% 													}
		% 											}
		% 										child {
		% 												node {Network can learn not use some layers by using identity layers}
		% 											}
		% 									}
		% 								child {
		% 										node {Shake shake}
		% 										child {
		% 												node {Randomly weight the paths for each mini-batch}
		% 											}
		% 									}
		% 							}
		% 						child {
		% 								node {Model Averaging}
		% 								child {
		% 										node {Dropout}
		% 										child {
		% 												node {Bagging of random neural subnetworks}
		% 												child {
		% 														node {Aims at reducing variance (error) of model}
		% 													}
		% 											}
		% 										child {
		% 												node {Dropout node multiplying output by zero}
		% 												child {
		% 														node {Only input and hidden nodes dropped out}
		% 													}
		% 												child {
		% 														node {Input unit included probability 0.8}
		% 													}
		% 												child {
		% 														node {Hidden unit probability 0.5}
		% 													}
		% 												child {
		% 														node {Computations}
		% 														child {
		% 																node {Forward computation}
		% 															}
		% 														child {
		% 																node {Back-propagation}
		% 																child {
		% 																		node {As usual multiplying activations by mask}
		% 																	}
		% 															}
		% 														child {
		% 																node {Inference}
		% 															}
		% 													}
		% 											}
		% 									}
		% 							}
		% 					}
		% 				child [level distance = 23cm] {
		% 						node {General methods}
		% 						child {
		% 								node {Weight decay}
		% 								child {
		% 										node {L2 Regularization}
		% 										child {
		% 												node {See older NN slides}
		% 											}
		% 									}
		% 								child {
		% 										node {L1 Regularization}
		% 										child {
		% 												node {See older NN slides}
		% 											}
		% 									}
		% 							}
		% 						child {
		% 								node {Data Augmentation}
		% 								child {
		% 										node {Apply random data augmentation online}
		% 										child {
		% 												node {Create different version of same sample at different epochs}
		% 											}
		% 										child {
		% 												node {Methods}
		% 												child {
		% 														node {Domain-specific manual policies}
		% 														child {
		% 																node {Cutout }
		% 															}
		% 														child {
		% 																node {Mixup}
		% 															}
		% 														child {
		% 																node {Cutmix}
		% 															}
		% 													}
		% 												child {
		% 														node {Trainable (data) augmentation policies/algorithms}
		% 														child {
		% 																node {A policy consists of 5 sub-policies}
		% 																child {
		% 																		node {Each sub-policy 2 image operations applied in sequence}
		% 																	}
		% 															}
		% 														child [level distance = 10cm] {
		% 																node {Image Augmentation Operations}
		% 																child {
		% 																		node {ShearX/Y}
		% 																	}
		% 																child {
		% 																		node {TranslateX/Y}
		% 																	}
		% 																child {
		% 																		node {Rotate}
		% 																	}
		% 																child {
		% 																		node {AutoContrast}
		% 																	}
		% 																child {
		% 																		node {Invert}
		% 																	}
		% 																child {
		% 																		node {Equalize}
		% 																	}
		% 																child {
		% 																		node {Solarize}
		% 																	}
		% 																child {
		% 																		node {Posterize}
		% 																	}
		% 																child {
		% 																		node {Contrast}
		% 																	}
		% 																child {
		% 																		node {Color}
		% 																	}
		% 																child {
		% 																		node {Brightness}
		% 																	}
		% 																child {
		% 																		node {Sharpness}
		% 																	}
		% 																child {
		% 																		node {Cutout}
		% 																	}
		% 																child {
		% 																		node {Sample Pairing}
		% 																	}
		% 															}
		% 													}
		% 											}
		% 									}
		% 							}
		% 						child {
		% 								node {Learning Dynamics}
		% 								child {
		% 										node {Early Stopping as a Regularizer}
		% 										child {
		% 												node {See older NN slides}
		% 											}
		% 									}
		% 							}
		% 					}
		% 			}
		% 		child {
		% 				node {Activation}
		% 				child {
		% 						node {Types}
		% 						child {
		% 								node {Rectified linear unit (ReLU)}
		% 							}
		% 						child {
		% 								node {Logistic Sigmoid Sigmoid (Binary Classification)}
		% 							}
		% 						child {
		% 								node {Softmax (Multi-Class Classification)}
		% 							}
		% 						child {
		% 								node {Hyperbolic Tangent}
		% 							}
		% 						child {
		% 								node {Linear}
		% 							}
		% 					}
		% 				child {
		% 						node {Importance of Non-linear activations}
		% 					}
		% 				child {
		% 						node {Nonlinear Mapping into new space (hidden layers) for nonlinear problems}
		% 					}
		% 			}
		% 		child {
		% 				node {Optimizing Neural Networks}
		% 				child {
		% 						node {Inference}
		% 					}
		% 				child {
		% 						node {Tranining}
		% 						child {
		% 								node {Compute gradients}
		% 								child {
		% 										node {Backpropagation}
		% 										child {
		% 												node {Algorithm}
		% 											}
		% 										child {
		% 												node {Forward pass}
		% 											}
		% 										child {
		% 												node {Backward pass}
		% 											}
		% 									}
		% 								child {
		% 										node {Chain-rule of Calculus}
		% 										child {
		% 												node {Fuction composition because of feedforward}
		% 											}
		% 									}
		% 								child {
		% 										node {Total Derivative}
		% 									}
		% 							}
		% 						child {
		% 								node {Update weights}
		% 								child {
		% 										node {SGD for Neural Networks}
		% 										child {
		% 												node {Algorithm}
		% 											}
		% 									}
		% 							}
		% 						% child {
		% 						% 		node {Gradient Descent}
		% 						% 	}
		% 						% child {
		% 						% 		node {Stochastic Gradient Descent}
		% 						% 	}
		% 					}
		% 			}
		% 	}
		child [level distance=20cm] {
				node {Convolutional Neural Networks}
				child {
						node {Analogy neural network}
						child {
								node {CNN: feature maps in, kernels being applied, feature maps out}
								child {
										node {In 3 aspects different}
										child {
												node {Spacity: no full connection between feature maps}
												child {
														node {Just have small sectional patch}
													}
											}
										child {
												node {Receptive field is hierarchically growing: See patterns occur in large portion of original image}
											}
										child {
												node {Reuse same weights, just learn one kernel}
												child {
														node {Same weights applied to each those patch locations}
													}
												child {
														node {Don't learn one weight for every pixel in feature map}
													}
											}
									}
							}
              child [level distance=8cm] {
								node {NN: activations in, weights w applied, activations out}
							}
					}
          child [level distance=70cm] {
						node {Parts}
						child {
								node {Image}
								child {
										node {Cuz it's 3-dimensional kernel also needs to be}
										child {
												node {In case of rgb depth 3}
											}
										child {
												node {Iteration over channel, num rows and columns}
											}
									}
							}
              child [level distance=20cm] {
								node {Kernels}
								child {
										node {Is 3-dimensional structure}
									}
								child {
										node {In case of convolution it's the same convolution being applied}
										child {
												node {Like NN's but weights of every neuron same}
												child {
														node {In neural networks each neuron has own weights}
													}
											}
										child {
												node {Don't look same because input different}
											}
									}
                  child [level distance=10cm] {
										node {Convolution: Sum of pairwise product all cells kernel with cells image}
										child {
												node {Connect only to neurons that are located nearby in a patch}
												child {
														node {Extract pattern that have a neighborhood relation}
														child {
																node {Extract particular aspects of image (patterns, edges)}
																child {
																		node {Detect local presences of spatially related patterns}
																	}
															}
													}
												child {
														node {Reducing number of connectivity}
														child {
																node {Reduces complexity drastically}
															}
													}
												child {
														node {Receptive field}
														child {
																node {If stack many layers on top of each other}
																child {
																		node {A very deep layer has interaction large array of pixels in original representation}
																	}
															}
													}
											}
										child {
												node {Number of rows and columns after applying convolutional not same as original image}
												child {
														node {Cannot move convolution outside of borders of image}
													}
											}
										child {
												node {Start index x + index on the patch, m going from 1 to m(l)}
												child {
														node {Applied on all the channels}
													}
												child {
														node {-1 because start from 1 and want have very first point x,y}
													}
											}
									}
								child {
										node {Once have convolution also apply non-linear activation}
										child {
												node {E.g. ReLU}
												child {
														node {Helps focus more on shapes and less on contrast}
													}
											}
									}
								child {
										node {Dim kernels, had I(l) many kernels, each those kernels had same depth I(l-1) with number channes of incoming image}
									}
								child {
										node {Kernel tensor of 4 dimensions}
									}
							}
						child {
								node {Feature Maps}
								child {
										node {is 2-dimensional}
									}
								child {
										node {Feature map is neuron in next layer}
										child {
												node {Input feature map is activations of previous layer}
											}
										child {
												node {Incoming activation previous layer not vector, two-dimensional matrix}
												child {
														node {Cuz patches being applied for every row and every column}
													}
											}
									}
								child {
										node {Feature maps as input, beginning image itself}
										child {
												node {For subsequent layers input is feature map of previous convolutional layer}
											}
										% child {
										% 		node {As many feature maps as kernels applied} % on original image
										% 	}
										child {
												node {For every kernel generate one feature map}
											}
									}
								child {
										node {Feature map always dim of number kernels applied previous layer and resulting dim image resolution}
									}
								child {
										node {Is output of convolution + non-linear activation}
									}
							}
						child [level distance=10cm] {
              node {Parameters}
								child {
										node {V are feature maps, equivalent of activations h}
									}
								child {
										node {I many kernels at every layer}
										child {
												node {NN's: M many neurons every layer}
											}
										child {
												node {I(0) number of (original) channels c, greyscale: 1, RGB: 3}
											}
										child {
												node {M is size of applied patch} % M is resolution of feature map
											}
										child {
												node {Actually x(l) and y(l) cuz can change at every layer}
												child {
														node {For simplicity assume resolution is same at every layers, same xy}
														child {
																node {Think of filled with 0's to preserve resolution}
															}
													}
											}
									}
							}
					}
				child {
						node {Identifying images}
						child {
								node {Extract image representations by CNN}
							}
						child {
								node {Prediction layer}
								child {
										node {Input: image representation, converts to categories (for classification)}
										child {
												node {Last layer as classifier for images}
												child {
														node {Flatten cells across all dimensions as vector, $i\cdot x\cdot y\times 1$ vector}
													}
												child {
														node {Flattened tensor can input it to standard DFN}
													}
											}
									}
							}
						child {
								node {Need derivative of loss with respect to all the kernels}
								% child {
								% 		node {Keep Gradient of J with respect to V(l) as gradient G and push it backwards}
								% 	}
								child {
										node {Have to start from prediction layer}
										child {
												node {G multiplied with gradient V(l) with respect to K(l)}
												child {
														node {G: Sum of G' multiplied with gradient of h(1) with respect to V(l)}
                            child {
                              node {G is gradient of J with respect to V(l)}
                              child {
                                node {in NN: h(l-1)}
                              }
                            }
														child {
																node {G': Gradient J with respect to h(1) of DFN} % 1st hidden layer
															}
													}
                        % child {
                        %   node {Only need on gradient of J with respect to kernel, cuz all the same}
                        % }
                        child {
                          node {Push backwards for gradient of J with respect to l-1 feature map, l-2 feature map etc.}
                          child {
                            node {Once have dJ/dV can compute gradient of loss with respect to kernels}
                          }
                        }
											}
									}
							}
					}
				% child {
				% 		node {Parameter Sharing or Tied-Weights (sharing weights)}
				% 		child {
				% 				node {Advantages}
				% 				child {
				% 						node {Smaller models}
				% 					}
				% 				child {
				% 						node {Lower complexity}
				% 					}
				% 				child {
				% 						node {Better generalization}
				% 					}
				% 			}
				% 	}
				% child {
				% 		node {Multi-layered Sparse Connectivity}
				% 		child {
				% 				node {Translation-equivariant capturing of patterns}
				% 			}
				% 	}
				% child {
				% 		node {Stacked Convolutions with Nonlinear Activations}
				% 	}
			}
		}
		}
		child {
				node {Objective Function}
				child [level distance=50cm] {
						node (loss) {Loss Functions}
						child {
								node (estimated target minus ground truth) {Estimated target $\hat y$ - Ground-truth target $y$}
								child {
										node {Maximum Likelihood Estimation
												% \resizebox{\textwidth}{!}{
												% 	\begin{minipage}[t]{12cm}
												% 		\begin{itemize}
												% 			\item probabilistic logic behind loss functions
												% 			\item minimizing least square error is equivalent to maximizing likelihood of observing the target given the data
												% 			\item and the mean of the errors is located at the estimated target
												% 			\item on average the highest density would be at the predicted value and then there's where you would have the smallest likelihood for an error, at the point of the estimation
												% 			\item mean is prediction of this linear model
												% 			\item mean value of predictions is going to be output of prediction model
												% 			\item likelihood of predicting y given X and the prediction model should be as large as possible
												% 			\item should be large on all the measurements that we have seen so it should be large on the first data point given the first observed features and the predictions of a model and it should be high on the second data point it should be high as well on all the other data points
												% 			\item have to maximize this likelihood in order for our prediction model to be accurate in estimating y
												% 			\item average value in normal distribution replaced with output of prediction model
												% 			\item as you see here maximizing the probabilistic likelihood of observing the targets if you assume that the errors are normally distributed that is equivalent to just learning a prediction model that minimizes the least square error
												% 			\item what does it mean to optimize the prediction model to minimize the sum of the losses of observed instances in other words to minimize the amount of error that we are doing in our prediction
												% 		\end{itemize}
												% 	\end{minipage}
												% }
											}
									}
								child {
										node {Regression}
										child {
												node {Least Squares}
											}
										child {
												node {L1}
											}
									}
								child {
										node {(Multi-class) Classification}
										child {
												node {Logloss}
												child {
														node {Softmax
																% \resizebox{\textwidth}{!}{
																% 	\begin{minipage}[t]{12cm}
																% 		\begin{itemize}
																% 			\item re-exrpess target variable as being as many binary targets as there categories
																% 			\item learrn one prediction model for every category
																% 			\item then connect together with (this) formula
																% 			\item prediction for class c and then normalize by sum of probabilies of all classes together
																% 			\item that ensures us that we have probability distribution of C
																% 			\item sum of probabilities is going to sum to 1
																% 			\item each between 0 and 1, because every component smaller than sum of all components
																% 			\item how well does proability for the 6th class match the true target of the sixth class, is either 1 or 0 because of definition above
																% 			\item if is 1 then want this log part be close to 0 by making ync be close to 1
																% 			\item this means this loss function is going to be small only in cases where prediction matches exactly the case of the instance of the class where it is one
																% 		\end{itemize}
																% 	\end{minipage}
																% }
															}
													}
											}
										child {
												node {Binary Classification}
												child {
														node {Logistic Loss
																% \resizebox{\textwidth}{!}{
																% 	\begin{minipage}[t]{12cm}
																% 		\begin{itemize}
																% 			\item if is 1, left handside, log($\hat y$) goes to 0 if $\hat y$ goes to 1, for values that go towards 0 the logarithm will be negative quantity, - in front, so large positive value, measure high error in cases where $\hat y$ is 0 and true y was one, in cases where true y was 1 and $\hat y$ was close to 1, log(1) comes 0, so do 0 error
																% 		\end{itemize}
																% 	\end{minipage}
																% }
															}
													}
												child {
														node {Hinge Loss}
													}
											}
									}
							}
						child [level distance=20cm] {
								node {Regularisation}
								child {
										node {Increase bias and reduce variance to decrease overall test error}
									}
								% child {
								% 		node {Add penalty term to empirical risk}
								% 	}
								% child {
								% 		node {Combining Regularizers as a Cocktail}
								% 		child {
								% 				node {Better than singular regularizer}
								% 			}
								% 	}
								child {
										node {Bias-Variance Tradeoff}
										child {
												node {Bias and Variance about family of models, not single model}
											}
										child {
												node (expected test error) {Expected test error}
												child {
														node {Variance}
														child {
																node (estimated model) {Estimated model}
															}
														child {
																node (expected prediction model) {Expected prediction model}
															}
														child {
																node {High variance}
																child {
																		node {Risk Overfitting}
																	}
															}
														child {
																node {Low variance}
																child {
																		node {Generalization}
																	}
															}
													}
												child [level distance=20cm] {
														node (bias) {Bias$^2$}
														child {
																node (expected target) {Expected Target}
																child {
																		node (expected value) {Expected Value}
																		child {
																				node {Properties of Expectation}
																			}
																	}
															}
														child {
																node {High bias}
																child {
																		node {Risk Underfitting}
																	}
															}
														child {
																node {Low bias}
																child {
																		node {Fitting}
																	}
															}
													}
												child {
														node (noise) {Noise}
														child {
																node {Real Target}
															}
														child {
																node {High noise}
																child {
																		node {Challenging Task}
																	}
															}
														child {
																node {Low noise}
																child {
																		node {Easy Task}
																	}
															}
													}
												child [level distance=15cm] {
														node {Model complexity will either}
														child {
																node {Increase bias}
																child {
																		node {Decrease variance}
																	}
															}
														child {
																node {Increase variance}
																child {
																		node {Decrease bias}
																	}
															}
													}
												child {
														node {Goal low bias and low variance}
													}
											}
									}
								child [level distance=40cm] {
										node {Generalization Performance}
										child {
												node {Underfitting}
												child {
														node {Model fails to fit the training data}
													}
												child {
														node {Both training and test errors descend, but get stuck}
													}
											}
										child {
												node {Generalizing}
												child {
														node {Model is accurate on test data}
													}
												child {
														node {Both training and test errors descend and approach a minimum}
													}
											}
										child {
												node {Overfitting}
												child {
														node {Model perfectly fits training data (incl. noise)}
													}
												child {
														node {Training error getting smaller, while test error increases}
													}
											}
										child {
												node {Training data and Test data}
											}
									}
								child {
										node {General Regularization Techniques}
										child [level distance=15cm] {
												node {Ensembling, Ensembles}
												child {
														node {Reduces variance}
													}
												child [level distance=15cm] {
														node {Bagging}
														child {
																node {Short for Bootstrap AGGregatING}
															}
														child {
																node {Sampling with replacement}
																child {
																		node {Reduce the final model’s variance}
																	}
																child {
																		node {Without affecting the bias much}
																	}
															}
														child {
																node {Requirements}
																child {
																		node {Low bias individual models}
																	}
																child {
																		node {Uncorrelated individual models}
																		child {
																				node {The errors of the models should be uncorrelated}
																				child {
																						node {So that the overall error gets reduced}
																					}
																			}
																	}
																child {
																		node {Bagged Trees}
																		child {
																				node {Low bias but might be correlated}
																			}
																		child {
																				node {Decorrelation by randomizing choice of features to split upon}
																			}
																	}
																child {
																		node {Random Forests}
																		child {
																				node {Randomness in terms of the split conditions}
																			}
																		child {
																				node {Algoritm}
																			}
																	}
															}
													}
												child {
														node {Boosting}
														child {
																node {Aggregate high-bias models to create a low-bias ensemble}
																child {
																		node {Train parameters next model keeping past models fixed}
																	}
															}
														child {
																node {Loss can be linearly approximated via the first-order Taylor expansion}
															}
														child {
																node {Algorithm: Gradient Boosting}
															}
														child {
																node {Algorithm: AdaBoost}
															}
														child {
																node {Gradient Boosted Decision Trees}
																child {
																		node {Hyperparameters}
																		child {
																				node {$\gamma$ is the weight of the penalty on the number of leaves}
																			}
																		child {
																				node {$\lambda$ is the weight of the penalty on the weights}
																			}
																	}
																child {
																		node {Complexity}
																	}
																child {
																		node {Advantages}
																		child {
																				node {No need for data preprocessing}
																			}
																		child {
																				node {Work well with categorical features}
																			}
																		child {
																				node {Ability to work with arbitrary loss functions}
																			}
																		child {
																				node {Very low-bias, yet relatively low-variance}
																			}
																		child {
																				node {Very fast algorithm $O(k\cdot M\cdot N\cdot log(T))$}
																			}
																	}
															}
													}
												child {
														node {Dropout}
													}
											}
										child {
												node (weight decay 2) {Weight decay}
												child {
														node {Add penalty term to empirical risk}
														child {
																node {$+ \alpha \Omega(\Theta)$}
															}
														child {
																node {Increase alpha}
																child {
																		node {Circle smaller, increasing bias}
																	}
															}
														child {
																node {Decrease alpha}
																child {
																		node {Circle bigger, model less constrained, does overfit more}
																	}
															}
														% child {
														%   node {Compromise close to loss function minimum and lying within radius of weights}
														% }
														child {
																node {Within permissible range of regularization want find point close to minimum as possible}
															}
													}
											}
										child {
												node {Universal, applied to Linear Models, Polynomial Regression Models, NN's}
											}
										child {
												node {L1}
											}
										child {
												node {L2}
											}
									}
							}
					}
			}
		child {
				node {Hyperparameter Optimization}
				child {
						node {Hyperparameters usually cannot be determined analytically}
					}
				child {
						node {Response function}
					}
				child {
						node {Algorithms}
						child {
								node {K-Fold Cross Validation}
							}
						child {
								node {Black-box Search}
								child {
										node {Grid and Random Search}
									}
								child {
										node {Sequential Model Based Optimization / Bayesian Optimization}
										child {
												node {Surrogate Model}
											}
										child {
												node {Exploration vs. Exploitation}
											}
										child {
												node {Acquisition function}
											}
										child {
												node {Gaussian Process Learning}
											}
									}
							}
						child {
								node {Gray-box Search}
								child {
										node {Hyperband}
										child {
												node {Learning Curve Tuning}
											}
										child {
												node {Successive Halving}
											}
									}
							}
					}
			}
		child {
				node {Fighting Overfitting}
				child {
						node {Basic Principles}
						child {
								node {Overfitting}
								child {
										node {Add more data}
									}
								child {
										node {Reduce model complexity}
										child {
												node {Feature engineering}
											}
										child {
												node {Increase regularization}
											}
									}
								child {
										node {Ensembling}
									}
							}
						child {
								node {Underfitting}
								child {
										node {Add informative features}
									}
								child {
										node {Decrease Regularization}
									}
								child {
										node {Increase model capacity}
									}
							}
					}
			}
		child {
				node {Error Measures}
				child {
						node {Classification and Regression}
						child {
								node {Classification}
								child {
										node {Binary}
										child {
												node {Miss-classification Rate (MCR)}
												child {
														node {Confusion matrix}
													}
											}
										child {
												node {Receiver Operating Characteristic curve (ROC curve)}
												child {
														node {Gives equal importance to positive and negative classes}
													}
												child {
														node {Area under the ROC curve (AUC)}
													}
											}
										child {
												node {Precision-Recall curve (PR curve)}
												child {
														node {Cares more about the positive class}
													}
												child {
														node {Area under the Precision-Recall (PR) curve}
													}
												child {
														node {F1}
														child {
																node {Harmonic mean between Precision and Recall}
															}
													}
											}
										child {
												node {Proxy Losses}
												child {
														node {Differentiable approximation of an error measure}
														child {
																node {Can’t use it for gradient descent}
															}
													}
												child {
														node {Should have same minimum point of original error measure}
													}
												child {
														node {Proxy of 0/1 error is logistic loss}
													}
												child {
														node {Imbalanced Target}
													}
											}
									}
							}
						child {
								node {Regression}
								child {
										node {Need no Proxies}
									}
								child {
										node {Scale-dependent errors}
										child {
												node {Mean Absolute Error}
											}
										child {
												node {Root Mean Square Error}
											}
									}
								child {
										node {Percentage errors}
										child {
												node {Independent to scale}
											}
										child {
												node {Mean Absolute Percentage Error}
											}
									}
								child {
										node {Mean Absolute Scaled Error (MAPE)}
										child {
												node {Alternative to percentage errors}
												child {
														node {Percentage errors are undefined when $y_n = 0$ and produce extreme values when $y_n\approx 0$}
													}
											}
										child {
												node {Nominator and denominator are same scale, cannot be perceived percentually}
											}
									}
							}
					}
				child {
						node {Ranking and Information Retrieval}
						child {
								node {Discounted cumulative gain (DCG)}
							}
						child {
								node {Normalized discounted cumulative gain (NDCG)}
								child {
										node {Ideal discounted cumulative gain (IDCG)}
									}
							}
						child {
								node {Pairwise Rank Loss}
								child {
										node {Algorithm}
									}
								child {
										node {Pairwise Loss is non-optimal for NDCG}
									}
							}
					}
			}
		child {
				node (target) {Instances, Features, Target}
			}
		child {
				node {Design Cycle
						% \resizebox{\textwidth}{!}{
						% 	\begin{minipage}[t]{14cm}
						% 		\begin{itemize}
						% 			\item those three steps here the steps before applying machine learning are actually crucial they're crucial because if you don't do them appropriately then your machine learning models will will not be very successful they will not be very accurate so you must clean the data well you must do diligent feature extraction and encoding and you must apply feature selection if it is necessary
						% 			\item we can merge those four steps (at the beginnning) in what we call end to end learning for example deep learning is typically one way of doing of doing end-to-end learning still in deep learning sometimes you have to do pre-processing but in essence the concept of end-to-end learning is that you provide the data in a in an as original manner as possible as you collected and then the feature extraction and the feature selection is done by the same model that does the prediction
						% 			\item ... so what it means is that the prediction it means is that the prediction model incorporates it the feature representation
						% 			\item ... which means it incorporates the extraction and the selection steps together
						% 			\item sometimes the prediction model by the nature of the way it is designed for example as a deep neural network might not need a lot of pre-processing so if you have lots of data and if your model is end to end learn end to end it is set up as an endtoend learning then you might need little pre-processing but pre-processing is is often used in practice because it improves accuracy
						% 		\end{itemize}
						% 		\begin{center}
						% 			\vspace{-0.35cm}
						% 			\color{SwitchColor}\rule{13cm}{0.25mm}
						% 		\end{center}
						% 		\begin{itemize}
						% 			\item the evaluation and model evaluation and model selection can also be more generic than that well in instance can be what if I don't know what machine learning algorithm do I use that is the part that we are we are actually also figuring out here so we can try a different machine learning algorithm and that means we move back to this step try this machine learning algorithm train it evaluate it measure its accuracy and calibrate its hyperparameters or its settings find the best settings for an algorithm and then you can try another algorithm uh maybe try first a linear model and see if you calibrate the the hyper parameters what accuracy you achieve then maybe you try a decision tree then maybe you try a neural network then maybe you try support Vector machine and in the end you keep the one that that works the best
						% 		\end{itemize}
						% 	\end{minipage}
						% }
					}
				child {
						node {1. Preprocessing
								% \resizebox{\textwidth}{!}{
								% 	\begin{minipage}[t]{12cm}
								% 		\begin{itemize}
								% 			\item in real life not offered clean data like matrix x that was in the dimensionality of n instances and m features, data scattered in relational data base that is found in different tables, find out way how to join all the features with primary keys of the tables
								% 			\item unbalanced target, cancer detection, 99% don't have it, very few cases to learn to classify the cancer, oversample minorty class, donwsample majority class
								% 			\item standardize data, maybe one column is normally distributed, maybe other column is not normally distributed, so might want to make all of them have a similar range for example that the column values they are then uh standardized to have a mean zero and a standard deviation of one
								% 		\end{itemize}
								% 	\end{minipage}
								% }
							}
						child {
								node {Collectng data}
							}
						child {
								node {Cleaning data}
							}
					}
				child {
						node {2. Feature extraction / encoding
								% \resizebox{\textwidth}{!}{
								% 	\begin{minipage}[t]{12cm}
								% 		\begin{itemize}
								% 			\item that you need to convert features which originally are not in a vector format you need to convert them to a vector format
								% 			\item want to engineer features, use the weight and the height Square in order to create a new feature
								% 			\item sometimes those features are not uh they are not defined in a continuous
								% 			\item domain for example you might have uh the city as a feature if you want to for example predict the price of a house and you have the location where the house is actually situated and that could be Berlin or friborg or London and these are not numerical features that you can train a prediction model so you have to convert them to some numerical feature and one way to handle it is the one hot encoding of categorical features so if you have 10 categories then instead of this 10 category features you create 10 different features each being either zero or one for example  location Berlin one it is if it is Berlin zero if it is not so you create one feature per each category and value zero or one if that particular value uh if that particular instance has the respective binary feature or not we call that one hot encoding of categorical features
								% 		\end{itemize}
								% 	\end{minipage}
								% }
							}
						child {
								node {Feature extraction}
							}
						child {
								node {Feature engineering}
							}
						child {
								node {Feature encoding}
								child {
										node {If feature with several categories}
										child {
												node {One-hot encoding}
												child {
														node {For each category different feature each either 0 or 1}
													}
												child {
														node {Where house situated not numerical feature}
														child {
																node {Convert to numerical feature}
															}
													}
											}
									}
							}
					}
				child {
						node {3. Feature selection
								% \resizebox{\textwidth}{!}{
								% 	\begin{minipage}[t]{12cm}
								% 		\begin{itemize}
								% 			\item we have many settings for the machine learning models and all the settings of those machine learning models which which we need to figure out a way how to handle they are part of the step of evaluation and model selection so we need to select what algorithm do I use how do I set those settings how do I design my experimental protocol with data in order to conduct that
								% 			\item you might have one feature for every product in your catalog and if you have millions of products well then you have lots of features and in essence sometimes for uh scalability reasons or also for accuracy reason reasons we might want to select only a subset of those features we call that a variable of feature subset selection process
								% 			\item some algorithms actually do it on their own for example decision trees so the step of feature selection is something that you don't always have to do uh you have to do it depending on what machine learning model you have to use and that comes with experience after you learn the characteristics of the prediction model then you can understand whether they do feature selection or not and then you decide whether to apply this step or not in the pipeline of Designing machine learning models
								% 		\end{itemize}
								% 	\end{minipage}
								% }
							}
					}
				child {
						node {4. Machine learning
								% \resizebox{\textwidth}{!}{
								% 	\begin{minipage}[t]{12cm}
								% 		\begin{itemize}
								% 			\item once you have finally a set of relevant features that have been extracted and pre-processed then you can train a prediction model that is the Box called that we are labeled labeling as machine learning that's the part where you define a prediction model and also you train the parameters of the prediction model in order to accurately estimate a Target variable
								% 			\item designing prediction models and actually optimizing the parameters of the prediction models in order to minimize a loss function
								% 		\end{itemize}
								% 	\end{minipage}
								% }
							}
					}
				child {
						node (evalmodsel) {5. Evaluation \& model selection
								% \resizebox{\textwidth}{!}{
								% 	\begin{minipage}[t]{16cm}
								% 		\begin{itemize}
								% 			\item we have to make sure that the models we train are carefully calibrated to achieve the highest accuracy
								% 			\item for example if you work with neuron networks you want to control the capacity the number of layers so how deep this networks have to be for some simpler tasks they don't have to be very deep for very complex tasks they need to be very deep so depending on the complexity of the task also the complexity of the prediction model has to be adapted and adapted means it needs to be configured
								% 			\item so this step is the step where we configure actually the settings of a prediction model which we otherwise call them hyperparameters
								% 			\item so remember prediction model has has parameters those are the thetas that we use to make prediction for instance in the linear model or the polinomial model
								% 			\item hyperparameters is a term that means those are parameters but they are not used to make prediction instead they are settings of the of the prediction model that's why we use
								% 			\item if you carefully select the hyper parameters then you achieve what we call generalization that the model can actually accurately predict the target value of new data
								% 			\item we don't want just a model that can correctly or accurately estimate the target variable of our previous collected data that is important of course because we learn from it however we don't want to to overfeed or we don't want to to actually feed to the noise of the recorded data so we want the model to be successful in the future because that's the purpose of machine learning we learn from the past so we can be accurate in the future and that is what we call generalization that we can generalize to new instances or generalize to Future instances and we we can design experiments how to control that we we are actually generalizing well to unseen instances
								% 		\end{itemize}
								% 	\end{minipage}
								% }
							}
					}
				child {
						node {6. Postprocessing
								% \resizebox{\textwidth}{!}{
								% 	\begin{minipage}[t]{12cm}
								% 		\begin{itemize}
								% 			\item if you have for example a sensitive attribute so let's say your attributes were were from X1 X2 until XM and actually one of those attributes is sensitive it could be for instance Ray so what what you want is is you want that the probability of estimating the target given if X2 the race is maybe is maybe africanamerican that this probability of Y given um it's equal to the probability of Y if X2 is maybe white American so you want to make sure that the that the estimations are done in way that those two probabilities U are as close as possible and that you can do by actually uh updating or um modifying the predictions in a way that that such fairness criterias hold
								% 			\item and for example some models don't yield  probabilities for example you might have models such as let's say support Vector machines which don't yield probababilities so you might want to adapt the predictions in a way that that they achieve probabilities and that is another post-processing step
								% 		\end{itemize}
								% 	\end{minipage}
								% }
							}
					}
			}
		child {
				node {Unsupervised Learning}
			}
	\end{mindmapcontent}
	\begin{edges}
		\edge{loss}{target}
		% \edge{expected prediction model}{expected value}
		% \edge{expected test error}{expected value}
		\edge{hyperparameters}{evalmodsel}
		\edge{expected prediction model}{bias}
		\edge{estimated target minus ground truth}{prediction models}
		\edge{expected target}{noise}
		\edge{no activation}{linear}
		\edge{binary}{sigmoid}
		\edge{multi-class}{softmax}
		\edge{weight decay 2}{weight decay}
	\end{edges}
	% \annotation{test}{annotation}
\end{mindmap}
\end{document}
