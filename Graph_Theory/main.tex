\documentclass{standalone}
\usepackage[english]{babel}
% https://tex.stackexchange.com/questions/570303/use-blacktriangleright-as-itemize-label
\usepackage{amssymb} % for black triangleright

\usepackage{graphicx}
% \graphicspath{figures/}

\renewcommand{\labelitemi}{$\textcolor{SwitchColor}{\bullet}$}
\renewcommand{\labelitemii}{$\textcolor{SwitchColor}{\blacktriangleright}$}
\renewcommand{\labelitemiii}{$\textcolor{SwitchColor}{\blacksquare}$}

% https://tex.stackexchange.com/questions/525959/prevent-latex-from-stretching-math
\setlength{\thinmuskip}{1\thinmuskip}
\setlength{\medmuskip}{1\medmuskip}
\setlength{\thickmuskip}{1\thickmuskip}

\usepackage{csquotes}
\usepackage{xcolor}
% \usepackage{anyfontsize}
\usepackage[export]{adjustbox}
% \usepackage[]{enumitem}
\usepackage{nicematrix}
\usepackage{tikz}
\usetikzlibrary{arrows.meta,positioning}
\usetikzlibrary{graphs}
\usetikzlibrary{patterns}
\usetikzlibrary{shadings}
\usetikzlibrary{mindmap, shadows, backgrounds} % , calc

\definecolor{SecondaryColor}{HTML}{BBAF01}
\definecolor{SecondaryColorDimmed}{HTML}{FEF684}
\definecolor{PrimaryColor}{HTML}{E95112}
\definecolor{PrimaryColorDimmed}{HTML}{F6AF91}
\colorlet{BoxColor}{gray!10!white}

% colored bold
% \newcommand\alert[1]{\textcolor{SwitchColor}{\textbf{#1}}}
\newcommand\alert[1]{\textcolor{SwitchColor}{#1}}

\newlength{\leveldistance}
\setlength{\leveldistance}{25cm}

\begin{document}
  \begin{tikzpicture}[
      auto,
      huge mindmap,
      fill opacity=0.6,
      draw opacity=0.8,
      concept color = PrimaryColorDimmed,
      every annotation/.style={fill=BoxColor, draw=none, align=center, fill = BoxColor, text width = 2cm},
      grow cyclic,
      level 1/.append style = {
        concept color=SecondaryColorDimmed,
        level distance=\leveldistance,
        sibling angle=360/\the\tikznumberofchildren,
        % https://tex.stackexchange.com/questions/501240/trying-to-use-the-array-environment-inside-a-tikz-node-with-execute-at-begin-no
        execute at begin node=\definecolor{SwitchColor}{named}{SecondaryColor},
      },
      level 2/.append style = {
        concept color=PrimaryColorDimmed,
        level distance=\leveldistance / 2,
        sibling angle=30,
        execute at begin node=\definecolor{SwitchColor}{named}{PrimaryColor},
      },
      level 3/.append style = {
        concept color=SecondaryColorDimmed,
        level distance=\leveldistance / 3,
        execute at begin node=\definecolor{SwitchColor}{named}{SecondaryColor},
      },
      level 4/.append style = {
        concept color=PrimaryColorDimmed,
        level distance=\leveldistance / 4,
        execute at begin node=\definecolor{SwitchColor}{named}{PrimaryColor},
      },
      level 5/.append style = {
        concept color=SecondaryColorDimmed,
        level distance=\leveldistance / 5,
        execute at begin node=\definecolor{SwitchColor}{named}{SecondaryColor},
      },
      level 6/.append style = {
        concept color=PrimaryColorDimmed,
        level distance=\leveldistance / 6,
        execute at begin node=\definecolor{SwitchColor}{named}{PrimaryColor},
      },
      level 7/.append style = {
        concept color=SecondaryColorDimmed,
        level distance=\leveldistance / 7,
        execute at begin node=\definecolor{SwitchColor}{named}{SecondaryColor},
      },
      level 8/.append style = {
        concept color=PrimaryColorDimmed,
        level distance=\leveldistance / 8,
        execute at begin node=\definecolor{SwitchColor}{named}{PrimaryColor},
      },
      concept connection/.append style = {
        color = BoxColor,
      },
  ]
  % damit Annotationen nicht auch eine Drop Shadow erhalten
  \begin{scope}[
      every node/.style = {concept, circular drop shadow}, % draw=none
      every child/.style={concept},
    ]
  \node (gt) at (current page.center) {Graph Theory}
  child {
    node {Dual Problems}
    child {
      node {Min-Cut and Maximum-Flow
        \resizebox{\textwidth}{!}{
          \begin{minipage}[t]{8cm}
            \begin{itemize}
              \item coming later in the lecture
            \end{itemize}
          \end{minipage}
        }
      }
    }
    child {
      node {Matching and Vertex cover
        \resizebox{\textwidth}{!}{
          \begin{minipage}[t]{8cm}
            \begin{itemize}
              \item  the  size of a \alert{vertex cover} is always at least as big as the size of a \alert{matching}
              \item \alert{Theorem (König und Egerváry):} For any bipartite graph $G$ the size of maximum matching equals the size of minimum vertex cover
            \end{itemize}
          \end{minipage}
        }
      }
      child {
        node {Matching
          \resizebox{\textwidth}{!}{
            \begin{minipage}[t]{8cm}
              \begin{itemize}
                \item A \alert{matching} $M$ in a graph $G$ is a set of non-loop edges with no shared endpoints
                \begin{itemize}
                  \item the vertices incident to the edges of a matching M are \alert{$M$-saturated}
                  \item the other vertices are \alert{$M$-unsaturated}
                  \item a \alert{perfect matching} saturates every vertex
                  \item the \alert{size} of a matching is given by the \alert{number of edges}
                  \item a \alert{maximal matching} in a graph is a matching that cannot be enlarged by adding an edge
                  \item a \alert{maximum matching} is a matching of maximum size (among all matchings in a graph)
                  \item $\alpha'(G)$ is the maximum size of matching of edges in $G$
                \end{itemize}
              \end{itemize}
            \end{minipage}
          }
        }
        child {
          node (mec) {Other Connections
            \resizebox{\textwidth}{!}{
              \begin{minipage}[t]{8cm}
                \begin{itemize}
                  \item \alert{Theorem(Gallai):} If G is a graph without isolated vertices, then $\alpha'(G) + \beta'(G) = n(G)$
                \end{itemize}
              \end{minipage}
            }
          }
        }
        child {
          node {Theorem (Berge)
            \resizebox{\textwidth}{!}{
              \begin{minipage}[t]{8cm}
                \begin{itemize}
                  \item  A matching $M$ in a graph $G$ is a maximum matching in $G$ \alert{iff} $G$ has no $M$-augmenting path
                \end{itemize}
              \end{minipage}
            }
          }
          child {
            node (apa) {Augmenting Path Algorithm
              \resizebox{\textwidth}{!}{
                \begin{minipage}[t]{8cm}
                  \begin{itemize}
                    \item  Repeatedly applying the Augmenting Path Algorithm to a bipartite graph produces a \alert{matching} and a \alert{vertex cover} of equal size
                  \end{itemize}
                \end{minipage}
              }
            }
          }
          child {
            node {Alternating and Augmenting Paths
              \resizebox{\textwidth}{!}{
                \begin{minipage}[t]{8cm}
                  \begin{itemize}
                    \item Given a matching $M$ in $G$, an \alert{$M$-alternating path} is a path that alternates between edges in $M$ and edges not in $M$
                    \item An $M$-alternating path whose endpoints are unsaturated by $M$ is an \alert{$M$-augmenting path}
                  \end{itemize}
                \end{minipage}
              }
            }
          }
        }
        child {
          node {Theorem (Hall)
            \resizebox{\textwidth}{!}{
              \begin{minipage}[t]{8cm}
                \begin{itemize}
                  \item An $X$, $Y$-bigraph has a matching that saturates $X$ if and only if $|N(S)| \ge |S|$ for all $S \subseteq X$
                \end{itemize}
              \end{minipage}
            }
          }
          child {
            node {X,Y-Bigraph
              \resizebox{\textwidth}{!}{
                \begin{minipage}[t]{8cm}
                  \begin{itemize}
                    \item a bipartite graph with vertex sets $X$ and $Y$ and edges only between $X$ and $Y$
                  \end{itemize}
                \end{minipage}
              }
            }
          }
        }
        child {
          node {Complete Graphs
            \resizebox{\textwidth}{!}{
              \begin{minipage}[t]{8cm}
                \begin{itemize}
                  \item $K_{2n}$ has $f_n = (2n-1)\cdot f_{n-1} = 1\cdot 3\cdot 5\cdot (2n-1)$ choices for choices for perfect matchings
                  \begin{itemize}
                    \item $K_n$ is a complete graph with $n$ vertices, i.e. a simple graph with maximum number of edges
                  \end{itemize}
                  \item $K_{2n+1}$ has no perfect matching because the node number of vertices is odd
                \end{itemize}
              \end{minipage}
            }
          }
        }
        child {
          node {Regular Bipartite graphs
            \resizebox{\textwidth}{!}{
              \begin{minipage}[t]{8cm}
                \begin{itemize}
                  \item for $k > 0$, every $k$-regular bipartite graph has a perfect matching
                \end{itemize}
              \end{minipage}
            }
          }
        }
        child {
          node {Symmetric Differences of Matchings
            \resizebox{\textwidth}{!}{
              \begin{minipage}[t]{8cm}
                \begin{itemize}
                  \item  Every component of the symmetric difference of two matchings
                  \begin{itemize}
                    \item is a path or
                    \item an even cycle
                  \end{itemize}
                \end{itemize}
              \end{minipage}
            }
          }
        }
      }
      child {
        node (vc) {Vertex cover
          \resizebox{\textwidth}{!}{
            \begin{minipage}[t]{8cm}
              \begin{itemize}
                \item a set $Q \subseteq V(G)$ that contains at least one endpoint of every edge
                \item the vertices $Q$ cover $E(G)$
                \begin{itemize}
                  \item $\beta(G)$ is the minimum size of vertex cover in $G$
                \end{itemize}
              \end{itemize}
            \end{minipage}
          }
        }
      }
    }
    child {
      node {Independence number and edge cover
        \resizebox{\textwidth}{!}{
          \begin{minipage}[t]{8cm}
            \begin{itemize}
              \item content
            \end{itemize}
          \end{minipage}
        }
      }
      child {
        node {Independence number
          \resizebox{\textwidth}{!}{
            \begin{minipage}[t]{8cm}
              \begin{itemize}
                \item Vertices are \alert{independent}, if they are not connected via an edge
                \begin{itemize}
                  \item $\alpha(G)$ is the maximum size of independent set of vertices in $G$
                \end{itemize} 
              \end{itemize}
            \end{minipage}
          }
        }
        child {
          node (isvc) {Other Connections
            \resizebox{\textwidth}{!}{
              \begin{minipage}[t]{8cm}
                \begin{itemize}
                  \item in a graph $G$ the set $S$ is an independent set \alert{iff} $\overline{S}$ is a vertex cover
                  \item $\alpha(G) + \beta(G) = n(G)$
                \end{itemize}
              \end{minipage}
            }
          }
        }
      }
      child {
        node (ec) {Edge cover 
          \resizebox{\textwidth}{!}{
            \begin{minipage}[t]{8cm}
              \begin{itemize}
                \item is a set $L$ of edges such that every vertex of $G$ is incident to some edge of $L$
                \begin{itemize}
                  \item $\beta'(G)$ is the minimum size of edge cover in $G$
                \end{itemize}
              \end{itemize}
            \end{minipage}
          }
        }
      }
    }
  }
  child {
    node {Undirected Graphs $G = (V, E, R)$
      \resizebox{\textwidth}{!}{
        \begin{minipage}[t]{10cm}
          \begin{itemize}
            \item \alert{vertex} set, e.g. $V(G) = \{v_1, v_2, \ldots\}$
            \item \alert{edge} set, e.g. $E(G) = \{e_1, e_2, \ldots\}$
            \item \alert{relation}, e.g. $R(G) = \{(e_1, \{v_1, v_2\}), \ldots\}$ that associates with each \alert{edge} a set of \alert{two vertices}
            \item one writes $e = uv$ (or $vu$) for an edge e with endpoints $u$ and $v$
          \end{itemize}
        \end{minipage}
      }
    }
      child {
        node {Types of graphs}
          child {
            node (bipartite graphs) {Bipartite Graphs
              \resizebox{\textwidth}{!}{
                \begin{minipage}[t]{8cm}
                  \begin{itemize}
                    \item A graph $G$ is \alert{bipartite} if $V(G)$ is the union of two disjoint independent sets
                  \end{itemize}
                \end{minipage}
              }
            }
            child {
              node {Chromatic number $\chi(G)$
                \resizebox{\textwidth}{!}{
                  \begin{minipage}[t]{8cm}
                    \begin{itemize}
                      \item is the \alert{minimum} number of colors such that \alert{adjacent vertices} receive \alert{different colors}
                    \end{itemize}
                  \end{minipage}
                }
              }
            }
            child {
              node {k-partite Graph
                \resizebox{\textwidth}{!}{
                  \begin{minipage}[t]{8cm}
                    \begin{itemize}
                      \item a graph $G$ is \alert{k-partite} if $V(G)$ can be expressed as the union of $k$ independent sets
                    \end{itemize}
                  \end{minipage}
                }
              }
            }
          }
          child {
            node (simple graph) {Simple Graph
              \resizebox{\textwidth}{!}{
                \begin{minipage}[t]{8cm}
                  \begin{itemize}
                    \item no loops
                    \item no multiple edges
                  \end{itemize}
                \end{minipage}
              }
            }
          }
      }
      child {
        node {Matrix representation
          \resizebox{\textwidth}{!}{
            \begin{minipage}[t]{8cm}
              \begin{itemize}
                \item \alert{loopless} graph $G$
                \item $V(G) = \{v_1, \ldots, v_n\}$
                \item $E(G) = \{e_1, \ldots e_m\}$
              \end{itemize}
              \includegraphics[width=0.3\textwidth, center]{./figures/matrix_representation.png}
            \end{minipage}
          }
        }
          child {
            node {Adjacency matrix $A(G)$
              \resizebox{\textwidth}{!}{
                \begin{minipage}[t]{8cm}
                  \begin{itemize}
                    \item is a $n\times n$ matrix with entries $a_{i,j}$
                    \item where $a_{i,j}$ is the number of edges with endpoints $\{v_i, v_j\}$
                  \end{itemize}
                  \includegraphics[width=0.3\textwidth, center]{./figures/adjacency_matrix.png}
                \end{minipage}
              }
            }
              child {
                node {Adjacency
                  \resizebox{\textwidth}{!}{
                    \begin{minipage}[t]{8cm}
                      \begin{itemize}
                        \item if $u$ and $v$ are endpoints of an edge they are \alert{adjacent}
                        \item short: $u\leftrightarrow v$
                      \end{itemize}
                    \end{minipage}
                  }
                }
              }
          }
          child {
            node {Incidence matrix $M(G)$
              \resizebox{\textwidth}{!}{
                \begin{minipage}[t]{8cm}
                  \begin{itemize}
                    \item is a $n\times m$ matrix with entries $m_{i,j}$ (that means the $m$ columns are edges)
                    \item where $m_{i,j}$ is 
                    \begin{itemize}
                      \item $1$ if $v_i$ is an endpoint of $e_j$ and
                      \item $0$ otherwise
                    \end{itemize}
                  \end{itemize}
                  \includegraphics[width=0.3\textwidth, center]{./figures/incidence_matrix.png}
                \end{minipage}
              }
            }
              child {
                node {Incidence
                  \resizebox{\textwidth}{!}{
                    \begin{minipage}[t]{8cm}
                      \begin{itemize}
                        \item $v$ and $e$ are \alert{incident} if $v$ is an endpoint of $e$
                      \end{itemize}
                    \end{minipage}
                  }
                }
                  child {
                    node {Degree
                      \resizebox{\textwidth}{!}{
                        \begin{minipage}[t]{8cm}
                          \begin{itemize}
                            \item the \alert{degree} of vertex $v$ is the number of incident edges
                            \item the number of $1$'s in a row of the incidence matrix
                          \end{itemize}
                        \end{minipage}
                      }
                    }
                  }
              }
          }
      }
      child {
        % https://latex.org/forum/viewtopic.php?t=8650
        node {Complement $\overline{G}$
          \resizebox{\textwidth}{!}{
            \begin{minipage}[t]{8cm}
            \begin{itemize}
              \item of a simple graph $G$ is the simple graph with:
              \begin{itemize}
                \item vertex set $V(G)$
                \item edge set $uv \in E(\overline{G}) \Leftrightarrow uv\not\in E(G)$
              \end{itemize}
            \end{itemize}
            \end{minipage}
          }
        }
          child {
            node {Clique
              \resizebox{\textwidth}{!}{
                \begin{minipage}[t]{8cm}
                  \begin{itemize}
                    \item a set of \alert{pairwise adjacent vertices}
                  \end{itemize}
                \end{minipage}
              }
            }
          }
          child {
            node (independant set) {Independant (stable) set
              \resizebox{\textwidth}{!}{
                \begin{minipage}[t]{8cm}
                  \begin{itemize}
                    \item is a set of pairwise nonadjacent vertices
                  \end{itemize}
                \end{minipage}
              }
            }
          }
      }
      child {
        node {Subgraph
          \resizebox{\textwidth}{!}{
            \begin{minipage}[t]{8cm}
              \begin{itemize}
                \item of a graph $G$ is a graph $H$ such that
                \begin{itemize}
                  \item $V(H)\subseteq V(G)$
                  \item $E(H)\subseteq E(G)$
                  \item the assignment of endpoints to edges in $H$ is the same as in $G$
                \end{itemize}
              \end{itemize}
            \end{minipage}
          }
        }
        child {
          node {Connections
          \resizebox{\textwidth}{!}{
            \begin{minipage}[t]{8cm}
              \begin{itemize}
                \item Graph $G$ is \alert{connected} if each pair of vertices in G belongs to a path
                \item otherwise Graph $G$ is \alert{disconnected}
              \end{itemize}
            \end{minipage}
          }}
        }
      }
      child {
        node {Connections in Graphs}
          child {
            node {Walk
              \resizebox{\textwidth}{!}{
                \begin{minipage}[t]{8cm}
                  \begin{itemize}
                    \item is a list $v_0, e_1, v_1, \ldots e_k, v_k$ of \alert{vertices} and \alert{edges}
                    \item for $1 \le i \le k$ the \alert{edge} $e_i$ has \alert{endpoints} $v_{i-1}$ and $v_i$
                    \item the \alert{length} is its \alert{number of edges}
                  \end{itemize}
                \end{minipage}
              }
            }
            child {
              node {u,v-walk
                \resizebox{\textwidth}{!}{
                  \begin{minipage}[t]{8cm}
                    \begin{itemize}
                      \item \alert a {walk} with \alert{first vertex} $u$ and \alert{last vertex} $v$
                    \end{itemize}
                  \end{minipage}
                }
              }
            }
            child {
              node {Closed walk
                \resizebox{\textwidth}{!}{
                  \begin{minipage}[t]{8cm}
                    \begin{itemize}
                      \item a walk with \alert{equal first} and \alert{last vertex}
                    \end{itemize}
                  \end{minipage}
                }
              }
            }
            child {
              node {Trail
                \resizebox{\textwidth}{!}{
                  \begin{minipage}[t]{8cm}
                    \begin{itemize}
                    \item a \alert{walk without repeating edges}
                    \end{itemize}
                  \end{minipage}
                }
              }
            }
          }
          child {
            node {Path
              \resizebox{\textwidth}{!}{
                \begin{minipage}[t]{8cm}
                  \begin{itemize}
                    \item a \alert{simple graph} whose vertices can be ordered such that:
                    \begin{itemize}
                      \item two vertices are adjacent $\Rightarrow$ they are consecutive in the circle
                    \end{itemize}
                    \item the \alert{length} is its \alert{number of edges}
                  \end{itemize}
                  \includegraphics[width=0.3\textwidth, center]{./figures/path.png}
                \end{minipage}
              }
            }
              child {
                node {u,v-path
                  \resizebox{\textwidth}{!}{
                    \begin{minipage}[t]{8cm}
                      \begin{itemize}
                        \item a path whose vertices of degree $1$ are $u$ and $v$ if $u\ne v$
                        \item if $u=v$ then the $u,v-path$ consists only of vertex $u$ and has no edges
                      \end{itemize}
                    \end{minipage}
                  }
                }
              }
              child {
                node {Cycle
                  \resizebox{\textwidth}{!}{
                    \begin{minipage}[t]{8cm}
                      \begin{itemize}
                        \item is a graph
                        \begin{enumerate}
                          \item with an \alert{equal} number of \alert{vertices} and \alert{edges}
                          \item whose \alert{vertices} can be placed around a \alert{circle} such that:
                          \begin{itemize}
                            \item two vertices are adjacent $\Rightarrow$ they are consecutive in the circle
                          \end{itemize}
                        \end{enumerate}
                      \end{itemize}
                      \includegraphics[width=0.3\textwidth, center]{./figures/cycle.png}
                    \end{minipage}
                  }
                }
              }
          }
      }
      child {
        node {Isomorphism
          \resizebox{\textwidth}{!}{
            \begin{minipage}[t]{8cm}
              \begin{itemize}
                \item from a \alert{simple graph} $G$ to a \alert{simple graph} $H$
                \item is a \alert{bijection} $f: V(G) \rightarrow V(H)$ such that
                \begin{itemize}
                  \item $uv\in E(G) \Rightarrow f(u)f(v)\in E(H)$
                \end{itemize}
                \begin{itemize}
                  \item one says $G$ \alert{is isomorphic to} $H$ if there's an \alert{isomorphism} from $G$ to $H$ ($G \cong H$)
                \end{itemize}
              \end{itemize}
            \end{minipage}
          }
        }
      }
      child {
        node {Decomposition of a graph
          \resizebox{\textwidth}{!}{
            \begin{minipage}[t]{8cm}
              \begin{itemize}
                \item is a \alert{list of subgraphs} such that \alert{each edge} appears in \alert{exactly one subgraph} in the list 
              \end{itemize}
            \end{minipage}
          }
        }
      }
    }
    child {
      node {Directed Graphs / Digraphs}
    };
  \end{scope}
  % ┌───────────────────┐
  % │ Verbindungslinien │
  % └───────────────────┘
  \begin{pgfonlayer}{background}
    \draw [concept connection]
  %     (commoncasefast) edge (amdahl)
  %     (branchpredictionbuffer) edge (2bitpredictor)
  %     (loadusedatahazard) edge (forwarding)
      (bipartite graphs) edge (independant set)
      (vc) edge (isvc)
      (vc) edge (apa)
      (ec) edge (mec);
  \end{pgfonlayer}
  % ┌──────────────┐
  % │ Annotationen │
  % └──────────────┘
  % https://tex.stackexchange.com/questions/302976/node-positioning-middle-point-mind-map-connection-bar
  \node [annotation, below] at (gt.south) {This mindmap is provided without guarantee of correctness and completeness!};
  % \path (measuringexecutiontime) -- node[annotation, above, align=center, pos=0.01] {Similiar to \textbf{Response Time:} How long it takes to do a task} (ca);
  % \path (performance) -- node[annotation, above, align=center, pos=0.01] {Similiar to \textbf{Throughput}: Total work done per time unit (e.g. tasks, transactions\ldots / per hour)} (ca);
  % \path (elapsedtime) -- node[annotation, above, align=center, pos=0.01] {Also called \textbf{Wall Clock Time} or \textbf{Real Time}} (ca);
  % \path (cputime) -- node[annotation, above, align=center, pos=0.01] {Also called \textbf{User Time}} (ca);
  % % \path (branchpredictionbuffer) -- node[annotation, below, align=center, pos=-0.06] {Also called Branch History Table} (ca);
  % \path (multicycle) -- node[annotation, above, align=center, pos=0.01] {Optimize space} (ca);
  % \path (pipelining) -- node[annotation, above, align=center, pos=0.01] {Optimize time} (ca);
  \end{tikzpicture}
\end{document}
