\documentclass[landscape, a4paper]{article}

%!Tex Root = ../main.tex

% ┌────────────┐
% │ Formatting │
% └────────────┘
\usepackage[english]{babel}
\usepackage[top=0cm,bottom=0cm,left=0cm,right=0cm]{geometry}
\usepackage[export]{adjustbox} % use c, l, r for images
\usepackage{csquotes}
\usepackage[parfill]{parskip}
\usepackage{fontspec}
% \usepackage{anyfontsize}
% \usepackage[]{enumitem}

% ┌──────┐
% │ Math │
% └──────┘
\usepackage{amssymb} % for black triangleright, https://tex.stackexchange.com/questions/570303/use-blacktriangleright-as-itemize-label
\usepackage{amsmath}
\usepackage{mathtools} % for \mathclap and 
\usepackage{breqn}

% ┌────────┐
% │ Tables │
% └────────┘
\usepackage{tabularray}
 % \UseTblrLibrary{diagbox}

% ┌────────┐
% │ Images │
% └────────┘
\usepackage{graphicx}
% \usepackage{float} % for the letter H
% \graphicspath{figures/}
\usepackage{subcaption}

% ┌────────┐
% │ Graphs │
% └────────┘
\usepackage{tikzit}
\usepackage{tikz}
\usetikzlibrary{backgrounds}
\usetikzlibrary{arrows}
\usetikzlibrary{shapes,shapes.geometric,shapes.misc}

% this style is applied by default to any tikzpicture included via \tikzfig
\tikzstyle{tikzfig}=[baseline=-0.25em,scale=0.5]

% these are dummy properties used by TikZiT, but ignored by LaTex
\pgfkeys{/tikz/tikzit fill/.initial=0}
\pgfkeys{/tikz/tikzit draw/.initial=0}
\pgfkeys{/tikz/tikzit shape/.initial=0}
\pgfkeys{/tikz/tikzit category/.initial=0}

% standard layers used in .tikz files
\pgfdeclarelayer{edgelayer}
\pgfdeclarelayer{nodelayer}
\pgfsetlayers{background,edgelayer,nodelayer,main}

% style for blank nodes
\tikzstyle{none}=[inner sep=0mm]

% include a .tikz file
\newcommand{\tikzfig}[1]{%
{\tikzstyle{every picture}=[tikzfig]
\IfFileExists{#1.tikz}
  {\input{#1.tikz}}
  {%
    \IfFileExists{./figures/#1.tikz}
      {\input{./figures/#1.tikz}}
      {\tikz[baseline=-0.5em]{\node[draw=red,font=\color{red},fill=red!10!white] {\textit{#1}};}}%
  }}%
}

% the same as \tikzfig, but in a {center} environment
\newcommand{\ctikzfig}[1]{%
\begin{center}\rm
  \tikzfig{#1}
\end{center}}

% fix strange self-loops, which are PGF/TikZ default
\tikzstyle{every loop}=[]


% ┌────────┐
% │ Citing │
% └────────┘
% \usepackage[style=authortitle]{biblatex}
% \addbibresource{./Graph_Theory.bib}
% \usepackage{cleveref}

% ┌──────────┐
% │ Diagrams │
% └──────────┘
% \usepackage{tikz}
% \usetikzlibrary{shadows, backgrounds} % , calc

% ┌──────────────────┐
% │ Multiple columns │
% └──────────────────┘
% \usepackage{multicol}

% ┌────────────────────┐
% │ Code hightligthing │
% └────────────────────┘
% \usepackage{minted}

% ┌────────────────────────┐
% │ Latex Programming Help │
% └────────────────────────┘
\usepackage{etoolbox}
\usepackage{xparse}
% https://tex.stackexchange.com/questions/358292/creating-a-subcounter-to-a-counter-i-created
\usepackage{chngcntr}

% ┌───────────────┐
% │ Pretty Boxes  │
% └───────────────┘
\usepackage{xcolor}
\usepackage{tcolorbox}
\tcbuselibrary{skins,theorems}

% ┌──────────────┐
% │ Pseudo Code  │
% └──────────────┘
\usepackage{pseudo}

% \usepackage{background}
\usepackage{gradient-text}
\usepackage{rotating}

% \newlength\mylen
% \setlength\mylen{\dimexpr\paperwidth/80\relax}
%
% \SetBgScale{1}
% \SetBgAngle{0}
% \SetBgColor{blue!30}
% \SetBgContents{\tikz{\draw[step=\mylen] (-.5\paperwidth,-.5\paperheight) grid (.5\paperwidth,.5\paperheight);}}

% ┌────────────┐
% │ Misc Tools │
% └────────────┘
\usepackage{lipsum}

%!Tex Root = ../main.tex

% ┌────────────┐
% │ Formatting │
% └────────────┘
% \setlength{\parskip}{0.4cm} % space between paragraphs, https://latexref.xyz/bs-par.html

% ┌───────┐
% │ Fonts │
% └───────┘
\usepackage{fontspec}
\newfontfamily\gyre{DejaVu Math TeX Gyre}
% colored bold
% \newcommand\alert[1]{\textcolor{SwitchColor}{\textbf{#1}}}
\newcommand\alert[1]{\textcolor{SwitchColor}{#1}}

% ┌──────────────┐
% │ Pseudo Code  │
% └──────────────┘
\newcounter{algorithm}
\setcounter{algorithm}{0}
\newtcbtheorem[use counter=algorithm]{algorithm}{\color{SecondaryColor}Algorithm}{pseudo/ruled}{alg}
% \newcommand{\ma}[1]{$\mathcal{#1}$}
% \renewcommand{\tt}[1]{{\small\texttt{#1}}}

% ┌────────┐
% │ Colors │
% └────────┘
\definecolor{PrimaryColor}{HTML}{800080}
\definecolor{PrimaryColorDimmed}{HTML}{D6D6F0}
\definecolor{SecondaryColor}{HTML}{006BB6}
\definecolor{SecondaryColorDimmed}{HTML}{E5F0F8}
\definecolor{SwitchColor}{named}{PrimaryColor}
\colorlet{BoxColor}{gray!10!white}

% ┌───────┐
% │ Links │
% └───────┘
\usepackage[allbordercolors=PrimaryColor, pdfborder={0 0 .2}]{hyperref}

% ┌─────────┐
% │ Mindmap │
% └─────────┘
\renewcommand{\labelitemi}{$\textcolor{SwitchColor}{\bullet}$}
\renewcommand{\labelitemii}{$\textcolor{SwitchColor}{\blacktriangleright}$}
\renewcommand{\labelitemiii}{$\textcolor{SwitchColor}{\blacksquare}$}

%!Tex Root = ../main.tex

% ┌─────────┐
% │ Mindmap │
% └─────────┘
\newlength{\leveldistance}
\setlength{\leveldistance}{25cm}

\newenvironment{edges}{\begin{pgfonlayer}{background}\draw [concept connection]}{;\end{pgfonlayer}}
\newcommand{\edge}[2]{(#1) edge (#2)}
\newcommand{\annotation}[2]{\path (#1) -- node[annotation, above, align=center, pos=0.03] {#2} (middle);}

\newenvironment{resettikz}{\pgfsetlayers{nodelayer,edgelayer}\tikzset{every node/.style={fill opacity=1.0, draw opacity=1.0, minimum size=0cm, inner sep=0pt}}}{}

\newenvironment{mindmap}{
	\begin{tikzpicture}[
			auto,
			huge mindmap,
			fill opacity=0.6,
			draw opacity=0.8,
			concept color = PrimaryColorDimmed,
			every annotation/.style={fill=BoxColor, draw=none, align=center, fill = BoxColor, text width = 2cm},
			grow cyclic,
			level 1/.append style = {
					concept color=SecondaryColorDimmed,
					level distance=\leveldistance,
					sibling angle=360/\the\tikznumberofchildren,
					% https://tex.stackexchange.com/questions/501240/trying-to-use-the-array-environment-inside-a-tikz-node-with-execute-at-begin-no
					execute at begin node=\definecolor{SwitchColor}{named}{SecondaryColor}\definecolor{SwitchColorDimmed}{named}{PrimaryColorDimmed},
				},
			level 2/.append style = {
					concept color=PrimaryColorDimmed,
					level distance=\leveldistance / 2,
					sibling angle=35,
					execute at begin node=\definecolor{SwitchColor}{named}{PrimaryColor}\definecolor{SwitchColorDimmed}{named}{SecondaryColorDimmed},
				},
			level 3/.append style = {
					concept color=SecondaryColorDimmed,
					level distance=\leveldistance / 3,
					execute at begin node=\definecolor{SwitchColor}{named}{SecondaryColor}\definecolor{SwitchColorDimmed}{named}{PrimaryColorDimmed},
				},
			level 4/.append style = {
					concept color=PrimaryColorDimmed,
					level distance=\leveldistance / 4,
					execute at begin node=\definecolor{SwitchColor}{named}{PrimaryColor}\definecolor{SwitchColorDimmed}{named}{SecondaryColorDimmed},
				},
			level 5/.append style = {
					concept color=SecondaryColorDimmed,
					level distance=\leveldistance / 5,
					execute at begin node=\definecolor{SwitchColor}{named}{SecondaryColor}\definecolor{SwitchColorDimmed}{named}{PrimaryColorDimmed},
				},
			level 6/.append style = {
					concept color=PrimaryColorDimmed,
					level distance=\leveldistance / 6,
					execute at begin node=\definecolor{SwitchColor}{named}{PrimaryColor}\definecolor{SwitchColorDimmed}{named}{SecondaryColorDimmed},
				},
			level 7/.append style = {
					concept color=SecondaryColorDimmed,
					level distance=\leveldistance / 7,
					execute at begin node=\definecolor{SwitchColor}{named}{SecondaryColor}\definecolor{SwitchColorDimmed}{named}{PrimaryColorDimmed},
				},
			level 8/.append style = {
					concept color=PrimaryColorDimmed,
					level distance=\leveldistance / 8,
					execute at begin node=\definecolor{SwitchColor}{named}{PrimaryColor}\definecolor{SwitchColorDimmed}{named}{SecondaryColorDimmed},
				},
			level 9/.append style = {
					concept color=SecondaryColorDimmed,
					level distance=\leveldistance / 9,
					execute at begin node=\definecolor{SwitchColor}{named}{SecondaryColor}\definecolor{SwitchColorDimmed}{named}{PrimaryColorDimmed},
				},
			concept connection/.append style = {
					color = BoxColor,
				},
		]
		}{
	\end{tikzpicture}
}

\newenvironment{mindmapcontent}{
	\begin{scope}[
			every node/.style = {concept, circular drop shadow}, % draw=none
			every child/.style={concept},
		]
		}{
		;\end{scope}
}

% ┌───────┐
% │ Boxes │
% └───────┘
\DeclareTotalTCBox{\inlinebox}{ s m }
{standard jigsaw,opacityback=0,colframe=SwitchColor,nobeforeafter,tcbox raise base,top=0mm,bottom=0mm,
	right=0mm,left=0mm,arc=0.1cm,boxsep=0.1cm}
{\IfBooleanTF{#1}%
	{\textcolor{PrimaryColor}{\setBold >\enspace\ignorespaces}#2}%
	{#2}}

\DeclareTotalTCBox{\inlineboxtwo}{ s m }
{standard jigsaw,opacityback=0,colframe=SwitchColorDimmed,nobeforeafter,tcbox raise base,top=0mm,bottom=0mm,
	right=0mm,left=0mm,arc=0.1cm,boxsep=0.1cm}
{\IfBooleanTF{#1}%
	{\textcolor{SwitchColorDimmed}{\setBold >\enspace\ignorespaces}#2}%
	{#2}}

% ┌──────────────────┐
% │ Case distinction │
% └──────────────────┘
% \newtoggle{absolute}
% % \toggletrue{absolute}
% \togglefalse{absolute}
% \newcommand{\lpathgraph}[1]{\iftoggle{absolute}{/home/areo/Documents/Studium/Summaries/x/}{./}#1}

% ┌───────┐
% │ Fixes │
% └───────┘
% https://tex.stackexchange.com/questions/89467/why-does-pdftex-hang-on-this-file
% \newcommand{\colon}{\mathrel{\mathop:}}

% ┌───────┐
% │ Paths │
% └───────┘
% \newcommand{\script}[2]{\href[page=#1]{}{\inlinebox{#2}}}
\newcommand{\script}[2]{\href{openpdf:/home/areo/Documents/Studium/Semester_1_Master/Hardware_Security_and_Trust/slides/Slides annotated/Hardware_Security_and_Trust_all_in_one.pdf:#1}{\inlinebox{#2}}}
\newcommand{\scripttwo}[2]{\href{openpdf:///home/areo/Documents/Studium/Semester_1_Master/Hardware_Security_and_Trust/slides/Slides annotated/bonus/12_Lecture_06Dec.pdf:#1}{\inlinebox{#2}}}
\newcommand{\videoeight}[2]{\href{https://youtu.be/YcHSlFjcndU?feature=shared&t=#1}{\inlineboxtwo{#2}}}
\newcommand{\videonine}[2]{\href{https://youtu.be/3dL-3EOIfJ8?si=l3OakqHOeCpnNayw&t=#1}{\inlineboxtwo{#2}}}
\newcommand{\videoten}[2]{\href{https://youtu.be/6oF737pa510?feature=shared&t=#1}{\inlineboxtwo{#2}}}
\newcommand{\videoeleven}[2]{\href{https://youtu.be/PJTqfzTIYJs?feature=shared&t=#1}{\inlineboxtwo{#2}}}
\newcommand{\videotwelve}[2]{\href{https://youtu.be/oDxAH7aO-Tk?feature=shared&t=#1}{\inlineboxtwo{#2}}}
\newcommand{\videothirteen}[2]{\href{https://youtu.be/3TkSXxe_Ty8?feature=shared&t=#1}{\inlineboxtwo{#2}}}
\newcommand{\videofourteen}[2]{\href{https://youtu.be/1Y3dZuJ0MHg?feature=shared&t=#1}{\inlineboxtwo{#2}}}


\begin{document}
\fontsize{4pt}{5pt}\selectfont

\begin{minipage}[t]{0.198\pagewidth}
	\fbox{General}
	\begin{betterlist}
		\item \alert{Hardware Security} goes beyond classical cryptography. It protects the implementations of cryptographic algorithms against physical attacks, side-channel attacks etc. Avoids tampering with devices
		\item \alert{Security} entails that the system fulfills certain security properties which an intelligent attacker seeks to undermine. No static and complete definition or classification. \alert{Important security properties (\enquote{CIAAN}) are:}
		\begin{betterlist}
			\item \alert{Confidentiality:} Protecting confidential information from being disclosed to unauthorized parties. (no unauthorized reads)
			\item \alert{Integrity:} Ensuring that information is only modified by authorized parties. (no unauthorized writes)
			\item \alert{Availability:} Making sure that information and systems are accessible to authorized parties when they need them (e.g. resistance to denial-of-service attacks).
			\item \alert{Authenticity:} Ensuring that information and communication come from the source they are supposed to come from.
			\item \alert{Non-repudiation:} Ensuring that nobody can deny having performed certain actions (like sending / receiving messages, changing data etc.).
		\end{betterlist}
		\item \alert{Vulnerability:} Weakness in the secure system
		\item \alert{Threat:} set of circumstances that has the potential to cause loss or harm
		\item \alert{Attack:} The act of a human exploiting the vulnerability in the system
		\item \alert{Safety:} System is designed without any error leading to unintended behavior. (\alert{design time})
		\item \alert{Reliability:} A correctly designed system continues to work correctly during its \alert{life-time}
		\begin{betterlist}
			\item \underline{relationship between security and safety / reliability:} Port for monitoring the system good for safety / reliability, but bad for confidentiality of the processed data. Safety problems (erroneous implementations) may also lead to security problems
		\end{betterlist}
		\item \underline{Possible Actions of Adversaries, Attacks:}
		\begin{betterlist}
			\item Pirating Intellectual Property (IP) – illegal use of IPs
			\begin{betterlist}
				\item E.g. system integrators, fabrication facilities
			\end{betterlist}
			\item Implementing and inserting Trojan horses
			\item Reverse engineering of ICs
			\item Spying by exploiting IC vulnerabilities (\enquote{backdoors})
			\item Physical attacks, side-channel attacks, fault injection
			\item Counterfeiting ICs by Recycling ICs and Cloning, overproducing ICs (e.g. fabrication facilities)
		\end{betterlist}
		\item \underline{Examples for Countermeasures to Attacks:}
		\begin{betterlist}
			\item Encrypting secret data
			\begin{betterlist}
				\item Encryption has to do with scrambling to hide
			\end{betterlist}
			\item Design locks or physical locks limiting the access
			\item Devices to verify the user identities
			\item Hiding signatures in the design files
			\item Intrusion detection
			\item Security policies
		\end{betterlist}
	\end{betterlist}
	\fbox{Supply Chain Vulnerabilities}
	\begin{betterlist}
		\item \underline{Problem 1:} Cost of Manufacturing
		\begin{betterlist}
			\item An \alert{untrusted foundry} may overproduce ICs and sell overproduced chips on its own account
			\item An \alert{untrusted assembly} may sell defective ICs, sell out-of-spec ICs (possibly noticed only by early aging, e.g.), mark correct ICs as defective and sell them on its own account
		\end{betterlist}
		\item \underline{Problem 2:} Design Complexity
		\begin{betterlist}
			\item buy design data as \alert{intellectual property} ($=$ \alert{IP}) from other companies
			\item \alert{IP Vendors} are located across the world. No control on the design process. Safety and security problem
		\end{betterlist}
		\item \underline{Several parties may be untrusted:} IP vendors (untrusted IP), System integrator (untrusted system, IP piracy), Fab (untrusted IC, IC Piracy (Counterfeiting)), Assembly (IC Piracy (Counterfeiting))
		\item  \alert{Counterfeit} chips are not restricted to Overproduced chips (foundry) and Defective / out-of-spec chips (assembly + test). The whole IC life cycle and supply chain is affected: \alert{Recycled ICs} and \alert{Cloned ICs}
	\end{betterlist}
	\includegraphics[width=\linewidth]{./figures/supply_chain_vulnerabilities.png}
	\fbox{\enquote{Classical} Cryptography}
	\begin{betterlist}
		\item \alert{Cryptology:} cryptography + cryptanalysis
		\item \alert{Cryptography:} art/science of keeping message secure. Is about algorithms protecting secret information
		\item \alert{Cryptanalysis:} art/science of breaking ciphertext
		\item \alert{Basic Cryptographic Scheme:} \alert{injective} and both sets have \alert{same cardinality}, so \alert{one-to-one}, \alert{bijective}, so it is \alert{reversable}, $D$ is the \alert{inverse function} of $E$ and by this also bijective
	\end{betterlist}
\end{minipage}
\begin{minipage}[t]{0.198\pagewidth}
	\fbox{Classification by type of encryption operations}
	\begin{betterlist}
		\item \alert{Substitution ciphers}
		\begin{betterlist}
			\item letters of $P$ replaced with other letters by $E()$
			% \item \underline{Sidenote:} One can say $key = 3$ or $key = \enquote{D}$, because $no(D) = 3$
			\item \underline{Effects:}
			\begin{betterlist}
				\item $C$ hides chars of $P$ (plaintext)
				\item if $> 1$ key alphabet (polyalphabetic), $C$ dissipates high frequency chars
			\end{betterlist}
		\end{betterlist}
		\begin{betterlist}
			\item \alert{General Monoalphabetic substitution ciphers}
			\begin{betterlist}
				\item each letter in $P$ is substituted by a fixed letter using a \alert{substitution table}
				\item the \alert{key} is the substitution table, $26!=2^{88}$ substitution tables (= keys)
				\item \underline{Attacks:}
				\begin{betterlist}
					\item \sout{Exhaustive search:} Search through $2^{88}$ keys (substitution tables) is completely infeasible with today‘s computers!
					\item \alert{Letter Frequency Attack:}
					\begin{betterlist}
						\item in practice, not only frequencies of individual letters (\alert{1-gram} (unigram) model of a language) can be used for an attack, but also the frequency of letter pairs, letter triples, etc
						\item  the longer the ciphertext $C$, the more effective statistical analysis would be
					\end{betterlist}
				\end{betterlist}
				\item need better concealing of statistical frequencies and probably also longer keys to avoid exhaustive search
			\end{betterlist}
			\begin{betterlist}
				\item \alert{Caesar Cipher}
				\begin{betterlist}
					\item each letter in $P$ is substituted by a fixed letter. In this special case the \alert{key} is of length $1$, it is the shift amount
					\item \underline{Attacks:}
					\begin{betterlist}
						\item \alert{Exhaustive search:} Try all possible keys until you find the right one. \enquote{Finding the right one} means \enquote{receiving a meaningful plain text}.
						\begin{betterlist}
							\item has $26$ possible keys
						\end{betterlist}
					\end{betterlist}
				\end{betterlist}
			\end{betterlist}
		\end{betterlist}
		\begin{betterlist}
			\item \alert{Polyalphabetic substitution ciphers}
			\begin{betterlist}
				\item several key alphabets, flatten (diffuse) somewhat the frequency distribution of letters by combining high and low distributions
				\item \underline{Attack:}
				\begin{betterlist}
					\item frequency of pairs is somewhat hidden. Works if the attacker doesn't know that polyalphabetic substitution with $n$ keys is used
					\item if one knows the $n$, then one can break the whole text into $n$ parts and for these a \alert{statistical analysis} works as for the monialphabetic substitution cipher with the unigram model
				\end{betterlist}
			\end{betterlist}
			\begin{betterlist}
				\item \alert{Vigenère Tableaux Method}
				\begin{betterlist}
					\item special case of polyalphabetic substitution with $n$ key alphabets. For each key alphabet the special case of Caesar cipher is chosen, i.e., each key alphabet can be represented by one letter
					\begin{betterlist}
						\item one can describe a row either by the shift or by saying which letter $c_i$ it mapped to which letter $c_j$ and takes $b_j$ as name of the row, choosing the last choice one can describe $n$ keys by a word of length $n$
					\end{betterlist}
					\item $26^n$ different keys, one has to choose $n$ large enough
					\item \underline{Attack:}
					\begin{betterlist}
						\item same problem as for polyalphabetic substitution ciphers
					\end{betterlist}
				\end{betterlist}
			\end{betterlist}
		\end{betterlist}
	\end{betterlist}
	\begin{betterlist}
		\item \alert{Transposition (permutation) ciphers}
		\begin{betterlist}
			\item order of letters in $P$ rearranged by $E()$
			% \begin{betterlist}
			% 	\item rearrange letters in plaintext to produce ciphertext
			% \end{betterlist}
			\item \underline{Effects:}
			\begin{betterlist}
				\item $C$ scrambles text, hides \alert{$n$-grams} for $n > 1$ (combinations of $n$ letters, e.g. th)
			\end{betterlist}
		\end{betterlist}
		\begin{betterlist}
			\item \alert{Rail-Fence Cipher}
			\begin{betterlist}
				\item columnar transposition
				\item \alert{key} = number of columns
				\item \underline{Attack:} Number of columns, the key space is restricted if the text is short
			\end{betterlist}
		\end{betterlist}
	\end{betterlist}
	\begin{betterlist}
		\item \alert{Product ciphers}
		\begin{betterlist}
			\item idea to combine two or more ciphers to enhance the security of the cryptosystem
			\begin{betterlist}
				\item built of multiple blocks, either \alert{Substitution} or \alert{Transposition}
			\end{betterlist}
			\item might not be stronger than individual components used separatly (two times ceasar is like one time ceaser with key being sum of keys) or as strong as individual components (two times ceasar with keys $3$ and $23$)
			\item \underline{Effects:}
			\begin{betterlist}
				\item can do all what Substitution and Transposition ciphers can, so more secure if used well
			\end{betterlist}
			\item \alert{Two-block product cipher}
			\begin{betterlist}
				\item $E2(E1(P, KE1), KE2)$, may be repeated to form several encryption rounds
			\end{betterlist}
			\item \alert{AES}
		\end{betterlist}
	\end{betterlist}
\end{minipage}
\begin{minipage}[t]{0.198\pagewidth}
	\fbox{Classification by key}
	\begin{betterlist}
		\item \alert{Crypto System with Keys:} $P = D(K_D, E(K_E, P))$
		\begin{betterlist}
			\item $P = D(K_D, C)$, $C = E(K_E, P)$, $D/E =$ set of d/encryption algorithms, $K_D/K_E$ selects $D_j/E_i \in D/E$
		\end{betterlist}
		\begin{betterlist}
			\item \underline{advantage of crypto systems with keys:}
			\begin{betterlist}
				\item keeping the encryption / decryption algorithm secret is not needed
				\item keys can regulary be changed, to increase security
			\end{betterlist}
		\end{betterlist}
	\end{betterlist}
	\begin{betterlist}
		\item \alert{Symmetric cryptosystems} $K_E = K_D$ (secret key encryption)
		\begin{betterlist}
			\item encipher and decipher using the same key or one key is easily derived from the other. Only sender $S$ and receiver $R$ know the \alert{secret key}
			\item as long as the key remains secret it also provides \alert{authentication} ($=$ proof of sender’s identity)
			% \item \underline{advantage:}
			%      \begin{betterlist}
			%
			%      \end{betterlist}
			\item \underline{disadvantage:}
			\begin{betterlist}
				\item one needs a \alert{key exchange}, the sender and the receiver have to agree on the same key and they key should be secret, one needs a way to transport the secret key via a secure channel from the sender to the receiver
			\end{betterlist}
			\item \alert{AES}
		\end{betterlist}
	\end{betterlist}
	\begin{betterlist}
		\item \alert{Asymmetric cryptosystems} $K_E \ne K_D$ (public key encryption)
		\begin{betterlist}
			\item encipher and decipher using different keys, computationally infeasible to derive one from other
			\item also other applications like \alert{authentication} (authenticity, able to sign message, so it can be sure it can only come from oneself)
			\item only owner of \alert{private key} $K_D$ can decode msgs that could be encoded by anybody with \alert{public key} $K_E$
			\item asymmetric schemes based on a \alert{one-way function} (based on mathematically hard problems) $f()$: Computing $y = f(x)$ easy, but $x = f^{-1}(y)$ is computationally infeasible
			\item \underline{advantage:}
			\begin{betterlist}
				% \item allows to share \alert{public key}, secure key exchange not needed (for \alert{confidentiality}, message should be kept secret)
				\item distribution of public key for encryption can be done over an insecure channel
			\end{betterlist}
			\item \underline{disadvantage:}
			\begin{betterlist}
				\item usually assymetric crypto systems are much slower than symmetric ones
			\end{betterlist}
			\item \alert{RSA}
		\end{betterlist}
		\item \alert{Hybrid Asymmetric-symm. Systems:}
		\begin{betterlist}
			\item Key exchange (for symmetric schemes) and digital signatures performed with (slow) asymmetric algorithms. Encryption of data is done using (fast) symmetric ciphers
		\end{betterlist}
		\includegraphics[width=\linewidth]{./figures/hybrid_assymetric-symmetric_systems.png}
	\end{betterlist}
	\fbox{RSA}
	\begin{betterlist}
		\item \alert{Key Generation:}
		\begin{enumerate}
			\item choose two large primes $p$, $q$
			\item compute $n = p \cdot q$
			\item compute $\Phi(n) = (p - 1) \cdot (q - 1)$
			\begin{betterlist}
				\item $\Phi(n) = |\{a \mid a\in \{0, \ldots, n-1\}, gcd(a, n) = 1\}|$, is equal to the number of invertible / relatively prime elements in $\mathbb{Z}/n\mathbb{Z}$
			\end{betterlist}
			\item select an arbitrary public exponent $e \in \{1, 2, \ldots, \Phi(n) - 1\}$ such that $gcd(e, \Phi(n) ) = 1$
			\begin{betterlist}
				\item computations are done in $\mathbb{Z}/ \Phi(n)\mathbb{Z}$, $e$ must have inverse in $\mathbb{Z}_{\Phi(n)}$
			\end{betterlist}
			\item compute the private key $d \in \{1, 2, \ldots , \Phi(n) - 1\}$ s.t. $d \cdot e \equiv 1 \mod \Phi(n)$
			\begin{betterlist}
				\item E.g. computed with extended Euklid algorithm computing $gcd(e, \Phi(n))$
				\item $d$ is the inverse of $e$ in $\mathbb{Z}_{\Phi(n)}$, $d$ has $4096$ bits, $d\le 2^{4096}-1$
			\end{betterlist}
			\item RETURN $k_{pub} = (n, e), k_{pr} = d$
		\end{enumerate}
		\item \alert{Ecnryption:} $y = x^e \mod n$, \alert{Descryption:} $x = y^d \mod n$
		\item \alert{Authentication / Signing:}
		\begin{betterlist}
			\item $(x^e)^d = x^{ed} = (x^d)^e \equiv x \mod n$, thus it is also possible to encrypt with the private key and decrypt with the public key
			\item[\color{PrimaryColor}\textbf{1.} ] A sends $(M, e_{K_{prA}}(M))$ to B
			\item[\color{PrimaryColor}\textbf{2.} ] B checks whether $M = d_{K_{pubA}}(e_{K_{prA}}(M))$ on recieving it
			\item to make it more efficient A uses a known hash function to compute a shorter $hash(M)$ and encrypt this: $e_{K_{prA}}(hash(M))$
		\end{betterlist}
		\item \alert{Attacks / Security:}
		\begin{betterlist}
			\item timing and accoustic \enquote{chosen cipher text} attack
			\item relies on the hardness to derive the \alert{private exponent} $d$ given the public key $(n, e)$
			\item if it would be possible to factorize $n$, then it would also be possible to compute $\Phi(n)= (p - 1) \cdot (q - 1)$. Use very big prime numbers $p$ and $q$ to make it hard to factorize $n$ into $p$ and $q$
			\item with the public key $e$ it would then be possible to compute the inverse of $e$ in $\mathbb{Z}_{\Phi(n)}$
			\begin{betterlist}
				\item with Euklid‘s algorithm $gcd(e, \Phi(n) ) = 1$ is computed as well as integers $x$ and $y$ with $x \cdot e + y \cdot \Phi(n) = 1$
				\item then $x$ mod $\Phi(n)$ is the multiplicative inverse, i.e., $x$ mod $\Phi(n)$ is the private key $d$
			\end{betterlist}
			\item
		\end{betterlist}
	\end{betterlist}
\end{minipage}
\begin{minipage}[t]{0.198\pagewidth}
	\fbox{Classification by way to process plaintext}
	\begin{betterlist}
		\item \alert{Stream Ciphers}
		\begin{betterlist}
			\item bitwise Encryption and Decryption
			\item encrypt bits individually
			\begin{betterlist}
				\item Encryption and decryption are simple additions modulo 2 (aka XOR).
				\item Encryption and decryption are the same functions
				\item \alert{Encryption:} $y_i = e_{si}(x_i) = s_i \oplus x_i$, where $x_i, y_i, s_i \in \{0, 1\}$. \alert{Decryption:} $x_i = e_{si}(y_i) = s_i \oplus y_i$
			\end{betterlist}
			\includegraphics[width=0.4\linewidth]{./figures/stream_ciphers.png}
			\begin{betterlist}
				\item \underline{Attack:}

				\includegraphics[width=\linewidth]{./figures/stream_cipher_attack.png}
				\begin{betterlist}
					\item if a prefix of plaintext is known, then $S_0, S_1, \ldots, S_k$ are known, then paramters of the random function are known and thus all $S_i$ are known
				\end{betterlist}
			\end{betterlist}
			% \item need a method to generate key stream efficiently, starting from some \enquote{seed}
			% \item usually small and fast common in embedded device
		\end{betterlist}
	\end{betterlist}
	\begin{betterlist}
		\item \alert{Block Ciphers}
		\begin{betterlist}
			\item always encrypt a full block (several bits)
			% \item are common for Internet applications
			\item Block Cipher Primitives
			\begin{betterlist}
				\item \alert{Claude Shannon:} There are two primitive operations with which strong encryption algorithms can be built:
				\begin{enumerate}
					\item \alert{Confusion:} An encryption operation where the \alert{relationship between key and ciphertext is obscured}
					\begin{betterlist}
						\item today, a common element for achieving confusion is substitution, which is found in AES and other ciphers
					\end{betterlist}
					\item \alert{Diffusion:} An encryption operation where the \alert{influence of one plaintext symbol is spread over many ciphertext symbols} with the goal of hiding statistical properties of the plaintext
					\begin{betterlist}
						\item a simple diffusion element is the \alert{bit permutation} (in other context known as Tranposition)
					\end{betterlist}
				\end{enumerate}
				\begin{betterlist}
					\item Both operations by themselves are suboptimal in providing security. A cipher must include confusion and diffusion elements
				\end{betterlist}
			\end{betterlist}
			\item \alert{polyalphabetic cipher} is a block cipher with (small) $8$-bit-blocks as char is encoded by $8$ bits
			\item \alert{AES}
		\end{betterlist}
	\end{betterlist}
	\fbox{Fast Modular Exponentiation:}
	\begin{betterlist}
		\item \alert{Square and Multiply algorithm:}
		\begin{betterlist}
			\item $4^6 = 4^{110} \rightarrow \underbrace{1^2 * 4 = 4}_{1} \rightarrow \underbrace{4^2 * 4 = 64}_{1} \rightarrow \underbrace{64^2 = 4096}_{0}$
		\end{betterlist}
		% \includegraphics[width=\linewidth]{./figures/square_multiply.png}
		\includegraphics[width=\linewidth]{./figures/square_and_multiply.png}
	\end{betterlist}
	\fbox{Cryptanalysis}
	\begin{enumerate}
		\item Ciphertext-only attack: Exhaustive search, Letter Frequency attack (correlation based method possible)
		\item Known plaintext attack: Analyst have $C$ and $P$ (or partial match between $C$ and $P$). Needs to deduce $E$ such that $C=E(P)$, then finds $D$ (e.g. header of file or todays date)
		\item Chosen plaintext attack: Analyst able to fabricate encrypted msgs for arbitrary plain texts
		\item Chosen ciphertext attack: Analyst is able to obtain decryptions of chosen ciphertexts. Purpose to find $K_D$
		\item Fault injection attack
	\end{enumerate}
\end{minipage}
\begin{minipage}[t]{0.198\pagewidth}
	\fbox{Advanced Encryption Standard (\alert{AES})}
	\begin{betterlist}
		\item \underline{requirements for all AES candidate submissions were:}
		\begin{betterlist}
			\item Block cipher with 128-bit block size
			\item Three supported key lengths: $128$, $192$ and $256$ bit
			\begin{betterlist}
				\item the number of AES rounds depends on the chosen key length ($10$, $12$, $14$)
			\end{betterlist}
			\item Security relative to other submitted algorithms
			\item Efficiency in software and hardware
			\begin{betterlist}
				\item \alert{Implementation in Software:} straightforward implementation is well suited for $8$-bit processors (e.g., smart cards) with small amount of memory, because operations are on bytes, but inefficient on $32$-bit or $64$-bit processors.
				\item \alert{T-Tables:} When executing MixColumn for one column of the state matrix one can speed up the calculation of the resulting vector which is made of $4$ \alert{columns} with one entry of the column of the state matrix multiplied with the values of one column of the $4\times 4$ matrix with $3$ exor operations between the columns. There are $2^8$ possible values for this one value from the column of the state matrix, thus one needs a lookup table with $2^8$ values of $32$-bit values concatenating the resulting values of one column of the calculation. One needs $4$ such lookup tables, because the $4\times 4$ matrix has $4$ columns. One can then calculate the resulting vector by $4$ lookups to the $4$ lookup tables and $3$ exor operations.
			\end{betterlist}
		\end{betterlist}
		\item \underline{Round Structure and Internal Structure:}
		\begin{betterlist}
			\item state $A$ (i.e., the $128$-bit data path) can be arranged in a $4\times 4$ \alert{state matrix} with $A_0,\ldots, A_{15}$ denoting the $16$-byte input of AES
		\end{betterlist}
		\begin{minipage}[b]{0.64\linewidth}
			\includegraphics[width=\linewidth]{./figures/aes_round_structure.png}
		\end{minipage}
		\begin{minipage}[b]{0.34\linewidth}
			\includegraphics[width=0.5\linewidth]{./figures/fourtimesfourmatrix.png}
			\includegraphics[width=\linewidth]{./figures/shiftrow_matrix.png}
			\includegraphics[width=\linewidth]{./figures/mixcolumn.png}
			\includegraphics[width=\linewidth]{./figures/aes_algorithm.png}
		\end{minipage}
		\begin{minipage}[b]{0.49\linewidth}
			\includegraphics[width=\linewidth]{./figures/key_schedule.png}
		\end{minipage}
		\begin{minipage}[b]{0.49\linewidth}
			\includegraphics[height=0.6\linewidth, angle=90]{./figures/functiong.png}
			\includegraphics[width=\linewidth]{./figures/aes_round_structure_details.png}
		\end{minipage}
		\begin{betterlist}
			\item \alert{Byte Substitution Layer:}
			\begin{betterlist}
				\item $16$ \alert{identical} S-Boxes that are \alert{bijective} (can be \alert{uniquely reversed})
				\item only \alert{nonlinear} elements of AES, i.e., $ByteSub(A_i) \oplus ByteSub(A_j) \ne ByteSub(A_i \oplus A_j)$, for $i, j = 0,\ldots,15$
				\item in software implementations, the S-Box is usually realized as a lookup table
				\item described as computing \alert{multiplicative inverse} in $GF(2^8)$
				\includegraphics[width=\linewidth]{./figures/importanceofsboxes.png}
			\end{betterlist}
			\item \alert{Diffusion Layer:} Provides diffusion over all input state bits. Performs a linear operation on state matrices $A$ and $B$, i.e., $DIFF(A) \oplus DIFF(B) = DIFF(A \oplus B)$
			\begin{betterlist}
				\item \alert{ShiftRows Sublayer:} Rows of the state matrix are shifted cyclically
				\item \alert{MixColumn Sublayer:} Linear transformation which mixes each column of the state matrix. All arithmetic is done in the Galois field $GF(2^8)$ ($\{0, 1\}^8$ with addition as bitwise exor and an \enquote{appropriate} multiplication). So one does multiplication with the same matrix and this matrix is fixed for AES. This matrix needs to have the property of being \alert{invertable}
			\end{betterlist}
			\item \alert{Key Addition Layer:}
			\begin{betterlist}
				\item $C \oplus k_i$, state matrix $C = C_0C_1\ldots C_{15}$, \alert{subkey} $k_i$ generated in the \alert{key schedule}, $\#subkeys = \#rounds + 1$
				% , different key schedules for the d{i}fferent key sizes
				\item \alert{word-oriented:} $1$ word = $32$ bits
				\item \alert{round coefficient} $RC$ represents an element of $GF(2^8)$. Until $RC[8]$ it is leftshift, then one can't shift anymore and chooses elements of the Galois field again, starting with $RC[1] = x^0 = (00000001)_2$
			\end{betterlist}
		\end{betterlist}
	\end{betterlist}
\end{minipage}

\newpage

\begin{minipage}[t]{0.198\pagewidth}
	\fbox{AES Continue}
	\begin{betterlist}
		\item \alert{Decryption:}
		\begin{betterlist}
			\item all layers must be inverted
			\item inverse S-Box: $A_i = S^{-1}(B_i) = S^{-1}(S(A_i))$ usually realized as a lookup table
			\item the product of the inverse of the $4\times 4$ and the $4\times 4$ matrix (in $GF(2^8)$) is the identity matrix ($1$es from top left to bottom right)
			\item Key Addition layer is its own inverse, Decryption key schedule. Before starting decryption, first compute all subkeys from the
			actual key (as done for encryption) and apply in reverse order
		\end{betterlist}
		\begin{minipage}[b]{0.5\linewidth}
			\includegraphics[width=\linewidth]{./figures/invshiftrowslayer.png}
		\end{minipage}
		\begin{minipage}[b]{0.5\linewidth}
			\includegraphics[width=\linewidth]{./figures/inversemixcolumnlayer.png}
		\end{minipage}
		\item \underline{Attacks / Security:}
		\begin{betterlist}
			\item \sout{Brute-force attack:} Due to the key length of $128$, $192$ or $256$ bits, a brute-force attack is not possible
			\item \sout{Analytical attacks:} There is no analytical attack known that is better than brute-force
			\item Side-channel attack
		\end{betterlist}
	\end{betterlist}
	\begin{betterlist}
		\item \underline{Encrypting longer plain text:}
		\begin{betterlist}
			\item \alert{ECB} = Electronic Code Book Mode (Straightforward application of block cipher)
			\begin{betterlist}
				\item Problem that with the same key identical plain text blocks are mapped to the same cipher text blocks (it is deterministic). Each pixel $\hat= 1$ Byte, $16$ pixels = $1$ AES-block
			\end{betterlist}
			\includegraphics[width=\linewidth]{./figures/ecb.png}
			\includegraphics[width=\linewidth]{./figures/ecb2.png}
			\begin{betterlist}
				\item \alert{Traffic analysis and substitution attack:} If know block with encoding of account number, without knowing how encryption works, one can replace ones own encrypted account number with account number of other account in other transmission
			\end{betterlist}
			\item \alert{CBC} = Cipher Block Chaining Mode
			\begin{betterlist}
				\item \underline{advantage:} no \alert{traffic analysis} possible
			\end{betterlist}

			\includegraphics[width=\linewidth]{./figures/cbc.png}
			\includegraphics[width=0.8\linewidth]{./figures/cbc2.png}
		\end{betterlist}
	\end{betterlist}
\end{minipage}
\begin{minipage}[t]{0.198\pagewidth}
	\fbox{Mathematical background}
	\begin{betterlist}
		\item \underline{Overview:}
		\begin{betterlist}
			\item \alert{finite field} $(\{0, 1\}^8, +, \cdot)=GF(2^8)$ is a special case of a Galois field $GF(p^k)$, with a prime number $p$ and a natural number $k$, having $p^k$ elements
			\begin{betterlist}
				\item AES needs fitting definitions of $+$ and $\cdot$ on $\{0, 1\}^8$ to turn it into a \alert{finite field}. Small finite field with $256$ elements
				\item The irreducible polynomial used in AES is Part of its specification. It is $g(X) = X^8 + X^4 + X^3 + X + 1$. Several irreducable polynomials of degree $8$
				\item one needs finite field property at point where one wants to invert the MixColumn Operation, one needs the property of having multiplicative inverse, to derive inverse matrix
				% \item AES Summary and Explanation, you can use this complete nice mathematical background for aes and just construct a multiplication table you just take two bytes use this mathematical approach and compute the product of two bytes which result in a byte and then you take this table and you forgot about all the mathematical background and you can still do computations in aes because in aes you Just need this table this is true for aes but this is not true for what we will see next this RSA for RSA we really cannot compress everything you must know into one table for this we need some some insight into mathematical background. Byte, length 8, degree 7, Irreducible then each row contains a 1, there see ninverse element to row element, apart from 0, Small hint key schedule of AES, remainder of polynomial division, MixColumn multiplication connection, single operations are always additions or multiplications of bytes
			\end{betterlist}
			\item $(\mathbb{Z}/m\mathbb{Z}, +, \cdot)$, is a \alert{field} and coincides with $GF(m)$ if $m$ is a prime number, else it is a Commutative ring with multiplicative identity $\overline{1}$, if $n$ is prime it is a field
			\begin{betterlist}
				\item to be useable for AES, $m$ must be a prime number, but $2^8-1$ isn't
				\item RSA, modulo arithmetic, calculations in the ring $(\mathbb{Z}/m\mathbb{Z}, +, \cdot)$. RSA is a huge residue class ring, $2.048$ Bits, in the order of $2^{2046}$, $m$ is product of two really huge prime numbers, just on table not possible
			\end{betterlist}
		\end{betterlist}
		\begin{betterlist}
			\item \alert{Groups}
			\begin{betterlist}
				\item $(R, *)$, $R$ is a non-empty set, $*$: $R \times R \rightarrow R$ is a \alert{group} \textit{iff}
				\begin{betterlist}
					\item $*$ is \alert{associative}
					\item an \alert{(unique) identity element} (unit element) $n \in R$ with $a * n = n * a = a \enspace\forall a\in R$
					\item for each $a\in R$ exists an \alert{inverse element} $a^{-1}\in R$ with $a * a^{-1} = a^{-1} * a = n$
				\end{betterlist}
				\item \alert{abelian group} \textit{iff} $*$ is commutative
			\end{betterlist}
			\begin{betterlist}
				\item \alert{Fields (germ. Körper)}
				\begin{betterlist}
					\item $(F, +, \cdot)$, $F$ is a non-empty set, $+,\cdot$: $F \times F \rightarrow F$ is a \alert{field} \textit{iff}
					\begin{betterlist}
						\item $(F, +)$ is an \alert{abelian group}
						\item \alert{Multiplication} is \alert{associative}
						\begin{betterlist}
							\item already required by \alert{abelian group} property
						\end{betterlist}
						\item \alert{Multiplication} is \alert{commutative}
						\begin{betterlist}
							\item already required by \alert{abelian group} property
						\end{betterlist}
						\item $(F\setminus \{0\}, \cdot)$ is an \alert{abelian group} ($0$ is the additive identity)
						\item \alert{Distribuitivity}
					\end{betterlist}
					\item \alert{Difference} to \alert{commutative rings} with (multiplicative) identity: For each element of $F\setminus\{0\}$ there is a multiplicative inverse
				\end{betterlist}
			\end{betterlist}
			\begin{betterlist}
				\item \alert{Rings}
				\begin{betterlist}
					\item $(R, +, \cdot)$, $R$ is a non-empty set, $+,\cdot$: $R \times R \rightarrow R$ is a \alert{ring} \textit{iff}
					\begin{betterlist}
						\item $(R, +)$ is an \alert{abelian group}
						\item \alert{Multiplication} is \alert{associative},
						\item \alert{(Special) Distributivity}: $a \cdot (b + c) = a \cdot b + a \cdot c$ and $(b + c)\cdot a = b \cdot a + c\cdot  a$ (commutativity of multiplication not required)
					\end{betterlist}
					\item \alert{commutative ring} \textit{iff} the multiplication is also commutative
				\end{betterlist}
			\end{betterlist}
		\end{betterlist}
		\fbox{Greatest common devisor (gcd)}
		\begin{betterlist}
			\item \underline{Prime factorization:}
			\begin{betterlist}
				\item $24 = 2\cdot 12 = 2\cdot 2\cdot 6 = \textcolor{PrimaryColor}{2}\cdot \textcolor{PrimaryColor}{2}\cdot 2\cdot \textcolor{PrimaryColor}{3}$
				\item $36 = 2\cdot 18 = 2\cdot 2\cdot 9 = \textcolor{PrimaryColor}{2}\cdot \textcolor{PrimaryColor}{2}\cdot \textcolor{PrimaryColor}{3}\cdot 3$
				\begin{betterlist}
					\item not $2\cdot 18 = 2 \cdot 3\cdot 6$, is only correct be coincidence, choose smallest possible prime number first
				\end{betterlist}
				\item $gcd(24, 36) = \textcolor{PrimaryColor}{2}\cdot \textcolor{PrimaryColor}{2}\cdot \textcolor{PrimaryColor}{3} = 12$
			\end{betterlist}
			\item \underline{Devisor series:}
			\begin{betterlist}
				\item $devseries(24) = (1, 2, 3, 4, 6, 8, \textcolor{PrimaryColor}{12}, 24)$
				\item $devseries(36) = (1, 2, 3, 4, 6, 9, \textcolor{PrimaryColor}{12}, 18, 36)$
				\item $gcd(24, 36) = \textcolor{PrimaryColor}{12}$
			\end{betterlist}
			\item \underline{Euklidian algorithm:}
			\begin{betterlist}
				\item $\begin{aligned}[t]
						36\mod 24                           & = 12 \\
						24\mod \textcolor{PrimaryColor}{12} & = 0
					\end{aligned}$
				\item $gcd(36,24) = 12$
			\end{betterlist}
			\item \alert{Least common multiple (lcm)}
			\begin{betterlist}
				\item $gcd(6, 8) = \dfrac{6\cdot 8}{lcm(6, 8)} = \dfrac{48}{24} = 2$
			\end{betterlist}
			\begin{betterlist}
				\item \underline{Prime factorisation:}
				\begin{betterlist}
					\item $84 = 2\cdot 2\cdot 3\cdot 7 = 2^2\cdot 3\cdot \textcolor{PrimaryColor}{7}$
					\item $120 = 2\cdot 2\cdot 2\cdot 3\cdot 5= \textcolor{PrimaryColor}{2^3}\cdot \textcolor{PrimaryColor}{3}\cdot \textcolor{PrimaryColor}{5}$
					\item $lcm(84, 120) = \textcolor{PrimaryColor}{2^3}\cdot \textcolor{PrimaryColor}{3}\cdot\textcolor{PrimaryColor}{5}\cdot \textcolor{PrimaryColor}{7} = 840$
				\end{betterlist}
			\end{betterlist}
			\begin{betterlist}
				\item \underline{Multiple series:}
				\begin{betterlist}
					\item $mulseries(6) = (6, 12, 18, \textcolor{PrimaryColor}{24}, 30, 36, 42, 48)$
					\item $mulseries(8) = (8, 16, \textcolor{PrimaryColor}{24}, 32, 40, 48)$
					\item $lcm(6, 8) = \textcolor{PrimaryColor}{24}$
				\end{betterlist}
			\end{betterlist}
			\begin{betterlist}
				\item \underline{Greatest common divisor (gcd):}
				\begin{betterlist}
					\item $lcm(6, 8) = \dfrac{6\cdot 8}{gcd(6, 8)} = \dfrac{48}{2} = 24$
				\end{betterlist}
			\end{betterlist}
		\end{betterlist}
	\end{betterlist}
\end{minipage}
\begin{minipage}[t]{0.198\pagewidth}
	\fbox{Residue class ring}
	\begin{betterlist}
		\item $\boxed{12 / 10 = 1 \wedge 12 \mod 10 = 2 \enspace(\text{remainder})} \Rightarrow 2 \equiv 12 \mod 10 \Leftrightarrow 2 + 1 \cdot 10 = 12 + 0 \cdot 10 \Leftrightarrow \boxed{2 + 1 \cdot 10 = 12}\Leftrightarrow 2 + 2 \cdot 10 = 12 + 1 \cdot 10$ ($12$ is the dividend, because it's larger than $2$, $0\le 2< 10$)
		\item \alert{Division with remainder in $\mathbb{Z}$:} For each $a \in \mathbb{Z}$ there is a unique integer $r$ with $0 \le r < m$ and $a = q \cdot m + r$ for some $q \in Z$ and $m \in \mathbb{N}, m > 1$
		\begin{betterlist}
			\item $10 / 4 = 1\enspace R:2$ (was bleibt übrig, wenn man Vielfache des Divisors so nah wie möglich aber kleiner an Dividend bekommen will) or $10 \mod 4 = 2$ (wieviel über Modul / Devisor drüber, wenn man Vielfache des Divisors so nah wie möglich aber kleiner an Dividend bekommen will), $4 / 10 = 0\enspace R:4$ (bei Vielfachen $0$ des Modul / Devisor  wieviel es bis zum Dividend ist) or $4 \mod 10 = 4$ (bei Vielfachen $0$ des Modul / Devisor um wieviel man darüber ist)
		\end{betterlist}
		\item $a, b \in Z$ are \alert{congruent modulo} $m$ ($a \equiv b \mod m$) \textit{iff} $a - b = q \cdot m$ for some $q \in Z$
		\begin{betterlist}
			\item $a \equiv b \mod m$ iff the division with remainder wrt. $m$ gives the same remainder for $a$ and $b$
		\end{betterlist}
		\item The \alert{residue class} of $r \in \mathbb{Z}$ with $0 \le r < m$ is the set $\overline{r} = \{a \in \mathbb{Z} \mid a \equiv r \mod m\} = \{q \cdot m + r \mid q \in \mathbb{Z}\}$
		\begin{betterlist}
			\item $(\mathbb{Z}/m\mathbb{Z}, +, \cdot), \mathbb{Z}/m\mathbb{Z} = \{\overline{0}, \overline{1}, \ldots, \overline{m-1}\}$
			\item for $a \in \overline{r}$ the integer $a \mod m$ is the unique element $r$ of $\overline{r}=\overline{a}=\{0\cdot m + r, 1\cdot m + r, \ldots, a,\ldots\}=\{q\cdot m + r \mid q\in\mathbb{Z}\}$ with $0 \le r < m$ (so in the example $\overline{13} = \overline{3}$)
			\item each $i \in \mathbb{Z}$ is in exactly one of $m$ pairwise disjoint residue classes: $\overline{0}, \overline{1}, \ldots, \overline{m-1}$
		\end{betterlist}
		\item If $a_1 \equiv a_2 \operatorname{mod} m$, $b_1 \equiv b_2 \operatorname{mod} m$ then
		\begin{betterlist}
			\item $a_1 + b_1 \equiv a_2 + b_2 \operatorname{mod} m$
			\item $a_1 \cdot b_1 \equiv a_2 \cdot b_2 \operatorname{mod} m$
		\end{betterlist}
		\item \underline{Addition and Multiplication well definied:} $\overline{a} + \overline{b} = \overline{a + b}$, $\overline{a} \cdot \overline{b} = \overline{a \cdot b}$
		\begin{betterlist}
			\item Addition and multiplication on $\mathbb{Z}/m\mathbb{Z}$ are \alert{well-defined}, since for the result it does not matter which representatives of $a$ and $b$ are chosen
			\item residue class ring $(\mathbb{Z}/m\mathbb{Z}, +, \cdot)$ forms a \alert{commutative ring} with \alert{multiplicative identity} $\overline{1}$ for all $\overline{p} \in \mathbb{Z}/m\mathbb{Z}$
			\item \underline{Example:} \alert{residue class ring:} $\mathbb{Z}/5\mathbb{Z} = \{\overline{0}, \overline{1}, \overline{2}, \overline{3}, \overline{4}\}$ and one \alert{residue class:} $\overline{2} = \{q\cdot 5 + 2 \mid q\in \mathbb{Z}\}$, if one does computation one does not handle infinite sets but handle representatives: $\overline{2} + \overline{3} + \overline{4} = \overline{2 + 3 + 4} = \overline{4}$, modulo operation gives smallest possible element in residue class
			\item \alert{Modular division:} $\overline{b}/\overline{a} = \overline{b}\cdot \overline{a}^{-1}$
			\item \alert{Modular reduction:}
			\begin{betterlist}
				\item $3^8 \mod 7 = ((81 \mod 7) \cdot (81 \mod 7)) \mod 7 = 4 \cdot 4 \mod 7 = 16 \mod 7 \equiv 2 \mod 7$
			\end{betterlist}
		\end{betterlist}
	\end{betterlist}
	\fbox{Finite / Galois Fields}
	\begin{betterlist}
		\item $GF(2) = (\mathbb{Z}/2\mathbb{Z}, +, \cdot)$ (same as $(\{0, 1\}, \oplus, \land)$) forms a field with additive identity $0$, multiplicative identity $1$ and $1$ as the multplicative inverse of $1$
		\item Making $(\{0, 1\}^k, +, \cdot)$ a finite field
		\begin{betterlist}
			\item define addition on $(\{0,1\}^k, +, \cdot)$ as bitwise Xor
			\begin{betterlist}
				\item Xor on $\{0,1\}$ is the same as $+$ on $GF(2) = \mathbb{Z}/2\mathbb{Z} = \{\overline{0}, \overline{1}\}$
			\end{betterlist}
			\item definition of $\cdot$ on $\{0,1\}^k$ such that the result becomes a finite field
			\begin{betterlist}
				\item the And operation is $\cdot$ on $GF(2)$
				\item to be done by reduction to the consideration of so-called polynomial rings over $GF(2)$
			\end{betterlist}
			\item \underline{Why not $\mathbb{Z}/2^k\mathbb{Z}$?:} an element $e$ that is even is not invertable in $\mathbb{Z}/2^k\mathbb{Z}$ and therefore can't be a field, all elements in $\overline{1} = \{1 + q\cdot 2^k \mid q\in \mathbb{Z}\}$ are odd as $q\cdot 2^k$ is always even, $a\cdot e$ is even, because $e$ is even, so it can't be in a set where all elements are odd
		\end{betterlist}
	\end{betterlist}
\end{minipage}
\begin{minipage}[t]{0.198\pagewidth}
	\fbox{Polynomial rings over $GF(2)$}
	\begin{betterlist}
		\item $F[X]$: All polynomials in one variable $X$ over the field $F$, coefficients are elements of $F$
		\item $F[X]_n$: Subset of $F[X]$ with $deg(g) < n$ for polynomials $g$
		\item we map $\{0,1\}^k$ bijectively to $GF(2)[X]_k$ by the mapping $\varphi(v_{k-1}, \ldots, v_0) = v_{k-1} X^{k-1} + \ldots + v_2 X^2 + v_1 X + v_0$ in order to be able to define a \alert{multiplication} on $\{0,1\}^k$
		\begin{betterlist}
			\item the usual \alert{polynomial addition} in $GF(2)[X]_k$ \enquote{is compatible} with bitwise exor on $\{0,1\}^k$
		\end{betterlist}
		\item \alert{Residue Class Ring Modulo a Polynomial:} $\overline{F(X)} = F[X] / g(X)$%$:=\{\overline{u(X)} \mid u(X) \in F[X]\}$
		\begin{betterlist}
			\item \alert{residue class of $u(X)$ modulo $g(X)$}: $\overline{u(X)}:=\{v(X) \in F[X] \mid v(X) \bmod g(X)=u(X) \bmod g(X)\}$,\quad$u(X),g(X) \in F[X] \text { with } \operatorname{deg}(g(X))\geq 1$
			\item exactly $2^k$ different residue classes modulo $g(X)\in GF(2)[X]_{k+1}$, the classes $\overline{r(X)}$ with $deg(r(X)) < k$ in the residue class ring $GF(2)[X]/g(X)$
			\item \alert{Addition and Multiplication in $F[X]/g(x)$:} $\overline{u(X)}+\overline{v(X)}:=\overline{u(X)+v(X)}$ and $\overline{u(X)} \cdot \overline{v(X)}:=\overline{u(X) \cdot v(X)}$
			\begin{betterlist}
				\item again well-defined, because it doesn't matter what representetives one chooses
				\item \alert{bijective mapping} $\psi:\{0,1\}^k\rightarrow G F(2)[X] / g(X)$\\
				with $\psi\left(v_{k-1}, \ldots, v_0\right)=\overline{v_{k-1} X^{k-1}+\ldots+v_1 X+v_0}$ and $g(X)\in GF(2)[X]_{k+1}$
				\item \alert{Addition} in $GF(2)[X]/g(X)$ is bitwise exor on $\{0,1\}^k$%, because if have polynomial with degree smaller than $k$ and a polynomial with degree smaller than $k$, than the division with $g(X)$ doesn't change anything, becase we already have something smaller than $k$, so the remainder is the element itself
				\item \alert{Multiplication}: $\left(v_{k-1}, \ldots, v_0\right) \cdot\left(w_{k-1}, \ldots, w_0\right):=\psi^{-1}\left(\psi\left(v_{k-1}, \ldots, v_0\right) \cdot \psi\left(w_{k-1}, \ldots, w_0\right)\right)$
				\begin{betterlist}
					\item multiplication in $\{0,1\}^k$ \enquote{via} multiplication in $GF(2)[X]/g(X)$, works since $GF(2)[X]/g(X)$ is closed under multiplication%. So product of two bitvectors defined by using the mapping into residue classes, doing the mulitplication of residue classes and then mapping back
					% \item Example, polymomial with $X^4$ would be out of the polynomial ring with degree smaller than $k=3$, therefore consider residue classes which means one takes result and reduce by polymomial with degree $k=3$, so remainder has a degree smaller than $3$
				\end{betterlist}
			\end{betterlist}
			\begin{betterlist}
				\item for $(GF(2)[X]/g(X), +, \cdot)$ to be a field, there must be multiplicative inverse for all elements $\ne 0$
				\begin{betterlist}
					% \item if $(GF(2)[X]/g(X), +, \cdot)$ is a field, then also $(\{0, 1\}^k, +, \cdot)$
					\item $g(X)$ must be \alert{irreducable}
					\item a polynomial $g(X) \in F[X]$ with $deg(g(X)) > 0$ is \alert{irreducible} \textit{iff} the following holds: If $g(X) = u(X) \cdot v(X)$ for $u(X), v(X) \in F[X]$ then $u(X) \in F$ or $v(X) \in F$
					\begin{betterlist}
						% \item \alert{irreducable} intuitively means that one doesn't have a decomposition of $g(X)$ into two non-trivial factors, so if one has $g(x) = u(X)\cdot v(X)$ then either $u(X)$ or $v(X)$ have to be a field element, trivial if one of them is a field element and non-trivial if one can really decompose it into two non-trivial parts, it's similiar to prime numbers, if have integer number, one can also ask whether one can decompose it into the product of two non-trivial numbers or whether for each decomposition into two factors at least one for them has to be the one, if one computes the decomposition into prime factors, one only has one prime factor, one cannot decompose the prime numbers into non-trivial factorisation.
						\item Field element $\hat =$ Polynomial of degreee $0$
					\end{betterlist}
				\end{betterlist}
			\end{betterlist}
			\item \underline{Polynomial Residue Class Ring:} Let $g(X)\in F[X]$ with $deg(g(X)) > 0$. $g(X)$ is \alert{irreducible} \textit{iff} $(F[X]/g(X), +, \cdot)$ is a field
			% \begin{betterlist}
			% 	\item \underline{Lemma: Polynomial Residue Class Ring:} For each $u(X), v(X) \in F[X] \setminus \{0\}$ with $max(deg(u(X), deg(v(X)) > 0$ there is a gcd $g(X) \in F[X]$ and there is a representation $g(X) = q_u(X) \cdot u(X) + q_v(X) ∙ v(X)$ with $q_u(X), q_v(X) \in F[X]$
			% \end{betterlist}
		\end{betterlist}
	\end{betterlist}
	\fbox{Invertible Elements}
	\begin{betterlist}
		\item \underline{Theorem (MA):} Let $n \in \mathbb{N}, n > 1$. $n$ is a prime number \textit{iff} $(\mathbb{Z}/n\mathbb{Z}, +, \cdot)$ is a field
		\begin{betterlist}
			% \item $n$ prime number $\hat =$ irreducable polynomial $g(X)$
			\item \alert{Prime Number}: A number $n \in \mathbb{N} \setminus \{0, 1\}$ is a prime number \textit{iff} the following holds: If $n = u \cdot v$ for $u, v \in N$ then $u = 1$ or $v = 1$
			% \item Proof, inverse of $\overline{1}$ is $\overline{1}$, for all elements besides $\overline{0}$ there's an inverse element, if $n$ would not be a prime number, then there would be a non-trivial decomposition into two numbers where both numbers are from this set: $\{2, \ldots, n-1\}$ and therefore we can conclude that $n$ must be a prime number
			\begin{betterlist}
				\item Theorem (MA') implies Theorem(MA)
			\end{betterlist}
		\end{betterlist}
		\item \underline{Theorem (MA’):} Let $n \in \mathbb{N}, n > 1$. The inverse of $\overline{a} \in (\mathbb{Z}_n, +, \cdot)$ exists \textit{iff} $gcd(a, n) = 1$
		\begin{betterlist}
			\item $a$ and $n$ are \alert{relatively prime} \alert{iff} $gcd(a, n) = 1$
			% \item this does not only say something about um n equal to a prime number but it also says something about other rings where n is not a prime number and for those rings this theorem exactly characterizes the invertible elements, this are the elements a bar where the greatest common divisor of a and n here is equal to one which means a and n are relatively prime
			% \begin{betterlist}
			% 	\item \underline{reason for $1$:} It is the neutral element of multiplication: $m \cdot m^{-1} = 1$
			% \end{betterlist}
			\item if $p$ is even then the greatest common devisor of $2^k$ and $p$ is at least $2$ which is not $1$ of course and therefore if $p$ is even then it's not invertible. If $p$ is odd then the prime factor decomposition doesn't contain $2$ which means the greatest common divisor of an odd $p$ and $2$ to the power of $k$ is $1$ and then we immediately have that this element is invertible % which means in this special case with this theorem we can completely characterize the even elements are not invertible, the odd elements are invertible
			\begin{betterlist}
				\item $2^k$ only has $2$ and $1$ as devisors, $2\cdot k$ has a lot of devisors including $1$ and $2$, thus the gcd is $1\cdot 2 = 2$, thus $2^k$ is not invertible
				\item $2^k$ only has $2$ and $1$ as devisors, $2\cdot k+1$ has a lot of devisors excluding $2$, thus the gcd is $1$, thus $2^k+1$ is invertible
			\end{betterlist}
			\item \underline{Lemma (MA):} For each $u, v \in \mathbb{Z} \setminus \{0\}$ there is a gcd $g \in \mathbb{N} \setminus \{0\} \subseteq \mathbb{Z} \setminus \{0\}$ and there is a representation $g = q_u \cdot u + q_v \cdot v$ with $q_u, q_v \in \mathbb{Z}$.
		\end{betterlist}
		\item \alert{Eulers Theorem:} Let $a\in \mathbb{Z}$, $n\in \mathbb{N}\setminus \{0\}$ with $gcd(a, n) = 1$. Then $a^{\Phi(n)}\equiv 1 \mod n$
		\begin{betterlist}
			\item If $gcd(a, n) = 1$, then $\overline{a^{\Phi(n)-1}}$ is the \alert{mulitplicative inverse} of $\overline{a}$ in $\mathbb{Z}/n\mathbb{Z}$
		\end{betterlist}
	\end{betterlist}
	\fbox{Polynomial Division}
	\begin{betterlist}
		\item in normal division one would substract, but since we are doing computations based on $GF(2)$, addition and substraction are the same, doing XOR and therefore each element is it's own inverse. Final result as soon as remainder is smaller than divisor. If one divides by a polynomial of degree $k$, one obtains a remainder with degree smaller than $k$
		% \item \underline{Lemma: Division with remainder:} For $u(X)$, $g(X) \in F[X]$ with $g(X)\ne 0$ there exist unique polynomials $q(X), r(X) \in F[X]$ with $u(X) = q(X) g(X) + r(X)$ and $deg(r(X)) < deg(g(X))$ or $r(X) = 0$
		% \begin{betterlist}
		% 	\item $u(X) \mod g(X) = r(X)$, $u(X) \operatorname{div} g(X) = q(X)$
		% 	\item $r(X)$ is a unique remainder
		% \end{betterlist}
	\end{betterlist}
	\fbox{Spectre:}
	\begin{betterlist}
		\item Leak information directly from user applications
		\item \underline{Preconditions for attacker:}
		\begin{enumerate}
			\item based on unexpected code executions due to speculative execution
			\begin{betterlist}
				\item Tries to predict whether branch is taken or not based on previous executions of branch instruction. Speculatively execute instructions at predicted position, avoid changes of architectural states during \alert{speculative execution}. If branch condition is finally evaluated and branch was mispredicted: Flush the effects of speculative executions (like pipeline flushing, invalidation of register writes in retiring phases etc.)%. Acceleration if branch correctly predicted, but at least correct architectural state if mispredicted
			\end{betterlist}
			\item can influence executed code of the victim
			\begin{betterlist}
				\item JavaScipt code is instrumented by web browser for memory protection (\enquote{sandboxing})
			\end{betterlist}
		\end{enumerate}
		\item \underline{Steps:}
		\begin{betterlist}
			\item \underline{instrumented code:} \verb|if (x < array1_size) y = probe_array[array1[x] << 12];|
			\begin{betterlist}
				\item \verb|array1| is an array of unsigned bytes of size \verb|array1_size|
				\item \verb|probe_array| is an array of unsigned bytes of size $1$ MB = $220$ Bytes
				% \item \verb|array1[x] << 12| cannot be larger than $2^{20}$ , so checking for access to \verb|probe_array| is not necessary
				\item choose \verb|x > array1_size| such that \verb|sd = array1[x]| is a secret data byte in the address space of the victim
				\item execute instrumented code several times with \verb|x < array1_size| until branch prediction predicts if-condition to be true
				% \item then \verb|x > array1_size| is executed such that \verb|sd = array1[x]| is secret data byte in the address space of the victim. \verb|y = probe_array[array1[x] << 12]| is executed by speculative execution
				\item speculative execution is aborted as soon as \verb|x < array1_size| evaluates to \verb|false|.
			\end{betterlist}
			\item Readout of cache information as in Meltdown
			\begin{betterlist}
				\item \verb|probe_array| is uncached (to enable readout of microarchitectural state). Caching by iterating the attack
				\item Uncaching by cache flushing or cache eviction. Evict certain cache line just by (at most) $A$ memory accesses with the same set number. Cache eviction instead of flushing needed in JavaScript code, since JavaScript lacks instructions for cache flushing
			\end{betterlist}
		\end{betterlist}
		\item \underline{Preconditions to win the \enquote{race}:}
		\begin{enumerate}
			\item \verb|array1_size| is uncached (to slow down if-condition checking)
			\item \verb|sd = array1[x]| is cached (to accelerate \verb|probe_array[array1[x] << 12]|)
		\end{enumerate}
		\item \underline{Possible countermeasures:}
		\begin{betterlist}
			\item Stricter sandboxing
			\begin{betterlist}
				\item Many modern browsers execute each website in a separate process
				\item No access to browser‘s address space possible
				% \item Spectre attack was demonstrated with Google Chrome version 62.0.3202
				% \item Modern Google Chrome browsers are not vulnerable anymore
			\end{betterlist}
			\item No speculative execution?
			\begin{betterlist}
				\item Huge performance degradation
			\end{betterlist}
		\end{betterlist}
	\end{betterlist}
\end{minipage}
\begin{minipage}[t]{0.2\pagewidth}
	\fbox{Attacks by Microarchitectural Data Sampling}
	\begin{betterlist}
		\item \alert{Instruction Set Architecture (ISA) of a processor:} interface it provides to the software it executes. Specifies set of instructions and registers and memory processed by the instructions
		\item \alert{Architectural states:} states defined by the ISA, i.e., memory cells and user-visible registers
		\item \alert{Microarchitectural states:} processor states not defined in the ISA, e.g. states of functional units, states of reservation stations, cache contents etc.
		\item It is ensured that effects of invalid out-of-order executions or discarded speculative executions do not change architectural states. However they may change microarchitectural states. Reading out microarchitectural states as side-channel information may leak information. Vulnerability on the hardware level, not on software level
	\end{betterlist}
	\fbox{Meltdown:}
	\begin{betterlist}
		\item uses Effects of \alert{out-of-order execution} to transport secret data into \alert{microarchitectural state} ($=$ buffers in reservation stations) and  uses \alert{data caching} to extract secret from microarchitectural state to \alert{architectural state}
		\begin{betterlist}
			\item breaks the isolation between user applications and the operating system. Allows a program to access the memory (of other programs and the operating system), even when it is not authorized to do so
		\end{betterlist}
		\item \underline{Preconditions for attacker}
		\begin{enumerate}
			% \item[\bfseries\color{PrimaryColor}$\bullet$] run on same machine and ability to start processes on this machine
			\item[\bfseries\color{PrimaryColor}$\bullet$] able to execute code on the attacked machine
			\item out-of-order execution with forwarding
			\begin{betterlist}
				\item \alert{out-of-order execution:} several instructions issued to several functional units in parallel, may be finished in an order different from the order in the program
				\item \alert{forwarding:} in Retiring phase after execution of the instruction $i$ by a FU $n$ a result token is generated which is sent (forwarded) to all reservation stations waiting for the result of FU $n$. Instruction can executed before register has been written in case of RAW (= read after write), if calculation result that should be written to this is forwarded to FU dealing with this instruction
			\end{betterlist}
			\item virtual memory managment with / based on paging and kernel space mapped into virtual address space of each process
			\begin{betterlist}
				% \item each process virtually uses it's complete address space (of e.g. $2^{64}$ addresses). Both \alert{virtual memory} and \alert{physical memory} are partitioned into \alert{pages} of the same size (e.g. $4 KB = 2^{12} Byte$). Most significant bits (e.g. $64 – 12 = 52$ bits) of virtual address = \alert{page number}. Most significant bits (e.g. $64 – 12 = 52$ bits) of physical address = \alert{page frame number}. Access to virtual page number $v$: Hardware Memory Management Unit (\alert{MMU}) checks in page table whether process has physical memory for page $v$. \underline{If yes:} MMU provides physical page frame number $p$ and makes access to $p$. \underline{If no:} \alert{Page fault}, operating system makes new physical memory page with page frame number $p$ available, enters $p$ for virtual page number $v$ into page table. Each process has its own \alert{page table}
				\item \underline{advantage of including kernel space into process page tables:} TLB does not need to be flushed for privileged accesses to kernel space, e.g. system calls
			\end{betterlist}
			\item use of caches not only for Translation Lookaside Buffer (TLB) but also for data and instruction and data caches for memory accesses, in particular data caching
			\begin{betterlist}
				\item modern CPUs use instruction and data caches for memory accesses to avoid slow memory accesses as far as possible
				\item fast cache memory storing page table entries recently accessed. Lookup in TLB first. Access page table only in case of TLB miss
			\end{betterlist}
		\end{enumerate}
		\item \underline{Steps:}
		\begin{enumerate}
			\item unauthorized READ of one byte from \verb|secret_kernel_address| is managed by Reservation Station $R_s$. The following READ in \verb|probe_array| is issued to Reservation Station $R_p$, before the READ from \verb|secret_kernel_address| has been finished. As soon as the secret data byte \verb|sd := M[secret_kernel_address]| arrives in $R_s$, it is forwarded to $R_p$ which reads \verb|probe_array[sd << 12]|. During the read of \verb|probe_array[sd << 12]| data are cached. The hardware memory protection raises an exception which is registered in the retiring phase of $R_s$ , writing into \verb|IN1| is prevented, exception is handled. However the secret data \verb|sd| left traces in the cache
			\item read out of the changed microarchitectural state: Read from \verb|probe_array[0 << 12]|, \ldots \verb|probe_array[255 << 12]|. Measure the access time in all $256$ cases. If access to address $i \ll 12$ is fast, all other accesses slow, then the leaked secret is $i$
			\begin{betterlist}
				\item left shift by $12$, since a processor does not read single bytes into the cache, but cache lines (surroundings of address)
				\item $256$ different access to \verb|probe_array| because the read byte can have $256$ different values and one just uses the \verb|probe_array| to find out which value has been read, because the address that can quickly be read corresponds to the forwarded value
				\item ensure that the contents of \verb|probe_array| are not cached before the attack. E.g. achieved by flushing the cache lines using x86 instruction \verb|clflush|
			\end{betterlist}
		\end{enumerate}
		\includegraphics[width=\linewidth]{./figures/meltdown_visualisation.png}
		\item \underline{Ways out of abortion by exception handling:}
		\begin{enumerate}
			\item \underline{Fork-and-crash:} Fork attacking application before accessing invalid memory location. Access invalid memory location only in child process. Recover the secret by probing in the parent process
			\item \underline{Exception handling:} Install an own signal handler for segmentation violation that does not crash the process
			\item \underline{Exception suppressing via TSX:} Use Intel‘s Transactional Synchronization Extensions (TSX). Transaction = Sequence of instructions that execute atomically. All instructions of a transaction are inverted, if one of them fails. But no exception raised!. Just wrap the attacking code into a TSX transaction ...
		\end{enumerate}
		\item \uline{only works, if race between exception handling and cache loading (reading \texttt{probe\_array[sd]}) is won by cache loading:}
		\begin{enumerate}
			\item try to read from \verb|secret_kernel_address| already before actual attack (e.g. by other processes). \underline{Goal:} Bring \verb|sd = M[secret_kernel_address]| into cache, so \verb|sd| is available earlier
			\item assuming that \verb|probe_array| starts at the beginning of a memory page and the page size is $2^{12}$, we first access $probe\_array[0 \ll 12 + 2^{11}], \ldots, probe\_array[255 \ll 12 + 2^{11}]$ (access all $256$ \verb|probe_array| pages with addresses in the middle of the pages). \underline{Goal:} Page addresses go into TLB, accelerated access to \verb|probe_array|...
			\item \sout{bringing physical address for \texttt{secret\_kernel\_address} into TLB won't help because it would accelerate both exception handling and cache loading}
		\end{enumerate}
		\item \underline{Possible countermeasures:}
		\begin{betterlist}
			\item Do not map kernel memory into address space of user processes. \underline{Disadvantages:}
			\begin{betterlist}
				\item For every system call the TLB has to be flushed. Performance problem
			\end{betterlist}
			\item fix vulnerability in hardware
			\begin{betterlist}
				\item check access rights before loading
				\item or implement hardware in a way that race is always won by access right checking + aborting the following instructions
				\item in case of right violation: Flush the cache completely or possibly flush the "last accessed" cache lines (needs some hardware modification)
			\end{betterlist}
		\end{betterlist}
	\end{betterlist}
\end{minipage}
\begin{minipage}[t]{0.2\pagewidth}
	\fbox{Associative Cache}

	\includegraphics[width=\linewidth]{./figures/associative_cache.png}
\end{minipage}
\begin{minipage}[t]{0.2\pagewidth}
	\fbox{Side Channel Attacks}
	\begin{betterlist}
		\item \underline{Observable side channel \alert{outputs} for side channel attacks:}
		\begin{betterlist}
			\item Power consumption
			\begin{betterlist}
				\item \underline{Two types:}
				\begin{betterlist}
					\item \alert{static power (data independent):} typically leakage
					\item \alert{dynamic power (data / activity dependent):} stems from circuit activity and each signal change consumes dynamic power
				\end{betterlist}
				\item \underline{Dynamic CMOS Power Consumption:}
				\begin{betterlist}
					\item \alert{Switching current} from VDD to load output capacitance when \alert{output} switches from $0$ to $1$
					\item \alert{Through current} from VDD to GND when both p-channel transistor and n-channel transistor turned on briefly during switch $0 \rightarrow 1$ or $1 \rightarrow 0$
				\end{betterlist}
			\end{betterlist}
			\item Electro-magnetic emissions
			\item Sound (accoustic)
			\item Light emissions (optical)
			\item Timing behavior, delay (timing and delay)
		\end{betterlist}
	\end{betterlist}
	\fbox{Power Attacks}
	\begin{betterlist}
		\item in \alert{smart cards}, one operation running at a time $\rightarrow$ simple power tracing is possible. In \alert{high-end processors} or \alert{FPGAs} typically parallel computations prevent visual SPA inspection $\rightarrow$ DPA
		\item \alert{Simple Power Analysis (SPA):}
		\begin{betterlist}
			\item targets variable instruction flow. Can reveal also data in special cases
			\item internal structure of \alert{DES} by looking at power trace (find out implemented algorithm by number of rounds etc.)
			\begin{betterlist}
				\item \sout{no direct key extraction possible for \alert{DES}}
			\end{betterlist}
			\item with higher resolution analysis on the level of single instructions (like conditional jump taken or not)
			\item \underline{data comparison:} involves string and memory comparison operations performing a conditional branch when a mismatch is found
			\item \underline{leak RSA key via square and multiply algorithm:}
			\begin{betterlist}
				\item Data dependent jump. Square always executed. Multiply depends on data. In the case of RSA, data $=$ private key
				\item \underline{Countermeasures:} in genereal execute dummy instructions to avoid leakage by power profile. Additional cost for increasing security
			\end{betterlist}
			\item \underline{Countermeasures:}
			\begin{betterlist}
				\item avoid procedures that use secret information for conditional branching operations via \enquote{creative coding} and \enquote{performance penalty}
				\item add code to hide power differences
				\item countermeasures from DPA and CPA
			\end{betterlist}
		\end{betterlist}
		\item \alert{Differential Power Analysis (DPA):}
		\begin{betterlist}
			\item targets data-dependence. Different operands present different power. Analyzes multiple execution flows. DPA typically able to make tiny differences in power consumption visible
			\item \underline{(statistical) aspect of DPA:} \alert{Averaging} over many traces to cancel out noise and \alert{form groups of measurements} and compute differences of averages
			\item \underline{Basic Principle:}
			\begin{betterlist}
				\item consider one target gate in the circuit (for hardware implementation) or one target operation (for software implementation)
				\item apply sequence of input patterns $P_0, P_1, \ldots, P_n$. Measure power consumed for each input pattern
				\item \underline{Create two sets:}
				\begin{betterlist}
					\item $S^+$ = set of patterns where output of target gate is $1$
					\begin{betterlist}
						\item if previous value of target gate was $0$, then target gate contributes to increased power consumption
						\item if previous value of target gate was $1$, then target gate does not contribute to increased power consumption
					\end{betterlist}
					\item $S^−$ = set of transitions where output of target gate is $0$
					\begin{betterlist}
						\item target gate does not contribute to increased power consumption (independently from previous value)
					\end{betterlist}
				\end{betterlist}
				% \item contribution of a single gate to overall power consumption is small, but nevertheless visible in the average power consumption of the two sets $S^+$ and $S^−$
				\item differences of single power consumptions for other gates cancel out between $S^+$ and $S^−$ by averaging
				\item $AveragePower(S^+) - AveragePower(S^−) > 0$
			\end{betterlist}
			\includegraphics[width=0.8\linewidth]{./figures/dpa_average.png}

			\item order of applying patterns not unimportant, as there has to be a switch from $0$ to $1$ in order to have a increased power consumption, so one has the best result if one applies always one pattern from $S^+$, then one from $S^-$. If order is random it should also be ok, since on average \enquote{half of the transitions in $S^+$} go from $0$ to $1$ and other half from $1$ to $1$, in $S^-$ there's never a $0\rightarrow 1$ transition
			\item \underline{\alert{one / single bit distiguisher} for partitioning into $S^+(k_s)$ and $S^−(k_s)$:} A location in the algorithm where knowledge on a small subset of key bits enables the computation of a bit value. Gate output, output bit of an operation which one can compute, if plain and / or cipher text + a (small) subset of the key is known. Hope something is correlated with key bit assumption, if correlation holds partitioning is ok, then can derive some information

			% \includegraphics[width=\linewidth]{./figures/distinguisher_not_always_working.png}
			\includegraphics[width=0.8\linewidth]{./figures/distinguisher_not_always_working_short.png}
			\begin{betterlist}
				\item effects cancel out, because transition happens at almost the same time at both places
			\end{betterlist}
			\item \alert{DPA attack on DES:}
			\begin{betterlist}
				\item with the correct assumption on $k_s$ the value of $(L_{15})_j (k_s, C_i)$ is correct with \enquote{probability $1$}
				\begin{betterlist}
					\item $AveragePower(S^+(k_s)) - AveragePower(S^−(k_s)) > 0$
				\end{betterlist}
				\item if assumption on $k_s$ is wrong, the value of $(L_{15})_j (k_s, C_i)$ is correct with \enquote{probability $\approx 0.5$}
				\begin{betterlist}
					\item $AveragePower(S^+(k_s)) - AveragePower(S^−(k_s)) \approx 0$
				\end{betterlist}
				\item since the point in time when exactly $(L_{15})_j$ is computed is unknown, DPA does not record \alert{single power values} for plain texts $P_i$, but \alert{power traces} over a certain amount of time
				% \item the power traces for different plain texts $P_i$ have to be precisely synchronized wrt. the start time of the recording
			\end{betterlist}
			\includegraphics[width=0.8\linewidth]{./figures/dpa_attack_on_des.png}

			\begin{betterlist}
				\item Compute average traces for $S^+(k_s)$ and $S^−(k_s)$. Subtract average traces. Zoom in for analyzing differential trace. Repeat for all keys $k_s$
				\item Method reveals $6$ bits of $k_{16}$. Repeat for different bit $j$ of $L_{15}$, revealing $6$ other bits. By this, $48$ out of $56$ bits of user key $k$ are known. \underline{Two options:} brute-force attack to compute remaining $8$ bits or continue with previous rounds
			\end{betterlist}
		\end{betterlist}
	\end{betterlist}
\end{minipage}
\begin{minipage}[t]{0.2\pagewidth}
	\fbox{Power Attacks Continue}
	\begin{betterlist}
		\item \alert{DPA attack on AES:}
		\begin{betterlist}
			\item use a bit (e.g. LSB) of the output of first S-box in round $1$ for partitioning. Depends on first key byte. Number of key hypotheses is $256$. Just repeat the attack for remaining $15$ S-Boxes and key bytes
			\item average traces for all $256$ possible key bytes with output bit $0$ and $1$ ($S^+(k_i)$ and $S^-(k_i)$) and then compute differential trace for all key bytes
		\end{betterlist}
		\item \underline{Countermeasures to DPA:} Difficult, but try to increase at least the number of samples needed for obtaining meaningful results by:
		\begin{betterlist}
			\item Signal size reduction
			\begin{betterlist}
				\item Use operations leaking less information in power consumption
				\item Reduce power consumption by different gate realizations (technology)
			\end{betterlist}
			\item Adding noise to power consumption measurements
			\begin{betterlist}
				\item Random noise cancels out by averaging (but more samples needed)
				\item Non-random contributions to hide key-dependent power consumption (\enquote{leak masking})
				\item Temporal noise (like varying clock speeds) for disturbing trace synchronization (shift measurements against each other, then can avoid sharp peak in differential power trace, e.g. random generator that influences clock speed)
			\end{betterlist}
			\item Balance power consumption (e.g. by dual-rail precharge logic, two transistors which always switch in opposite direction, 0 $\rightarrow$ 1 and 1 $\rightarrow$ 0 have same power consumption)
			\item Power supply filters (simplest just capacitors which just consume / cancel the peaks)
			\item Frequent key exchange to prevent attacks with large amount of samples
		\end{betterlist}
	\end{betterlist}
	\begin{betterlist}
		\item \alert{Correlation Power Analysis (CPA):}
		\begin{betterlist}
			\item goes beyond the usage of binary distinguishers (as for DPA)
			\item often less traces are needed than with DPA
			\item \underline{basic idea:} compute a correlation between \alert{power consumption} and \alert{processed values} (over several bits)
			\begin{betterlist}
				\item consider \alert{Hamming weight (HW)} of the intermediate value $x$ (e.g. a byte) that directly \alert{correlates with power consumption}
				\begin{betterlist}
					\item larger $HW(x)$, larger power consumption
					\item smaller $HW(x)$, smaller power consumption
					\item \underline{example:} Hamming weight of output of first S-box in round $1$ of AES
				\end{betterlist}
				% \item $X$ = power consumptions (for $n$ different \alert{plain texts}). $Y$ = hamming weights of some intermediate value (like output of first S-box in round $1$ of AES based on a certain key byte assumption)%. Choose the key byte hypothesis where correlation for some point in time is maximized. Repeat the previous steps for every key part in order to recover the full key
				\item assume we have $n$ \alert{power traces (of plain texts)}, each with $p$ data points $d_j^k$ $(1 \le k \le n, 1 \le j \le p)$. There are $m$ subkey guesses, for each subkey guess $i$ and each power trace the \alert{hamming weight of the guessed intermdiate value} (like output of first S-box in round $1$ of AES based on a certain key byte assumption) is $h_i^k$ $(1 \le k \le n, 1 \le i \le m)$. We have to compute $p \cdot m$ correlation coefficients $ρ_{j,i}$ for each time $j$ and each subkey guess $i$:

				\includegraphics[width=0.4\linewidth]{./figures/pearson_correlation_coefficient.png}
				% \item \alert{Pearson‘s correlation coefficient:} $−1 \le \rho_{X,Y} \le 1$
				%
				% \includegraphics[width=0.1\linewidth]{./figures/pearson_symbol.png}
				% \includegraphics[width=0.4\linewidth]{./figures/pearson_correlation_coefficient.png}
				% \begin{betterlist}
				% 	\item $X$ = Power consumptions (for $n$ different plain texts)
				% 	\item $Y$ = Hamming weights of some intermediate value (like output of first S-box in round $1$ of AES based on a certain key byte assumption)
				% \end{betterlist}
				\item we select the key guess $i$ with the highest $\rho_{j,i}$. Repeat the previous steps for every key part in order to recover the full key
			\end{betterlist}
		\end{betterlist}
	\end{betterlist}
	\fbox{Timing Attacks}
	\begin{betterlist}
		\item \underline{Timing attack against RSA:}
		\begin{betterlist}
			\item simple timing attack over number of $1$'s in Square and Multiply. If number of $1$'s is \alert{very small} or \alert{very high}, then such a simple timing attack enables \alert{brute-force attack}
			\item similiar attack possible as for accoustic chosen ciphertext attack
		\end{betterlist}
		\item similar to SPA countermeasures
	\end{betterlist}
	\fbox{Attack over electromagnetic side channels}
	\begin{betterlist}
		\item electrical transitions induce EM field captured by inductive probe. Obtain current signal on the die by integrating the recorded signal. Changes / derivation of current $\hat=$ EM field $\rightarrow$ similar to power analysis
		% \item can be used for \enquote{reverse engineering} of chip properties. Areas with high activity = CPU, functional units. Local information (resolution depends on probe). 
		\item \underline{Advantage:} Allows localized readings. \underline{Disadvantages:} Experimentally complicated. Geometrical scanning can be tedious. Low level and noisy signals (decapsulation required)
		\item \underline{Attacks:} approaches similar to power analysis possible (differential, correlation)
		\item \underline{Countermeasures:} E.g. shielding against electro-magnetic radiation, Faraday cages, ...
	\end{betterlist}
	\fbox{Attack over acoustic side channels}
	\begin{betterlist}
		\item \alert{Acoustic attack against RSA}
		\begin{betterlist}
			\item attack on modular exponentiation step $y^{d \mod (q−1)} \mod q$ to compute $q$ (attacking $p$ would be possible as well)
      \item if $q$ is known, then $p = n/q$ is known and $\Phi(n)= (p − 1) \cdot (q − 1)$ is known and $d$ can be computed from $e$
			\item assume that before round $i$ the $i$ most significant bits of $q$, i.e., $(q_{2047} \ldots q_{2048−i})$ are already known
			\item choose ciphertext $y_i = (q_{2047} \ldots q_{2048−i} 011 \ldots 1)$
			\item \alert{Case 1:} $q_{2048−i−1} = 1$, then $q > y^i$, then $(y^i)_q=y^i$ goes into square and multiply algorithm with exponent $d \mod (q − 1)$
			\item \alert{Case 2:} $q_{2048−i−1} = 0$, then $q \le y^i$, then $(y^i)_q=(y^i - q)$ goes into square and multiply algorithm with exponent $d \mod (q − 1)$
      \item square and multiply algorithm always multiplies with $(y^i)_q$. Different acoustic behaviour depending on $(y^i)_q$, because in the two different cases $(y^i)_q$ have vastly different values
      \item after estimating $q_{2048−i−1}$ from acoustic analysis go to round $i + 1$ with known $i + 1$ most significant bits $(q_{2047} \ldots q_{2048−i−1})$
      
		\end{betterlist}
		\item \underline{Countermeasures:}
		\begin{betterlist}
			\item acoustic shielding
			\item carefully designed acoustic noise generator to hide acoustic signal?
			\item parallel software load for hiding? (multi-core processors, FPGA-implementations)
			\item \enquote{cipher text randomization} to compromise chosen cipher text approach: Instead of decrypting cipher text $y$. Choose random text $r$ (not completely random, but invertible). Decrypt $r^e \cdot y$, i.e., $r^e \cdot y$ is used for exponentiation instead of $y$. Multiply the result by $r^{−1}$. $(r^e \cdot y)^d \cdot r^{−1} \mod n \equiv r^{ed} \cdot r^{−1} \cdot y^d \mod n \equiv r \cdot r^{−1} \cdot y^d \mod n \equiv y^d \mod n$
		\end{betterlist}
	\end{betterlist}
	\fbox{Chinese Remainder Theorem (CRT)}
	\begin{betterlist}
		\item accelerated exponentiation
		\item the number of operations is not reduced, but the bit width of the operands
		\item possible computation of $x = y^d \mod n$ with $n = p \cdot q$, $p$, $q$ primes
		\begin{betterlist}
			\item \underline{Transformation into CRT domain:} $y_p \equiv y \mod p, y_q \equiv y \mod q$
			\item \underline{Exponentiation in CRT domain:}\\
			$x_p \equiv y_p^{d \mod (p-1)} \mod p, x_q \equiv y_q^{d \mod (q-1)} \mod q$
			\item \underline{Backtransformation using CRT}: $x \equiv [q \cdot c_p] \cdot x_p + [p \cdot c_q] \cdot x_q \mod n$
			\begin{betterlist}
				\item where $c_p$ is the inverse of $q$ in $\mathbb{Z}/p\mathbb{Z}$, $c_q$ is the inverse of $p$ in $\mathbb{Z}/q\mathbb{Z}$.
			\end{betterlist}
		\end{betterlist}
	\end{betterlist}
\end{minipage}
\begin{minipage}[t]{0.2\pagewidth}
	\fbox{Fault Injection Attacks}
	\begin{betterlist}
		\item \underline{Usable side channel \alert{inputs} for fault injection:}
		\begin{betterlist}
			\item Supply voltage
			\item Clock frequency
			\item Temperature
			\item Light
		\end{betterlist}
	\end{betterlist}
\end{minipage}
\end{document}
