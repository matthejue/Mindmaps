%!Tex Root = ../main.tex

% ┌─────────┐
% │ Mindmap │
% └─────────┘
\newlength{\leveldistance}
\setlength{\leveldistance}{25cm}

\newenvironment{edges}{\begin{pgfonlayer}{background}\draw [concept connection]}{;\end{pgfonlayer}}
\newcommand{\edge}[2]{(#1) edge (#2)}
\newcommand{\annotation}[2]{\path (#1) -- node[annotation, above, align=center, pos=0.03] {#2} (middle);}

\newenvironment{resettikz}{\pgfsetlayers{nodelayer,edgelayer}\tikzset{every node/.style={fill opacity=1.0, draw opacity=1.0, minimum size=0cm, inner sep=0pt}}}{}

\newenvironment{mindmap}{
	\begin{tikzpicture}[
			auto,
			huge mindmap,
			fill opacity=0.6,
			draw opacity=0.8,
			concept color = PrimaryColorDimmed,
			every annotation/.style={fill=BoxColor, draw=none, align=center, fill = BoxColor, text width = 2cm},
			grow cyclic,
			level 1/.append style = {
					concept color=SecondaryColorDimmed,
					level distance=\leveldistance,
					sibling angle=360/\the\tikznumberofchildren,
					% https://tex.stackexchange.com/questions/501240/trying-to-use-the-array-environment-inside-a-tikz-node-with-execute-at-begin-no
					execute at begin node=\definecolor{SwitchColor}{named}{SecondaryColor}\definecolor{SwitchColorDimmed}{named}{PrimaryColorDimmed},
				},
			level 2/.append style = {
					concept color=PrimaryColorDimmed,
					level distance=\leveldistance / 2,
					sibling angle=60,
					execute at begin node=\definecolor{SwitchColor}{named}{PrimaryColor}\definecolor{SwitchColorDimmed}{named}{SecondaryColorDimmed},
				},
			level 3/.append style = {
					concept color=SecondaryColorDimmed,
					level distance=\leveldistance / 3,
					execute at begin node=\definecolor{SwitchColor}{named}{SecondaryColor}\definecolor{SwitchColorDimmed}{named}{PrimaryColorDimmed},
				},
			level 4/.append style = {
					concept color=PrimaryColorDimmed,
					level distance=\leveldistance / 4,
					execute at begin node=\definecolor{SwitchColor}{named}{PrimaryColor}\definecolor{SwitchColorDimmed}{named}{SecondaryColorDimmed},
				},
			level 5/.append style = {
					concept color=SecondaryColorDimmed,
					level distance=\leveldistance / 5,
					execute at begin node=\definecolor{SwitchColor}{named}{SecondaryColor}\definecolor{SwitchColorDimmed}{named}{PrimaryColorDimmed},
				},
			level 6/.append style = {
					concept color=PrimaryColorDimmed,
					level distance=\leveldistance / 6,
					execute at begin node=\definecolor{SwitchColor}{named}{PrimaryColor}\definecolor{SwitchColorDimmed}{named}{SecondaryColorDimmed},
				},
			level 7/.append style = {
					concept color=SecondaryColorDimmed,
					level distance=\leveldistance / 7,
					execute at begin node=\definecolor{SwitchColor}{named}{SecondaryColor}\definecolor{SwitchColorDimmed}{named}{PrimaryColorDimmed},
				},
			level 8/.append style = {
					concept color=PrimaryColorDimmed,
					level distance=\leveldistance / 8,
					execute at begin node=\definecolor{SwitchColor}{named}{PrimaryColor}\definecolor{SwitchColorDimmed}{named}{SecondaryColorDimmed},
				},
			level 9/.append style = {
					concept color=SecondaryColorDimmed,
					level distance=\leveldistance / 9,
					execute at begin node=\definecolor{SwitchColor}{named}{PrimaryColor}\definecolor{SwitchColorDimmed}{named}{SecondaryColorDimmed},
				},
			level 10/.append style = {
					concept color=PrimaryColorDimmed,
					level distance=\leveldistance / 10,
					execute at begin node=\definecolor{SwitchColor}{named}{PrimaryColor}\definecolor{SwitchColorDimmed}{named}{SecondaryColorDimmed},
				},
			level 11/.append style = {
					concept color=SecondaryColorDimmed,
					level distance=\leveldistance / 11,
					execute at begin node=\definecolor{SwitchColor}{named}{PrimaryColor}\definecolor{SwitchColorDimmed}{named}{SecondaryColorDimmed},
				},
			level 12/.append style = {
					concept color=PrimaryColorDimmed,
					level distance=\leveldistance / 12,
					execute at begin node=\definecolor{SwitchColor}{named}{PrimaryColor}\definecolor{SwitchColorDimmed}{named}{SecondaryColorDimmed},
				},
			concept connection/.append style = {
					color = BoxColor,
				},
		]
		}{
	\end{tikzpicture}
}

\newenvironment{mindmapcontent}{
	\begin{scope}[
			every node/.style = {concept, circular drop shadow}, % draw=none
			every child/.style={concept},
		]
		}{
		;\end{scope}
}

% ┌───────┐
% │ Boxes │
% └───────┘
\DeclareTotalTCBox{\inlinebox}{ s m }
{standard jigsaw,opacityback=0,colframe=SwitchColor,nobeforeafter,tcbox raise base,top=0mm,bottom=0mm,
	right=0mm,left=0mm,arc=0.1cm,boxsep=0.1cm}
{\IfBooleanTF{#1}%
	{\textcolor{PrimaryColor}{\setBold >\enspace\ignorespaces}#2}%
	{#2}}

\DeclareTotalTCBox{\inlineboxtwo}{ s m }
{standard jigsaw,opacityback=0,colframe=SwitchColorDimmed,nobeforeafter,tcbox raise base,top=0mm,bottom=0mm,
	right=0mm,left=0mm,arc=0.1cm,boxsep=0.1cm}
{\IfBooleanTF{#1}%
	{\textcolor{SwitchColorDimmed}{\setBold >\enspace\ignorespaces}#2}%
	{#2}}

% ┌──────────────────┐
% │ Case distinction │
% └──────────────────┘
% \newtoggle{absolute}
% % \toggletrue{absolute}
% \togglefalse{absolute}
% \newcommand{\lpathgraph}[1]{\iftoggle{absolute}{/home/areo/Documents/Studium/Summaries/x/}{./}#1}

% ┌───────┐
% │ Fixes │
% └───────┘
% https://tex.stackexchange.com/questions/89467/why-does-pdftex-hang-on-this-file
% \newcommand{\colon}{\mathrel{\mathop:}}

% ┌───────┐
% │ Paths │
% └───────┘
\newcommand{\script}[2]{\href{openpdf:/home/areo/Documents/Studium/Semester_1_Master/Verification_of_Digital_Circuits/slides/annotated/Verification_of_Digital_Circuits_all_in_one.pdf:#1}{\inlinebox{#2}}}
\newcommand{\scripttwo}[2]{\href{openpdf:///home/areo/Nextcloud/Verification of Digital Circuits - WS2324/Slides/annotated/bonus/lecture08_sat_aig.pdf:#1}{\inlinebox{#2}}}
\newcommand{\videoseven}[2]{\href{https://youtu.be/ILBLSr7zO70?feature=shared&t=#1}{\inlineboxtwo{#2}}}
\newcommand{\videoeight}[2]{\href{https://youtu.be/DCDNlu0I8GA?feature=shared&t=#1}{\inlineboxtwo{#2}}}
\newcommand{\videonine}[2]{\href{https://youtu.be/DCDNlu0I8GA?feature=shared&t=#1}{\inlineboxtwo{#2}}}
\newcommand{\videoeleven}[2]{\href{https://youtu.be/dKszaFJQF3I?si=v1Qhw6OC-VxS7z7A&t=#1}{\inlineboxtwo{#2}}}
\newcommand{\videotwelve}[2]{\href{https://youtu.be/uVedKmgkKxU?si=Ne2tds-ok_b_k6i2&t=#1}{\inlineboxtwo{#2}}}
