%!Tex Root = ../main.tex
% ./Packete.tex
% ./Design.tex
% ./Vorbereitung.tex
% ./Aufgabe1.tex
% ./Aufgabe2.tex
% ./Aufgabe3.tex
% ./Aufgabe4.tex
% ./Appendix.tex

\begin{mindmap}
  \begin{mindmapcontent}
    \node (be) at (current page.center) {Beweisen
      \resizebox{\textwidth}{!}{
        \begin{minipage}[t]{16cm}
          \begin{itemize}
            \item Ein Satz oder Theorem ist in der Mathematik eine widerspruchsfreie logische Aussage, die mittels eines Beweises als wahr erkannt, das heißt, aus Axiomen, Definitionen und bereits bekannten Sätzen hergeleitet werden kann
            \item Ein Satz wird nach seiner Rolle, seiner Bedeutung oder seinem Kontext oft auch anders bezeichnet:
            \begin{itemize}
              \item Lemma (oder Hilfssatz) für eine Aussage, die nur im Beweis anderer Sätze im gleichen Werk verwendet wird und unabhängig davon keine Bedeutung hat,
              \item Proposition für eine ebenfalls hauptsächlich lokal bedeutsame Aussage, etwa einen Hilfssatz, der in mehr als einem Beweis verwendet wird,
              \item Satz (oder Theorem) für eine wesentliche Erkenntnis, die im Werk dargestellt wird, und
              \item Korollar (oder Folgesatz) für eine triviale Folgerung, die sich aus einem Satz oder einer Definition ohne großen Aufwand ergibt.
            \end{itemize}
          \end{itemize}
        \end{minipage}
      }
    }
    child {
      node {Induktion}
      child {
        node {Vollständige Induktion
          \resizebox{\textwidth}{!}{
            \begin{minipage}[t]{16cm}
              \begin{itemize}
                \item für \alert{Allbehauptungen}: $\forall n\in\mathbb{N}:\mathcal{E}(n)$ (Eigenschaft $\mathcal{E}$)
                \item die allgemeine Form wird als \alert{Strukturelle Induktion} bezeichnet, \alert{Vollständige Induktion} ist zum Beweis von Aussagen für alle Natürlichen Zahlen
                \begin{itemize}
                  \item Menge der \alert{natürlichen Zahlen} $\mathbb{N}$ weißt geignete Struktur auf, da auf ihnen die Nachfolgerbeziehung besteht
                \end{itemize}
                \item man führt Teilbewise für $2$ Fälle: 
                \begin{itemize}
                  \item \alert{Basisfall:} ${\mathcal{E}}(0)$, d.h. die natürliche Zahl $0$ hat die Eigenschaft $\epsilon$
                  \item \alert{Induktionsfall:} $\forall i\in\mathbb{N} \left(\mathcal{E}(i)\Rightarrow\mathcal{E}(i+1)\right)$, d.h. die Eigenschaft $\epsilon$ überträgt sich von jeder natürlichen Zahl $i$ auf ihren Nachfolger $i + 1$
                \end{itemize}
                \item bei beweisender Allbehauptung für den Induktionsfall wird Implikation: Sei $i\in\mathbb{N}$ beliebig. Zu zeigen: $\epsilon(i)\Rightarrow\epsilon(i+1)$ zerlegt in: Sei $i\in\mathbb{N}$ beliebig. Es gelte: $\epsilon(i)$. Zu zeigen: $\epsilon(i+1)$
                \item Diese Darstellung des Induktionsfalls legt seine Zerlegung in die üblichen drei Bestandteile \alert{Induktionsannahme}, \alert{Induktionsbehauptung}, \alert{Induktionsschritt} nahe:
              \includegraphics[width=\linewidth]{./figures/vollstaendige_induktion.png}
              \end{itemize}
              \begin{itemize}
                \item Induktionsfall beweist Allbehauptung mit darin enthaltener Implikationsbehauptung 
                \item Wenn für Eigenschaft $\epsilon$ Basisfall und Induktionsfall bewiesen sind, gilt, dass jede natürliche Zahl die Eigenschaft $\epsilon$ hat, denn:
                \begin{enumerate}
                  \setcounter{enumi}{-1}
                  \item Wegen des Basisfalls gilt $\epsilon(0)$.
                  \item Mit $\epsilon(0)$ gilt wegen des Induktionsfalls auch $\epsilon(1)$.
                  \item Mit $\epsilon(1)$ gilt wegen des Induktionsfalls auch $\epsilon(2)$.
                  \item Mit $\epsilon(2)$ gilt wegen des Induktionsfalls auch $\epsilon(3)$.
                  \item $\ldots$
                \end{enumerate}
              \end{itemize}
              \includegraphics[width=0.6\textwidth]{./figures/vollständige_induktion_1.png}\\
              \includegraphics[width=0.8\textwidth]{./figures/vollständige_induktion_2.png}
            \end{minipage}
          }
        }
      }
    }
    child {
      node {Indirekter Beweis}
      child {
        node {Beweis durch Widerspruch
          \resizebox{\textwidth}{!}{
            \begin{minipage}[t]{16cm}
              \begin{itemize}
                \item anstatt einen mathematischen Satz $S$ direkt zu beweisen, kann man seine Negation $\neg S$ durch logische Schlussfolgerungen zu einem Widerspruch führen.
                \item Wenn man die Widerspruchsanahme $\neg S$ zu einem Widerspruch geführt hat, weiß man, dass $\neg S$ immer falsch sein muss. Damit ist die doppelte Negation $\neg\neg S$ von $S$ wahr. Da $\neg\neg S \Leftrightarrow S$ eine Tautologie ist, ist $\neg\neg S$ genau dann wahr, wenn $S$ wahr ist. Damit muss $S$ wahr sein.
                \item \alert{Zerlegung der Implikation:}
                \begin{align*}
                  &V o r a u s s e t z u n g_{1}\wedge\cdot\cdot\cdot\wedge\;V o r a u s s e t z u n g_{n}\rightarrow B e h a u p t u n g\\
                  &\Leftrightarrow \neg\left(V o r a u s s e t z u n g_{1}\wedge\cdot\cdot\cdot\wedge\ V o r a u s s e t z u n g_{n}\right)\vee B e h a u p t u n g\\
                  &\Leftrightarrow \neg\left(\left(V o r a u s s e t z u n g_{1}\wedge\cdot\cdot\cdot\wedge\;V o r a u s s e t z u n g_{n}\right)\wedge\neg B e h a u p t u n g\right)\\
                  &\Leftrightarrow V o r a u s s e t z u n g_{1}\wedge\cdot\cdot\cdot\wedge\ V o r a u s s e t z u n g_{n}\wedge\lnot B e h a u p t u n g \rightarrow \perp
                \end{align*}
                \begin{itemize}
                  \item \alert{Anders gesagt:} Aus den Vorraussetzungen folgt logisch die Behauptung \textit{genau dann wenn} $V o r a u s s e t z u n g_{1}\wedge\cdot\cdot\cdot\wedge\ V o r a u s s e t z u n g_{n}\wedge\lnot B e h a u p t u n g$ \textit{NICHT} gilt.
                \end{itemize}
              \end{itemize}
              \begin{itemize}
                \item es gibt neben \alert{expliziten Voraussetzungen} auch \alert{implizite Voraussetzungen}, die nicht ausdrücklich genannt werden (Rechenregeln und Standard-Definitionen)
                \item In der \alert{Behauptung} steht immer eine \alert{wahre Aussage}. Hat man eine Aussage, die \alert{nicht wahr} ist, muss man sie \alert{negiert} in die Behauptung schreiben.
              \end{itemize}
              \includegraphics[width=\textwidth, center]{./figures/beweis_durch_widerspruch.png}\\
              \includegraphics[width=0.7\textwidth, center]{./figures/beweis_durch_widerspruch_beispiel.png}
            \end{minipage}
          }
        }
      }
      child {
        node {Beweis durch Kontraposition
          \resizebox{\textwidth}{!}{
            \begin{minipage}[t]{8cm}
              \begin{itemize}
                \item $\mathcal{F}\rightarrow\mathcal{G}\Leftrightarrow\neg\mathcal{F}\vee\mathcal{G}\Leftrightarrow\neg\neg\mathcal{G}\vee¬\mathcal{F}\Leftrightarrow\neg\mathcal{G}\rightarrow\neg\mathcal{F}$
                \item Konstraposition der Behauptung, die eine Implikation ist wird bewiesen. Immer mit Beweismuster für Implikation kombiniert, da Kontraposition der Behauptung auch eine Implikation ist
              \end{itemize}
              \includegraphics[width=0.6\textwidth, center]{./figures/kontraposition.png}\\
              \includegraphics[width=0.8\textwidth, center]{./figures/kontraposition_example.png}
            \end{minipage}
          }
        }
      }
    };
  \end{mindmapcontent}
  % \begin{edges}
  %   \edge{test}{re}
  % \end{edges}
  \annotation{be.south}{This mindmap is provided without guarantee of correctness and completeness!}
  \annotation{be.north}{\href{/tmp/current.pdf}{go back}}
\end{mindmap}
