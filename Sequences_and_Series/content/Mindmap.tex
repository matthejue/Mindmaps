%!Tex Root = ../main.tex
% ./Packete.tex
% ./Design.tex
% ./Vorbereitung.tex
% ./Aufgabe1.tex
% ./Aufgabe2.tex
% ./Aufgabe3.tex
% ./Aufgabe4.tex
% ./Appendix.tex

\begin{mindmap}
  \begin{mindmapcontent}
    \node (re) at (current page.center) {Sequences and Series}
    child {
      node {Sequences}
      child {
        node {Arithmetic Sequence
          \resizebox{\textwidth}{!}{
            \begin{minipage}[t]{8cm}
              \begin{itemize}
                \item $(\overset{a_1}{3}, \overset{a_2}{7}, \overset{a_3}{11}, \overset{a_4}{15}, \overset{a_5}{19}, \overset{a_6}{23}, \overset{a_7}{27})$
                \item \alert{arithmetic mean:} $\frac{a+b}{2} \overset{e.g.}{=} \frac{7+23}{2} = 15$
                \item \alert{explizit formula:} $a_n = a_1 + (n-1) \cdot d \overset{e.g.}{=} 3 + (5-1)\cdot 4 = 19 = a_5$
                \item \alert{recursive formula:} $a_{i+1} = a_n + d, \quad a_1 = s,\quad i\ge 0$
              \end{itemize}
            \end{minipage}
          }
        }
      }
      child {
        node {Geometric Sequence
          \resizebox{\textwidth}{!}{
            \begin{minipage}[t]{8cm}
              \begin{itemize}
                \item $(\overset{a_1}{3}, \overset{a_2}{6}, \overset{a_3}{12}, \overset{a_4}{24}, \overset{a_5}{48}, \overset{a_6}{96}, \overset{a_7}{192})$
                \item \alert{geometric mean:} $\sqrt{a\cdot b} \overset{e.g.}{=}  \sqrt{6\cdot 96} = 24$
                \item \alert{explizit formula:} $a_n = a_1 \cdot (m)^{n-1} \overset{e.g.}{=} 3\cdot 2^{6-1} = 3\cdot 32 = 96 = a_5$
                \item \alert{recursive formula:} $a_{i+1} = a_i \cdot m, \quad a_1 = s,\quad i\ge 0$
              \end{itemize}
            \end{minipage}
          }
        }
      }
    }
    child {
      node {Series}
      child {
        node {Geometric Series
          \resizebox{\textwidth}{!}{
            \begin{minipage}[t]{8cm}
              \begin{itemize}
                \item $\overset{a_1}{3} + \overset{a_2}{6} + \overset{a_3}{12} + \overset{a_4}{24} + \overset{a_5}{48} + \overset{a_6}{96} + \overset{a_7}{192}$
                \item \alert{partial sum:} $S_n = \frac{a_1(1-m^n)}{1-m} \overset{e.g.}{=} \frac{3\cdot (1-2^6)}{1-2} = 189 = S_6$
                \item \enquote{Geometrische Summenformel}: $\frac{(m^{n+1}-1)}{m-1}$
                \begin{itemize}
                  \item $\overset{a_0}{2^0} + \overset{a_1}{2^1} + \overset{a_2}{4} + \overset{a_3}{8} = \overset{a_0}{1} + S_3 = \overset{a_0}{1} + \frac{2\cdot(1-2^3)\cdot (-1)}{(1-2)\cdot (-1)} = \overset{a_0}{1} + \frac{2\cdot(2^3-1)}{2-1} = 1 + 2^{3+1}-2 = 2^{3+1}-1 = \frac{(2^{3+1}-1)}{2-1} = 15$
                \end{itemize}
              \end{itemize}
            \end{minipage}
          }
        }
      }
      child {
        node {Arithmetic Series
          \resizebox{\textwidth}{!}{
            \begin{minipage}[t]{8cm}
              \begin{itemize}
                \item $\overset{a_1}{3} + \overset{a_2}{7} + \overset{a_3}{11} + \overset{a_4}{15} + \overset{a_5}{19} + \overset{a_6}{23} + \overset{a_7}{27}$
                \item \alert{partial sum:} $S_n = (\frac{a_1 + a_n}{2})\cdot n \overset{e.g.}{=} \frac{3+27}{2}\cdot 7 = 105 = S_7$
                \item \enquote{Gaußsche Summenformel}: $\frac{(n+1) \cdot n}{2}$
                \begin{itemize}
                  \item $\overset{a_1}{1} + \overset{a_2}{2} + \overset{a_3}{3} + \overset{a_4}{4} + \overset{a_5}{5} + \overset{a_6}{6} + \overset{a_7}{7} = \frac{(1 + 7)\cdot 7}{2} = 28$
                \end{itemize}
              \end{itemize}
            \end{minipage}
          }
        }
      }
    };
  \end{mindmapcontent}
  % \begin{edges}
  %   \edge{test}{re}
  % \end{edges}
  \annotation{re.south}{This mindmap is provided without guarantee of correctness and completeness!};
\end{mindmap}
