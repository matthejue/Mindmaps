%!Tex Root = ../main.tex
% ./Packete.tex
% ./Design.tex
% ./Vorbereitung.tex
% ./Aufgabe1.tex
% ./Aufgabe2.tex
% ./Aufgabe3.tex
% ./Aufgabe4.tex
% ./Appendix.tex

\newlength{\leveldistance}
\setlength{\leveldistance}{17cm}

\newcounter{algorithm}
\setcounter{algorithm}{0}
\newtcbtheorem[use counter=algorithm]{algorithm}{\color{SecondaryColor}Algorithm}{pseudo/ruled}{alg}

\newenvironment{mindmap}{
  \begin{tikzpicture}[
      auto,
      huge mindmap,
      fill opacity=0.6,
      draw opacity=0.8,
      concept color = PrimaryColorDimmed,
      every annotation/.style={fill=BoxColor, draw=none, align=center, fill = BoxColor, text width = 2cm},
      grow cyclic,
      level 1/.append style = {
        concept color=SecondaryColorDimmed,
        level distance=\leveldistance,
        sibling angle=360/\the\tikznumberofchildren,
        % https://tex.stackexchange.com/questions/501240/trying-to-use-the-array-environment-inside-a-tikz-node-with-execute-at-begin-no
        execute at begin node=\definecolor{SwitchColor}{named}{SecondaryColor},
      },
      level 2/.append style = {
        concept color=PrimaryColorDimmed,
        level distance=\leveldistance / 2,
        % sibling angle=60,
        % sibling angle=360/\the\tikznumberofchildren,
        execute at begin node=\definecolor{SwitchColor}{named}{PrimaryColor},
      },
      level 3/.append style = {
        concept color=SecondaryColorDimmed,
        level distance=\leveldistance / 3,
        execute at begin node=\definecolor{SwitchColor}{named}{SecondaryColor},
      },
      level 4/.append style = {
        concept color=PrimaryColorDimmed,
        level distance=\leveldistance / 4,
        execute at begin node=\definecolor{SwitchColor}{named}{PrimaryColor},
      },
      level 5/.append style = {
        concept color=SecondaryColorDimmed,
        level distance=\leveldistance / 5,
        execute at begin node=\definecolor{SwitchColor}{named}{SecondaryColor},
      },
      level 6/.append style = {
        concept color=PrimaryColorDimmed,
        level distance=\leveldistance / 6,
        execute at begin node=\definecolor{SwitchColor}{named}{PrimaryColor},
      },
      level 7/.append style = {
        concept color=SecondaryColorDimmed,
        level distance=\leveldistance / 7,
        execute at begin node=\definecolor{SwitchColor}{named}{SecondaryColor},
      },
      level 8/.append style = {
        concept color=PrimaryColorDimmed,
        level distance=\leveldistance / 8,
        execute at begin node=\definecolor{SwitchColor}{named}{PrimaryColor},
      },
      concept connection/.append style = {
        color = BoxColor,
      },
  ]
  % damit Annotationen nicht auch eine Drop Shadow erhalten
}{
\end{tikzpicture}
}

\newenvironment{mindmapcontent}{
  \begin{scope}[
      every node/.style = {concept, circular drop shadow}, % draw=none
      every child/.style={concept},
    ]
}{
  \end{scope}
}

\newenvironment{edges}{\begin{pgfonlayer}{background}\draw [concept connection]}{;\end{pgfonlayer}}
\newcommand{\edge}[2]{(#1) edge (#2)}
\newcommand{\annotation}[2]{\path (#1) -- node[annotation, above, align=center, pos=0.03] {#2} (sa);}
