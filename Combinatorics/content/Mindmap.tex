%!Tex Root = ../main.tex
% ./Packete.tex
% ./Design.tex
% ./Vorbereitung.tex
% ./Aufgabe1.tex
% ./Aufgabe2.tex
% ./Aufgabe3.tex
% ./Aufgabe4.tex
% ./Appendix.tex

\begin{mindmap}
  \begin{mindmapcontent}
    \node (co) at (current page.center) {Kombinatorik (combinatorics)
      \resizebox{\textwidth}{!}{
        \begin{minipage}[t]{25cm}
          \begin{itemize}
            \item \alert{Anordnung:} (auch Permutation) alle Elemente der Grundmenge werden betrachtet
            \item \alert{Auswahl:} (auch Variation oder Kombination) eine Stichprobe der Grundmenge wird betrachtet
            \item es wird immer die Anzahl möglicher Relationen zwischen der Menge mit $k$ Elementen und der Menge mit $n$ Elementen mit den entsprechenden Beschränkungen für Urnen-, Fächermodell und (ohne) Wiederholung berechnet
            \item \alert{Auswahl (Variationen und Kombinationen):}
            \begin{itemize}
              \item \alert{Urnenmodell:} 
                \begin{itemize}
                  \item jede Relation muss \alert{bijektiv} sein
                  \item \alert{ohne Wiederholung:} 
                    \begin{itemize}
                      \item jede Relation muss \alert{rechtseindeutig} sein (ein Ball darf \alert{nicht} an mehreren Stellen des Podiums stehen)
                      \item \alert{\enquote{ohne Wiederholung}} entspricht \alert{\enquote{ohne Zurücklegen}} 
                      \item \alert{\enquote{mit Wiederholung}} entspricht \alert{\enquote{mit Zurücklegen}} 
                    \end{itemize}
                \end{itemize}
              \item \alert{Fächermodell:} 
                \begin{itemize}
                  \item jede Relation muss eine \alert{Funktion} sein
                  \item \alert{ohne Wiederholung:} 
                    \begin{itemize}
                      \item jede Relation muss \alert{injektiv} / linkseindeutig sein (mehrere unterschiedliche Bälle dürfen \alert{nicht} ins gleiche Fach)
                    \end{itemize}
                \end{itemize}
            \end{itemize}
            \item \alert{Anordnung (Permutationen):}
            \begin{itemize}
              \item \alert{Urnenmodell:}
              \begin{itemize}
                \item jede Relation muss eine bijektive Funktion sein
                \item \alert{mit Wiederholung:}
                \begin{itemize}
                  \item \underline{Sonderregelung zur Vereinfachung:} \alert{ohne Zurücklegen}. Man könnte es alternativ mit Zurücklegen machen und beschränken, dass man jedes unterscheidbare Objekt nur $k_i$ mal zurücklegen darf. Dies ist allerdings schwierig als Relationseigenschafft zu formulieren. Daher gibt man anstelle des $k_i$ maligen Zurücklegens die Grundmenge bereits mit $k_i$ mal identisch enthaltenten Objekten an und tauscht diese nur bijektiv an neue Positionen.
                  \item dieses Problem fällt etwas aus dem üblichen Schema raus, da man normalerweise von einer Grundmenge aus $n$ untescheidbaren Objekten ausgeht, aber dann landet man einfach nur bei Variationen mit Wiederholung mit $n=k$ und das stellt keinen besonderen Spezialfall da. Daher wird hier ein spezielles neues Problem betrachtet, bei dem man Gruppen von identischen Objekten hat. Dieses Problem lässt sich sich eigentlich für beliebige Auswahlen mit Wiederholung formulieren, sodass Auswahlen möglich sind bei denen die verschiendene Gruppen von identischen Objekten sich nicht genau zu $n$ aufsummieren, aber man beschränkt sich hier auf den einfachsten Fall von \enquote{Permutationen mit Wiederholung} mit $\sum_{i=1}^{m} k_i = n$, da nur dieser ein relevantes häufiges Problem darstellt. Das $n$ ist jetzt nicht mehr die Anzahl Elemente einer Grundmenge von unterscheidbaren Objekten wie sonst üblich, sondern eine Grundmenge von verschiedenen Gruppen identischer Objekte. Dieses Problem bei dem $n$ die Anzahl unterscheidbarer Objekte ist und man diese unterscheidbaren Objekte nur $k_i$ mal zurücklegen kann stellt nämlich kein relevantes, häufiges Problem dar.
                \end{itemize}
              \end{itemize}
            \end{itemize}
          \end{itemize}
        \end{minipage}
      }
    }
    child {
      node {Variationen (partial permutations)
        \resizebox{\textwidth}{!}{
          \begin{minipage}[t]{12cm}
            \begin{itemize}
            \item \underline{Urnenmodell:} \alert{Auswahl} von Bällen aus der Urne für ein Podium bei der die \alert{Reihenfolge} ($\approx$ Tupel) der Podiumsplätze \alert{berücksichtigt} wird
            % \begin{itemize}
            %   \item auch geordnete Stichprobe 
            % \end{itemize}
                \begin{resettikz}
                  \ctikzfig{example_urn_problem_order_important}
                \end{resettikz}
              \item \underline{Fächermodell:} Bälle sind \alert{unterscheidbar}
                \begin{resettikz}
                  \ctikzfig{example_box_problem_destinguishable}
                \end{resettikz}
            \end{itemize}
          \end{minipage}
        }
      }
      child {
        node {Variationen mit Wiederholung
          \resizebox{\textwidth}{!}{
            \begin{minipage}[t]{12cm}
              \begin{itemize}
                \item \alert{Anzahl:} $n^k$, $k$ beliebig ($k>n$ ist erlaubt)
              \end{itemize}
            \end{minipage}
          }
        }
        child {
          node {Urnenmodell
            \resizebox{\textwidth}{!}{
              \begin{minipage}[t]{8cm}
                \begin{itemize}
                  \item \alert{Menge möglicher Relationen:}\\ $U = \{\{(1, 1), (2, 2)\}, \{(2, 1), (3, 2)\}, \{(1, 1), (3, 2)\},$\\
                            $\{(2, 1), (1, 2)\}, \{(3, 1), (2, 2)\}, \{(3, 1), (1, 2)\},$\\
                    $\{(1, 1), (1, 2)\}, \{(2, 1), (2, 2)\}, \{(3, 1), (3, 2)\}\}$
                  \item \alert{Variationen:} $\{(1, 2), (2, 3), (1, 3), (2, 1), (3, 2), (3, 1),$\\
                    $(1, 1), (2, 2), (3, 3)\}$
                  \item $|U| = 3^2 = 9$
                \end{itemize}
              \end{minipage}
            }
          }
        }
        child {
          node {Fächermodell
            \resizebox{\textwidth}{!}{
              \begin{minipage}[t]{8cm}
                \begin{itemize}
                  \item \alert{Menge möglicher Relationen:}\\ $F = \{\{(1, 1), (2, 2)\}, \{(1, 2), (2, 3)\}, \{(1, 1), (2, 3)\},$\\
                    $\{(1, 2), (2, 1)\}, \{(1, 3), (2, 2)\}, \{(1, 3), (2, 1)\},$\\
                    $\{(1, 1), (2, 1)\}, \{(1, 2), (2, 2)\}, \{(1, 3), (2, 3)\}\}$
                  \item \alert{Variationen:} $\{(1, 2), (2, 3), (1, 3), (2, 1), (3, 2), (3, 1),$\\
                    $(1, 1), (2, 2), (3, 3)\}$
                  \item $|F| = 3^2 = 9$
                  \item in Fächern selbst spielt \alert{Reihenfolge} der Kugeln darin \alert{keine Rolle}, nur welche Kugeln darin sind spielt eine Rolle (jedes Fach verhält sich also wie eine \alert{gewöhnliche Menge}, $\{(1, 3), (2, 3)\}$ und $\{(2, 3), (1, 3)\}$ werden \alert{nicht unterschieden})
                  \item es spielt hingegeben eine Rolle, welche Kugel welchem Fach zugeordnet ist ($\{(1, 3), (2, 2)\}$ und $\{(1, 2), (2, 3)\}$ werden \alert{unterschieden})
                \end{itemize}
              \end{minipage}
            }
          }
        }
        % child {
        %   node {Eindeutige Permutationen von Wörtern
        %     \resizebox{\textwidth}{!}{
        %       \begin{minipage}[t]{12cm}
        %         \begin{itemize}
        %           \item \alert{Anzahl:} $\dbinom{n}{k_1, k_2, \ldots, k_m} = \dfrac{n!}{k_1! \cdot k_2! \cdot \ldots \cdot k_m} $
        %           \begin{itemize}
        %             \item man will immer die Anzahl \alert{Permutationen} eines Wortes der Länge $n$ mit $m$ verschiedenen Symbolen, die jeweils $k_i$ mal im Wort vorkommen herausbekommen. Die nichtnegative Indices $k_1$ bis $k_m$ ergeben in Summe $n$: $\displaystyle \sum_{i=1}^{m} k_i = n$, da man immer die Anzahl Permutationen herausbekommen will, also die Länge $k$ der Permutationen gleich ist, wie die Länge $n$ des ursprünglichen Wortes: $k=n$.  % TODO: korrekt?
        %             % TODO: wieso Multiplikation im Nenner?
        %             % \item Ein $k_i$ gibt an, wie oft ein bestimmtes Symbol eines Wortes im Wort vorkommt 
        %             \item \alert{Herleitung:} 
        %             \begin{itemize}
        %               % \item im Gegensatz zum Binomialkoeffizienten ist das $(n-k)!$ nicht notwenig, da man immer die Anzahl Permutationen in diesem Spezialfall herausbekommen will, bei denen $k=n$ ist und somit keine Faktoren von $n!$ rausgekürzt werden müssen
        %               \item mit dem Produkt der Fakultäten $k_1!$ bis $k_m$! im Nenner sorgt man dafür, dass die verschiedenen Permutationen eines $k_i$ mal im Wort vorkommenden Symbols, die von $n!$ mitgezählt werden als eine Kombination gezählt werden
        %               \item da die verschiedenen Permutationen eines Symbols mit den Permutationen eines anderen Symbols weitere Permutationen bildet, wird multipliziert
        %             \end{itemize}
        %           \item hat man nur zwei verschiedene Symbole in einem Wort, kann man einfach den üblichen Binomialkoeffizienten $\dbinom{n}{k}$ verwenden, wobei $k!$ für das eine Symbol steht und $(n-k)!$ für das andere Symbol, dass $n-k$ mal im Wort vorkommt, welches die Länge $n$ hat: $\dbinom{n}{k_1, k_2} = \dbinom{n}{k_1, n-k_1} = \dfrac{n!}{k_1!(n-k_1)!} = \dbinom{n}{k_1}$
        %           \end{itemize}
        %         \end{itemize}
        %       \end{minipage}
        %     }
        %   }
        % }
      }
      child {
        node {Variationen ohne Wiederholung
          \resizebox{\textwidth}{!}{
            \begin{minipage}[t]{12cm}
              \begin{itemize}
                \item \alert{Anzahl:} $\dfrac{n!}{(n-k)!}$, $k \le n$
                \begin{itemize}
                  \item \alert{Herleitung:} für die erste Stelle gibt es $n$ Möglichkeiten, für die zweite $n-1$ Möglichkeiten, also $n!$, wenn man bis 1 runtergeht. Allerdings gibt es nur $k\le n$ Stellen, daher muss man 1 bis $n-k$, also $(n-k)!$ in den Nenner schreiben, damit sich $1$ bis $n-k$ im Zähler damit rauskürzen
                \end{itemize}
              \end{itemize}
            \end{minipage}
          }
        }
        child {
          node {Urnenmodell
            \resizebox{\textwidth}{!}{
              \begin{minipage}[t]{8cm}
                \begin{itemize}
                  \item \alert{Menge möglicher Relationen:}\\ $U = \{\{(1, 1), (2, 2)\}, \{(2, 1), (3, 2)\}, \{(1, 1), (3, 2)\},$\\
                    $\{(2, 1), (1, 2)\}, \{(3, 1), (2, 2)\}, \{(3, 1), (1, 2)\}\}$
                  \item \alert{Variationen:} $\{(1, 2), (2, 3), (1, 3), (2, 1), (3, 2), (3, 1)\}$
                  \item $|U| = \dfrac{3!}{(3-2)!} = \dfrac{6}{1} = 6$
                \end{itemize}
              \end{minipage}
            }
          }
        }
        child {
          node {Fächermodell
            \resizebox{\textwidth}{!}{
              \begin{minipage}[t]{8cm}
                \begin{itemize}
                  \item \alert{Menge möglicher Relationen:}\\ $F = \{\{(1, 1), (2, 2)\}, \{(1, 2), (2, 3)\}, \{(1, 1), (2, 3)\},$\\
                    $\{(1, 2), (2, 1)\}, \{(1, 3), (2, 2)\}, \{(1, 3), (2, 1)\}\}$
                  \item \alert{Variationen:} $\{(1, 2), (2, 3), (1, 3), (2, 1), (3, 2), (3, 1)\}$
                  \item $|F| = \dfrac{3!}{(3-2)!} = \dfrac{6}{1} = 6$
                \end{itemize}
              \end{minipage}
            }
          }
        }
      }
    }
    child {
      node {Permutationen (permutations)
        \resizebox{\textwidth}{!}{
          \begin{minipage}[t]{12cm}
            \begin{itemize}
              \item \underline{Urnenmodell:} \alert{Anordnung} aller Bälle aus der Urne auf einem Podium bei dem die \alert{Reihenfolge} ($\approx$ Tupel) \alert{aller} Podiumsplätze  \alert{berücksichtigt} wird (ohne Wiederholung) oder die \alert{Reihenfolge} mancher \alert{Gruppen von Podiumsplätzen nicht berücksichtigt} wird (mit Wiederholung)
                \begin{itemize}
                  \item der Urne werden alle Bälle entnommen
                \end{itemize}
                \begin{resettikz}
                  \ctikzfig{example_urn_problem_order_important_permutations}
                \end{resettikz}
              \item \underline{Fächermodell:} Neu-\alert{Anordnung} der Bälle, die \alert{alle unterscheidbar} sind (ohne Wiederholung) oder bei denen \alert{Gruppen von Bällen nicht unterscheidbar} sind (mit Wiederholung)
                \begin{resettikz}
                  \ctikzfig{example_box_problem_destinguishable_permutations}
                \end{resettikz}
            \end{itemize}
          \end{minipage}
        }
      }
      child {
        node {Permutationen mit Wiederholung
          \resizebox{\textwidth}{!}{
            \begin{minipage}[t]{14cm}
              \begin{itemize}
                \item \alert{Anzahl:} $\dbinom{n}{k_1, k_2, \ldots, k_m} = \dfrac{n!}{k_1! \cdot k_2! \cdot \ldots \cdot k_m!}$, $\displaystyle \sum_{i=1}^{m} k_i = n$
                \begin{itemize}
                  \item \alert{Multinomialkoeffizient} für die Anzahl Tupel der Länge $n$ mit $m$ verschiedenen Gruppen von identischen Objekten, die jeweils $k_i>0$ mal im Tupel vorkommen
                  \item \alert{Herleitung:}
                  \begin{itemize}
                    \item mit dem Produkt der Fakultäten $k_1!$ bis $k_m$! im Nenner sorgt man dafür, dass die verschiedenen Permutationen eines $k_i$ mal im Tupel vorkommenden Objektes, die von $n!$ mitgezählt werden als eine Kombination gezählt werden
                    \item da die verschiedenen Permutationen eines $k_i$ mal im Tupel vorkommenden Objektes untereinander auch nochmal verschieden kombiniert werden k˝nnen, werden nach der Produktregel der Kombinatorik im Nenner die $k_i$ multipliziert
.                    \item hat man nur zwei verschiedene Objekte ($m=2$) in einem Tupel, kann man einfach den üblichen Binomialkoeffizienten $\dbinom{n}{k}$ verwenden, wobei $k!$ für das eine Objekt steht und $(n-k)!$ für das andere Objekt, dass $n-k$ mal im Tupel vorkommt, welches die Länge $n$ hat: $\dbinom{n}{k_1, k_2} = \dbinom{n}{k_1, n-k_1} = \dfrac{n!}{k_1!(n-k_1)!} = \dbinom{n}{k_1}$
                  \end{itemize}
                \end{itemize}
        %         \begin{itemize}
        %           \begin{itemize}
        %             % TODO: wieso Multiplikation im Nenner?
        %             \begin{itemize}
                    % \item im Gegensatz zum Binomialkoeffizienten ist das $(n-k)!$ nicht notwenig, da man immer die Anzahl Permutationen in diesem Spezialfall herausbekommen will, bei denen $k=n$ ist und somit keine Faktoren von $n!$ rausgekürzt werden müssen
        %             \end{itemize}
        %           \end{itemize}
        %         \end{itemize}
              \end{itemize}
            \end{minipage}
          }
        }
        child {
          node {Urnenmodell
            \resizebox{\textwidth}{!}{
              \begin{minipage}[t]{8cm}
                \begin{itemize}
                  \item \alert{Menge möglicher Relationen:}\\ $U = \{\{(1, 1), (2, 2), (3, 2)\}, \{(1, 2), (2, 1), (3, 2)\},$\\
                    $\{(1, 2), (2, 2), (3, 1)\}\}$
                  \item \alert{Permutationen:} $\{(1, 2, 2), (2, 1, 2), (2, 2, 1)\}$
                  \item $|U| = \dfrac{3!}{1!2!} = \dfrac{6}{2} = 3$
                \end{itemize}
              \end{minipage}
            }
          }
        }
        child {
          node {Fächermodell
            \resizebox{\textwidth}{!}{
              \begin{minipage}[t]{8cm}
                \begin{itemize}
                  \item \alert{Menge möglicher Relationen:}\\ $F = \{\{(1, 1), (2, 2), (2, 3)\}, \{(2, 1), (1, 2), (2, 3)\},$\\
                    $\{(2, 1), (2, 2), (1, 3)\}\}$
                  \item \alert{Permutationen:} $\{(1, 2, 2), (2, 1, 2), (2, 2, 1)\}$
                  \item $|F| = \dfrac{3!}{1!2!} = \dfrac{6}{2} = 3$
                \end{itemize}
              \end{minipage}
            }
          }
          child {
            node {Mit Produktregel der Kombinatorik
              \resizebox{\textwidth}{!}{
                \begin{minipage}[t]{12cm}
                  \begin{resettikz}
                    \ctikzfig{./figures/permutationtree}
                  \end{resettikz}
                  \begin{itemize}
                    \item \alert{Permutationen:} 1122, 1212, 1221, 2112, 2211, 2121
                    \item wenn $k_1! \cdot k_2!$ nicht im Nenner stehen würde, dann würde man davon ausgehen, dass die zwei $1$er und zwei $2$er unterscheidbar sind (hier durch jeweils $1$, $2$ und $3$, $4$ dargestellt) und auch nocch gegenseitig weiter kombiniert werden können:
                    \begin{itemize}
                      \item 1233, 1324, 1342, 3124, 3412, 3141
                      \item 1243, 1423, 1432, 4123, 4312, 4131
                      \item 2134, 2314, 2341, 3214, 3421, 3241
                      \item 2143, 2413, 2431, 4213, 4321, 4231
                    \end{itemize}
                  \end{itemize}
                \end{minipage}
              }
            }
          }
        }
        % child {
        %   node {Eindeutige Permutationen von Wörtern
        %     \resizebox{\textwidth}{!}{
        %       \begin{minipage}[t]{12cm}
        %         \begin{itemize}
        %           \item \alert{Anzahl:} $\dbinom{n}{k_1, k_2, \ldots, k_m} = \dfrac{n!}{k_1! \cdot k_2! \cdot \ldots \cdot k_m} $
        %           \begin{itemize}
        %             \item man will immer die Anzahl \alert{Permutationen} eines Wortes der Länge $n$ mit $m$ verschiedenen Symbolen, die jeweils $k_i$ mal im Wort vorkommen herausbekommen. Die nichtnegative Indices $k_1$ bis $k_m$ ergeben in Summe $n$: $\displaystyle \sum_{i=1}^{m} k_i = n$, da man immer die Anzahl Permutationen herausbekommen will, also die Länge $k$ der Permutationen gleich ist, wie die Länge $n$ des ursprünglichen Wortes: $k=n$.  % TODO: korrekt?
        %             % TODO: wieso Multiplikation im Nenner?
        %             % \item Ein $k_i$ gibt an, wie oft ein bestimmtes Symbol eines Wortes im Wort vorkommt 
        %             \item \alert{Herleitung:} 
        %             \begin{itemize}
        %               % \item im Gegensatz zum Binomialkoeffizienten ist das $(n-k)!$ nicht notwenig, da man immer die Anzahl Permutationen in diesem Spezialfall herausbekommen will, bei denen $k=n$ ist und somit keine Faktoren von $n!$ rausgekürzt werden müssen
        %               \item mit dem Produkt der Fakultäten $k_1!$ bis $k_m$! im Nenner sorgt man dafür, dass die verschiedenen Permutationen eines $k_i$ mal im Wort vorkommenden Symbols, die von $n!$ mitgezählt werden als eine Kombination gezählt werden
        %               \item da die verschiedenen Permutationen eines Symbols mit den Permutationen eines anderen Symbols weitere Permutationen bildet, wird multipliziert
        %             \end{itemize}
        %           \item hat man nur zwei verschiedene Symbole in einem Wort, kann man einfach den üblichen Binomialkoeffizienten $\dbinom{n}{k}$ verwenden, wobei $k!$ für das eine Symbol steht und $(n-k)!$ für das andere Symbol, dass $n-k$ mal im Wort vorkommt, welches die Länge $n$ hat: $\dbinom{n}{k_1, k_2} = \dbinom{n}{k_1, n-k_1} = \dfrac{n!}{k_1!(n-k_1)!} = \dbinom{n}{k_1}$
        %           \end{itemize}
        %         \end{itemize}
        %       \end{minipage}
        %     }
        %   }
        % }
      }
      child {
        node {Permutationen ohne Wiederholung
          \resizebox{\textwidth}{!}{
            \begin{minipage}[t]{12cm}
              \begin{itemize}
                \item \alert{Anzahl:} $\dfrac{n!}{(n-n)!} = \dfrac{n!}{1} = n!$
              \end{itemize}
            \end{minipage}
          }
        }
        child {
          node {Urnenmodell
            \resizebox{\textwidth}{!}{
              \begin{minipage}[t]{8cm}
                \begin{itemize}
                  \item \alert{Menge möglicher Relationen:}\\ $U = \{\{(1, 1), (2, 2)\}, \{(1, 2), (2, 1)\}\}$
                  \item \alert{Permutationen:} $\{(1, 2), (2, 1)\}$
                  \item $|U| = 2! = 2$
                \end{itemize}
              \end{minipage}
            }
          }
        }
        child {
          node {Fächermodell
            \resizebox{\textwidth}{!}{
              \begin{minipage}[t]{8cm}
                \begin{itemize}
                  \item \alert{Menge möglicher Relationen:}\\ $F = \{\{(1, 1), (2, 2)\}, \{(2, 1), (1, 2)\}\}$
                  \item \alert{Permutationen:} $\{(1, 2), (2, 1)\}$
                  \item $|F| = 2! = 2$
                \end{itemize}
              \end{minipage}
            }
          }
        }
      }
    }
    child {
      node {Kombinationen (combinations)
        \resizebox{\textwidth}{!}{
          \begin{minipage}[t]{12cm}
            \begin{itemize}
            \item \underline{Urnenmodell:} \alert{Auswahl} für ein Podium bei der die \alert{Reihenfolge} ($\approx$ Menge) der Podiumsplätze \alert{nicht berücksichtigt} wird
              % \begin{itemize}
              %   \item auch ungeordnete Stichprobe 
              % \end{itemize}
                \begin{resettikz}
                  \ctikzfig{example_urn_problem_order_unimportant}
                \end{resettikz}
              \item \underline{Fächermodell:} Bälle sind \alert{nicht unterscheidbar}
                \begin{resettikz}
                  \ctikzfig{example_box_problem_indestinguishable}
                \end{resettikz}
            \end{itemize}
          \end{minipage}
        }
      }
      child {
        node {Kombinationen ohne Wiederholung
          \resizebox{\textwidth}{!}{
            \begin{minipage}[t]{12cm}
              \begin{itemize}
                \item \alert{Anzahl:} $\dbinom{n}{k} = \dfrac{n!}{k!\cdot(n-k)!} = \dfrac{n\cdot (n-1) \cdot \ldots \cdot (n-k+1)}{k!}$, $k\le n$
                \begin{itemize}
                  \item \alert{Binomialkoeffizient} für Anzahl \alert{gewöhnlicher Untermengen} mit $k$ Elementen
                  \item \alert{Herleitung:} $(n-k)!$ ist wie bei der Herleitung der Anzahl von Variationen ohne Wiederholung nur dazu da von $n!$ die $n-k$ Faktoren $(n-k)\cdot \ldots \cdot 1$ herauszukürzen, sodass da nur noch die Faktoren $n\cdot (n-1) \cdot \ldots \cdot (n-k+1)$ für die $k$ Stellen übrig bleiben. Das $k!$ im Nenner ist die Anzahl der verschiedenen Permutationen einer Variation / Permutation mit $k$ Stellen und hat die Aufgabe dafür zu sorgen, dass alle $k!$ verschiedenen Permutationen einer Variation / Permutation, die in $n!$ gezählt werden als nur eine Kombination zählen.
                  \item \alert{Alternative Herleitung}: Der Binomialkoeffizient berechnet ebenfalls die Anzahl Permutationen mit Wiederholung mit nur $2$ Gruppen von identischen Objekten. Die eine Gruppe entspricht der Auswahl der Länge $k$, die man bei Kombinationen mit Wiederholung trifft und die anderen Gruppe ergibt sich automatisch aus allen anderen $n-k$ identischen Objekten, die es braucht um auf die Länge $n$ aufzufüllen. Die Anzahl Permutationen mit Wiederholung mit nur $2$ Gruppen von identischen Objekten ist daher immer gleich der Anzahl Untermengen der Länge $k$, was man auch darin sieht, dass deren Formel zur Berechnung der Anzahl gleichgesetzt werden können, wenn es beim Multinomialkoeffizient nur $2$ Gruppen von identischen Objekten gibt: $\dbinom{n}{k_1, k_2} = \dbinom{n}{k_1}$, da $\displaystyle\sum_{i=1}^{2}k_i = n \Leftrightarrow k_2 = n-k_1$. Für alles genauere siehe Permutationen mit Wiederholung.
                  \item $\dbinom{n}{n} = \dfrac{n!}{n!\cdot(n-n)!} = \dfrac{n!}{n!\cdot(n-n)!} = \dfrac{n!}{n!\cdot 1} = 1$, $k=n$
                \end{itemize}
              \end{itemize}
            \end{minipage}
          }
        }
        child {
          node {Fächermodell
            \resizebox{\textwidth}{!}{
              \begin{minipage}[t]{8cm}
                \begin{itemize}
                  \item \alert{Menge möglicher Relationen:}\\ $F = \{\{(1, x), (2, x)\}, \{(2, x), (3, x)\}, \{(1, x), (3, x)\}\}$
                  \item \alert{Kombinationen:} $\{(1, 2), (2, 3), (1, 3)\}$
                  \begin{itemize}
                    \item man kann anstelle von Tupelklammern $()$ auch Mengenklammern $\{\}$ verwenden, da die Reihenfolge egal ist
                  \end{itemize}
                  \item $|F| = \dfrac{3!}{2!(3-2)!} = \dfrac{6}{2} = 3$
                \end{itemize}
              \end{minipage}
            }
          }
        }
        child {
          node {Urnenmodell
            \resizebox{\textwidth}{!}{
              \begin{minipage}[t]{8cm}
                \begin{itemize}
                  \item \alert{Menge möglicher Relationen:}\\ $U = \{\{(x, 1), (x, 2)\}, \{(x, 2), (x, 3)\}, \{(x, 1), (x, 3)\}\}$
                  \item \alert{Kombinationen:} $\{(1, 2), (2, 3), (1, 3)\}$
                  \begin{itemize}
                    \item man kann anstelle von Tupelklammern $()$ auch Mengenklammern $\{\}$ verwenden, da die Reihenfolge egal ist
                  \end{itemize}
                  \item $|U| = \dfrac{3!}{2!(3-2)!} = \dfrac{6}{2} = 3$
                \end{itemize}
              \end{minipage}
            }
          }
        }
        child {
          node {Vollständige Graphen
            \resizebox{\textwidth}{!}{
              \begin{minipage}[t]{8cm}
                \begin{itemize}
                  \item Anzahl Kanten in einem vollständigen Graphen bzw. maixmal Anzahl Kanten in einem einfachen Graphen
                    \begin{itemize}
                      \item erster betrachteter Knoten hat $n-1$ Kanten zu allen anderen Knoten
                      \item zweiter betrachteter Knoten hat $n-1$ Kanten zu allen anderen Knoten, aber es gibt bereits eine Kante zum ersten betrachteten Knoten, also $n-2$
                      \item also $\displaystyle n - 1 + \ldots + 1 = \sum_{i=1}^{n-1} i = \frac{n(n-1)}{2} = \binom{n}{2}$
                    \end{itemize}
                \end{itemize}
                \begin{resettikz}
                  \ctikzfig{./figures/completegraph}
                \end{resettikz}
              \end{minipage}
            }
          }
        }
        child {
          node {Binärbäume
            \resizebox{\textwidth}{!}{
              \begin{minipage}[t]{10cm}
                \begin{itemize}
                  \item Anzahl Pfade, bei denen ein Ereignis genau $k$ mal auftritt
                  \begin{itemize}
                    \item alle Untermengen von $\{1, \ldots, n\}$ der Länge $k$, also $\dbinom{n}{k}$
                    \item im Beispiel unten ist die Anzahl Pfade folglich: $\dbinom{3}{2} = \dfrac{3\cdot 2}{2} = 3$
                  \end{itemize}
                \end{itemize}
                \begin{resettikz}
                  \ctikzfig{./figures/tree}
                \end{resettikz}
              \end{minipage}
            }
          }
        }
        child {
          node {Bezug zu Summenformel
            \resizebox{\textwidth}{!}{
              \begin{minipage}[t]{10cm}
                \begin{itemize}
                  \item $\dbinom{n}{2} = \dfrac{n!}{2!(n-2)!} = \dfrac{n(n-1)(n-2)!}{2!(n-2)!} = \dfrac{n(n-1)}{2}$
                  \item $\dbinom{n+1}{2} = \dfrac{(n+1)!}{2!(n+1-2)!} = \dfrac{(n+1)n(n-1)!}{2!(n-1)!} = \dfrac{n(n+1)}{2}$
                \end{itemize}
              \end{minipage}
            }
          }
        }
      }
      child {
        node {Kombinationen mit Wiederholung
          \resizebox{\textwidth}{!}{
            \begin{minipage}[t]{12cm}
              \begin{itemize}
                \item \alert{Anzahl:} $\dbinom{n+k-1}{k} = \dfrac{(n+k-1)!}{k!\cdot(n+k-1-k)!} = \dfrac{(n+k-1)!}{k!\cdot(n-1)!} = \dfrac{n\cdot (n+1)\cdot \ldots \cdot (n+k-1)}{k!}$, $k$ beliebig ($k>n$ ist erlaubt) 
                \begin{itemize}
                  \item spezieller \alert{Binomialkoeffizient} $\left(\!\!\dbinom{n}{k}\!\!\right) = \dbinom{n + k - 1}{k} = \dbinom{n+k-1}{n-1}$ für Anzahl \alert{Multimengen} mit $k$ Elementen
                \end{itemize}
              \end{itemize}
            \end{minipage}
          }
        }
        child {
          node {Fächermodell
            \resizebox{\textwidth}{!}{
              \begin{minipage}[t]{8cm}
                \begin{itemize}
                   \item \alert{Menge möglicher Relationen:}\\ $F = \{\{(x, 1), (x, 2)\}, \{(x, 2), (x, 3)\}, \{(x, 1), (x, 3)\},$\\
                     $\{(x, 1)\}, \{(x, 2)\}, \{(x, 3)\}\}$
                   \item \alert{Kombinationen:} $\{(1, 2), (2, 3), (1, 3), (1, 1), (2, 2), (3, 3)\}$
                    \begin{itemize}
                      \item man kann anstelle von Tupelklammern $()$ auch Mengenklammern $\{\}$ verwenden, da die Reihenfolge egal ist. Man muss eine Menge mit nur einem Element $\{x\}$ dann nur korrekt als $(x, x)$ interpretieren
                    \end{itemize}
                   \item $|F| = \dfrac{(3+2-1)!}{2!(3-1)!} = \dfrac{4\cdot 3}{2} = 6$
                   \item nur die \alert{Anzahl Kugeln} in einem Fach \alert{spielt} eine \alert{Rolle}, da die Kugeln \alert{nicht unterscheidbar} sind
                \end{itemize}
              \end{minipage}
            }
          }
        }
        child {
          node {Urnenmodell
            \resizebox{\textwidth}{!}{
              \begin{minipage}[t]{8cm}
                \begin{itemize}
                  \item \alert{Menge möglicher Relationen:}\\ $U = \{\{(1, x), (2, x)\}, \{(2, x), (3, x)\}, \{(1, x), (3, x)\},$\\
                    $\{(1, x)\}, \{(2, x)\}, \{(3, x)\}\}$
                   \item \alert{Kombinationen:} $\{(1, 2), (2, 3), (1, 3), (1, 1), (2, 2), (3, 3)\}$
                    \begin{itemize}
                      \item man kann anstelle von Tupelklammern $()$ auch Mengenklammern $\{\}$ verwenden, da die Reihenfolge egal ist. Man muss eine Menge mit nur einem Element $\{x\}$ dann nur korrekt als $(x, x)$ interpretieren
                    \end{itemize}
                   \item $|U| = \dfrac{(3+2-1)!}{2!(3-1)!} = \dfrac{4\cdot 3}{2} = 6$
                 \end{itemize}
              \end{minipage}
            }
          }
        }
      }
    }
    child {
      node {Allgemeines Zählprinzip
        \resizebox{\textwidth}{!}{
          \begin{minipage}[t]{14cm}
            \begin{itemize}
              \item auch \alert{Produktregel der Kombinatorik}
              \item ist weder eine Anordnung noch eine Auswahl, da das Kartesische Produkt zweier potentiel unterschiedlicher Mengen berechnet wird. Für die Anzahl ist die Reihenfolge in der das Kartesische Produkt zwischen jeweils zwei der Mengen angewandt wird egal, nur werden sich bei jeder unterschiedlichen Reihenfolge andere Tupel ergeben
              \item bei $k$ Mengen $M_1, M_2, \ldots, M_k$, die jeweils $n_1, n_2, \ldots, n_k$ Elemente enthalten, lassen sich $n_1\cdot n_2\cdot \ldots \cdot n_k$ verschiedene $k$-Tupel $(x_1, x_2,\ldots, x_k)$ zusammenstellen
            \end{itemize}
            \begin{resettikz}
              \ctikzfig{./figures/combinatorics}
            \end{resettikz}
            \begin{itemize}
              \item die Anzahl \textcolor{PrimaryColor}{blauer} Blätter im oberen Teilbauem ist gleich der Anzahl verschiedener 2er Tupel aus $\{1, 2\}$ und $\{1, 2\}$ und der Anzahl verschiedener Pfade im oberen Teilbaum. Die Anzahl \textcolor{SecondaryColor}{roter} Blätter ist gleich der Anzahl verschiedener 2er Tupel aus $\{3, 4\}$ und $\{1, 2, 3\}$ und der Anzahl verschiedener Pfade im oberen Teilbaum
              \begin{itemize}
                \item wenn zwei Teilbäume \alert{identisch} sind, kann man deren gemeinsame Blätteranzahl mithilfe der \alert{Produktregel der Kombinatorik} berechnen
                \item wenn zwei Teilbäume \alert{unterschiedlich} sind, muss man deren Blätteranzahl \alert{seperat berechnen} und dann \alert{addieren}
              \end{itemize}               
            \end{itemize}
          \end{minipage}
        }
      }
      child {
        node {Potenzieren
            \resizebox{\textwidth}{!}{
              \begin{minipage}[t]{14cm}
                \begin{itemize}
                  \item haben alle Knoten in $b$ Teilbäumen \alert{gleich viele Nachfolgerknoten} $b$ kann man die Anzahl der Blätter im allen Teilbäumen mithilfe der \alert{Tiefe} aller Teilbäume zusammen $k$ und der \alert{Anzahl Nachfolgeknoten} $b$ berechnen als $b^k = 2^3 = 8$
                \end{itemize}
                \begin{resettikz}
                  \ctikzfig{./figures/combinatoricspower}
                \end{resettikz}
              \end{minipage}
          }
        }
      }
    }               ;
  \end{mindmapcontent}
  % \begin{edges}
  %   \edge{lc}{bc}
  % \end{edges}
  \annotation{co.south}{This mindmap is provided without guarantee of correctness and completeness!};
\end{mindmap}

% die Sache mit injektiver Funktion bei Variationen und bijektiver Funktion bei Permutationen
