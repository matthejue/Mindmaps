%!Tex Root = ../main.tex
% ./Packete.tex
% ./Design.tex
% ./Vorbereitung.tex
% ./Aufgabe1.tex
% ./Aufgabe2.tex
% ./Aufgabe3.tex
% ./Aufgabe4.tex
% ./Appendix.tex

\begin{mindmap}
  \begin{mindmapcontent}
    \node (co) at (current page.center) {Kombinatorik (combinatorics)}
    child {
      node {Variationen und Permutationen (partial permutations and permutations)
        \resizebox{\textwidth}{!}{
          \begin{minipage}[t]{8cm}
            \begin{itemize}
              \item \underline{Urnenmodell:} \alert{Reihenfolge berücksichtigt} ($\approx$ Folge)
                \vspace{-2cm}
                \begin{resettikz}
                  \ctikzfig{example_urn_problem_order_important}
                \end{resettikz}
                \vspace{-2cm}
              \item \underline{Fächermodell:} \alert{unterscheidbar}
                \vspace{-2cm}
                \begin{resettikz}
                  \ctikzfig{example_box_problem_destinguishable}
                \end{resettikz}
                \vspace{-2cm}
              \item \alert{Variation} wenn $k<n$
              \item \alert{Permutation} wenn $k=n$
            \end{itemize}
          \end{minipage}
        }
      }
      child {
        node {Variationen / Permutationen mit Wiederholung
          \resizebox{\textwidth}{!}{
            \begin{minipage}[t]{12cm}
              \begin{itemize}
                \item \alert{mit Zurücklegen} in Bezug auf Urnenmodell
                \item \alert{Anzahl:} $n^k$, $k$ beliebig ($k>n$ ist erlaubt)
                  \begin{itemize}
                    \item Anzahl \alert{Folgen}, wobei Folgenglieder sich \alert{nicht unterscheiden} müssen
                  \end{itemize}
                \item \alert{bijektive Relation} für Urnenmodell
                \begin{itemize}
                  \item \alert{nicht rechtseindeutig} (also wie Multifunktion)
                \end{itemize}
                \item \alert{Funktion} für Fächermodell
                \begin{itemize}
                  \item \alert{nicht injektiv} (linkseindeutig)
                \end{itemize}
              \end{itemize}
            \end{minipage}
          }
        }
        child {
          node {Urnenmodell
            \resizebox{\textwidth}{!}{
              \begin{minipage}[t]{8cm}
                \begin{itemize}
                  \item $\Omega = \{\{(1, 1), (2, 2)\}, \{(2, 1), (3, 2)\}, \{(1, 1), (3, 2)\},$\\
                    $\{(2, 1), (1, 2)\}, \{(3, 1), (2, 2)\}, \{(3, 1), (1, 2)\},$\\
                    $\{(1, 1), (1, 2)\}, \{(2, 1), (2, 2)\}, \{(3, 1), (3, 2)\}\}$
                  \item $|\Omega| = 3^2 = 9$
                \end{itemize}
              \end{minipage}
            }
          }
        }
        child {
          node {Fächermodell
            \resizebox{\textwidth}{!}{
              \begin{minipage}[t]{8cm}
                \begin{itemize}
                  \item $\Omega = \{\{(1, 1), (2, 2)\}, \{(1, 2), (2, 3)\}, \{(1, 1), (2, 3)\},$\\
                    $\{(1, 2), (2, 1)\}, \{(1, 3), (2, 2)\}, \{(1, 3), (2, 1)\},$\\
                    $\{(1, 1), (2, 1)\}, \{(1, 2), (2, 2)\}, \{(1, 3), (2, 3)\}\}$
                  \item $|\Omega| = 3^2 = 9$
                  \item in Fächern selbst spielt \alert{Reihenfolge} der Kugeln darin \alert{keine Rolle}, nur welche Kugeln darin sind spielt eine Rolle (jedes Fach verhält sich also wie eine \alert{gewöhnliche Menge}, $\{(1, 3), (2, 3)\}$ und $\{(2, 3), (1, 3)\}$ werden \alert{nicht unterschieden})
                  \item es spielt hingegeben eine Rolle, welche Kugel welchem Fach zugeordnet ist ($\{(1, 3), (2, 2)\}$ und $\{(1, 2), (2, 3)\}$ werden \alert{unterschieden})
                \end{itemize}
              \end{minipage}
            }
          }
        }
      }
      child {
        node {Variationen / Permutationen ohne Wiederholung
          \resizebox{\textwidth}{!}{
            \begin{minipage}[t]{12cm}
              \begin{itemize}
                \item \alert{ohne Zurücklegen} in Bezug auf Urnenmodell
                \item \alert{Anzahl:} $\dfrac{n!}{(n-k)!}$, $k\le n$
                \begin{itemize}
                  \item $\dfrac{n!}{(n-n)!} = \dfrac{n!}{1} = n!$, $k=n$
                  \item Anzahl \alert{Folgen}, wobei Folgenglieder sich alle \alert{unterscheiden} müssen
                  \item \alert{Herleitung:} für die erste Stelle gibt es $n$ Kombinationen, für die zweite $n-1$ Kombinationen, also $n!$, wenn man bis 1 runtergeht. Allerdings gibt es nur $k\le n$ Stellen, daher muss man 1 bis $n-k$, also $(n-k)!$ in den Nenner schreiben, damit sich $1$ bis $n-k$ im Zähler damit rauskürzen
                \end{itemize}
                \item \alert{partielle, bijektive Funktion} und \alert{bijektive Funktion} wenn $k=n$ für Urnenmodell
                \begin{itemize}
                  \item \alert{rechtseindeutig} (also wie partielle Funktion)
                \end{itemize}
                \item \alert{injektive Funktion} und \alert{bijektive Funktion} wenn $k=n$ für Fächermodell
                \begin{itemize}
                  \item \alert{injektiv} (linkseindeutig)
                \end{itemize}
              \end{itemize}
            \end{minipage}
          }
        }
        child {
          node {Urnenmodell
            \resizebox{\textwidth}{!}{
              \begin{minipage}[t]{8cm}
                \begin{itemize}
                  \item $\Omega = \{\{(1, 1), (2, 2)\}, \{(2, 1), (3, 2)\}, \{(1, 1), (3, 2)\},$\\
                    $\{(2, 1), (1, 2)\}, \{(3, 1), (2, 2)\}, \{(3, 1), (1, 2)\}\}$
                  \item $|\Omega| = \dfrac{3!}{(3-2)!} = \dfrac{6}{1} = 6$
                \end{itemize}
              \end{minipage}
            }
          }
        }
        child {
          node {Fächermodell
            \resizebox{\textwidth}{!}{
              \begin{minipage}[t]{8cm}
                \begin{itemize}
                  \item $\Omega = \{\{(1, 1), (2, 2)\}, \{(1, 2), (2, 3)\}, \{(1, 1), (2, 3)\},$\\
                    $\{(1, 2), (2, 1)\}, \{(1, 3), (2, 2)\}, \{(1, 3), (2, 1)\}\}$
                  \item $|\Omega| = \dfrac{3!}{(3-2)!} = \dfrac{6}{1} = 6$
                \end{itemize}
              \end{minipage}
            }
          }
        }
      }
    }
    child {
      node {Kombinationen (combinations)
        \resizebox{\textwidth}{!}{
          \begin{minipage}[t]{8cm}
            \begin{itemize}
              \item \underline{Urnenmodell:} \alert{Reihenfolge nicht berücksichtigt} ($\approx$ Menge)
                \vspace{-2cm}
                \begin{resettikz}
                  \ctikzfig{example_urn_problem_order_unimportant}
                \end{resettikz}
                \vspace{-2cm}
              \item \underline{Fächermodell:} \alert{nicht unterscheidbar}
                \vspace{-2cm}
                \begin{resettikz}
                  \ctikzfig{example_box_problem_indestinguishable}
                \end{resettikz}
                \vspace{-2cm}
            \end{itemize}
          \end{minipage}
        }
      }
      child {
        node {Kombinationen ohne Wiederholung
          \resizebox{\textwidth}{!}{
            \begin{minipage}[t]{12cm}
              \begin{itemize}
                \item \alert{ohne Zurücklegen} in Bezug auf Urnenmodell
                \item \alert{Anzahl:} $\dbinom{n}{k} = \dfrac{n!}{k!\cdot(n-k)!} = \dfrac{n\cdot (n-1) \cdot \ldots \cdot (n-k+1)}{k!}$, $k\le n$
                \begin{itemize}
                  \item \alert{Binomialkoeffizient} für Anzahl \alert{gewöhnlicher Untermengen} der Länge $k$
                  \item $\dbinom{n}{n} = \dfrac{n!}{n!\cdot(n-n)!} = \dfrac{n!}{n!\cdot(n-n)!} = \dfrac{n!}{n!\cdot 1} = 1$, $k=n$
                \end{itemize}
                \item \alert{partielle, bijektive Funktion} und \alert{bijektive Funktion} wenn $k=n$ für Urnenmodell
                \begin{itemize}
                  \item \alert{rechtseindeutig} (also wie partielle Funtion)
                \end{itemize}
                \item \alert{injektive Funktion} und \alert{bijektive Funktion} wenn $k=n$ für Fächermodell
                \begin{itemize}
                  \item \alert{injektiv} (linkseindeutig)
                \end{itemize}
              \end{itemize}
            \end{minipage}
          }
        }
        child {
          node {Fächermodell
            \resizebox{\textwidth}{!}{
              \begin{minipage}[t]{8cm}
                \begin{itemize}
                  \item $\Omega = \{\{(1, x), (2, x)\}, \{(2, x), (3, x)\}, \{(1, x), (3, x)\}\}$
                  \item $|\Omega| = \dfrac{3!}{2!(3-2)!} = \dfrac{6}{2} = 3$
                \end{itemize}
              \end{minipage}
            }
          }
        }
        child {
          node {Urnenmodell
            \resizebox{\textwidth}{!}{
              \begin{minipage}[t]{8cm}
                \begin{itemize}
                  \item $\Omega = \{\{(x, 1), (x, 2)\}, \{(x, 2), (x, 3)\}, \{(x, 1), (x, 3)\}\}$
                  \item $|\Omega| = \dfrac{3!}{2!(3-2)!} = \dfrac{6}{2} = 3$
                \end{itemize}
              \end{minipage}
            }
          }
        }
      }
      child {
        node {Kombinationen mit Wiederholung
          \resizebox{\textwidth}{!}{
            \begin{minipage}[t]{12cm}
              \begin{itemize}
                \item \alert{mit Zurücklegen} in Bezug auf Urnenmodell
                \item \alert{Anzahl:} $\dbinom{n+k-1}{k} = \dfrac{(n+k-1)!}{k!\cdot(n+k-1-k)!} = \dfrac{(n+k-1)!}{k!\cdot(n-1)!} = \dfrac{n\cdot (n+1)\cdot \ldots \cdot (n+k-1)}{k!}$, $k$ beliebig ($k>n$ ist erlaubt) 
                \begin{itemize}
                  \item spezieller \alert{Binomialkoeffizient} $\left(\!\!\dbinom{n}{k}\!\!\right) = \dbinom{n + k - 1}{k} = \dbinom{n+k-1}{n-1}$ für Anzahl \alert{Multimengen} der Länge $k$
                \end{itemize}
                \item \alert{surjektive Relation} für Urnenmodell
                \begin{itemize}
                  \item \alert{nicht rechtseindeutig} (also wie Multifunktion)
                \end{itemize}
                \item \alert{Funktion} für Fächermodell
                \begin{itemize}
                  \item \alert{nicht injektiv} (linkseindeutig)
                \end{itemize}
              \end{itemize}
            \end{minipage}
          }
        }
        child {
          node {Fächermodell
            \resizebox{\textwidth}{!}{
              \begin{minipage}[t]{8cm}
                \begin{itemize}
                   \item $\Omega = \{\{(x, 1), (x, 2)\}, \{(x, 2), (x, 3)\}, \{(x, 1), (x, 3)\},$\\
                     $\{(x, 1)\}, \{(x, 2)\}, \{(x, 3)\}\}$
                   \item $|\Omega| = \dfrac{(3+2-1)!}{2!(3-1)!} = \dfrac{4\cdot 3}{2} = 6$
                   \item nur die \alert{Anzahl Kugeln} in einem Fach \alert{spielt} eine \alert{Rolle}, da die Kugeln \alert{nicht unterscheidbar} sind
                \end{itemize}
              \end{minipage}
            }
          }
        }
        child {
          node {Urnenmodell
            \resizebox{\textwidth}{!}{
              \begin{minipage}[t]{8cm}
                \begin{itemize}
                  \item $\Omega = \{\{(1, x), (2, x)\}, \{(2, x), (3, x)\}, \{(1, x), (3, x)\},$\\
                    $\{(1, x)\}, \{(2, x)\}, \{(3, x)\}\}$
                   \item $|\Omega| = \dfrac{(3+2-1)!}{2!(3-1)!} = \dfrac{4\cdot 3}{2} = 6$
                 \end{itemize}
              \end{minipage}
            }
          }
        }
      }
    };
  \end{mindmapcontent}
  % \begin{edges}
  %   \edge{lc}{bc}
  % \end{edges}
  \annotation{co.south}{This mindmap is provided without guarantee of correctness and completeness!};
\end{mindmap}

% die Sache mit injektiver Funktion bei Variationen und bijektiver Funktion bei Permutationen
