%!Tex Root = ../main.tex
% ./Packete.tex
% ./Design.tex
% ./Vorbereitung.tex
% ./Aufgabe1.tex
% ./Aufgabe2.tex
% ./Aufgabe3.tex
% ./Aufgabe4.tex
% ./Appendix.tex

\begin{mindmap}
  \begin{mindmapcontent}
    \node (lo) at (current page.center) {Logic
      \resizebox{\textwidth}{!}{
        \begin{minipage}[t]{16cm}
          \begin{itemize}
            \item Logic defines \href[page=223]{/home/areo/Documents/Studium/Summaries/Logic/Foundations_of_AI_all_in_one_with_go_back.pdf}{Syntax and Semantics}
            \begin{itemize}
              \item sentence (germ. Aussage)
            \end{itemize}
          \end{itemize}
        \end{minipage}
      }
    }
    child {
      node {Predicate Logic / Prädikatenlogik}
    }
    child {
      node {Propositional Logic / Aussagenlogik
        \resizebox{\textwidth}{!}{
          \begin{minipage}[t]{12cm}
            \begin{itemize}
              \item \href[page=230]{/home/areo/Documents/Studium/Summaries/Logic/Foundations_of_AI_all_in_one_with_go_back.pdf}{Syntax and operator precedence}
              \item \alert{logic formula:} e.g. $P \vee Q$, \alert{atomic formula / atomic proposition / atom:} e.g. $P$
              \begin{itemize}
                \item atomic propositions can be \alert{true} (T) or \alert{false} (F)
                \item the truth of a formula follows from the truth of its atomic propositions (\alert{truth assignment} or \alert{interpretation}) and the connectives
              \end{itemize}
              \item \alert{literal:} (possibly negated) atomic formula
              \item \alert{clause:} disjunction of literals
            \end{itemize}
            % \begin{itemize}
            %   \item \alert{countable alphabet} of \alert{atomic propositions}: e.g. $\Sigma = \{$P, Q, R, \dots$\}$
            %   \begin{itemize}
            %     \item atomic propositions can be \alert{true} (T) or \alert{false} (F)
            %     \item the truth of a formula follows from the truth of its atomic propositions (\alert{truth assignment} or \alert{interpretation}) and the connectives.
            %   \end{itemize}
            %   \item \alert{logic formula:} e.g. $P \vee Q$, \alert{atomic formula / atom:} e.g. $P$
            %   \begin{itemize}
            %     \item \alert{operator precedence:} $\neg > \wedge > \vee > \Rightarrow > \Leftrightarrow$
            %   \end{itemize}
            %   \item \alert{literal:} (possibly negated) atomic formula
            %   \item \alert{clause:} disjunction of literals
            % \end{itemize}
          \end{minipage}
        }
      }
      child {
        node {Boolesche Algebra
          \resizebox{\textwidth}{!}{
            \begin{minipage}[t]{8cm}
              \begin{itemize}
                \item \href[page=224]{/home/areo/Documents/Studium/Summaries/Technische_Informatik/Technische_Informatik_all_in_one_with_go_back.pdf}{Definition, Axiome}
                \item \href[page=15]{/home/areo/Documents/Studium/Summaries/Technische_Informatik/Technische_Informatik_all_in_one_with_go_back.pdf}{Konventionen, Präzidenzregeln}
                \item aus den Axiomen \href[page=231]{/home/areo/Documents/Studium/Summaries/Technische_Informatik/Technische_Informatik_all_in_one_with_go_back.pdf}{ableitbare Rechenregeln}
              \end{itemize}
            \end{minipage}
          }
        }
        child {
          node {Boolesche Algebra der Funktionen in n Variablen $(\mathbb{B}_n, \cdot, +, \neg)$
            \resizebox{\textwidth}{!}{
              \begin{minipage}[t]{8cm}
                \begin{itemize}
                  \item \href[page=228]{/home/areo/Documents/Studium/Summaries/Technische_Informatik/Technische_Informatik_all_in_one_with_go_back.pdf}{Definition}
                \end{itemize}
              \end{minipage}
            }
          }
        }
        child {
          node {Boolesche Algebra der Teilmengen von S $(Pot(S), \cap, \cup, \neg)$
            \resizebox{\textwidth}{!}{
              \begin{minipage}[t]{8cm}
                \begin{itemize}
                  \item \href[page=229]{/home/areo/Documents/Studium/Summaries/Technische_Informatik/Technische_Informatik_all_in_one_with_go_back.pdf}{Definition}
                \end{itemize}
              \end{minipage}
            }
          }
        }
        child {
          node {Dualitätsprinzip bei booleschen Algebren
            \resizebox{\textwidth}{!}{
              \begin{minipage}[t]{8cm}
                \begin{itemize}
                  \item \href[page=233]{/home/areo/Documents/Studium/Summaries/Technische_Informatik/Technische_Informatik_all_in_one_with_go_back.pdf}{Definition}
                \end{itemize}
              \end{minipage}
            }
          }
        }
      }
      % child {
      %   node {Syntax and Semantics
      %     \resizebox{\textwidth}{!}{
      %       \begin{minipage}[t]{12cm}
      %         \begin{itemize}
      %           \item \alert{Sentences} are expressed according to the syntax of the representation language
      %           \begin{itemize}
      %             \item \alert{Syntax} specifies all the sentences that are well-formed
      %             \begin{itemize}
      %               \item $x + y = 4$ is a well-formed sentence
      %               \item $x4y+ =$ is not a well-formed sentence
      %             \end{itemize}
      %             \item A logic also defines the \alert{semantics} (meaning) of sentences
      %             \begin{itemize}
      %               \item Defines the \alert{truth} of a sentence with respect to each possible world
      %               \item E.g., specifies that the sentence $x + y = 4$ is true in a world in which $x = 2$ and $y = 2$, but not in a world in which $x = 1$ and $y = 1$
      %             \end{itemize}
      %           \end{itemize}
      %         \end{itemize}
      %       \end{minipage}
      %     }
      %   }
      % }
      child {
        node {Logical Entailment / Logische Ableitung bzw. Deduktion
          \resizebox{\textwidth}{!}{
            \begin{minipage}[t]{8cm}
              \begin{itemize}
                \item $\alpha \models \beta$ \alert{iff} $M(\alpha) \subseteq M(\beta)$ \alert{iff} in every model in which $\alpha$ is true, $\beta$ is also true
                \begin{itemize}
                  \item when a sentence $\beta$ \alert{\enquote{logically follows}} from another sentence $\alpha$, $\alpha$ \alert{\enquote{entails}} $\beta$
                  \item \alert{semantic relation} between models of a theorie (set of sentences) or a single sentence and models of a sentence
                  \item \href[page=224]{/home/areo/Documents/Studium/Summaries/Logic/Foundations_of_AI_all_in_one_with_go_back.pdf}{reference 1} and \href[page=226]{/home/areo/Documents/Studium/Summaries/Logic/Foundations_of_AI_all_in_one_with_go_back.pdf}{reference 2}
                \end{itemize}
                  \item $I \models \varphi$: Interpretation $I$ \alert{\enquote{satisfies}} a formula $\varphi$ or $\varphi$ \alert{\enquote{is true under}} $I$ when $I(\varphi) = T$
              \end{itemize}
            \end{minipage}
          }
        }
          child {
            node (models) {Models
              \resizebox{\textwidth}{!}{
                \begin{minipage}[t]{8cm}
                  \begin{itemize}
                    \item If a sentence $\alpha$ is true in a possible \alert{world} $m$, we say that $m$ \alert{satisfies} $\alpha$ or $m$ is a \alert{model} of $\alpha$
                    \begin{itemize}
                      \item We denote the set of all models of $\alpha$ by $M(\alpha)$ ($\hat=$ On-Menge)
                    \end{itemize}
                  \end{itemize}
                \end{minipage}
              }
            }
          }
          child {
            node {Interpretation
              \resizebox{\textwidth}{!}{
                \begin{minipage}[t]{8cm}
                  \begin{itemize}
                    \item \alert{truth assignment} of the atoms in $\Sigma$, corresponds to semantics
                    \item $I: \Sigma \mapsto \{T, F\}$
                    \item \href[page=232]{/home/areo/Documents/Studium/Summaries/Logic/Foundations_of_AI_all_in_one_with_go_back.pdf}{inductive definition} of the interpretation function
                  \end{itemize}
                \end{minipage}
              }
            }
          }
          child {
            node {Inference
              \resizebox{\textwidth}{!}{
                \begin{minipage}[t]{8cm}
                  \begin{itemize}
                    \item $theorie \vdash_i \alpha$: generate / derive a sentence $\alpha$ with an inference method $i$
                    \item \href[page=226]{/home/areo/Documents/Studium/Summaries/Logic/Foundations_of_AI_all_in_one_with_go_back.pdf}{reference}
                    \item like to have inference algorithms that derive only sentences that are entailed (\alert{soundness}) and all of them (\alert{completeness})
                  \end{itemize}
                \end{minipage}
              }
            }
          }
      }
    };
  \end{mindmapcontent}
  \begin{edges}
    % \edge{lo}{models} % just test
  \end{edges}
  \annotation{lo.south}{This mindmap is provided without guarantee of correctness and completeness!}
  \annotation{lo.north}{\href{/tmp/current.pdf}{go back}}
\end{mindmap}
