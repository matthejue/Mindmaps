\documentclass{standalone}
\usepackage[english]{babel}
% https://tex.stackexchange.com/questions/570303/use-blacktriangleright-as-itemize-label

\usepackage{amssymb} % for black triangleright

\usepackage{amsmath}

\usepackage{graphicx}
% \graphicspath{figures/}

\renewcommand{\labelitemi}{$\textcolor{SwitchColor}{\bullet}$}
\renewcommand{\labelitemii}{$\textcolor{SwitchColor}{\blacktriangleright}$}
\renewcommand{\labelitemiii}{$\textcolor{SwitchColor}{\blacksquare}$}

% https://tex.stackexchange.com/questions/525959/prevent-latex-from-stretching-math
\setlength{\thinmuskip}{1\thinmuskip}
\setlength{\medmuskip}{1\medmuskip}
\setlength{\thickmuskip}{1\thickmuskip}

\usepackage{csquotes}
\usepackage{xcolor}
% \usepackage{anyfontsize}
\usepackage[export]{adjustbox}
% \usepackage[]{enumitem}
\usepackage{tikz}
\usetikzlibrary{arrows.meta,positioning}
\usetikzlibrary{graphs}
\usetikzlibrary{patterns}
\usetikzlibrary{shadings}
\usetikzlibrary{mindmap, shadows, backgrounds} % , calc

\definecolor{SecondaryColor}{HTML}{BBAF01}
\definecolor{SecondaryColorDimmed}{HTML}{FEF684}
\definecolor{PrimaryColor}{HTML}{E95112}
\definecolor{PrimaryColorDimmed}{HTML}{F6AF91}
\definecolor{SwitchColor}{named}{PrimaryColor}
\colorlet{BoxColor}{gray!10!white}

\usepackage[allbordercolors=PrimaryColor, pdfborder={0 0 .2}]{hyperref}

% colored bold
% \newcommand\alert[1]{\textcolor{SwitchColor}{\textbf{#1}}}
\newcommand\alert[1]{\textcolor{SwitchColor}{#1}}

\newlength{\leveldistance}
\setlength{\leveldistance}{25cm}

% to input other file start
\usepackage{pseudo}
\usepackage{xparse}
\usepackage{tcolorbox}
\tcbuselibrary{skins,theorems}

\newcounter{algorithm}
\setcounter{algorithm}{0}
\newtcbtheorem[use counter=algorithm]{algorithm}{\color{SecondaryColor}Algorithm}{pseudo/ruled}{alg}

\newcommand{\ma}[1]{$\mathcal{#1}$}
\renewcommand{\tt}[1]{{\small\texttt{#1}}}

% https://tex.stackexchange.com/questions/600708/input-inside-tikzpicture-differs-from-inserting-manually
\ExplSyntaxOn
\cs_new:Npn \expandableinput #1
  { \use:c { @@input } { \file_full_name:n {#1} } }
\AddToHook{env/tikzpicture/begin}
  { \cs_set_eq:NN \input \expandableinput }
\ExplSyntaxOff

\usepackage{fontspec}
\newfontfamily\gyre{DejaVu Math TeX Gyre}

\NewDocumentCommand{\bfs}{s}{
  \begin{algorithm}{\pr{Breadth-First Search as Graph-Search}($G$, $s$, $t$)}{\thetcbcounter}
    \begin{pseudo}[indent-mark,kw,hl-warn=false]
      \ma{Q} $\leftarrow$ \tt{new.Queue()}\\
      \tt{counter} $\leftarrow 1$\\
      \ma{Q}\tt{.enqueue($s$)}; \tn{mark $s$}\\
      \tt{s.count} $\leftarrow$ \tt{counter}; \tt{counter} $\leftarrow$ \tt{counter} $+ 1$ \\
      As long as \tn{not \ma{Q}.empty()} do\\+
        $u \leftarrow$ \ma{Q}\tt{.dequeue()}\IfBooleanTF#1{\\[hl]}{\\} 
        if $u$ is $t$ then\IfBooleanTF#1{\\+[hl]}{\\+} 
          return \cn{true}\\-
        for each \tn{adjavent node $v$ of $u$ in $G$} do\\+
          if $v$ \tn{not marked} then\\+
            \ma{Q}\tt{.enqueue($v$)}; \tn{mark $v$}\\
            $v$\tt{.count} $\leftarrow$ \tt{counter}; \tt{counter} $\leftarrow$ \tt{counter} $+ 1$ \IfBooleanTF#1{\\---[hl]}{\\---}
      return \cn{false}
    \end{pseudo}
  \end{algorithm}
}

\NewDocumentCommand{\dfs}{s}{
  \begin{algorithm}{\pr{Depth-First Search as Graph Search}($G$, $s$, $t$)}{\thetcbcounter}
    \begin{pseudo}[indent-mark,kw,hpad=0.6cm]
      \ma{S} $\leftarrow$ \tt{new.Stack()} \\
      \tt{counter} $\leftarrow 1$\\
      \ma{S}\tt{.push($s$);} \tn{mark} $s$\\
      $s$\tt{.start} $\leftarrow$ \tt{counter;} \tt{counter} $\leftarrow$ \tt{counter} $+ 1$\IfBooleanTF#1{\\[hl]}{\\}
      if $s$ is $t$ then\IfBooleanTF#1{\\+[hl]}{\\+}
        return \cn{true}\\-
      As long as \tn{not \ma{S}\tt{.empty()}} do\\+
        % $u \leftarrow$ \ma{S}\tt{.pop();} \ma{S}\tt{.push($u$)}\\
        $u \leftarrow$ \ma{S}\tt{.look\_at\_top()}\\
        if \tn{a not marked adjacent node $v$ of $u$ exists in $G$} then\\+
          \ma{S}\tt{.push($v$);} \tn{mark} $v$\IfBooleanTF#1{\\[hl]}{\\}
          if $v$ \tn{is} $t$ then\IfBooleanTF#1{\\+[hl]}{\\+}
            return \cn{true}\\-
          $v$\tt{.start} $\leftarrow$ \tt{counter;} \tt{counter} $\leftarrow$ \tt{counter} $+ 1$ \\-
        else\\+
          $u$ $\leftarrow$ \ma{S}\tt{.pop()}\\
          $u$\tt{.end} $\leftarrow$ \tt{counter;} \tt{counter} $\leftarrow$ \tt{counter} $+ 1$\IfBooleanTF#1{\\--[hl]}{\\--}
      return \cn{false}
    \end{pseudo}
  \end{algorithm}
}

\NewDocumentCommand{\dfsrec}{s}{
  \begin{algorithm}{\pr{Recursive Depth-First Search}($G$, $s$, $t$, $1$)}{\thetcbcounter}
    \begin{pseudo}[indent-mark,kw,hl-warn=false]
    \fn{Rec-DFS}\tn{($G$, $u$, $t$, \tt{counter})}\IfBooleanTF#1{\\+[hl]}{\\+}
      if $u$ is $t$ then\IfBooleanTF#1{\\+[hl]}{\\+}
        return \cn{True}\\-
      \fn{mark} $u$\\
      $u$\tt{.start} $\leftarrow$ \tt{counter;} \tt{counter} $\leftarrow$ \tt{counter} $+ 1$\\
      for each \tn{adjacent node $v$ of $u$ in $G$} do\\+
        if \tn{$v$ not marked} then\\+
          \tt{result}, \tt{counter} $\leftarrow$ \fn{Rec-DFS}\tn{($G$, $v$, $t$, \tt{counter})}\IfBooleanTF#1{\\[hl]}{\\}
          if \tn{$result$ is} \cn{True} then\IfBooleanTF#1{\\+[hl]}{\\+}
            return \tt{True}, \tt{counter}\\---
      $u$\tt{.end} $\leftarrow$ \tt{counter;} \tt{counter} $\leftarrow$ \tt{counter} $+ 1$\IfBooleanTF#1{\\[hl]}{\\}
      return \cn{False}, \tt{counter}\\--
    \end{pseudo}
  \end{algorithm}
}

\newcommand{\sourcesone}{
  \resizebox{\textwidth}{!}{
    \begin{minipage}[t]{6cm}
      \tiny \cite{AnswerWhatDifference2018}, \cite{russell2010artificial}, \cite{Pseudocode2023}, \cite{ziggystarAnswerWhatDifference2013}
    \end{minipage}
  }
}

\newcommand{\bd}{
  \resizebox{\textwidth}{!}{
    \begin{minipage}[t]{10cm}
      \begin{itemize}
        \item number of \alert{Vertices} ${\mid}V{\mid}$ and \alert{Edges} ${\mid}E{\mid}$ are for \alert{explicit graphs}
        \item \alert{maximal branching factor} (maximal out-degree) $b$, \alert{depth of a target node} $d$, \alert{maximum depth of the search tree} $m$, \alert{cost of the optimal path} $C$ and \alert{minimal weight of an edge} $\epsilon$ are for \alert{implicit graphs}, whose vertices or edges are not represented as explicit objects in a computer's memory, but rather are determined algorithmically from some other input, e.g. a computable function(the states/nodes are generated). That might be needed when working with graphs that are too large to store explicitly (or infinite)
        \item ${\mid}E{\mid}$ may vary between $1$ and ${\mid}V{\mid}^2$ for simple graphs
      \end{itemize}
    \end{minipage}
  }
}

% to input other file end

\begin{document}
  \begin{tikzpicture}[
      auto,
      huge mindmap,
      fill opacity=0.6,
      draw opacity=0.8,
      concept color = PrimaryColorDimmed,
      every annotation/.style={fill=BoxColor, draw=none, align=center, fill = BoxColor, text width = 2cm},
      grow cyclic,
      level 1/.append style = {
        concept color=SecondaryColorDimmed,
        level distance=\leveldistance,
        sibling angle=360/\the\tikznumberofchildren,
        % https://tex.stackexchange.com/questions/501240/trying-to-use-the-array-environment-inside-a-tikz-node-with-execute-at-begin-no
        execute at begin node=\definecolor{SwitchColor}{named}{SecondaryColor},
      },
      level 2/.append style = {
        concept color=PrimaryColorDimmed,
        level distance=\leveldistance / 2,
        sibling angle=30,
        execute at begin node=\definecolor{SwitchColor}{named}{PrimaryColor},
      },
      level 3/.append style = {
        concept color=SecondaryColorDimmed,
        level distance=\leveldistance / 3,
        execute at begin node=\definecolor{SwitchColor}{named}{SecondaryColor},
      },
      level 4/.append style = {
        concept color=PrimaryColorDimmed,
        level distance=\leveldistance / 4,
        execute at begin node=\definecolor{SwitchColor}{named}{PrimaryColor},
      },
      level 5/.append style = {
        concept color=SecondaryColorDimmed,
        level distance=\leveldistance / 5,
        execute at begin node=\definecolor{SwitchColor}{named}{SecondaryColor},
      },
      level 6/.append style = {
        concept color=PrimaryColorDimmed,
        level distance=\leveldistance / 6,
        execute at begin node=\definecolor{SwitchColor}{named}{PrimaryColor},
      },
      level 7/.append style = {
        concept color=SecondaryColorDimmed,
        level distance=\leveldistance / 7,
        execute at begin node=\definecolor{SwitchColor}{named}{SecondaryColor},
      },
      level 8/.append style = {
        concept color=PrimaryColorDimmed,
        level distance=\leveldistance / 8,
        execute at begin node=\definecolor{SwitchColor}{named}{PrimaryColor},
      },
      concept connection/.append style = {
        color = BoxColor,
      },
  ]
  % damit Annotationen nicht auch eine Drop Shadow erhalten
  \begin{scope}[
    every node/.style = {concept, circular drop shadow}, % draw=none
    every child/.style={concept},
    ]
    \node (gt) at (current page.center) {Graph Theory
        \resizebox{\textwidth}{!}{
          \begin{minipage}[t]{18cm}
            \begin{itemize}
              \item the order (germ. Knotenzahl) $n(G)$ of a graph $G$ is the number of vertices
              \item the size (germ. Kantenzahl) $e(G)$ of a graph $G$ is the number of edges
            \end{itemize}
          \end{minipage}
        }
      }
      child {
        node {Search Algorithms}
        % START1
        % END1
        % START2
        % END2
        % START3
        % END3
      }
      child {
        node {Dual Problems}
        child {
          node {Chromatic number $\chi(G)$
            \resizebox{\textwidth}{!}{
              \begin{minipage}[t]{8cm}
                \begin{itemize}
                  \item is the \alert{minimum} number of colors such that \alert{adjacent vertices} receive \alert{different colors}
                \end{itemize}
              \end{minipage}
            }
          }
        }
        child {
          node {Min-Cut and Maximum-Flow
            \resizebox{\textwidth}{!}{
              \begin{minipage}[t]{8cm}
                \begin{itemize}
                  \item coming later in the lecture
                \end{itemize}
              \end{minipage}
            }
          }
        }
        child {
          node {Matching and Vertex cover
            \resizebox{\textwidth}{!}{
              \begin{minipage}[t]{8cm}
                \begin{itemize}
                  \item the  size of a \alert{vertex cover} is always at least as big as the size of a \alert{matching}
                  \item \alert{Theorem (König und Egerváry):} For any bipartite graph $G$ the size of maximum matching equals the size of minimum vertex cover
                \end{itemize}
              \end{minipage}
            }
          }
          child {
            node {Matching
              \resizebox{\textwidth}{!}{
                \begin{minipage}[t]{8cm}
                  \begin{itemize}
                    \item a \alert{matching} $M$ in a graph $G$ is a set of non-loop edges with no shared endpoints
                      \begin{itemize}
                        \item the vertices incident to the edges of a matching M are \alert{$M$-saturated}
                        \item the other vertices are \alert{$M$-unsaturated}
                        \item a \alert{perfect matching} saturates every vertex
                        \item the \alert{size} of a matching is given by the \alert{number of edges}
                        \item a \alert{maximal matching} in a graph is a matching that cannot be enlarged by adding an edge
                        \item a \alert{maximum matching} is a matching of maximum size (among all matchings in a graph)
                        \item $\alpha'(G)$ is the maximum size of matching of edges in $G$
                      \end{itemize}
                  \end{itemize}
                \end{minipage}
              }
            }
            child {
              node (mec) {Other Connections
                \resizebox{\textwidth}{!}{
                  \begin{minipage}[t]{8cm}
                    \begin{itemize}
                      \item \alert{Theorem(Gallai):} If G is a graph without isolated vertices, then $\alpha'(G) + \beta'(G) = n(G)$
                    \end{itemize}
                  \end{minipage}
                }
              }
            }
            child {
              node {Theorem (Berge)
                \resizebox{\textwidth}{!}{
                  \begin{minipage}[t]{8cm}
                    \begin{itemize}
                      \item a matching $M$ in a graph $G$ is a maximum matching in $G$ \alert{iff} $G$ has no $M$-augmenting path
                    \end{itemize}
                  \end{minipage}
                }
              }
              child {
                node (apa) {Augmenting Path Algorithm
                  \resizebox{\textwidth}{!}{
                    \begin{minipage}[t]{8cm}
                      \begin{itemize}
                        \item  Repeatedly applying the Augmenting Path Algorithm to a bipartite graph produces a \alert{matching} and a \alert{vertex cover} of equal size
                      \end{itemize}
                    \end{minipage}
                  }
                }
              }
              child {
                node {Alternating and Augmenting Paths
                  \resizebox{\textwidth}{!}{
                    \begin{minipage}[t]{8cm}
                      \begin{itemize}
                        \item given a matching $M$ in $G$, an \alert{$M$-alternating path} is a path that alternates between edges in $M$ and edges not in $M$
                        \item an $M$-alternating path whose endpoints are unsaturated by $M$ is an \alert{$M$-augmenting path}
                      \end{itemize}
                    \end{minipage}
                  }
                }
              }
            }
            child {
              node {Theorem (Hall)
                \resizebox{\textwidth}{!}{
                  \begin{minipage}[t]{8cm}
                    \begin{itemize}
                      \item An $X$, $Y$-bigraph has a matching that saturates $X$ if and only if $|N(S)| \ge |S|$ for all $S \subseteq X$
                    \end{itemize}
                  \end{minipage}
                }
              }
              child {
                node {X,Y-Bigraph
                  \resizebox{\textwidth}{!}{
                    \begin{minipage}[t]{8cm}
                      \begin{itemize}
                        \item a bipartite graph with vertex sets $X$ and $Y$ and edges only between $X$ and $Y$
                      \end{itemize}
                    \end{minipage}
                  }
                }
              }
            }
            child {
              node {Complete Graphs
                \resizebox{\textwidth}{!}{
                  \begin{minipage}[t]{8cm}
                    \begin{itemize}
                      \item $K_{2n}$ has $f_n = (2n-1)\cdot f_{n-1} = 1\cdot 3\cdot 5\cdot (2n-1)$ choices for choices for perfect matchings
                        \begin{itemize}
                          \item $K_n$ is a complete graph with $n$ vertices, i.e. a simple graph with maximum number of edges
                        \end{itemize}
                      \item $K_{2n+1}$ has no perfect matching because the node number of vertices is odd
                    \end{itemize}
                  \end{minipage}
                }
              }
            }
            child {
              node {Regular Bipartite graphs
                \resizebox{\textwidth}{!}{
                  \begin{minipage}[t]{8cm}
                    \begin{itemize}
                      \item for $k > 0$, every $k$-regular bipartite graph has a perfect matching
                    \end{itemize}
                  \end{minipage}
                }
              }
            }
            child {
              node {Symmetric Differences of Matchings
                \resizebox{\textwidth}{!}{
                  \begin{minipage}[t]{8cm}
                    \begin{itemize}
                      \item  Every component of the symmetric difference of two matchings
                        \begin{itemize}
                          \item is a path or
                          \item an even cycle
                        \end{itemize}
                    \end{itemize}
                  \end{minipage}
                }
              }
            }
          }
          child {
            node (vc) {Vertex cover
              \resizebox{\textwidth}{!}{
                \begin{minipage}[t]{8cm}
                  \begin{itemize}
                    \item a set $Q \subseteq V(G)$ that contains at least one endpoint of every edge
                    \item the vertices $Q$ cover $E(G)$
                      \begin{itemize}
                        \item $\beta(G)$ is the minimum size of vertex cover in $G$
                      \end{itemize}
                  \end{itemize}
                \end{minipage}
              }
            }
          }
        }
        child {
          node {Independence number and edge cover
            \resizebox{\textwidth}{!}{
              \begin{minipage}[t]{8cm}
                \begin{itemize}
                  \item content
                \end{itemize}
              \end{minipage}
            }
          }
          child {
            node {Independence number
              \resizebox{\textwidth}{!}{
                \begin{minipage}[t]{8cm}
                  \begin{itemize}
                    \item Vertices are \alert{independent}, if they are not connected via an edge
                      \begin{itemize}
                        \item $\alpha(G)$ is the maximum size of independent set of vertices in $G$
                      \end{itemize} 
                  \end{itemize}
                \end{minipage}
              }
            }
            child {
              node (isvc) {Other Connections
                \resizebox{\textwidth}{!}{
                  \begin{minipage}[t]{8cm}
                    \begin{itemize}
                      \item in a graph $G$ the set $S$ is an independent set \alert{iff} $\overline{S}$ is a vertex cover
                      \item $\alpha(G) + \beta(G) = n(G)$
                    \end{itemize}
                  \end{minipage}
                }
              }
            }
          }
          child {
            node (ec) {Edge cover 
              \resizebox{\textwidth}{!}{
                \begin{minipage}[t]{8cm}
                  \begin{itemize}
                    \item a set $L$ of edges such that every vertex of $G$ is incident to some edge of $L$
                      \begin{itemize}
                        \item $\beta'(G)$ is the minimum size of edge cover in $G$
                      \end{itemize}
                  \end{itemize}
                \end{minipage}
              }
            }
          }
        }
      }
      child {
        node {(Undirected) Graphs $G = (V, E, R)$
          \resizebox{\textwidth}{!}{
            \begin{minipage}[t]{10cm}
              \begin{itemize}
                \item \alert{vertex set}, e.g. $V(G) = \{v_1, v_2, \ldots\}$
                \item \alert{edge set}, e.g. $E(G) = \{e_1, e_2, \ldots\}$
                \item \alert{relation}, e.g. $R(G) = \{(e_1, \{v_1, v_2\}), \ldots\}$ that associates with each \alert{edge} a set of two vertices, called its \alert{endpoints}
                \item a \alert{loop} is an edge with equal endpoints 
                \item \alert{multiple edges} are edges with same pair of endpoints                       
                \item \underline{\href[page=37]{/home/areo/Documents/Studium/Summaries/Graph_Theory/Graphentheorie_english_all_in_one_with_go_back.pdf}{interesting properties}:}
                  \begin{itemize}
                    \item $\displaystyle \sum_{v\in V(G)} d(v) = 2\cdot e(G)$
                    \item in a graph $G$ the average vertex degree is $\displaystyle \frac{\sum_{v\in V(G)} d(v)}{\mathrm{n}(\mathrm{G})}$ and hence $\displaystyle \delta(G) \leq \frac{2 \cdot e(G)}{n(G)} \leq \Delta(G)$
                    \item every graph has an even number of vertices of odd degree
                    \item a $k$-regular graph with $n$ vertices has $\frac{n\cdot k}{2}$ edges
                    \item the maximum number of edges in a simple graph is $\dbinom{n}{2}$
                    \item the minimum number of edges in a connected graph with $n$ vertices is $n-1$
                    \item if $G$ is a simple $n$-vertex graph with $\delta(G) \ge \dfrac{n-1}{2}$ then $G$ is connected
                  \end{itemize}
              \end{itemize}
            \end{minipage}
          }
        }
        child {
          node {Trees
            \resizebox{\textwidth}{!}{
              \begin{minipage}[t]{12cm}
                \begin{itemize}
                  \item a graph with no cycle is \alert{acyclic}
                  \item a \alert{forest} is an acyclic graph
                  \item a \alert{tree} is a connected acyclic graph
                  \item a \alert{leaf} is a vertex of degree $1$
                  \item \underline{\href[page=59]{/home/areo/Documents/Studium/Summaries/Graph_Theory/Graphentheorie_english_all_in_one_with_go_back.pdf}{interesting properties}}
                    \begin{itemize}
                      \item every tree with at least two vertices has at least two leaves
                      \item deleting a leaf from an $n$-vertex tree produces a tree with $n-1$ vertices
                      \item for an $n$-vertex simple graph $G$ (with $n\ge 1$) the following are equivalent:
                        \begin{enumerate}
                          \item $G$ is connected and has no cycles
                          \item $G$ is connected and has $n-1$ edges
                          \item $G$ has $n-1$ edges and no cycles
                          \item for $u, v \in V(G)$, $G$ has exactly one $u, v$-path
                        \end{enumerate}
                      \item every edge of a tree is a cut-edge
                      \item adding one edge to a tree forms exactly one cycle
                    \end{itemize}
                \end{itemize}
              \end{minipage}
            }
          }
          child {
            node {Spanning Trees
              \resizebox{\textwidth}{!}{
                \begin{minipage}[t]{12cm}
                  \begin{itemize}
                    \item a \alert{spanning subgraph} of $G$ is a subgraph with vertex set $V(G)$
                    \item a \alert{spanning tree} is a spanning subgraph that is a tree
                    \item \underline{\href[page=69]{/home/areo/Documents/Studium/Summaries/Graph_Theory/Graphentheorie_english_all_in_one_with_go_back.pdf}{interesting properties}:}
                      \begin{itemize}
                        \item every connected graph contains a spanning tree
                        \item if $T, T'$ are spanning trees of a connected graph $G$ and $e\in E(T) - E(T')$ then there is a edge $e'\in E(T') - E(T)$ such that $T-e+e'$ is a spanning tree of $G$
                      \end{itemize}
                  \end{itemize}
                \end{minipage}
              }
            }
            child {
              node {Minimum Spanning Trees
                \resizebox{\textwidth}{!}{
                  \begin{minipage}[t]{12cm}
                    \begin{itemize}
                      \item a \alert{spanning tree} whose sum of \alert{edge weights} is as \alert{small as possible}
                      \item for minimum spanning trees we consider only \alert{non-negative edge weights}
                        % \item \alert{weighted graph}: a graph with numerical labels on the edges, so-called edge weights
                    \end{itemize}
                  \end{minipage}
                }
              }
            }
            child {
              node {Kruskal’s Algorithm
                \resizebox{\textwidth}{!}{
                  \begin{minipage}[t]{12cm}
                    \begin{itemize}
                      \item in a connected weighted graph $G$ Kruskal’s algorithm constructs a minimum-weight spanning tree
                      \item \href[page=75]{/home/areo/Documents/Studium/Summaries/Graph_Theory/Graphentheorie_english_all_in_one_with_go_back.pdf}{algorithm}
                    \end{itemize}
                  \end{minipage}
                }
              }
            }
          }
          child {
            node {Distance, Diameter, Eccentricity, Radius and Center
              \resizebox{\textwidth}{!}{
                \begin{minipage}[t]{12cm}
                  \begin{itemize}
                    \item \alert{distance} $d_G(u, v)$: the least (shortest) length of a $u, v$-path (if $G$ has a $u, v$-path)
                      \begin{itemize}
                        \item if $G$ has no such path then $d_G(u, v) = \infty$
                      \end{itemize}
                    \item \alert{diameter}: $diam(G) := max_{u, v\in V(G)} d(u, v)$
                      \begin{itemize}
                        \item for a simple graph $G$: $diam(G)\ge 3 \Rightarrow diam(\overline{G}) \le 3$
                      \end{itemize}
                    \item \alert{eccentricity}: $\epsilon(u) := max_{v\in V(G)} d(u, v)$
                      % \item \alert{radius}: $rad(G) := min_{u\in V(G)} \epsilon(u, v)$
                    \item \alert{radius}: $rad(G) := min_{u\in V(G)} \epsilon(u)$
                    \item the \alert{center} of a graph $G$ is the subgraph induced by the vertices of minimum eccentricity
                      \begin{itemize}
                        \item the \alert{center of a tree} is a vertex or an edge
                      \end{itemize}
                  \end{itemize}
                \end{minipage}
              }
            }
          }
        }
        child {
          node {Eulerian graphs
            \resizebox{\textwidth}{!}{
              \begin{minipage}[t]{12cm}
                \begin{itemize}
                  \item  a graph is Eulerian if it has a closed trail (i.e. circuit) containing all edges
                  \item  a graph $G$ is Eulerian \alert{iff}:
                    \begin{itemize}
                      \item it has at most one nontrivial component
                      \item and its vertices all have even degree
                    \end{itemize}
                \end{itemize}
              \end{minipage}
            }
          }
        }
        child {
          node {Types of graphs}
          child {
            node (simple graph) {Simple Graph
              \resizebox{\textwidth}{!}{
                \begin{minipage}[t]{8cm}
                  \begin{itemize}
                    \item \alert{has:}
                      \begin{itemize}
                        \item no loops
                        \item no multiple edges
                      \end{itemize}
                    \item one writes $e = uv$ (or $vu$) for an edge e with \alert{endpoints} $u$ and $v$
                  \end{itemize}
                \end{minipage}
              }
            }
          }
          child {
            node (regular graphs) {Regular graphs
              \resizebox{\textwidth}{!}{
                \begin{minipage}[t]{12cm}
                  \begin{itemize}
                    \item $G$ is regular if $\Delta(G) = \delta(G)$
                  \end{itemize}
                \end{minipage}
              }
            }
          }
          child {
            node {Special graphs
              \resizebox{\textwidth}{!}{
                \begin{minipage}[t]{12cm}
                  \begin{itemize}
                    \item \href[page=24]{/home/areo/Documents/Studium/Summaries/Graph_Theory/Graphentheorie_english_all_in_one_with_go_back.pdf}{list of special graphs}
                  \end{itemize}
                \end{minipage}
              }
            }
          }
          child {
            node (bipartite graphs) {Bipartite Graphs
              \resizebox{\textwidth}{!}{
                \begin{minipage}[t]{8cm}
                  \begin{itemize}
                    \item a graph $G$ is \alert{bipartite} if $V(G)$ is the union of two disjoint independent sets
                    \item \alert{Theorem of König:} A graph is bipartite \alert{iff} it has no odd cycle
                  \end{itemize}
                \end{minipage}
              }
            }
            child {
              node {k-partite Graph
                \resizebox{\textwidth}{!}{
                  \begin{minipage}[t]{8cm}
                    \begin{itemize}
                      \item a graph $G$ is \alert{k-partite} if $V(G)$ can be expressed as the union of $k$ independent sets
                    \end{itemize}
                  \end{minipage}
                }
              }
            }
          }
        }
        child {
          node {Matrix representation
            \resizebox{\textwidth}{!}{
              \begin{minipage}[t]{8cm}
                \begin{itemize}
                  \item a \alert{loopless} graph $G$
                    \begin{itemize}
                      \item $V(G) = \{v_1, \ldots, v_n\}$
                      \item $E(G) = \{e_1, \ldots e_m\}$
                    \end{itemize}
                    used as example graph for the child nodes
                \end{itemize}
                \includegraphics[width=0.3\textwidth, center]{./figures/matrix_representation.png}
              \end{minipage}
            }
          }
          child {
            node {Adjacency matrix $A(G)$
              \resizebox{\textwidth}{!}{
                \begin{minipage}[t]{8cm}
                  \begin{itemize}
                    \item a $n\times n$ matrix with entries $a_{i,j}$
                    \item where $a_{i,j}$ is the number of edges with endpoints $\{v_i, v_j\}$
                  \end{itemize}
                  \includegraphics[width=0.3\textwidth, center]{./figures/adjacency_matrix.png}
                \end{minipage}
              }
            }
            child {
              node {Adjacency
                \resizebox{\textwidth}{!}{
                  \begin{minipage}[t]{8cm}
                    \begin{itemize}
                      \item if $u$ and $v$ are endpoints of an edge they are \alert{adjacent} and \alert{neighbors}
                      \item one writes: $u\leftrightarrow v$
                    \end{itemize}
                  \end{minipage}
                }
              }
            }
          }
          child {
            node {Incidence matrix $M(G)$
              \resizebox{\textwidth}{!}{
                \begin{minipage}[t]{8cm}
                  \begin{itemize}
                    \item a $n\times m$ matrix with entries $m_{i,j}$ (that means the $m$ columns are edges)
                    \item where $m_{i,j}$ is 
                      \begin{itemize}
                        \item $1$ if $v_i$ is an endpoint of $e_j$ and
                        \item $0$ otherwise
                      \end{itemize}
                  \end{itemize}
                  \includegraphics[width=0.3\textwidth, center]{./figures/incidence_matrix.png}
                \end{minipage}
              }
            }
            child {
              node {Incidence
                \resizebox{\textwidth}{!}{
                  \begin{minipage}[t]{8cm}
                    \begin{itemize}
                      \item $v$ and $e$ are \alert{incident} if $v$ is an endpoint of $e$
                    \end{itemize}
                  \end{minipage}
                }
              }
              child {
                node (degree) {Degree $d(v)$ or $d_G(v)$
                  \resizebox{\textwidth}{!}{
                    \begin{minipage}[t]{8cm}
                      \begin{itemize}
                        \item the \alert{degree} of vertex $v$ is the number of incident edges
                          \begin{itemize}
                            \item except for loops then the edge counts twice
                          \end{itemize}
                        \item \alert{maximum degree} is $\Delta(G)$
                        \item \alert{minimum degree} is $\delta(G)$
                        \item the number of $1$'s in a row of the incidence matrix
                      \end{itemize}
                    \end{minipage}
                  }
                }
              }
            }
          }
        }
        child {
          % https://latex.org/forum/viewtopic.php?t=8650
          node {Complement $\overline{G}$
            \resizebox{\textwidth}{!}{
              \begin{minipage}[t]{8cm}
                \begin{itemize}
                  \item of a simple graph $G$ is the simple graph with:
                    \begin{itemize}
                      \item vertex set $V(G)$
                      \item edge set $uv \in E(\overline{G}) \Leftrightarrow uv\not\in E(G)$
                    \end{itemize}
                \end{itemize}
              \end{minipage}
            }
          }
        }
        child {
          node {Subgraph
            \resizebox{\textwidth}{!}{
              \begin{minipage}[t]{8cm}
                \begin{itemize}
                  \item a \alert{subgraph }of a graph $G$ is a graph $H$ such that:
                    \begin{itemize}
                      \item $V(H)\subseteq V(G)$
                      \item $E(H)\subseteq E(G)$
                      \item the assignment of endpoints to edges in $H$ is the same as in $G$
                    \end{itemize}
                  \item \href[page=29]{/home/areo/Documents/Studium/Summaries/Graph_Theory/Graphentheorie_english_all_in_one_with_go_back.pdf}{other definition for subgraph}
                \end{itemize}
              \end{minipage}
            }
          }
          child {
            node {Decomposition of a graph
              \resizebox{\textwidth}{!}{
                \begin{minipage}[t]{8cm}
                  \begin{itemize}
                    \item germ. Kantenzerlegung (Dekomposition)
                    \item a \alert{list of subgraphs} such that \alert{a edge} appears in \alert{exactly one subgraph} in the list 
                  \end{itemize}
                \end{minipage}
              }
            }
          }
          child {
            node {Clique
              \resizebox{\textwidth}{!}{
                \begin{minipage}[t]{8cm}
                  \begin{itemize}
                    \item a set of \alert{pairwise adjacent vertices}
                  \end{itemize}
                \end{minipage}
              }
            }
          }
          child {
            node (independant set) {Independant (stable) set
              \resizebox{\textwidth}{!}{
                \begin{minipage}[t]{8cm}
                  \begin{itemize}
                    \item a set of pairwise nonadjacent vertices
                  \end{itemize}
                \end{minipage}
              }
            }
          }
        }
        child {
          node {Connected / Disconnected
            \resizebox{\textwidth}{!}{
              \begin{minipage}[t]{8cm}
                \begin{itemize}
                  \item a graph $G$ is \alert{connected} if each pair of vertices in G belongs to a path
                  \item otherwise $G$ is \alert{disconnected}
                \end{itemize}
              \end{minipage}
            }
          }
          child {
            node {Components
              \resizebox{\textwidth}{!}{
                \begin{minipage}[t]{12cm}
                  \begin{itemize}
                    \item the (connected) \alert{components} of a graph $G$ are its maximal connected subgraphs
                    \item a component is \alert{trivial} if it has no edges
                    \item an \alert{isolated} vertex is a vertex of degree $0$
                    \item every graph with $n$ vertices and k edges has at least $n-k$ components
                  \end{itemize}
                \end{minipage}
              }
            }
            child {
              node {Cut-edge
                \resizebox{\textwidth}{!}{
                  \begin{minipage}[t]{12cm}
                    \begin{itemize}
                      \item an edge whose deletion increases the number of components
                        \begin{itemize}
                          \item an edge is a cut-edge if and only if it belongs to no cycle
                        \end{itemize}
                    \end{itemize}
                  \end{minipage}
                }
              }
            }
            child {
              node {Cut-vertex
                \resizebox{\textwidth}{!}{
                  \begin{minipage}[t]{12cm}
                    \begin{itemize}
                      \item a vertex whose deletion increases the number of components
                    \end{itemize}
                  \end{minipage}
                }
              }
            }
          }
        }
        child {
          node {Paths, Walks, Cycles, Circuits
            \resizebox{\textwidth}{!}{
              \begin{minipage}[t]{12cm}
                \begin{itemize}
                  \item the \alert{length} is its \alert{number of edges}
                \end{itemize}
              \end{minipage}
            }
          }
          child {
            node {Walk
              \resizebox{\textwidth}{!}{
                \begin{minipage}[t]{8cm}
                  \begin{itemize}
                    \item a list $v_0, e_1, v_1, \ldots, e_k, v_k$ of \alert{vertices} and \alert{edges}
                      \begin{itemize}
                        \item for $1 \le i \le k$ the \alert{edge} $e_i$ has \alert{endpoints} $v_{i-1}$ and $v_i$
                      \end{itemize}
                    \item a \enquote{\alert{walk $W$ contains a path $P$}} if the vertices and edges of $P$ occur as a sublist of the vertices and edges of $W$
                  \end{itemize}
                \end{minipage}
              }
            }
            child {
              node {u,v-walk
                \resizebox{\textwidth}{!}{
                  \begin{minipage}[t]{8cm}
                    \begin{itemize}
                      \item \alert a {walk} with \alert{first vertex} $u$ and \alert{last vertex} $v$
                    \end{itemize}
                  \end{minipage}
                }
              }
            }
            child {
              node {Closed walk
                \resizebox{\textwidth}{!}{
                  \begin{minipage}[t]{8cm}
                    \begin{itemize}
                      \item a walk with \alert{equal first} and \alert{last vertex}
                    \end{itemize}
                  \end{minipage}
                }
              }
            }
            child {
              node {Trail
                \resizebox{\textwidth}{!}{
                  \begin{minipage}[t]{8cm}
                    \begin{itemize}
                      \item a walk \alert{without repeating edges}
                    \end{itemize}
                  \end{minipage}
                }
              }
              child {
                node {Circuit
                  \resizebox{\textwidth}{!}{
                    \begin{minipage}[t]{12cm}
                      \begin{itemize}
                        \item a \alert{closed trail}, i.e. the \alert{first} and the \alert{last vertex} are \alert{equal}
                      \end{itemize}
                    \end{minipage}
                  }
                }
              }
            }
          }
          child {
            node {Path
              \resizebox{\textwidth}{!}{
                \begin{minipage}[t]{8cm}
                  \begin{itemize}
                    \item a \alert{simple graph} whose vertices can be ordered such that:
                      \begin{itemize}
                        \item two vertices are adjacent \alert{iff} they are consecutive in the list
                      \end{itemize}
                  \end{itemize}
                  \includegraphics[width=0.3\textwidth, center]{./figures/path.png}
                \end{minipage}
              }
            }
            child {
              node {u,v-path
                \resizebox{\textwidth}{!}{
                  \begin{minipage}[t]{8cm}
                    \begin{itemize}
                      \item a path whose vertices of \alert{degree} $1$ are $u$ and $v$ if $u\ne v$
                      \item if $u=v$ then the $u,v$-path consists only of vertex $u$ and has no edges
                    \end{itemize}
                  \end{minipage}
                }
              }
            }
            child {
              node {Cycle
                \resizebox{\textwidth}{!}{
                  \begin{minipage}[t]{8cm}
                    \begin{itemize}
                      \item a simple graph
                        \begin{enumerate}
                          \item with an \alert{equal} number of \alert{vertices} and \alert{edges}
                          \item whose \alert{vertices} can be placed around a \alert{circle} such that:
                            \begin{itemize}
                              \item two vertices are adjacent \alert{iff} they are consecutive in the circle
                            \end{itemize}
                        \end{enumerate}
                    \end{itemize}
                    \includegraphics[width=0.3\textwidth, center]{./figures/cycle.png}
                  \end{minipage}
                }
              }
            }
          }
        }
        child {
          node {Isomorphism
            \resizebox{\textwidth}{!}{
              \begin{minipage}[t]{8cm}
                \begin{itemize}
                  \item an \alert{isomorphism} from a \alert{simple graph} $G$ to a \alert{simple graph} $H$ is a \alert{bijection} $f: V(G) \rightarrow V(H)$ such that
                    \begin{itemize}
                      \item $uv\in E(G)$ \alert{iff} $f(u)f(v)\in E(H)$
                    \end{itemize}
                  \item one says \enquote{\alert{$G$ is isomorphic to $H$}} if there is an \alert{isomorphism} from $G$ to $H$ 
                  \item one writes: $G \cong H$
                \end{itemize}
              \end{minipage}
            }
          }
        }
      }
      child {
        node {Directed Graphs / Digraphs $G = (V, E, f)$
          \resizebox{\textwidth}{!}{
            \begin{minipage}[t]{14cm}
              \begin{itemize}
                \item \alert{vertex set}, e.g. $V(G) = \{v_1, v_2, \ldots\}$
                \item \alert{edge set}, e.g. $E(G) = \{e_1, e_2, \ldots\}$
                \item a \alert{function}, .e.g. $f = \{e_1 \mapsto (v_1, v_2), \ldots\}$ assigning each edge an ordered pair of vertices
                \item the first vertex of the ordered pair is called \alert{tail}, the second vertex is called \alert{head}, together they are called the \alert{endpoints}
                \item when a digraph \alert{models a relation} each ordered pair is the pair for at most one edge. Like simple graphs we then treat the pair of endpoints as the edge
                \item in a digraph a \alert{loop} is an edge with equal endpoints
                \item \alert{multiple edges} have the same ordered pair of endpoints
                \item in an edge from $u$ to $v$
                  \begin{itemize}
                    \item $v$ is the \alert{successor} of $u$
                    \item $u$ is the \alert{predecessor} of $v$
                  \end{itemize}
                \item we write $u \rightarrow v$ for an edge from $u$ to $v$
                \item \underline{\href[page=50]{/home/areo/Documents/Studium/Summaries/Graph_Theory/Graphentheorie_english_all_in_one_with_go_back.pdf}{interesting properties}:}
                  \begin{itemize}
                    \item $\displaystyle \sum_{v\in V(G)} d^+(v) = e(G) = \sum_{v\in V(G)} d^-(v)$
                    \item if $G$ is a digraph with $\delta^+(G)\ge 1$ or $\delta^-(G)\ge 1$ then $G$ contains a cycle
                  \end{itemize}
              \end{itemize}
            \end{minipage}
          }
        }
        child {
          node {Directed Trees
            \resizebox{\textwidth}{!}{
              \begin{minipage}[t]{12cm}
                \begin{itemize}
                  \item a \alert{rooted tree} $T$ is a simple digraph with a vertex $r$ chosen as root such that for each vertex $v\in V(T)$ there's a unique path \alert{$P(v)$}  from $r$ to $v$
                  \item the \alert{parent} of $v\ne r$ is the predecessor in $P(v)$
                  \item the \alert{children} of $v$ are the \alert{sucessor set} $N^+(v)$ of $v$
                  \item the \alert{ancestors} of $v$ are all vertices of $P(v) - v$
                  \item the \alert{descendants} of $v$ are all vertices $u\ne v$ such that $P(u)$ contains the vertex $v$
                  \item the \alert{leaves} are vertices with no children
                  \item a \alert{rooted plane tree} or \alert{planted tree} is a rooted tree with a given (left/right) ordering or labeling specified for all children of each vertex
                \end{itemize}
                \includegraphics[width=0.6\textwidth, center]{./figures/directed_tree.png}
              \end{minipage}
            }
          }
          child {
            node {Binary Tree
              \resizebox{\textwidth}{!}{
                \begin{minipage}[t]{12cm}
                  \begin{itemize}
                    \item a \alert{binary tree} is
                      \begin{itemize}
                        \item a rooted plane tree
                        \item where each vertex has at most two children
                        \item and each child of a vertex is designated as its
                          \begin{itemize}
                            \item \alert{left child} or
                            \item \alert{right child}
                          \end{itemize}
                      \end{itemize}
                    \item the subtrees rooted at the root are
                      \begin{itemize}
                        \item the \alert{left subtree} and
                        \item the \alert{right subtree}
                      \end{itemize}
                    \item a \alert{$k$-ary tree} allows each vertex up to $k$ children
                  \end{itemize}
                  \includegraphics[width=0.6\textwidth, center]{./figures/binary_tree.png}
                \end{minipage}
              }
            }
          }
          child {
            node {Dominator Tree
              \resizebox{\textwidth}{!}{
                \begin{minipage}[t]{12cm}
                  \begin{itemize}
                    \item rooted tree described by the root vertex $r$ and the edges described by $u \operatorname{idom} v$
                    \item for a flowgraph $G=(V,E,r)$ a vertex $v$ \alert{dominates} a vertex $w \ne v$: ($v \operatorname{dom} w$) if every path in $G$ from $r$ to $w$ contains $v$
                    \item for a flowgraph $G=(V,E,r)$ a vertex $v$ is an \alert{immediate dominator} of $w$: ($v \operatorname{idom} w$) if ($v \operatorname{dom} w$) and there is no vertex $u$ such that ($v \operatorname{dom} u$) and ($u \operatorname{dom} w$)
                    \item \href[page=93]{/home/areo/Documents/Studium/Summaries/Graph_Theory/Graphentheorie_english_all_in_one_with_go_back.pdf}{Lemma}:
                      \begin{itemize}
                        \item $(u \operatorname{dom} v) \operatorname{and} (v \operatorname{dom} w) \Rightarrow (u \operatorname{dom} w)$
                        \item $(u \operatorname{dom} w) \operatorname{and} (v \operatorname{dom} w) \Rightarrow (u \operatorname{dom} v) \operatorname{or} (v \operatorname{dom} u)$
                        \item $(u \operatorname{dom} v) \Rightarrow \neg (v \operatorname{dom} u)$
                        \item $\forall v \ne r: \exists! u: (u \operatorname{idom} v)$
                      \end{itemize}
                    \item \href[page=96]{/home/areo/Documents/Studium/Summaries/Graph_Theory/Graphentheorie_english_all_in_one_with_go_back.pdf}{algorithm}
                  \end{itemize}
                \end{minipage}
              }
            }
            child {
              node {Flowgraph
                \resizebox{\textwidth}{!}{
                  \begin{minipage}[t]{12cm}
                    \begin{itemize}
                      \item a \alert{flowgraph} $G=(V,E,r)$ is 
                        \begin{itemize}
                          \item a digraph $(V, E)$
                          \item with start vertex $r$
                          \item such that for any vertex $v \in V$
                          \item there is a path from $r$ to $v$
                        \end{itemize}
                      \item a \alert{program flowgraph} is
                        \begin{itemize}
                          \item a flowgraph with maximum outdegree $2$
                        \end{itemize}
                    \end{itemize}
                  \end{minipage}
                }
              }
            }
            child {
              node {Control Flow Graph (CFG)
                \resizebox{\textwidth}{!}{
                  \begin{minipage}[t]{12cm}
                    \begin{itemize}
                      \item \alert{Control Flow Graph (CFG):}
                        \begin{itemize}
                          \item Given a program code $P$
                          \item each reachable line represents a vertex in the flowgraph
                          \item the first executed line is $r$ of the flowgraph
                          \item edge $(u,v)$ is in the flowgraph
                            \begin{itemize}
                              \item if and only if the control can jump from line $u$ to line $v$
                            \end{itemize}
                          \item alia for determining \alert{dominators}, i.e. Code lines guaranteed to be executed before other code
                        \end{itemize}
                      \item \alert{well structured / reducible CFG:} edges are either forward edges or back edges
                        \begin{itemize}
                          \item \alert{forward edges} form an acyclic graph reached from the entry node
                          \item \alert{back edges} consist only of edges whose targets dominate their sources
                        \end{itemize}
                    \end{itemize}
                  \end{minipage}
                }
              }
            }
          }
          child {
            node {Code Tree
              \resizebox{\textwidth}{!}{
                \begin{minipage}[t]{12cm}
                  \begin{itemize}
                    \item \href[page=83]{/home/areo/Documents/Studium/Summaries/Graph_Theory/Graphentheorie_english_all_in_one_with_go_back.pdf}{definition}
                    \item example \href[page=84]{/home/areo/Documents/Studium/Summaries/Graph_Theory/Graphentheorie_english_all_in_one_with_go_back.pdf}{Huffman Coding}
                  \end{itemize}
                \end{minipage}
              }
            }
          }
        }
        child {
          node {Eulerian graphs
            \resizebox{\textwidth}{!}{
              \begin{minipage}[t]{12cm}
                \begin{itemize}
                  \item an \alert{Eulerian trail} in a digraph (or graph) is a trail containing all edges
                  \item an \alert{Eulerian circuit} is a closed trail containing all edges
                  \item a digraph is Eulerian \alert{if} it has an Eulerian circuit
                  \item a digraph is Eulerian \alert{iff} 
                    \begin{itemize}
                      \item $d^+(v) = d^-(v)$ for all nodes $v$
                      \item the underlying graph has at most one non-trivial component
                    \end{itemize}
                \end{itemize}
              \end{minipage}
            }
          }
        }
        child {
          node {Subgraphs, Isomorphism, decomposition, union
            \resizebox{\textwidth}{!}{
              \begin{minipage}[t]{12cm}
                \begin{itemize}
                  \item are like in (undirected) graphs
                \end{itemize}
              \end{minipage}
            }
          }
        }
        child {
          node {Matrix representation}
          child {
            node {Incidence matrix $M(G)$
              \resizebox{\textwidth}{!}{
                \begin{minipage}[t]{12cm}
                  \begin{itemize}
                    \item loopless digraph $G$
                    \item $m_{i,j} = +1$ if $v_i$ is the tail of $e_j$
                    \item $m_{i,j} = -1$ if $v_i$ is the head of $e_j$
                  \end{itemize}
                  \includegraphics[width=0.8\textwidth, center]{./figures/incidence_matrix_dir.png}
                \end{minipage}
              }
            }
            child {
              node {Degree
                \resizebox{\textwidth}{!}{
                  \begin{minipage}[t]{12cm}
                    \begin{itemize}
                      \item the \alert{outdegree} $d^+(v)$ is the number of edges with tail $v$
                      \item the \alert{indegree} $d^-(v)$ is the number of edges with head $v$
                      \item the \alert{out-neighborhood (successor set)} $N^+(v) := \{x \in V(G) : v \rightarrow x\}$
                      \item the \alert{in-neighborhood (predecessor set)} $N^-(v) := \{x \in V(G) : x \rightarrow v\}$
                    \end{itemize}
                  \end{minipage}
                }
              }
            }
          }
          child {
            node {Adjacency matrix $A(G)$
              \resizebox{\textwidth}{!}{
                \begin{minipage}[t]{12cm}
                  \begin{itemize}
                    \item digraph $G$
                    \item the entry in position $a_{i,j}$ is the number of edges from $v_i$ to $v_j$
                  \end{itemize}
                  \includegraphics[width=0.8\textwidth, center]{./figures/adjacence_matrix_dir.png}
                \end{minipage}
              }
            }
          }
        }
        child {
          node {Paths, Walks, Cycles, Circuits
            \resizebox{\textwidth}{!}{
              \begin{minipage}[t]{12cm}
                \begin{itemize}
                  \item the \alert{length} is its \alert{number of edges}
                  \item trail, walk and circuit are the same in (undirected) graphs and digraphs
                \end{itemize}
              \end{minipage}
            }
          }
          child {
            node {Path
              \resizebox{\textwidth}{!}{
                \begin{minipage}[t]{12cm}
                  \begin{itemize}
                    \item a \alert{simple digraph} whose vertices can be linearly ordered so that: 
                      \begin{itemize}
                        \item there is an edge with tail $u$ and head $v$ \alert{iff} $v$ immediately follows $u$
                      \end{itemize}
                  \end{itemize}
                \end{minipage}
              }
            }
            child {
              node {u,v-path 
                \resizebox{\textwidth}{!}{
                  \begin{minipage}[t]{12cm}
                    \begin{itemize}
                      \item a path whose only vertex of \alert{indegree} $0$ is $u$ and \alert{outdegree} $0$ is $v$
                    \end{itemize}
                  \end{minipage}
                }
              }
            }
            child {
              node {Cycle
                \resizebox{\textwidth}{!}{
                  \begin{minipage}[t]{12cm}
                    \begin{itemize}
                      \item a simple digraph whose vertices can be arranged on a circle such that: 
                        \begin{itemize}
                          \item edges exist \alert{iff} nodes follow each other according to the (without loss of generality) clockwise orientation
                        \end{itemize}
                    \end{itemize}
                  \end{minipage}
                }
              }
            }
          }
          % child {
          %   node {Walk
          %     \resizebox{\textwidth}{!}{
          %       \begin{minipage}[t]{12cm}
          %         \begin{itemize}
          %           \item a list $v_0, e_1, v_1, \ldots, e_k, v_k$ of vertices and edges 
          %           \begin{itemize}
          %             \item for $1 \le i \le k$ the edge $e_i$ has tail (from) $v_{i-1}$ and head (to) $v_i$
          %           \end{itemize}
          %         \end{itemize}
          %       \end{minipage}
          %     }
          %   }
          %   child {
          %     node {Closed walk
          %       \resizebox{\textwidth}{!}{
          %         \begin{minipage}[t]{12cm}
          %           \begin{itemize}
          %             \item a walk with \alert{equal first} and \alert{last vertex}
          %           \end{itemize}
          %         \end{minipage}
          %       }
          %     }
          %   }
          %   child {
          %     node {Trail
          %       \resizebox{\textwidth}{!}{
          %         \begin{minipage}[t]{12cm}
          %           \begin{itemize}
          %             \item a walk \alert{without repeating edges}
          %           \end{itemize}
          %         \end{minipage}
          %       }
          %     }
          %     child {
          %       node {Circuit
          %         \resizebox{\textwidth}{!}{
          %           \begin{minipage}[t]{12cm}
          %             \begin{itemize}
          %               \item a \alert{closed trail}, i.e. the \alert{first} and the \alert{last vertex} are \alert{equal}
          %               \item like a circuit in computer engineering
          %             \end{itemize}
          %           \end{minipage}
          %         }
          %       }
          %     }
          %   }
          %   child {
          %     node {u,v-walk
          %       \resizebox{\textwidth}{!}{
          %         \begin{minipage}[t]{12cm}
          %           \begin{itemize}
          %             \item has \alert{first vertex} $u$ and \alert{last vertex} $v$
          %             % \item \alert a {walk} with \alert{first vertex} $u$ and \alert{last vertex} $v$
          %           \end{itemize}
          %         \end{minipage}
          %       }
          %     }
          %   }
          % }
        }
        child {
          node {(Strongly) Connected
            \resizebox{\textwidth}{!}{
              \begin{minipage}[t]{12cm}
                \begin{itemize}
                  \item a digraph is \alert{weakly connected}
                    \begin{itemize}
                      \item if its underlying graph is connected
                    \end{itemize}
                  \item a digraph is \alert{strongly connected}
                    \begin{itemize}
                      \item if for each ordered pair $u$, $v$ there is a path from $u$ to $v$
                    \end{itemize}
                \end{itemize}
              \end{minipage}
            }
          }
        }
        child {
          node {Types of graphs}
          child {
            node {Underlying Graph
              \resizebox{\textwidth}{!}{
                \begin{minipage}[t]{12cm}
                  \begin{itemize}
                    \item The underlying graph of a digraph $D$ is the graph $G$
                      \begin{itemize}
                        \item with $V(G) = V(D)$
                        \item $E(G) = E(D)$
                        \item each ordered pairs of an edge become an unordered pair
                      \end{itemize}
                  \end{itemize}
                \end{minipage}
              }
            }
          }
          child {
            node {Simple Graph
              \resizebox{\textwidth}{!}{
                \begin{minipage}[t]{12cm}
                  \begin{itemize}
                    \item a digraph is simple
                      \begin{itemize}
                        \item if each ordered pair is the head and tail of at most one edge
                        \item one loop may be present at each vertex
                      \end{itemize}
                    \item in a simple digraph $uv$ denotes an edge with tail $u$ and head $v$
                  \end{itemize}
                \end{minipage}
              }
            }
          }
        }
      };
    \end{scope}
  % ┌───────────────────┐
  % │ Verbindungslinien │
  % └───────────────────┘
  \begin{pgfonlayer}{background}
    \draw [concept connection]
  %     (commoncasefast) edge (amdahl)
  %     (branchpredictionbuffer) edge (2bitpredictor)
  %     (loadusedatahazard) edge (forwarding)
      (bipartite graphs) edge (independant set)
      (vc) edge (isvc)
      (vc) edge (apa)
      (regular graphs) edge (degree)
      (gt) edge (degree)
      (ec) edge (mec);
  \end{pgfonlayer}
  % ┌──────────────┐
  % │ Annotationen │
  % └──────────────┘
  % https://tex.stackexchange.com/questions/302976/node-positioning-middle-point-mind-map-connection-bar
  \node [annotation, below] at (gt.south) {This mindmap is provided without guarantee of correctness and completeness!};
  \node [annotation, below] at (gt.north) {\href{https://www.youtube.com/playlist?list=PLmsC317bB1b0m9-ZCKGNR6iM6UTPUO__r}{Recordings} where the different topics of this mindmap get explained\\\href{/tmp/current.pdf}{go back}};
  % \path (measuringexecutiontime) -- node[annotation, above, align=center, pos=0.01] {Similiar to \textbf{Response Time:} How long it takes to do a task} (ca);
  % \path (performance) -- node[annotation, above, align=center, pos=0.01] {Similiar to \textbf{Throughput}: Total work done per time unit (e.g. tasks, transactions\ldots / per hour)} (ca);
  % \path (elapsedtime) -- node[annotation, above, align=center, pos=0.01] {Also called \textbf{Wall Clock Time} or \textbf{Real Time}} (ca);
  % \path (cputime) -- node[annotation, above, align=center, pos=0.01] {Also called \textbf{User Time}} (ca);
  % % \path (branchpredictionbuffer) -- node[annotation, below, align=center, pos=-0.06] {Also called Branch History Table} (ca);
  % \path (multicycle) -- node[annotation, above, align=center, pos=0.01] {Optimize space} (ca);
  % \path (pipelining) -- node[annotation, above, align=center, pos=0.01] {Optimize time} (ca);
  \end{tikzpicture}
\end{document}
