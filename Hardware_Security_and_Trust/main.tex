\documentclass{standalone}

%!Tex Root = ../main.tex

% ┌────────────┐
% │ Formatting │
% └────────────┘
\usepackage[english]{babel}
\usepackage[top=0cm,bottom=0cm,left=0cm,right=0cm]{geometry}
\usepackage[export]{adjustbox} % use c, l, r for images
\usepackage{csquotes}
\usepackage[parfill]{parskip}
\usepackage{fontspec}
% \usepackage{anyfontsize}
% \usepackage[]{enumitem}

% ┌──────┐
% │ Math │
% └──────┘
\usepackage{amssymb} % for black triangleright, https://tex.stackexchange.com/questions/570303/use-blacktriangleright-as-itemize-label
\usepackage{amsmath}
\usepackage{mathtools} % for \mathclap and 
\usepackage{breqn}

% ┌────────┐
% │ Tables │
% └────────┘
\usepackage{tabularray}
 % \UseTblrLibrary{diagbox}

% ┌────────┐
% │ Images │
% └────────┘
\usepackage{graphicx}
% \usepackage{float} % for the letter H
% \graphicspath{figures/}
\usepackage{subcaption}

% ┌────────┐
% │ Graphs │
% └────────┘
\usepackage{tikzit}
\usepackage{tikz}
\usetikzlibrary{backgrounds}
\usetikzlibrary{arrows}
\usetikzlibrary{shapes,shapes.geometric,shapes.misc}

% this style is applied by default to any tikzpicture included via \tikzfig
\tikzstyle{tikzfig}=[baseline=-0.25em,scale=0.5]

% these are dummy properties used by TikZiT, but ignored by LaTex
\pgfkeys{/tikz/tikzit fill/.initial=0}
\pgfkeys{/tikz/tikzit draw/.initial=0}
\pgfkeys{/tikz/tikzit shape/.initial=0}
\pgfkeys{/tikz/tikzit category/.initial=0}

% standard layers used in .tikz files
\pgfdeclarelayer{edgelayer}
\pgfdeclarelayer{nodelayer}
\pgfsetlayers{background,edgelayer,nodelayer,main}

% style for blank nodes
\tikzstyle{none}=[inner sep=0mm]

% include a .tikz file
\newcommand{\tikzfig}[1]{%
{\tikzstyle{every picture}=[tikzfig]
\IfFileExists{#1.tikz}
  {\input{#1.tikz}}
  {%
    \IfFileExists{./figures/#1.tikz}
      {\input{./figures/#1.tikz}}
      {\tikz[baseline=-0.5em]{\node[draw=red,font=\color{red},fill=red!10!white] {\textit{#1}};}}%
  }}%
}

% the same as \tikzfig, but in a {center} environment
\newcommand{\ctikzfig}[1]{%
\begin{center}\rm
  \tikzfig{#1}
\end{center}}

% fix strange self-loops, which are PGF/TikZ default
\tikzstyle{every loop}=[]


% ┌────────┐
% │ Citing │
% └────────┘
% \usepackage[style=authortitle]{biblatex}
% \addbibresource{./Graph_Theory.bib}
% \usepackage{cleveref}

% ┌──────────┐
% │ Diagrams │
% └──────────┘
% \usepackage{tikz}
% \usetikzlibrary{shadows, backgrounds} % , calc

% ┌──────────────────┐
% │ Multiple columns │
% └──────────────────┘
% \usepackage{multicol}

% ┌────────────────────┐
% │ Code hightligthing │
% └────────────────────┘
% \usepackage{minted}

% ┌────────────────────────┐
% │ Latex Programming Help │
% └────────────────────────┘
\usepackage{etoolbox}
\usepackage{xparse}
% https://tex.stackexchange.com/questions/358292/creating-a-subcounter-to-a-counter-i-created
\usepackage{chngcntr}

% ┌───────────────┐
% │ Pretty Boxes  │
% └───────────────┘
\usepackage{xcolor}
\usepackage{tcolorbox}
\tcbuselibrary{skins,theorems}

% ┌──────────────┐
% │ Pseudo Code  │
% └──────────────┘
\usepackage{pseudo}

% \usepackage{background}
\usepackage{gradient-text}
\usepackage{rotating}

% \newlength\mylen
% \setlength\mylen{\dimexpr\paperwidth/80\relax}
%
% \SetBgScale{1}
% \SetBgAngle{0}
% \SetBgColor{blue!30}
% \SetBgContents{\tikz{\draw[step=\mylen] (-.5\paperwidth,-.5\paperheight) grid (.5\paperwidth,.5\paperheight);}}

% ┌────────────┐
% │ Misc Tools │
% └────────────┘
\usepackage{lipsum}

%!Tex Root = ../main.tex

% ┌────────────┐
% │ Formatting │
% └────────────┘
% \setlength{\parskip}{0.4cm} % space between paragraphs, https://latexref.xyz/bs-par.html

% ┌───────┐
% │ Fonts │
% └───────┘
\usepackage{fontspec}
\newfontfamily\gyre{DejaVu Math TeX Gyre}
% colored bold
% \newcommand\alert[1]{\textcolor{SwitchColor}{\textbf{#1}}}
\newcommand\alert[1]{\textcolor{SwitchColor}{#1}}

% ┌──────────────┐
% │ Pseudo Code  │
% └──────────────┘
\newcounter{algorithm}
\setcounter{algorithm}{0}
\newtcbtheorem[use counter=algorithm]{algorithm}{\color{SecondaryColor}Algorithm}{pseudo/ruled}{alg}
% \newcommand{\ma}[1]{$\mathcal{#1}$}
% \renewcommand{\tt}[1]{{\small\texttt{#1}}}

% ┌────────┐
% │ Colors │
% └────────┘
\definecolor{PrimaryColor}{HTML}{800080}
\definecolor{PrimaryColorDimmed}{HTML}{D6D6F0}
\definecolor{SecondaryColor}{HTML}{006BB6}
\definecolor{SecondaryColorDimmed}{HTML}{E5F0F8}
\definecolor{SwitchColor}{named}{PrimaryColor}
\colorlet{BoxColor}{gray!10!white}

% ┌───────┐
% │ Links │
% └───────┘
\usepackage[allbordercolors=PrimaryColor, pdfborder={0 0 .2}]{hyperref}

% ┌─────────┐
% │ Mindmap │
% └─────────┘
\renewcommand{\labelitemi}{$\textcolor{SwitchColor}{\bullet}$}
\renewcommand{\labelitemii}{$\textcolor{SwitchColor}{\blacktriangleright}$}
\renewcommand{\labelitemiii}{$\textcolor{SwitchColor}{\blacksquare}$}

%!Tex Root = ../main.tex

% ┌─────────┐
% │ Mindmap │
% └─────────┘
\newlength{\leveldistance}
\setlength{\leveldistance}{25cm}

\newenvironment{edges}{\begin{pgfonlayer}{background}\draw [concept connection]}{;\end{pgfonlayer}}
\newcommand{\edge}[2]{(#1) edge (#2)}
\newcommand{\annotation}[2]{\path (#1) -- node[annotation, above, align=center, pos=0.03] {#2} (middle);}

\newenvironment{resettikz}{\pgfsetlayers{nodelayer,edgelayer}\tikzset{every node/.style={fill opacity=1.0, draw opacity=1.0, minimum size=0cm, inner sep=0pt}}}{}

\newenvironment{mindmap}{
	\begin{tikzpicture}[
			auto,
			huge mindmap,
			fill opacity=0.6,
			draw opacity=0.8,
			concept color = PrimaryColorDimmed,
			every annotation/.style={fill=BoxColor, draw=none, align=center, fill = BoxColor, text width = 2cm},
			grow cyclic,
			level 1/.append style = {
					concept color=SecondaryColorDimmed,
					level distance=\leveldistance,
					sibling angle=360/\the\tikznumberofchildren,
					% https://tex.stackexchange.com/questions/501240/trying-to-use-the-array-environment-inside-a-tikz-node-with-execute-at-begin-no
					execute at begin node=\definecolor{SwitchColor}{named}{SecondaryColor}\definecolor{SwitchColorDimmed}{named}{PrimaryColorDimmed},
				},
			level 2/.append style = {
					concept color=PrimaryColorDimmed,
					level distance=\leveldistance / 2,
					sibling angle=35,
					execute at begin node=\definecolor{SwitchColor}{named}{PrimaryColor}\definecolor{SwitchColorDimmed}{named}{SecondaryColorDimmed},
				},
			level 3/.append style = {
					concept color=SecondaryColorDimmed,
					level distance=\leveldistance / 3,
					execute at begin node=\definecolor{SwitchColor}{named}{SecondaryColor}\definecolor{SwitchColorDimmed}{named}{PrimaryColorDimmed},
				},
			level 4/.append style = {
					concept color=PrimaryColorDimmed,
					level distance=\leveldistance / 4,
					execute at begin node=\definecolor{SwitchColor}{named}{PrimaryColor}\definecolor{SwitchColorDimmed}{named}{SecondaryColorDimmed},
				},
			level 5/.append style = {
					concept color=SecondaryColorDimmed,
					level distance=\leveldistance / 5,
					execute at begin node=\definecolor{SwitchColor}{named}{SecondaryColor}\definecolor{SwitchColorDimmed}{named}{PrimaryColorDimmed},
				},
			level 6/.append style = {
					concept color=PrimaryColorDimmed,
					level distance=\leveldistance / 6,
					execute at begin node=\definecolor{SwitchColor}{named}{PrimaryColor}\definecolor{SwitchColorDimmed}{named}{SecondaryColorDimmed},
				},
			level 7/.append style = {
					concept color=SecondaryColorDimmed,
					level distance=\leveldistance / 7,
					execute at begin node=\definecolor{SwitchColor}{named}{SecondaryColor}\definecolor{SwitchColorDimmed}{named}{PrimaryColorDimmed},
				},
			level 8/.append style = {
					concept color=PrimaryColorDimmed,
					level distance=\leveldistance / 8,
					execute at begin node=\definecolor{SwitchColor}{named}{PrimaryColor}\definecolor{SwitchColorDimmed}{named}{SecondaryColorDimmed},
				},
			level 9/.append style = {
					concept color=SecondaryColorDimmed,
					level distance=\leveldistance / 9,
					execute at begin node=\definecolor{SwitchColor}{named}{SecondaryColor}\definecolor{SwitchColorDimmed}{named}{PrimaryColorDimmed},
				},
			concept connection/.append style = {
					color = BoxColor,
				},
		]
		}{
	\end{tikzpicture}
}

\newenvironment{mindmapcontent}{
	\begin{scope}[
			every node/.style = {concept, circular drop shadow}, % draw=none
			every child/.style={concept},
		]
		}{
		;\end{scope}
}

% ┌───────┐
% │ Boxes │
% └───────┘
\DeclareTotalTCBox{\inlinebox}{ s m }
{standard jigsaw,opacityback=0,colframe=SwitchColor,nobeforeafter,tcbox raise base,top=0mm,bottom=0mm,
	right=0mm,left=0mm,arc=0.1cm,boxsep=0.1cm}
{\IfBooleanTF{#1}%
	{\textcolor{PrimaryColor}{\setBold >\enspace\ignorespaces}#2}%
	{#2}}

\DeclareTotalTCBox{\inlineboxtwo}{ s m }
{standard jigsaw,opacityback=0,colframe=SwitchColorDimmed,nobeforeafter,tcbox raise base,top=0mm,bottom=0mm,
	right=0mm,left=0mm,arc=0.1cm,boxsep=0.1cm}
{\IfBooleanTF{#1}%
	{\textcolor{SwitchColorDimmed}{\setBold >\enspace\ignorespaces}#2}%
	{#2}}

% ┌──────────────────┐
% │ Case distinction │
% └──────────────────┘
% \newtoggle{absolute}
% % \toggletrue{absolute}
% \togglefalse{absolute}
% \newcommand{\lpathgraph}[1]{\iftoggle{absolute}{/home/areo/Documents/Studium/Summaries/x/}{./}#1}

% ┌───────┐
% │ Fixes │
% └───────┘
% https://tex.stackexchange.com/questions/89467/why-does-pdftex-hang-on-this-file
% \newcommand{\colon}{\mathrel{\mathop:}}

% ┌───────┐
% │ Paths │
% └───────┘
% \newcommand{\script}[2]{\href[page=#1]{}{\inlinebox{#2}}}
\newcommand{\script}[2]{\href{openpdf:/home/areo/Documents/Studium/Semester_1_Master/Hardware_Security_and_Trust/slides/Slides annotated/Hardware_Security_and_Trust_all_in_one.pdf:#1}{\inlinebox{#2}}}
\newcommand{\scripttwo}[2]{\href{openpdf:///home/areo/Documents/Studium/Semester_1_Master/Hardware_Security_and_Trust/slides/Slides annotated/bonus/12_Lecture_06Dec.pdf:#1}{\inlinebox{#2}}}
\newcommand{\videoeight}[2]{\href{https://youtu.be/YcHSlFjcndU?feature=shared&t=#1}{\inlineboxtwo{#2}}}
\newcommand{\videonine}[2]{\href{https://youtu.be/3dL-3EOIfJ8?si=l3OakqHOeCpnNayw&t=#1}{\inlineboxtwo{#2}}}
\newcommand{\videoten}[2]{\href{https://youtu.be/6oF737pa510?feature=shared&t=#1}{\inlineboxtwo{#2}}}
\newcommand{\videoeleven}[2]{\href{https://youtu.be/PJTqfzTIYJs?feature=shared&t=#1}{\inlineboxtwo{#2}}}
\newcommand{\videotwelve}[2]{\href{https://youtu.be/oDxAH7aO-Tk?feature=shared&t=#1}{\inlineboxtwo{#2}}}
\newcommand{\videothirteen}[2]{\href{https://youtu.be/3TkSXxe_Ty8?feature=shared&t=#1}{\inlineboxtwo{#2}}}
\newcommand{\videofourteen}[2]{\href{https://youtu.be/1Y3dZuJ0MHg?feature=shared&t=#1}{\inlineboxtwo{#2}}}


\begin{document}
\begin{mindmap}
	\begin{mindmapcontent}
		\node (middle) at (current page.center) {Hardware Security and Trust
			\resizebox{\textwidth}{!}{
				\begin{minipage}[t]{16cm}
					\begin{itemize}
						\item \alert{Hardware Security} goes beyond classical cryptography, does not only consider the algorithm itself, but it considers the implementation of the algorithms (e.g. AES is believed to be secure, but attacks against the implementation like the side-channel attack: power analysis are pretty successful). % It protects the implementations of cryptographic algorithms against physical attacks, side-channel attacks etc.
						% - Avoids tampering with devices
					\end{itemize}
				\end{minipage}
			}
		}
		child {
				node {Hardware Security Attacks and Countermeasures
						\resizebox{\textwidth}{!}{
							\begin{minipage}[t]{12cm}
								\begin{itemize}
									\item \script{255}{Side-channel and fault injection} and \script{359}{again}, \videothirteen{3263}{Non- or semi-invasive}, no damage, \videothirteen{3326}{Sometimes side channel for fault injection}
									\begin{itemize}
										\item \script{360}{Illustration} and \videoseventeen{3694}{Explanation}
									\end{itemize}
								\end{itemize}
							\end{minipage}
						}
					}
				child {
						node {Attacks via Test Infrastructure}
					}
				child [level distance = 35cm] {
						node (fault injection) {Fault Injection
								\resizebox{\textwidth}{!}{
									\begin{minipage}[t]{12cm}
										\begin{itemize}
											\item Fault injections are a special case of \alert{physical attacks}, \videoseventeen{3928}{Borders are fluid}
											\begin{itemize}
												\item \videoseventeen{3651}{One definition}, both side channel attacks, fault injection is side channel attack
											\end{itemize}
											\item \script{380}{Definition}
											\item \script{255}{Short Definition} and \script{362}{again}, using side-channel inputs, \videoten{2131}{Other definition}, \videoseventeen{3673}{In lecture other definition}
											\item \script{256}{Side channel inputs for fault injection} and \script{361}{again}, \videothirteen{3358}{Decrease supply voltage}, delay slower, information not arrive at register, \videothirteen{3408}{Clock frequency}, increasing frequency similiar to decreasing supply voltage, \videothirteen{3462}{Temperature}, influence speed, \videothirteen{3491}{Common goal}
											\item \script{387}{Relevance}
											\item \script{364}{Categories below}, \videoseventeen{3995}{Here not important how fault injections was done physically}, algorithms
										\end{itemize}
									\end{minipage}
								}
							}
						child {
								node {Countermeasures
										\resizebox{\textwidth}{!}{
											\begin{minipage}[t]{12cm}
												\begin{itemize}
													\item \script{389}{Definition}
												\end{itemize}
											\end{minipage}
										}
									}
								child {
										node {Probing Attempt Detector
												\resizebox{\textwidth}{!}{
													\begin{minipage}[t]{12cm}
														\begin{itemize}
															\item \script{390}{Definition (f.)}
														\end{itemize}
													\end{minipage}
												}
											}
									}
								child {
										node {Top-layer Sensor Meshes
												\resizebox{\textwidth}{!}{
													\begin{minipage}[t]{12cm}
														\begin{itemize}
															\item \script{392}{Definition} and \script{394}{Countering}
														\end{itemize}
													\end{minipage}
												}
											}
									}
								child {
										node {Error Detection / Error Correction
												\resizebox{\textwidth}{!}{
													\begin{minipage}[t]{12cm}
														\begin{itemize}
															\item \script{395}{Error Detection / Error Correction}
															\item \script{397}{Self-Checking Design}
															\item \script{398}{Error-Detecting Codes}
														\end{itemize}
													\end{minipage}
												}
											}
									}
							}
						child [level distance = 5cm] {
								node {Examples for attacks using fault injection}
								child {
										node {Differential Fault Injection Attacks
												\resizebox{\textwidth}{!}{
													\begin{minipage}[t]{12cm}
														\begin{itemize}
															\item \script{379}{Fundamental Idea}
														\end{itemize}
													\end{minipage}
												}
											}
									}
								child {
										node {Simple Fault Attack
												\resizebox{\textwidth}{!}{
													\begin{minipage}[t]{12cm}
														\begin{itemize}
															\item \script{381}{Example (ff.)}
															\begin{itemize}
																\item \script{383}{Precision}, fault model
															\end{itemize}
														\end{itemize}
													\end{minipage}
												}
											}
									}
								child {
										node {RSA
												\resizebox{\textwidth}{!}{
													\begin{minipage}[t]{12cm}
														\begin{itemize}
															\item \script{384}{Attack (ff.)}
															\begin{itemize}
																\item \script{386}{Example}
															\end{itemize}
														\end{itemize}
													\end{minipage}
												}
											}
									}
							}
						child {
								node {Different methods to inject faults}
								child {
										node {Non-Invasive Fault Injection Attacks
												\resizebox{\textwidth}{!}{
													\begin{minipage}[t]{12cm}
														\begin{itemize}
															\item Lowest cost (need oscilloscopes, waveform generators, ...)
															\item No damage to the target IC
														\end{itemize}
													\end{minipage}
												}
											}
										child {
												node {Clock Glitching
														\resizebox{\textwidth}{!}{
															\begin{minipage}[t]{12cm}
																\begin{itemize}
																	\item \script{365}{Definition} and \videoseventeen{4060}{Explanation}, \videoseventeen{4130}{Impact of increasing clock speed}, \videoseventeen{4234}{Find out longest path}, find out maximum delay, resolt not yet there, \videoseventeen{4341}{Skip instructions}, instruction fetch phase, prevent instruction from being executed because not stored in instruction register, clock glitching at exactly right moment, \videoseventeen{4388}{Change way instruction executed}, jump instruction, jump target by some computation, could avoid storing of target address and by this avoid jump, e.g. jump after password check, \videoseventeen{4489}{Need to know a lot about target architecture}, how build, if has instruction pipeline, \videoseventeen{4551}{In many cases CPU crashes}, \videoseventeen{4640}{Premature latching explained}, \videoseventeen{4694}{By several methods}, capacitive coupling
																\end{itemize}
															\end{minipage}
														}
													}
											}
										child {
												node {Voltage Glitching
														\resizebox{\textwidth}{!}{
															\begin{minipage}[t]{12cm}
																\begin{itemize}
																	\item \script{366}{Definition}
																\end{itemize}
															\end{minipage}
														}
													}
											}
									}
								child [level distance = 12cm] {
										node {Semi-Invasive Fault Injection Attacks
												\resizebox{\textwidth}{!}{
													\begin{minipage}[t]{12cm}
														\begin{itemize}
															\item Intermediate cost (need better oscilloscopes, logic analyzers, test boards, ...)
															\item Minimal physical tampering required
															\item Preceded by decapsulation of the IC ($\Rightarrow$ semi-invasive)
															\begin{itemize}
																\item Decapsulation with \alert{mechanical} or \alert{chemical} methods
															\end{itemize}
															\item \script{367}{Local Heating, Flash Glitching, Laser Glitching}
														\end{itemize}
													\end{minipage}
												}
											}
										child {
												node {Local Heating
														\resizebox{\textwidth}{!}{
															\begin{minipage}[t]{12cm}
																\begin{itemize}
																	\item \script{371}{Definition}
																\end{itemize}
															\end{minipage}
														}
													}
											}
										child {
												node {Flash Glitching
														\resizebox{\textwidth}{!}{
															\begin{minipage}[t]{12cm}
																\begin{itemize}
																	\item \script{371}{Definition}
																\end{itemize}
															\end{minipage}
														}
													}
											}
										child {
												node {Laser Glitching
														\resizebox{\textwidth}{!}{
															\begin{minipage}[t]{12cm}
																\begin{itemize}
																	\item \script{371}{Definition}
																\end{itemize}
															\end{minipage}
														}
													}
											}
										child {
												node {Backside Imaging
														\resizebox{\textwidth}{!}{
															\begin{minipage}[t]{12cm}
																\begin{itemize}
																	\item \script{369}{Definition}
																	\item Backside imaging with infrared light
																\end{itemize}
															\end{minipage}
														}
													}
											}
										child {
												node {Laser Scanning
														\resizebox{\textwidth}{!}{
															\begin{minipage}[t]{12cm}
																\begin{itemize}
																	\item \script{369}{Definition}
																\end{itemize}
															\end{minipage}
														}
													}
											}
									}
								child {
										node {Invasive Fault Injection Attacks
												\resizebox{\textwidth}{!}{
													\begin{minipage}[t]{12cm}
														\begin{itemize}
															\item High cost
															\begin{itemize}
																\item E.g. Removing layers of an IC, scanning by electron microscopes, small microprobes
															\end{itemize}
															\item Destructive
															\item Modifies attacked IC itself
														\end{itemize}
													\end{minipage}
												}
											}
										child {
												node {Signal injection
														\resizebox{\textwidth}{!}{
															\begin{minipage}[t]{12cm}
																\begin{itemize}
																	\item \script{372}{Definition}
																\end{itemize}
															\end{minipage}
														}
													}
											}
										child {
												node {Disconnect nets in the circuit by laser cutting}
											}
										child {
												node {Modify nets by wire bonding}
											}
										child {
												node {Destroy transistors / wires by high voltage between two probes}
											}
										child {
												node {Decapsulation}
											}
										child {
												node {Laser Cutting
														\resizebox{\textwidth}{!}{
															\begin{minipage}[t]{12cm}
																\begin{itemize}
																	\item \script{373}{Definition}
																\end{itemize}
															\end{minipage}
														}
													}
											}
										child {
												node {Test Point Creation
														\resizebox{\textwidth}{!}{
															\begin{minipage}[t]{12cm}
																\begin{itemize}
																	\item \script{373}{Definition}
																\end{itemize}
															\end{minipage}
														}
													}
											}
										child {
												node {Wire Bonding
														\resizebox{\textwidth}{!}{
															\begin{minipage}[t]{12cm}
																\begin{itemize}
																	\item \script{373}{Definition}
																\end{itemize}
															\end{minipage}
														}
													}
											}
										child {
												node {Reverse Engineering}
												child {
														node {Get physical access to all structures of the IC
																\resizebox{\textwidth}{!}{
																	\begin{minipage}[t]{12cm}
																		\begin{itemize}
																			\item \script{376}{Definition}
																		\end{itemize}
																	\end{minipage}
																}
															}
													}
												child {
														node {Reverse Engineering
																\resizebox{\textwidth}{!}{
																	\begin{minipage}[t]{12cm}
																		\begin{itemize}
																			\item \script{376}{Definition}
																		\end{itemize}
																	\end{minipage}
																}
															}
													}
												child {
														node {Memory Extraction
																\resizebox{\textwidth}{!}{
																	\begin{minipage}[t]{12cm}
																		\begin{itemize}
																			\item \script{376}{Definition}
																		\end{itemize}
																	\end{minipage}
																}
															}
													}
											}
										child {
												node {Side Channels by Micro Probing
														\resizebox{\textwidth}{!}{
															\begin{minipage}[t]{12cm}
																\begin{itemize}
																	\item \script{377}{Definition}
																\end{itemize}
															\end{minipage}
														}
													}
											}
									}
							}
					}
				child {
						node (side channel) {Side Channel Attacks
								\resizebox{\textwidth}{!}{
									\begin{minipage}[t]{12cm}
										\begin{itemize}
											\item Side Channel Attacks are a special case of \alert{physical attacks}
											\item \script{255}{Definition}, observing side channel outputs, \videoten{2114}{Other definition}
											\item \script{256}{Side channel outputs for side channel attacks}
										\end{itemize}
									\end{minipage}
								}
							}
						child {
								node {Side Channels}
								child {
										node {Power Consumption
												\resizebox{\textwidth}{!}{
													\begin{minipage}[t]{12cm}
														\begin{itemize}
															\item \script{258}{Definition} and \videothirteen{3672}{Explanation}, power consumption depends on next input, the more switching in cmos logic circuit, the more power consumption, derive about input applied, e.g. doesn't change, \videothirteen{3825}{Processor power cons. depends on instructions executed}
														\end{itemize}
													\end{minipage}
												}
											}
										child {
												node {Power attacks
														\resizebox{\textwidth}{!}{
															\begin{minipage}[t]{12cm}
																\begin{itemize}
																	\item \script{265}{Definition} and \videothirteen{4990}{Explanation idea}, \videothirteen{3579}{Historical order}, more clever, \videothirteen{5156}{Can't predict power completely}, unkown / uncontrolled influences, \videothirteen{5239}{Several measurements}, hypothesis correct or not, cancel out noice by statistical analysis and taking average, randomness cancels out, \videothirteen{5274}{Example}, signals with small signal to noise ratio, \videofourteen{138}{Hypothesis of the key}, subkey or one bit in key, \videofourteen{171}{Decision maybe only one bit}, then repeat attack and extract other bits
																\end{itemize}
															\end{minipage}
														}
													}
												child {
														node {Simple Power Analysis (SPA)
																\resizebox{\textwidth}{!}{
																	\begin{minipage}[t]{12cm}
																		\begin{itemize}
																			\item \script{270}{Definition}
																			\item \script{290}{Countermeasures}, \videofourteen{2759}{Do coding in a way}, not derive information from power profile or timing, \videofourteen{2787}{Create Coding lead to performance penalty}
																			\begin{itemize}
																				\item \script{289}{Operations having high SPA Risk}, \videofourteen{2590}{Could be check password}, \videofourteen{2610}{Password successful to certain degree}, \videofourteen{2636}{Fix of vulnerability}, result flag, init to true, hide even more with else case where assignment to flag as well, \videofourteen{2721}{Keeping number iterations constant}, no data dependant
																			\end{itemize}
																			\item \script{291}{Conclusions}, \videofourteen{2843}{Detect key change}
																		\end{itemize}
																	\end{minipage}
																}
															}
														child {
																node (des) {Data Encryption Standard (DES)
																		\resizebox{\textwidth}{!}{
																			\begin{minipage}[t]{18cm}
																				\begin{itemize}
																					\item \script{271}{Most important facts}
																					\item \script{272}{Basic Structure}, \videofourteen{733}{DES Structure}, \videofourteen{755}{Contrast to AES plaintext 128 bits}, \videofourteen{860}{Feistel network only place use key}, subkey round one, \videofourteen{909}{Key Schedule}, predefined selection mechanism, \videofourteen{966}{Feistel network}, \videofourteen{991}{Similar S-Boxes in AES}, \videofourteen{1014}{What is R1}, \videofourteen{1051}{Rounds same, only different keys used}
																					\item \script{273}{SPA on DES}, \videofourteen{1097}{Flow}
																					\begin{itemize}
																						\item \script{274}{Power Trace (ff.)}, \videofourteen{1166}{What does it tell us}, \videofourteen{1226}{Input and Output permutation}, \videofourteen{1274}{Round 2 and 3}, \videofourteen{1378}{Spikes different in rounds}, rotations, \videofourteen{1509}{Cycle 6 difference}, \videofourteen{1544}{Jump taken or not can depend on the key}, \videofourteen{1566}{Smart card not implemented in hardware}, \videofourteen{1576}{Don't have left rotation on registers}, \videofourteen{1586}{What doing instead}, \videofourteen{1615}{May be conditional jump}
																						\item \script{280}{Intermediary result}, \videofourteen{1709}{8 Bit instructions and registers}, \videofourteen{1770}{28 bits mask out leading 1}
																					\end{itemize}
																					\item \script{301}{Back to DES (ff.)} and \videofifteen{563}{Explanation}, \videofifteen{605}{8 bits omitted}, \videofifteen{610}{Why k16}, what one knows
																					\begin{itemize}
																						\item \script{301}{What is known} and \videofifteen{723}{Explanation}, \videofifteen{817}{If know key, then L15 is known}, use 1 bit of L15 for partitioning, \videofifteen{861}{If would depend on all key bits...}, then would have to guess complete subkey
																						\item \script{303}{Feistel network}, \videofifteen{878}{1 Bit of L15 depends only on 6 bits of the key}, Feistel netowrk, can enumerate all possible assignments, \videofifteen{988}{Fixed rule how expansion is done}, not important how done in particular, \videofifteen{1029}{8 6 bit parts}, \videofifteen{1043}{Similar S-Box from AES}, non-linear mapping of 6 bits into 4 bits, \videofifteen{1101}{What important to understand attack}, 1 output bit of feistel network depends on 6 key bits, i.e. 6 bits of round key $k_i$
																						\item \script{304}{Single output bit}, \videofifteen{1215}{How know L15j}, \videofifteen{1335}{What can use this for power attack}
																					\end{itemize}
																					\item \script{305}{DPA on DES (ff.)}, \videofifteen{1381}{Main idea}, \videofifteen{1427}{$2^6$ possible assumptions}, only $1$ is correct, 64, \alert{die $6$ bits bei DES sind von ihrer Position fest, aber nicht ihrem Wert ($0$ oder $1$), daher $2^6$}, \videofifteen{1471}{L15j that depends}, \videofifteen{1487}{Doing by simulation}
																					\begin{itemize}
																						\item \script{306}{Case distinction}, \videofifteen{1569}{If assumption not correct}, random distribution, \videofifteen{1607}{If assumption correct}, s+ larger than s-, \enquote{assumption propably choice of $k_S$}, \videofifteen{1649}{Assumption wrong in remaning 63 cases}, wrong assumption, \videofifteen{1737}{If n large enough random property cancels out noise}
																						\item \script{307}{Illustration} and \videofifteen{1752}{Explanation}
																						\item \script{308}{Order applying patterns sidenote} and \videofifteen{1968}{Explanation}, if random should be ok, best thing if alternating s+ and s-, \alert{can only observe large power consumption for transitions from 0 to 1, it has to switch to from 1 to 0 in between, else it is transition from 1 to 1}, \videofifteen{2149}{Bad choice}, \videofifteen{2210}{Since doing computation in advance}, usually one measurement and then distribution into k+ and k- based on key assumption
																						\item \script{309}{Don't know point in time when switching occurs}
																						\item \script{310}{Key Extraction Flow} and \videofifteen{2570}{Explanation}
																						\item \script{311}{Hypothesis Testing (f.)}, \videofifteen{2737}{See not completely 0}, not cancel out completely, \videofifteen{2753}{Hint most likely}, this key assumption correct one, most probably here the event on l5j will occur, at this point in time partitioning with respect to key assumption k1 was correct
																						\item \script{313}{Finding out all bits}, \videofifteen{2834}{5 (3?) other outputs depend on same 6 key bits}, all others depend on other set of 6 key bits, \videofifteen{2910}{Key deterministic}, if know 48 bits of key, then can compute? wich one in the original key they are, \videofifteen{2941}{Brut force $2^8$ remaining possibilities}, \videofifteen{2972}{Same situation with L15 and R15}
																					\end{itemize}
																				\end{itemize}
																			\end{minipage}
																		}
																	}
															}
														child {
																node {RSA
																		\resizebox{\textwidth}{!}{
																			\begin{minipage}[t]{12cm}
																				\begin{itemize}
																					\item \script{281}{Definition}, \videofourteen{1899}{Only algorithm secure not implementation}, may be insecure, \videofourteen{2001}{Square and Multiply complexity}, linear in number of bits in e and d, \videofourteen{2025}{Don't attack public key}, because it's public
																					\item \script{282}{Square and Multiply}
																					\item \script{285}{SPA on Square and Multiply}
																					\begin{itemize}
																						\item \script{283}{Power Trace (f.)}, \videofourteen{2210}{Derive exponent from power trace}, derive private key, \videofourteen{2331}{Problem data dependant jump}
																					\end{itemize}
																					\item \script{286}{Countermeasures (ff.)}, \videofourteen{2390}{Dummy multiplication in every case}, \videofourteen{2481}{Bad to make jumps based on data would like to protect}
																				\end{itemize}
																			\end{minipage}
																		}
																	}
															}
													}
												child {
														node (dpa) {Differential Power Analysis (DPA)
																\resizebox{\textwidth}{!}{
																	\begin{minipage}[t]{18cm}
																		\begin{itemize}
																			\item \script{293}{Definition with Traces (ff.)}, \videofourteen{3006}{Data-dependence}, \videofourteen{3059}{Cancel out noise by looking at average of many power profiles}, everything not correlated somehow cancels out, \videofourteen{3174}{Addition several power profiles}, not so easy to say then there's e.g. 2nd round of left rotation, \videofourteen{3239}{Why DPA helps here}, \videofourteen{3303}{Contribution second algorithm goes away}, noise cancels out, average several runs, \videofourteen{3347}{Another thing}, maybe 2nd process cancels out in difference of average run with key a and average run of key b, because it possible does same for key a and key b, \videofourteen{3404}{Averaging and comparing 2 averages to each other}, \videofourteen{3610}{Summary DPA}, form groups, in groups compute averages to cancel out noise and difference show rly small differences in those averages, \videofifteen{71}{If operands depend on the key}, \videosixteen{53}{Another summary}
																			\begin{itemize}
																				\item \script{294}{Power Traces} and \videofourteen{3469}{Explanation}, \videofourteen{3550}{Computing difference make small differences visible}, can zoom into difference, \videofourteen{3568}{Precondition to be applicable to cryptanalysis}, form meaningful groups, extract information from difference of those groups
																			\end{itemize}
																			\item \script{299}{Basic Principle (f.)} and \videofourteen{4643}{Explanation}, \videofourteen{4700}{From sequence of input patterns get sequence of power values}, \videofourteen{4741}{Don't talk about traces}, talk about single power values at first, for each pattern have a value, \videofourteen{4767}{Model of circuit}, \videofourteen{4863}{Target gate is 0}, contribution to overall power consumption small, \videofourteen{5333}{Main idea}, \videofifteen{282}{Select one output bit}, \videofifteen{298}{Choosing target gate}, property if know key and if know input (plain text or cipher text) then one knows the value at the output of this gate, \videofifteen{4507}{Idea of DPA}, to cancel out single affects by averaging
																			\begin{itemize}
																				\item \script{300}{2nd part}, \videofourteen{4951}{What plan to do}, average power in s+ larger than average power in s-, if contribution of all other gates somehow cancels out, noise canceling, \videofourteen{5142}{Compute difference}, \videofourteen{5252}{In DPA attack hypothesis depends on key}, if assumption on key was correct, this partitioning was correct and one can see a difference. If hypothesis was not correct, then yellow contribution are distributed more or less equally between s+ and s- and therefore the difference is 0, \videofifteen{410}{Average over pattern for left side larger}, then for the right side with high probability, because include tiny additional component of power consumption due to this special target gate
																			\end{itemize}
																			\item \script{322}{Countermeasures}, \videofifteen{4414}{Signal to noise ratio reducing}, signals are more looking like noise, \videofifteen{4438}{Effect of low power designs}, the more successful are in reducing power consumption and reducing peaks, the more difficult DPA will be, \videofifteen{4477}{Do something in parallel}, \videofifteen{4534}{Leak masking}, if using non-random contributions to power consumption, then could be possible to hide the key dependant power consumption, \videofifteen{4567}{Doesn't necessarily work}, same as multicores, but even multicores can be attacked with DPA if have enough patterns to apply, \videofifteen{4597}{Countermeasures always can avoided by more samples}, have this property, \videofifteen{4617}{Temporal noise explained}, if able to shift measurements against each other then can avoid sharp peak in differential power trace, \videofifteen{4695}{Random generator that influences clock speed}, \videofifteen{4747}{Balance power consumption}, more expensive than Cmos, varying power consumption 0 to 1 (bigger) and 1 to 0 (smaller), \videofifteen{4812}{Instead of 1 transistor? always have 2}, doubling area of implementation, \videofifteen{4866}{Power supply filters}, impossible measure real power consumption, if measure plug in the wall have capacitors that cancel out peaks, power supply filters are capacitors in simplest case consuming / cancel out peaks
																			% , \videofifteen{4727}{Temporal noise pretty often used}
																			% , \videofifteen{5072}{AES secure against DPA?}
																		\end{itemize}
																	\end{minipage}
																}
															}
														child {
																node {Dynamic CMOS Power Consumption
																		\resizebox{\textwidth}{!}{
																			\begin{minipage}[t]{12cm}
																				\begin{itemize}
																					\item \script{296}{Definition (f.)}, \videofourteen{3663}{CMOS Inverter}, \videofourteen{3734}{Not considering static behaviour}, measure what happens if change from 0 to 1 or 1 to 0, \videofourteen{3948}{What happens measuring current at this point}, high current loading output to 1 for 1 to 0, stops when arrive at 1 at this capacitor, \videofourteen{4007}{Transition 0 to 1, output 1 to 0}, 2nd case, capacitor C is discharged to ground, \videofourteen{4050}{Little peak in power profile}, changing transistors from conductive to non-conductive or vice versa there's point in time where both are conducting a little bit, then have little bit of current from vdd to ground, one is transistor open and the other is not closed perfectly, \videofourteen{4161}{Reason is that measuring here}, with oscilloscope, else other way round, then switching current when go from 1 to 0, \videofourteen{4188}{Asymmetry comes from the point where on is measuring}, \videofifteen{203}{VDD}, this part not conductive, almost no current, \videofourteen{4204}{Always measure at VDD connection}, \videofifteen{248}{Always assuming measurements at VDD 2}
																					\item \script{298}{Example: Xor}, \videofourteen{4320}{Don't possible to observe real output of XOR gate}, only derive from power consumption
																				\end{itemize}
																			\end{minipage}
																		}
																	}
															}
														child {
																node {AES
																		\resizebox{\textwidth}{!}{
																			\begin{minipage}[t]{12cm}
																				\begin{itemize}
																					\item \script{314}{DPA on AES (ff.)}, \videofifteen{3103}{What one needs}, one bit distinguisher, depend on restricted number of key bits / cases, not possible if number of candidate bits too large, \videofifteen{3237}{Last attack was only cipher text}, plain text played no role
																					\begin{itemize}
																						\item \script{316}{Round Structure}, \videofifteen{3346}{Not in the middle}, not able to compute this bit, \videofifteen{3411}{Simplest possible locations}, key addition layer, \videofifteen{3439}{Select one bit}, output only depends on one key bit and one plain text bit, \videofifteen{3509}{Pretty similar way also do it at the end}, know cipher text and one bit of key n, then know bit, \videofifteen{3556}{Other possibility assumption on 8 bits}, one output of a S-Box, Byte Substitution Layer, \videofifteen{3658}{Mapping known and be be inverted}, \videofifteen{3663}{First 8 Bits of the key}, one bit at output
																						\item \script{317}{Traces (ff.)} and \videofifteen{3787}{Traces explained}, \videofifteen{3719}{Is arbitrary whether to choose LSB and first S-box}, one output of S-Box only depends on $a_0$ and $a_0$ by key addition only depends on first byte of $k_0$, in this special case $k_0$ is the same as $k$ if one has 128 bit AES, \videofifteen{3772}{Same game}, only number possible key assumptions is not $64$ but $2^8=256$, \videofifteen{3867}{Most probably this is the correct key}, usually (differential)? trace with largest peaks is the right one, \videofifteen{3922}{Argumentation behind waving}, at all other places everything random, there are also other correlations with the key assumption
																						\item \script{320}{Work with every \alert{single bit distinguisher}} and \videofifteen{3961}{Explanation}, does every single bit distinguisher work, \videofifteen{4080}{Example not every distinguisher working}, effects cancel out, \videofifteen{4190}{Don't make use of}, don't know exact implementation, hope something is correlated with the key bit assumption
																						\item \script{321}{Using S-Box} and \videofifteen{4293}{Explanation}, \videofifteen{4314}{How obtain remaining key bytes}, choosing another S-Box, \videofifteen{4329}{No Brootforce needed}
																					\end{itemize}
																				\end{itemize}
																			\end{minipage}
																		}
																	}
															}
													}
												child {
														node {Correlation Power Analysis
																\resizebox{\textwidth}{!}{
																	\begin{minipage}[t]{12cm}
																		\begin{itemize}
																			\item \script{324}{Definition (ff.)}, idea \videofifteen{5027}{Needing less samples}, \videosixteen{434}{Processed values maybe several bits}, \videosixteen{467}{Example: Hamming weight explained}, \videosixteen{524}{Hamming weight high means high power consumption}, based on observation that switchings to 1 correspond to high power consumption, \videosixteen{553}{Generalization of DPA}
																			\begin{itemize}
																				\item \script{325}{Measure Correlation}, \videosixteen{742}{If key byte assumption wrong}, possibly not any point in time where correlation is large, \videosixteen{752}{If key byte assumption is correct}, if took right time interval one will find the point in time where there's a large correlation
																				\item \script{326}{Pearson‘s Correlation Coefficient}, \videosixteen{864}{Correlation coefficient}, \videosixteen{905}{Doesn't depend on $X$ and $Y$}, \videosixteen{989}{What is covariance}, \videosixteen{1145}{If would consider $X$ instead of $Y$}, if $y_i$ equal to $x_i$, \videosixteen{1188}{Extreme case}, series $Y$ equal to series $X$, \videosixteen{1229}{Summary}, \videosixteen{1257}{Problem with covariation}, $X$ and $3\cdot X$, covariance would increase by factor $3$, \videosixteen{1350}{Special cases}, \videosixteen{1363}{All $y_i$ equal to $x_i$}, \videosixteen{1416}{Also for X and $3\cdot X$}, \videosixteen{1892}{Ommited 1/n}
																				\item \script{327}{Procedure}, \videosixteen{1624}{In practise not for each plain text single value}, but trace over points in time, \videosixteen{1645}{p points in time}, \videosixteen{1680}{Example subkey guesses}, \videosixteen{1722}{Exactly same as for DPA}, \videosixteen{1779}{Fix point in time and fix key guess}, \videosixteen{1820}{Obtain $n$ values}, \videosixteen{1841}{$n$ values different plain texts with fixed key guess}, \videosixteen{1924}{Sum up over power traces}, fixed point in time and fixed key guess, \videosixteen{2043}{$i$ gives key guess}, output S-Box depends on key guess, \videosixteen{2124}{$p\times m$ correlation coefficients}, $p$ points in time, $m$ subkey guesses, \videosixteen{2143}{How one continues}, \videosixteen{2219}{Question why not actual S-Box outputs}, previous value not guaranteed to be $0$, good enough to differentiate between wrong and right key guesses, \videosixteen{2465}{Question}, what is subkey, \videosixteen{2607}{Hope effects cancel out}, power consumption also influenced by other S-Box outputs, slighly larger correlation with right key assumption compared to wrong key assumptions, \videosixteen{2805}{Sure that highest correlation corresponds to correct key assumption}, not guaranteed, assumptions remainder of circuit, affect that invalidated correlation between power values and hamming weight of output, 3 best correlations, brute force analysis, additional effects in whole circuit
																			\end{itemize}
																		\end{itemize}
																	\end{minipage}
																}
															}
													}
											}
										child {
												node {Power Consumption of a Circuit
														\resizebox{\textwidth}{!}{
															\begin{minipage}[t]{12cm}
																\begin{itemize}
																	\item \script{266}{Definition}, \videofourteen{208}{Cmos}, static power low, if no switching one of the connections not conductive, in reality small static power flowing from vdd to ground also if no switching, transistors a little bit conductive, static power not data dependant, \videofourteen{294}{Dynamic power information about switchings}, in circuit, \videofourteen{310}{Why switching consumes power}, charge capacitors
																\end{itemize}
															\end{minipage}
														}
													}
											}
										child {
												node {Measuring Phase
														\resizebox{\textwidth}{!}{
															\begin{minipage}[t]{12cm}
																\begin{itemize}
																	\item \script{267}{Definition}, \videofourteen{386}{Measure with Oscilloscope}, \videofourteen{422}{Also important where on measures}, goal as close to attacked device as possible
																\end{itemize}
															\end{minipage}
														}
													}
											}
									}
								child {
										node {EM Side Channels
												\resizebox{\textwidth}{!}{
													\begin{minipage}[t]{12cm}
														\begin{itemize}
															\item \script{259}{Definition (n.l.)}, \videothirteen{3906}{Inductive coupling}, electric / magnetic field, \videothirteen{3933}{Capacitive coupling}, charge, \videothirteen{3955}{Modulated signals}, \videothirteen{3986}{Similar to analyzing power consumption}, measuring current also measure power consumption
															\begin{itemize}
																\item \script{335}{Probe design}, localized readings, \videosixteen{3750}{Electromagnetic field}, \videosixteen{3792}{Advantage}, localized readings, power consumption is just averaging about all possible locations on the chip
																\item \script{336}{Signal}, \videosixteen{3898}{Changes of current correspond to derivation of current}, if have EM curve do integration to get current curve, \videosixteen{3935}{Can basically do same as in power analysis}, if voltage is fixed, then current corresponds to power, \videosixteen{3966}{Localised power analysis}
																\item \script{337}{Spatial Positioning} and \script{338}{Spectral density}, \videosixteen{4027}{Find out something about layout of chip}, \videosixteen{4090}{Resolution depends size of coil}, \videosixteen{4149}{Decapsulated of the chip}, remove shielding effects, remove package, attack on chip surface
																\item \script{339}{Features, Drawbacks, Countermeasures}
															\end{itemize}
														\end{itemize}
													\end{minipage}
												}
											}
									}
								child {
										node {Acoustic Side Channels
												\resizebox{\textwidth}{!}{
													\begin{minipage}[t]{12cm}
														\begin{itemize}
															\item \script{260}{Definition and Examples (n.l.)}, \videothirteen{3640}{Extract keys from ciphers}, \videothirteen{4068}{Each key different sound}, \videothirteen{4247}{Cooler}, not directly connected with power consumption, load, \videothirteen{4286}{Sound cannot hear}, \videothirteen{4349}{Called coil whining}
															\begin{itemize}
																\item \script{342}{Measurement and \enquote{coil whining}}, \videosixteen{4409}{Resolution not rly high}, even tough fan connected to power consumption, not correlated on fine granular basis
																\item \script{343}{Microphone Readout (f.)}, \videosixteen{4480}{Picture from paper explained}, \videosixteen{4587}{Different instructions make different noises}, \videosixteen{4654}{Much is in spectrum one can not hear}
															\end{itemize}
															\item \script{356}{Countermeasures}, \videoseventeen{2760}{Acoustic shielding}, not always possible remove all sounds on spectrum one can not hear, \videoseventeen{2831}{If random can counteract with many measurements}, to cancel out noise, \videoseventeen{2936}{Cipher text randomization}, famous countermeasure, \videoseventeen{3003}{Not only give $y$ to encryption algorithm}, \videoseventeen{3017}{Modulo $n$ always not mentioned}, \videoseventeen{3147}{Text $r$ not completely random}, must be invertible, so not completely random, \videoseventeen{3191}{So should be relatively prime with $n$}, then it's also invertible, not have $p$ or $q$ as factor, then it's enough
														\end{itemize}
													\end{minipage}
												}
											}
										child {
												node {RSA
														\resizebox{\textwidth}{!}{
															\begin{minipage}[t]{18cm}
																\begin{itemize}
																	\item \script{341}{Extract RSA Key (from GnuPG)}
																	\begin{itemize}
																		\item \script{342}{Acoustic Side Channel and MicrophoneHuman Readout (ff.)}
																		\item \script{345}{Attacking RSA (f.)}, \videosixteen{4705}{RSA operations executed on usual processor}, \videosixteen{4722}{Multiplication with large numbers}, map multiplications into software multiplications, \videosixteen{4793}{Attacking signing and decryption operations}, operations using private keys
																		\begin{itemize}
																			\item \script{346}{Chinese Remainder Theorem (CRT)}, \videosixteen{4980}{Advantage don't have to make exponentiation with $d$}, \videosixteen{5930}{Need to know this trick}, without this trick not possible
																			\item \script{347}{Cost estimation} and \videosixteen{5239}{Explanation}, \videosixteen{5691}{Not save any exponentiations or multiplications}, \videosixteen{5735}{First sight}, \videosixteen{5780}{Reason}, modulo p and q half of bits in n, same number of operations but on shorter numbers, \videosixteen{5828}{Multiplication has to be done in software}, map multiplication of large numbers by software to several multiplications of small numbers, \videosixteen{5856}{Software libraries runtime}, multiplication of numbers with $n$ bits $n^2$ or best $n^log(3)$, \videoseventeen{252}{Real numbers}, 4096, around 2000
																		\end{itemize}
																		\item \script{348}{Acoustic Side Channel (f.)} and \videoseventeen{344}{Explanation}, switching from this to this operation, \videoseventeen{419}{See difference if change key}, yes, \videoseventeen{430}{Signing or decryption not relevant}
																		\item \script{350}{Key Extraction (ff.)}, \videoseventeen{556}{Chosen-ciphertext attack}, choose ciphertext in intelligent way to get information leaked about key, corresponding plain text could be complete nonsense, \videoseventeen{608}{Adaptive}, choose different cipher texts in rounds, what choose in next round depends on what found out in previous round, \videoseventeen{633}{What one needs}, access to decryption mechanism, able to force victim to decrypt chosen ciphertexts, \videoseventeen{771}{Only second part, don't care about first part}, could also attack $p$ and not $q$, just decided this way, \videoseventeen{818}{If know $q$ one has broken the decryption}, then one can compute $p$, \videoseventeen{856}{Factorization problem}, also express as if find out $q$..., 2: \videoseventeen{985}{Choosen ciphertext attack in rounds}, in each round $1$ additional bit of $q$, starting with most significant bits, \videoseventeen{1027}{Before the first round one msb is already known}, \videoseventeen{1048}{Chosen cipher text attack}, $y_i$ ciphertext used in round $i$, part of $q$ one already knows, then $0$ and a sequence of $1$es, \videoseventeen{1087}{By this choice}, depending on value of next bit, the input to the exponentiation algorithm will differ pretty much, 3: \videoseventeen{1131}{Case distinction explained}, \videoseventeen{1238}{Back to exponentiation operation}, \videoseventeen{1275}{Because $y_i$ strictly smaller than $q$}, the modulo operation doesn't change the $y_i$, \videoseventeen{1348}{2nd case}, \videoseventeen{1390}{Prefix of $q$ and $y_i$ is the same}, \videoseventeen{1446}{Modulo operation changes something}, \videoseventeen{1506}{Substract $q$}, \videoseventeen{1763}{Use assumption of case distinction}, \videoseventeen{1948}{Why that has to be the result}, \videoseventeen{1968}{$Q$ has leading bit $1$}, \videoseventeen{1990}{In both cases calculated number}, that goes into square and multiply algorithm with exponent..., \videoseventeen{2080}{Assume $q$ looking random}, \videoseventeen{2117}{Hope sound different for both cases}, 5: \videoseventeen{2223}{Authors paper argue there's acoustic difference}, \videoseventeen{2283}{$p$ part no difference}, for $q$ part see difference, \videoseventeen{2314}{Should be able to detect whether it is $0$ or $1$}, analyzing many sound profile to know that looks like $0$ or $1$, remarks: \videoseventeen{2375}{Remark: Could also do Timing attack}, similar way like acoustic attacks, \videoseventeen{2406}{Similar timing attack explained}, Software based multiplication, multiplication with smaller numbers should need less time, \videoseventeen{2446}{Advantage acoustic over timing}, timing close to internals of CPU, operations in software library, \videoseventeen{2506}{Question realistic}, 2048 emails, next email depends on result of analysis, send on after the other, someone would notice that he's attacked, \videoseventeen{2636}{Nonsense text}, plain text corresponding to cipher text is nonsense, \videoseventeen{2665}{Must somehow use decryption engine of victim}, cause victim to do decryptions in some way, \videoseventeen{2695}{Another disadvantage}, cannot put everything into one ciphertext, the weakness of this attack, \videoseventeen{3490}{Question: How know which is correct bit}, don't know, assume zero or one have certain feature in profile
																		\begin{itemize}
																			\item \script{355}{Work if CRT-trick not used?} and \videoseventeen{3231}{Explanation}, \videoseventeen{3260}{Assume library uses CRT trick}, otherwise don't have different computations for $q$ in $2$ different residue class rings, \videoseventeen{3329}{Instead would do this computation}
																		\end{itemize}
																	\end{itemize}
																\end{itemize}
															\end{minipage}
														}
													}
											}
									}
								child {
										node {Optical Side Channels
												\resizebox{\textwidth}{!}{
													\begin{minipage}[t]{12cm}
														\begin{itemize}
															\item \script{261}{Definition}
														\end{itemize}
													\end{minipage}
												}
											}
									}
								child {
										node {Timing and Delay
												\resizebox{\textwidth}{!}{
													\begin{minipage}[t]{12cm}
														\begin{itemize}
															\item \script{262}{Definition} and \videothirteen{4533}{Explanation}
														\end{itemize}
													\end{minipage}
												}
											}
										child {
												node {Timing attacks
														\resizebox{\textwidth}{!}{
															\begin{minipage}[t]{12cm}
																\begin{itemize}
																	\item \script{329}{Definition}
																\end{itemize}
															\end{minipage}
														}
													}
												child {
														node {Square and Multiply
																\resizebox{\textwidth}{!}{
																	\begin{minipage}[t]{12cm}
																		\begin{itemize}
																			\item \script{330}{Timing Attack on Square and Multiply}, \videosixteen{3217}{For decryption $k$ is equal to right key}
																			\begin{itemize}
																				\item \script{331}{Relationship between Power and Timing}, \videosixteen{3288}{$0$ just squaring}, \videosixteen{3294}{If have $1$, squaring and multiplication}, squaring and multiplication different shape, \videosixteen{3387}{Influence on length of power curve}, switching one $0$ to $1$, \videosixteen{3419}{Can find out number of 1es in key}
																				\item \script{332}{Observation, Better Timing Attacks on RSA}, \videosixteen{3606}{Turn acoustic attack into timing attack}, \videosixteen{3629}{Summary timing attack}
																			\end{itemize}
																			\item \script{333}{Countermeasures} and \videosixteen{3676}{Explanation}, modify code need same amount of time independant from $0$ or $1$
																		\end{itemize}
																	\end{minipage}
																}
															}
													}
											}
									}
							}
						child {
								node {Hardware Targets
										\resizebox{\textwidth}{!}{
											\begin{minipage}[t]{12cm}
												\begin{itemize}
													\item \script{263}{Definition} and \videothirteen{4657}{Explanation}, \videothirteen{4675}{8-Bit width}, \videothirteen{4742}{Difficulty of attacks different}, fpga hardware parallelism, the more complicated device, the more difficult is attack
												\end{itemize}
											\end{minipage}
										}
									}
								child {
										node {Smart Cards
												\resizebox{\textwidth}{!}{
													\begin{minipage}[t]{12cm}
														\begin{itemize}
															\item \script{264}{Definition}, \videothirteen{4825}{Power supply outside}, can manipulate power supply, \videothirteen{4864}{No clock}, card reading machine clock outside privided, fault injection attacks, \videothirteen{4903}{Simple processor}, no multicore, only observe encryption algorithm, no multicore, only observe encryption algorithm, only thing running, \videothirteen{5059}{Another advantage smart card}, measure directly at source, microprocessor capacitors make signal smother, the closer to device the better measurements are
														\end{itemize}
													\end{minipage}
												}
											}
									}
							}
					}
				child {
						node {Attacks by Microarchitectural Data Sampling (MDS)
								\resizebox{\textwidth}{!}{
									\begin{minipage}[t]{18cm}
										\begin{itemize}
											\item \alert{Instruction Set Architecture} (ISA) of a processor = interface it provides to the software it executes. It specifies:
											\begin{itemize}
												\item Set of instructions
												\item Registers and memory processed by the instructions
											\end{itemize}
											\item \alert{Architectural states} = states defined by the ISA, i.e., memory cells and user-visible registers, \videoten{2501}{Architectural states visible to the programmer}, \videoten{2618}{Example for no architectural state}
											\item \alert{Microarchitectural states} = processor states not defined in the ISA, \videoten{2935}{Microarchitectural states occurence} due to certain implementation of ISA, e.g.:
											\begin{itemize}
												\item \videoten{2669}{In general}
												\item \videoten{2707}{States of Functional Units}
												\item \videoten{2753}{States of reservation stations}, internal datastructure of out of order processors
												\item \videoten{2781}{Cache contents}
												\item etc.
											\end{itemize}
										\end{itemize}
									\end{minipage}
								}
							}
						child {
								node {Meltdown and Spectre
										\resizebox{\textwidth}{!}{
											\begin{minipage}[t]{16cm}
												\begin{itemize}
													\item \script{202}{Main idea}, exploited vulnerabilties, \videoten{2892}{In other words}, caches, \videoten{3263}{Fulfill correct semantics}, result / effect architectural level / state never incorrect, result has to be same as in sequential execution, invariant, \videoten{3347}{Effects of invalid out of order and discarded speculative executions}, change microarchitectural states, \videoten{3419}{In other words 2}
													\item \script{252}{Summary}, \videothirteen{2987}{Hardware designers not think about microarchitectural states}, thought noone can read
												\end{itemize}
											\end{minipage}
										}
									}
								child {
										node {Meltdown
												\resizebox{\textwidth}{!}{
													\begin{minipage}[t]{18cm}
														\begin{itemize}
															\item based on out-of-order executions (superscalar processor), \script{202}{Note}
															\item \script{203}{Overview}
															\item \script{229}{What it uses}
															\item \underline{Implementation details}, \videoten{3537}{First step operating system}
															\begin{itemize}
																\item \script{231}{Step 1: Unauthorized read, probe\_array and cache data (f.)}
																\begin{itemize}
																	\item \script{230}{Precondition and Code}, \videoeleven{5166}{If memory managment works as it should...}, \videoeleven{5195}{Problem is...}, forwarding will put data into load reservation station before being finished with check for unprivliged access, \videoeleven{5292}{Then be fast}, read access, in cache access to secret address, \videoeleven{5321}{Probe array property}, all addresses lead to cache misses besides one address that has a future hit, \videoeleven{5373}{Other detail has to be solved}, \videotwelve{258}{Read after Write situation}, often forwarding to reservation station that does the load from the probe array at this address in IN1, \videotwelve{322}{If forwarding would not happen attack impossible}, \videotwelve{357}{Idea is use secret data to do this read access}, before it came clear that it was not allowed to read from this address, \videotwelve{2479}{Question for understanding}, not possible to read secret data in other way, cannot say reach this cache line, secret data used as address for doing access to probe array, therefore one does not evaluate cache itself by it's contents, but by the fact where one has fast access / cache hit, \alert{probe array has $256$ entries, because one reads one \alert{byte} from the kernel which are $2^8=256$ and the probe array entry that can be read fast is going to correspond to the byte that has been read, a byte can only be between $0$ and $256$}
																	\item \videotwelve{422}{$R_s$ and $R_p$ are load reservation stations}, \videotwelve{469}{First and second reservation stations wait for}, \videotwelve{542}{How in Tomasulo Algorithm}, \videotwelve{564}{Forward to $R_p$ no problem}, no architectural state, internal state, \videotwelve{585}{Assume doing read}, \videotwelve{675}{Not important that sd written into cache}, important that cache is accessed with address sd, \videotwelve{820}{To make this effect visible}
																\end{itemize}
																\item \script{233}{Step 2: Flushing the cache lines, Measure access times}, \videotwelve{895}{Not possible to read data from cache directly}, \videotwelve{984}{Size of probe array}, \videotwelve{1159}{Access first address}, \videotwelve{1174}{Fast access at $i$}, \videotwelve{1215}{Why left shift $12$ positions}, \videotwelve{1254}{For seperating cache lines}, with block of size $2^12$, with accesing $0$ would also fill cache at position $1$, \videotwelve{1309}{Cache lines don't have $2^{12}$}, \videotwelve{1393}{Can this work?}
																\item \script{235}{Exception handling (pr.)}, \videotwelve{1548}{What fork does}, \videotwelve{1579}{How can execution in parent and child process be different}, \videotwelve{1697}{Method with return value}, \videotwelve{1738}{Reason why this works}, cache not private to some process, but is a global structure in the whole processor, \videotwelve{1787}{Need rights to install own signal handler}, \videotwelve{1842}{TSX group instructions}
																\item \script{236}{Race condition}, \videotwelve{2111}{Can ensure probe\_array starts at beginning of memory page}, \videotwelve{2160}{What effect of this}, address translations with respect to probe array fast, because already have in TLB, \videotwelve{2193}{Reason for 12}, page size not cache line size, \videotwelve{2324}{By access to blue bytes}, addresses stored in TLB, \videotwelve{2351}{TLB summary} accelerated translation of virtual to real addresses or short: Acceleration of address translation, only flushed when doing context switch between 2 processes, \videotwelve{2399}{One needs accelerated access to probe\_array}, \videotwelve{2444}{Secret data available earlier}, secret kernel address in cache, so provided faster and thus forwarded faster
																\begin{itemize}
																	\item \script{237}{Steps in the race (f.)} and \videotwelve{2670}{Explanation}, \videotwelve{2941}{Get cache miss because}, flushed cache before, \videotwelve{2965}{But not important what one reads}, it's important that one is reading from $b$, \videotwelve{3038}{What doesn't help}, bringing address a into TLB would accelerate both, but one only wants to accelerate sequence $2$
																\end{itemize}
															\end{itemize}
															\item \script{238}{Leak whole memory}, \videotwelve{3165}{What helps to leak whole memory}, reference to page in physical memory coming from kernel, \videotwelve{3190}{Add second reference to physical page}, which is from user process page table, if allocate memory by user process, \videotwelve{3207}{Kernel page table is simple}, \videotwelve{3278}{Access all possible physical page frames}, via kernel, read complete physical memory if can read kernel
															\item \script{239}{Summary and Visualization} and \videotwelve{3378}{Explanation} and \videothirteen{44}{Explanation 2}, \videotwelve{3442}{Main Problem of Meltdown}, \videotwelve{3512}{Assumption not necessary to clean Microarchitectural states}
															% \item \script{240}{(Affected Processors)}, \videotwelve{3583}{AMD not}, possibly something different about timing, \videotwelve{4001}{Updates not always applied}, precondition is having malware on machine
															\item \script{241}{Countermeasures}, \videotwelve{3644}{Not all coutermeasures published}, \videotwelve{3687}{Why not mapping kernel memory works}, \videotwelve{3759}{Before Meltdown kernel memory}, in each page table of each process, \videotwelve{3866}{Problem with checking access rights before loading}, \videotwelve{3917}{Race always won by... explained}, \videothirteen{164}{Critique about access rights before loading}, slow down every memory access, \videothirteen{238}{Race always won can't be done by firmware update}, \videothirteen{249}{Access right checking takes so long}, can be that access right bits not written in TLB, solution then to store in TLB, then fast enough, \videothirteen{442}{Another proposal}, access right violation just flush the cache, prevnt reading information out in next step, comes at cost, one process with frequent acess right violations would slow down all other processes, because cache global data structure, \videothirteen{590}{Flush only cache line containing secret address}, queue of last addresses that go to the cache, flush lines stored in this queue on access right violation, number addresses in queue depend on implementation rates between access right violation and read out
															\item \underline{Preconditions:}
															\begin{enumerate}
																\item[\bfseries\color{PrimaryColor}$\bullet$] \videoten{3563}{Precondition on same machine}, able to start processes on this machine
																\item \videoten{3610}{Out-of-order execution} or more precisely \videoeleven{3049}{Out-of-order with forwarding}%, \videoeleven{220}{Predondition 1 repreated}
																\item \videoeleven{3059}{Virtual memory managment with / based on paging} and \videotwelve{108}{Kernel space mapped into virtual address space}, of each process
																\item \videoeleven{4850}{Use of caches}, not only for tlb but also for data and instruction and data caches for memory accesses, \videotwelve{135}{In particular data caching}
															\end{enumerate}
														\end{itemize}
													\end{minipage}
												}
											}
										child {
												node {Parallel Accelerated Execution
														\resizebox{\textwidth}{!}{
															\begin{minipage}[t]{12cm}
																\begin{itemize}
																	\item \script{204}{Overview}, \videoten{3651}{Available resources meaning}
																\end{itemize}
															\end{minipage}
														}
													}
												child {
														node {Pipelining
																\resizebox{\textwidth}{!}{
																	\begin{minipage}[t]{12cm}
																		\begin{itemize}
																			\item \videoten{3705}{Definition}, \videoten{3843}{In-order}
																		\end{itemize}
																	\end{minipage}
																}
															}
													}
												child {
														node {Out-of-order execution
																\resizebox{\textwidth}{!}{
																	\begin{minipage}[t]{12cm}
																		\begin{itemize}
																			\item \script{205}{Definition}, \videoten{2974}{In other words}, \videoten{3239}{Superscalar processors}, happens in them, \videoten{3876}{Combined with in-order}, \videoten{4046}{If something faster or waiting for operands}
																		\end{itemize}
																	\end{minipage}
																}
															}
														child {
																node {Reservation stations (RSs) and Functional Units (FUs)
																		\resizebox{\textwidth}{!}{
																			\begin{minipage}[t]{12cm}
																				\begin{itemize}
																					\item \script{206}{Definition} and \videoten{4140}{Explanation}, \videoeleven{310}{Decentralized control for execution of instruction}
																					\begin{itemize}
																						\item \script{207}{Illustration}, \videoten{4341}{Register flags result invalid}, \videoten{4539}{Reservation stations internal states}
																						\item \videoten{4604}{\alert{Retiring Phase}}, put result from functional unit or load buffer on common data bus, \alert{result token}, \videoten{4657}{Forwarding}, \videoten{4760}{Exception}
																					\end{itemize}
																				\end{itemize}
																			\end{minipage}
																		}
																	}
															}
														child {
																node {Tomasulo Scheduling
																		\resizebox{\textwidth}{!}{
																			\begin{minipage}[t]{12cm}
																				\begin{itemize}
																					\item \script{208}{Example (ff.)}, instance of out-of-order execution, \videoeleven{828}{To issue}, \videoeleven{838}{Issue always in-order}
																					\begin{itemize}
																						\item \videoeleven{520}{Explanation Different parts}
																						\begin{itemize}
																							\item \videoeleven{529}{Register file}, \videoeleven{561}{Valid bit}, \videoeleven{577}{Reference to Reservation Station}, where future contents of register produced
																							\item \videoeleven{665}{At least 3 reservation stations}, but here functional unit several reservation stations, \videoeleven{718}{InFU and other column titles}, \videoeleven{774}{Common databus}, connect everything, \videoeleven{874}{RS1 and RS2 Columns}
																						\end{itemize}
																						\item \videoeleven{943}{Explanation Procedure}
																						\begin{itemize}
																							\item \videoeleven{1034}{Change register status}, \videoeleven{1191}{One Cycle, next Executed}, remaining cycles in FU after cycle 2, \videoeleven{1318}{Reg invalidated}, \videoeleven{1450}{Operation finishes}, Register 2 waiting for result of RS 1, \videoeleven{1621}{Forwarding}, immediately read what is on common data bus for register 2, take as early as possible, else would have to delay addition for one cycle, \videoeleven{1996}{Confusing}, register file and reservation stations values after cycle 6 and common data bus values during cycle 6, \videoeleven{2060}{WAR-hazard}, reservation stations read operands immediately, \videoeleven{2246}{Two places waiting for result}, of multiplication
																						\end{itemize}
																						\item \videoeleven{1830}{Disadvantage of original Tomasulo Scheduling}, common databus bottleneck
																					\end{itemize}
																				\end{itemize}
																			\end{minipage}
																		}
																	}
																child {
																		node {Data Dependencies
																				\resizebox{\textwidth}{!}{
																					\begin{minipage}[t]{12cm}
																						\begin{itemize}
																							\item \script{221}{RAW} and \videoeleven{2451}{Example in Tomasulo}, Forwarding is exploited in Meltdown
																							\item \script{222}{WAR} and \videoeleven{2749}{Example in Tomasulo}, \videoeleven{2690}{In Tomasulo solution is early reading into RSs}, \videoeleven{2938}{If Read in between}
																							\item \script{223}{WAW}
																							\item \videoeleven{2627}{Read after Read is no problem}
																						\end{itemize}
																					\end{minipage}
																				}
																			}
																	}
															}
													}
											}
										child {
												node {Virtual memory management with paging
														\resizebox{\textwidth}{!}{
															\begin{minipage}[t]{12cm}
																\begin{itemize}
																	\item \script{224}{Definition} and \videoeleven{3073}{Explanation}
																	\item \script{225}{Adress Translation Illustration} and \videoeleven{3311}{Explanation}, \videoeleven{3419}{Page table pointer}, \videoeleven{3539}{Normally care about size of page table}, \videoeleven{3581}{Every process own page table}, context switch between two processes, then just change page table pointer, \videoeleven{3626}{Interesting question}, how Meltdown can read memory of other processes or main memory of the kernel, in this simple version, \videoeleven{3703}{Case of page fault}, \videoeleven{3796}{What do in Address translation}, 2 memory accesses, main memory accesses pretty slow, cache virtual addresses used in the past, Translation Lookaside buffer
																	\item Each process has its own page table
																	\item For efficiency reasons, each process page table contains kernel address space as well (at least until Meltdown)
																	\begin{itemize}
																		\item \script{227}{Virtual address space of processes}, Advantage of including kernel space into process page tables: TLB does not need to be flushed for privileged accesses to kernel space, e.g. system calls, \videoeleven{4125}{What do with kernel addresses}, programs have system calls and during them addresses of kernel space can be read, \videoeleven{4125}{What do with kernel addresses}, programs have system calls and during them addresses of kernel space can be read, \videoeleven{4179}{Could do same mechanism with system calls}, flush TLB and have seperate page table during syscall that belongs to OS, like context switch, also when comes back, \videoeleven{4328}{Don't have to change page table and TLB}, when doing syscall, efficiency reasons
																		\item \script{228}{When an exception is thrown}, \videoeleven{4410}{Additional contents in page table}, memory managment unit, \videoeleven{4519}{Cause interrupt}, run into segmentation fault, \videoeleven{4581}{Syscalls implement with care}, from designers of the OS, \videoeleven{4714}{Vulnerability of Meltdown}, microachitectural states are not clean, \videoeleven{4776}{Problem that Meltdown has to solve}, Segmentation fault
																	\end{itemize}
																\end{itemize}
															\end{minipage}
														}
													}
												child {
														node {Translation Lookaside Buffer (TLB)
																\resizebox{\textwidth}{!}{
																	\begin{minipage}[t]{12cm}
																		\begin{itemize}
																			\item \script{226}{Definition} and \videoeleven{3877}{Explanation}, \videoeleven{3960}{Most expansive memory}, \videoeleven{3973}{With virtual address}, parallel comparison to all entries in TLB, \videoeleven{4014}{What has to happen on context switch with TLB}, flush TLB on context switch
																		\end{itemize}
																	\end{minipage}
																}
															}
													}
											}
										child {
												node {Instruction and data caches
														\resizebox{\textwidth}{!}{
															\begin{minipage}[t]{12cm}
																\begin{itemize}
																	\item \script{229}{Definition}, \alert{cache hit} and \alert{cache miss}, \videoeleven{4941}{Cache miss also put result into cache}, for the next access to the same address
																\end{itemize}
															\end{minipage}
														}
													}
											}
									}
								child {
										node {Spectre
												\resizebox{\textwidth}{!}{
													\begin{minipage}[t]{16cm}
														\begin{itemize}
															\item based on speculative executions, \script{202}{Note}
															\item \script{242}{Overview}, \videotwelve{4152}{More difficult to exploit because...}, more preconditions
															\item \script{244}{Preconditions}:
															\begin{enumerate}
																\item Speculative Execution, \videotwelve{4530}{What unexpected code executions means}
																\item \videotwelve{4570}{Make victim to execute some code}, Attacker not start process
															\end{enumerate}
															\item \script{245}{Code (f.)}, \videotwelve{4785}{Sandboxing}, \videotwelve{4894}{Instrumentation}, \videotwelve{4925}{Side remark for probe\_array}, \videotwelve{5004}{Everything in address space of victim}, \videotwelve{5099}{Attack tries to influence outcome of speculation}, \videotwelve{5143}{Increase probability that attack works}, \videotwelve{5247}{Again traces in the cache}, 2nd part same as in meltdown, \videotwelve{5265}{Difference to Meltdown}, Not crash, \alert{access to memory that belongs to executing process itself}, \videothirteen{1053}{Remainder same as in meltdown}
															\item \script{247}{Win the \enquote{race}}, \videothirteen{1176}{Similiar to meltdown approach}, \videothirteen{1236}{How caching and uncaching work}, 2nd attack not change speculation if enough access with x smaller array1\_size, \videothirteen{1339}{In Javascript not possible to use cache flush}
															\item \script{250}{Other Application Scenarios}, \videothirteen{2434}{Gadgets}, \videothirteen{2469}{How could work}, \videothirteen{2510}{But also preconditions and evaluation}, of probe\_array, \videothirteen{2647}{Need 3 things}
															\item \script{251}{Countermeasures}, \videotwelve{5437}{Avoid by separate process for javascript code}, \videotwelve{5459}{In new browsers fixed}, \videothirteen{2735}{Sandboxing}, \videothirteen{2757}{Each website in separate process}, \alert{stronger sandboxing}, don't additional if conditions, \videothirteen{2784}{Different processes also completely different page tables}, no access to browser address space, \videothirteen{2956}{Somtimes have to sacrafice performance for security}
														\end{itemize}
													\end{minipage}
												}
											}
										child {
												node {Speculative execution
														\resizebox{\textwidth}{!}{
															\begin{minipage}[t]{12cm}
																\begin{itemize}
																	\item \script{243}{Definition}, \videoten{2995}{In other words}, \videotwelve{4230}{Branches}, \videotwelve{4293}{Avoid changes to architectural states that can't revert}, \videotwelve{4384}{Avoid writing memory or register file}, \videotwelve{4447}{Should not retire speculative instruction}, before not clear whehter jump should be taken or not, \videotwelve{4505}{If miss predicted ensure architectural state correct}
																\end{itemize}
															\end{minipage}
														}
													}
											}
										child {
												node {A-Way Set-Associative Cache
														\resizebox{\textwidth}{!}{
															\begin{minipage}[t]{12cm}
																\begin{itemize}
																	\item \script{248}{Example} and \videothirteen{1405}{Explanation}, \videothirteen{1416}{Letters}, \videothirteen{1578}{A-Way}, \videothirteen{1714}{Additional step for number of byte}, because in datablock complete cache line stored, \videothirteen{1784}{Valid Bits}, \videothirteen{1813}{Access in Direct Mapped Cache}, \videothirteen{1840}{Fully-associative cache}, $s=0$, \videothirteen{1902}{Cache replacement strategies}, all address prefixes different from S and all A data blocks occupied
																\end{itemize}
															\end{minipage}
														}
													}
												child {
														node {Cache Eviction
																\resizebox{\textwidth}{!}{
																	\begin{minipage}[t]{12cm}
																		\begin{itemize}
																			\item \script{249}{Definition} and \videothirteen{2015}{Explanation}, \videothirteen{2072}{Number accesses depending on replacement strategy}, with A accesses in A-way associative can be sure evicted contents, \videothirteen{2095}{probe\_array completely uncached number of accesses}, \#num\_address\_in\_array * A, \videothirteen{2141}{Example}, can also be directly replaced, all A are different, random replacement strategy can be never throw out
																		\end{itemize}
																	\end{minipage}
																}
															}
													}
											}
									}
							}
					}
			}
		child {
				node {(\enquote{Classical}) Cryptography
						\resizebox{\textwidth}{!}{
							\begin{minipage}[t]{12cm}
								\begin{itemize}
									\item \alert{Cryptology:} Cryptography + cryptanalysis
									\begin{itemize}
										\item \alert{Cryptography:} Art/science of keeping message secure
										\begin{itemize}
											\item is about algorithms protecting secret information
											\item \script{67}{Basic Cryptographic Scheme}
											\item \script{68}{Formal Notation}, $E$ \alert{injective} and both setes have \alert{same cardinality}, so \alert{one-to-one}, \alert{bijective}, so it is \alert{reversable}, $D$ is the \alert{inverse function} of $E$ and by this also bijective
											\item \script{69}{Practice: Sending and Receiving secure messages}
										\end{itemize}
										\item \alert{Cryptanalysis:} Art/science of breaking ciphertext
										\begin{itemize}
											\item \script{194}{Different possible situations (pr. ff.)} and \videoten{1126}{Explanation}
											\begin{itemize}
												\item \videoten{1216}{Ciphertext-only}
												\item \videoten{1327}{Known plaintext}, \videoten{1373}{E.g. date or file header at beginning}
												\item \videoten{1457}{Chosen plaintext}
												\item \videoten{1591}{Chosen ciphertext}
												\item \videoten{1676}{Fault injection}
											\end{itemize}
										\end{itemize}
									\end{itemize}
									\item \script{194}{Criteria for \enquote{Good} Ciphers} and \videoten{1741}{Explanation}, \videoten{1976}{Only has to be good modulo the resources able to invest}, the target field of application fixes the required security level and also the cost one is able to invest
								\end{itemize}
							\end{minipage}
						}
					}
				child {
						node {Mathematical background
								\resizebox{\textwidth}{!}{
									\begin{minipage}[t]{14cm}
										\begin{itemize}
											\item \script{131}{\alert{finite field} $(\{0, 1\}^8, +, \cdot)=GF(2^8)$}, special case of \inlinebox{Galois field} $GF(p^k)$, with a prime number $p$ and a natural number $k$, having $p^k$ elements
											\begin{itemize}
												\item AES needs fitting definitions of $+$ and $\cdot$ on $\{0, 1\}^8$ to turn it into a \alert{finite field}. Small finite field with $256$ elements
												\item The irreducible polynomial used in AES is \videoeight{179}{Part of its specification}. It is $g(X) = X^8 + X^4 + X^3 + X + 1$. \videoeight{209}{Several irreducable polynomials of degree $8$}
												\item one needs finite field property at point where one wants to invert the MixColumn Operation, one needs the property of having multiplicative inverse, to derive inverse matrix
												\item \script{170}{AES Summary} and \videoeight{23}{Explanation}, you can use this complete nice mathematical background for aes and just construct a multiplication table you just take two bytes use this mathematical approach and compute the product of two bytes which result in a byte and then you take this table and you forgot about all the mathematical background and you can still do computations in aes because in aes you \videoeight{225}{Just need this table} this is true for aes but this is not true for what we will see next this RSA for RSA we really cannot compress everything you must know into one table for this we need some some insight into mathematical background. \videoeight{731}{Byte, length 8, degree 7}, \videoeight{767}{Irreducible then each row contains a 1}, there see ninverse element to row element, apart from 0, \videoeight{857}{Small hint key schedule of AES}, remainder of polynomial division, \videoeight{1074}{MixColumn multiplication connection}, single operations are always additions or multiplications of bytes
											\end{itemize}
											\item \script{131}{$(\mathbb{Z}/m\mathbb{Z}, +, \cdot)$}, is a \alert{field} and coincides with $GF(m)$ if $m$ is a prime number, else it is a \videoeight{1168}{Commutative ring with multiplicative identity $\overline{1}$}, if $n$ is prime it is a field
											\begin{itemize}
												\item to be useable for AES, $m$ must be a prime number, but $2^8-1$ isn't
												\item RSA, modulo arithmetic, calculations in the ring $(\mathbb{Z}/m\mathbb{Z}, +, \cdot)$. RSA is a huge residue class ring, $2.048$ Bits, in the order of $2^{2046}$, $m$ is product of two really huge prime numbers, just on table not possible
											\end{itemize}
										\end{itemize}
									\end{minipage}
								}
							}
						child {
								node {Greatest common devisor (gcd)}
								child {
										node {Prime factorization
												\resizebox{\textwidth}{!}{
													\begin{minipage}[t]{12cm}
														\begin{itemize}
															\item $24 = 2\cdot 12 = 2\cdot 2\cdot 6 = \textcolor{PrimaryColor}{2}\cdot \textcolor{PrimaryColor}{2}\cdot 2\cdot \textcolor{PrimaryColor}{3}$
															\item $36 = 2\cdot 18 = 2\cdot 2\cdot 9 = \textcolor{PrimaryColor}{2}\cdot \textcolor{PrimaryColor}{2}\cdot \textcolor{PrimaryColor}{3}\cdot 3$
															\begin{itemize}
																\item not $2\cdot 18 = 2 \cdot 3\cdot 6$, is only correct be coincidence, choose smallest possible prime number first
															\end{itemize}
															\item $gcd(24, 36) = \textcolor{PrimaryColor}{2}\cdot \textcolor{PrimaryColor}{2}\cdot \textcolor{PrimaryColor}{3} = 12$
														\end{itemize}
													\end{minipage}
												}
											}
									}
								child {
										node {Devisor series
												\resizebox{\textwidth}{!}{
													\begin{minipage}[t]{12cm}
														\begin{itemize}
															\item $devseries(24) = (1, 2, 3, 4, 6, 8, \textcolor{PrimaryColor}{12}, 24)$
															\item $devseries(36) = (1, 2, 3, 4, 6, 9, \textcolor{PrimaryColor}{12}, 18, 36)$
															\item $gcd(24, 36) = \textcolor{PrimaryColor}{12}$
														\end{itemize}
													\end{minipage}
												}
											}
									}
								child {
										node {Euklidian algorithm
												\resizebox{\textwidth}{!}{
													\begin{minipage}[t]{12cm}
														\begin{itemize}
															\item $\begin{aligned}[t]
																	36:24                           & = 1\quad R: 12 \\
																	24:\textcolor{PrimaryColor}{12} & = 2\quad R: 0  \\
																\end{aligned}$
															\item $gcd(36,24) = 12$
														\end{itemize}
													\end{minipage}
												}
											}
									}
								child {
										node {Least common multiple (lcm)
												\resizebox{\textwidth}{!}{
													\begin{minipage}[t]{12cm}
														\begin{itemize}
															\item germ. \enquote{kleinstes gemeinsames Vielfaches (kgv)}
															\item $gcd(6, 8) = \dfrac{6\cdot 8}{lcm(6, 8)} = \dfrac{48}{24} = 2$
														\end{itemize}
													\end{minipage}
												}
											}
										child {
												node {Prime factorisation
														\resizebox{\textwidth}{!}{
															\begin{minipage}[t]{12cm}
																\begin{itemize}
																	\item $84 = 2\cdot 2\cdot 3\cdot 7 = 2^2\cdot 3\cdot \textcolor{PrimaryColor}{7}$
																	\item $120 = 2\cdot 2\cdot 2\cdot 3\cdot 5= \textcolor{PrimaryColor}{2^3}\cdot \textcolor{PrimaryColor}{3}\cdot \textcolor{PrimaryColor}{5}$
																	\item $lcm(84, 120) = \textcolor{PrimaryColor}{2^3}\cdot \textcolor{PrimaryColor}{3}\cdot\textcolor{PrimaryColor}{5}\cdot \textcolor{PrimaryColor}{7} = 840$
																\end{itemize}
															\end{minipage}
														}
													}
											}
										child {
												node {Multiple series
														\resizebox{\textwidth}{!}{
															\begin{minipage}[t]{12cm}
																\begin{itemize}
																	\item $mulseries(6) = (6, 12, 18, \textcolor{PrimaryColor}{24}, 30, 36, 42, 48)$
																	\item $mulseries(8) = (8, 16, \textcolor{PrimaryColor}{24}, 32, 40, 48)$
																	\item $lcm(6, 8) = \textcolor{PrimaryColor}{24}$
																\end{itemize}
															\end{minipage}
														}
													}
											}
										child {
												node {Greatest common divisor (gcd)
														\resizebox{\textwidth}{!}{
															\begin{minipage}[t]{12cm}
																\begin{itemize}
																	\item germ. \enquote{kleinstes gemeinsames Vielfaches (kgv)}
																	\item $lcm(6, 8) = \dfrac{6\cdot 8}{gcd(6, 8)} = \dfrac{48}{2} = 24$
																\end{itemize}
															\end{minipage}
														}
													}
											}
									}
							}
						child {
								node {Groups
										\resizebox{\textwidth}{!}{
											\begin{minipage}[t]{14cm}
												\begin{itemize}
													\item \script{133}{Definition and Example}
													\item $(R, *)$, $R$ is a non-empty set, $*$: $R \times R \rightarrow R$ is a \alert{group} \textit{iff}
													\begin{itemize}
														\item $*$ is \alert{associative}
														\item an \alert{(unique) identity element} (unit element) $n \in R$ with $a * n = n * a = a \enspace\forall a\in R$
														\item for each $a\in R$ exists an \alert{inverse element} $a^{-1}\in R$ with $a * a^{-1} = a^{-1} * a = n$
													\end{itemize}
													\item \alert{abelian group} \textit{iff} $*$ is commutative
												\end{itemize}
											\end{minipage}
										}
									}
								child {
										node {Fields (germ. Körper)
												\resizebox{\textwidth}{!}{
													\begin{minipage}[t]{12cm}
														\begin{itemize}
															\item \script{139}{Definition}
															\item $(F, +, \cdot)$, $F$ is a non-empty set, $+,\cdot$: $F \times F \rightarrow F$ is a \alert{field} \textit{iff}
															\begin{itemize}
																\item $(F, +)$ is an \alert{abelian group}
																\item \alert{Multiplication} is \alert{associative}
																\begin{itemize}
																	\item already required by \alert{abelian group} property
																\end{itemize}
																\item \alert{Multiplication} is \alert{commutative}
																\begin{itemize}
																	\item already required by \alert{abelian group} property
																\end{itemize}
																\item $(F\setminus \{0\}, \cdot)$ is an \alert{abelian group} ($0$ is the additive identity)
																\item \alert{Distribuitivity}
															\end{itemize}
															\item \alert{Difference} to \alert{commutative rings} with (multiplicative) identity: For each element of $F\setminus\{0\}$ there is a multiplicative inverse
														\end{itemize}
													\end{minipage}
												}
											}
										child {
												node (finitefields) {Finite / Galois Fields
														\resizebox{\textwidth}{!}{
															\begin{minipage}[t]{14cm}
																\begin{itemize}
																	\item $(GF(2), +, \cdot) = (\mathbb{Z}/2\mathbb{Z}, +, \cdot)$ (or $\{0, 1\}$ with $\oplus$ as $+$ and $\wedge$ as $\cdot$)forms a field with additive identity $0$, multiplicative identity $1$ and $1$ as the multplicative inverse of $1$
																	\begin{itemize}
																		\item \script{140}{Example}
																	\end{itemize}
																	\item \script{138}{Making $(\{0, 1\}^k, +, \cdot)$ a finite field}
																	\begin{itemize}
																		\item define addition on $(\{0,1\}^k, +, \cdot)$ as bitwise Xor
																		\begin{itemize}
																			\item Xor on $\{0,1\}$ is the same as $+$ on $GF(2) = \mathbb{Z}/2\mathbb{Z} = \{\overline{0}, \overline{1}\}$
																		\end{itemize}
																		\item definition of $\cdot$ on $\{0,1\}^k$ such that the result becomes a finite field
																		\begin{itemize}
																			\item the And operation is $\cdot$ on $GF(2)$
																			\item to be done by reduction to the consideration of so-called polynomial rings over $GF(2)$
																			\item \script{141}{Why not $\mathbb{Z}/2^k\mathbb{Z}$?}, an element that is even is not invertable, therefore can't be a field, all elements in $\overline{1}$ are odd, $a\cdot p$ is even, because $p$ is even, so it can't be in a set where all elements are odd
																		\end{itemize}
																	\end{itemize}
																\end{itemize}
															\end{minipage}
														}
													}
											}
									}
								child {
										node {Rings
												\resizebox{\textwidth}{!}{
													\begin{minipage}[t]{12cm}
														\begin{itemize}
															\item \script{132}{Definition and Example}, example has 2 additional prperties that one are not required for it to be a ring
															\item $(R, +, \cdot)$, $R$ is a non-empty set, $+,\cdot$: $R \times R \rightarrow R$ is a \alert{ring} \textit{iff}
															\begin{itemize}
																\item $(R, +)$ is an \alert{abelian group}
																\item \alert{Multiplication} is \alert{associative}
																\item \alert{Distributivity}, because commutativity of multiplication is not required for a ring and therefore one needs second version of the law of distributivity
															\end{itemize}
															\item \alert{commutative ring} \textit{iff} the multiplication is also commutative
														\end{itemize}
													\end{minipage}
												}
											}
										child {
												node {Residue class ring (germ. Restklassenring)
														\resizebox{\textwidth}{!}{
															\begin{minipage}[t]{14cm}
																\begin{itemize}
																	\item \script{134}{Definitions (ff.)}
																	\begin{itemize}
																		% \item $2 \equiv 12 \mod 10 \overset{\text{mod gilt auf}}{\underset{\text{beiden Seiten}}{\Leftrightarrow}} 12 \equiv 2 \mod 2 \Leftrightarrow 12 + q_1\cdot m \equiv 2 + q_2\cdot m\Leftrightarrow 12 + 0\cdot m \equiv 2 + q_2\cdot m\Leftrightarrow 12 \equiv 2 + q_2\cdot m$
																		\item $\boxed{12 / 10 = 1 \wedge 12 \mod 10 = 2 \enspace(\text{remainder})} \Rightarrow 2 \equiv 12 \mod 10 \Leftrightarrow 2 + 1 \cdot 10 = 12 + 0 \cdot 10 \Leftrightarrow \boxed{2 + 1 \cdot 10 = 12}\Leftrightarrow 2 + 2 \cdot 10 = 12 + 1 \cdot 10$ ($12$ is the dividend, because it's larger than $2$, $0\le 2< 10$)
																		\item \underline{Division with remainder in $\mathbb{Z}$:} For each $a \in \mathbb{Z}$ there is a unique integer $r$ with $0 \le r < m$ and $a = q \cdot m + r$ for some $q \in Z$ and $m \in \mathbb{N}, m > 1$
																		\begin{itemize}
																			\item \script{134}{Terms}
																			\item $10 / 4 = 1\enspace R:6$ (wieviel bis Dividend) or $10 \mod 4 = 6$ (wieviel über Modul / Devisor darüber), $4 / 10 = 0\enspace R:4$ (bei bestimmtem Vielfachen von Modul / Devisor um wieviel man darüber ist oder wieviel es dann bis zum Dividend ist)
																		\end{itemize}
																		\item \underline{Congruent modulo $m$:} Definition: $a, b \in Z$ are congruent modulo $m$ ($a \equiv b \mod m$) \textit{iff} $a − b = q \cdot m$ for some $q \in Z$
																		\begin{itemize}
																			\item $a \equiv b \mod m$ iff the division with remainder wrt. $m$ gives the same remainder for $a$ and $b$
																			\item \script{134}{Proof}, \underline{forward:} because $r_a = r_b$ are equal the second part reduces to $0$, \underline{backward:} because devision of remainder gives unique result, then one can conclude that $r_a = r_b$
																		\end{itemize}
																	\end{itemize}
																	\item \script{135}{Residue class with example}: The residue class of $r \in \mathbb{Z}$ with $0 \le r < m$ is the set $\overline{r} = \{a \in \mathbb{Z} \mid a \equiv r \mod m\} = \{q \cdot m + r \mid q \in \mathbb{Z}\}$
																	\begin{itemize}
																		\item $(\mathbb{Z}/m\mathbb{Z}, +, \cdot), \mathbb{Z}/m\mathbb{Z} = \{\overline{0}, \overline{1}, \ldots, \overline{m-1}\}$
																		\item for $a \in \overline{r}$ the integer $a \mod m$ is the unique element $r$ of $\overline{r}=\overline{a}=\{0\cdot m + r, 1\cdot m + r, \ldots, a,\ldots\}=\{q\cdot m + r \mid q\in\mathbb{Z}\}$ with $0 \le r < m$ (so in the example $\overline{13} = \overline{3}$)
																		\item each $i \in \mathbb{Z}$ is in exactly one of $m$ pairwise disjoint residue classes: $\overline{0}, \overline{1}, \ldots, \overline{m-1}$
																	\end{itemize}
																	\item \script{136}{Lemma and proof}: If $a_1 \equiv a_2 \operatorname{mod} m$, $b_1 \equiv b_2 \operatorname{mod} m$ then
																	\begin{itemize}
																		\item $a_1 + b_1 \equiv a_2 + b_2 \operatorname{mod} m$
																		\item $a_1 \cdot b_1 \equiv a_2 \cdot b_2 \operatorname{mod} m$
																		\item Addition and multiplication on $\mathbb{Z}/m\mathbb{Z}$ are \alert{well-defined}, since for the result it does not matter which representatives of $a$ and $b$ are chosen
																	\end{itemize}
																	\item \script{137}{Addition and Multiplication well definied}
																	\begin{itemize}
																		\item $\overline{a} + \overline{b} = \overline{a + b}$, $\overline{a} \cdot \overline{b} = \overline{a \cdot b}$
																		\item residue class ring $(\mathbb{Z}/m\mathbb{Z}, +, \cdot)$ forms a \alert{commutative ring} with \alert{multiplicative identity} $\overline{1}$ for all $\overline{p} \in \mathbb{Z}/m\mathbb{Z}$
																		\item \underline{Example:} \alert{residue class ring:} $\mathbb{Z}/5\mathbb{Z} = \{\overline{0}, \overline{1}, \overline{2}, \overline{3}, \overline{4}\}$ and one \alert{residue class:} $\overline{2} = \{q\cdot 5 + 2 \mid q\in \mathbb{Z}\}$, if one does computation one does not handle infinite sets but handle representatives: $\overline{2} + \overline{3} + \overline{4} = \overline{2 + 3 + 4} = \overline{4}$, modulo operation gives smallest possible element in residue class
																	\end{itemize}
																\end{itemize}
															\end{minipage}
														}
													}
												child {
														node (polrings) {Polynomial rings over $GF(2)$
																\resizebox{\textwidth}{!}{
																	\begin{minipage}[t]{20cm}
																		\begin{itemize}
																			\item \script{143}{$GF(2)[X]$ and $GF(2)[X]_k$ and example for addition}
																			\begin{itemize}
																				\item $F[X]$: All polynomials in one variable $X$ over the field $F$, coefficients are elements of $F$
																				\item $F[X]_n$: Subset of $F[X]$ with $deg(g) < n$ for polynomials $g$
																				\item we map $\{0,1\}^k$ bijectively to $GF(2)[X]_k$ by the mapping $\varphi(v_{k−1}, \ldots, v_0) = v_{k−1} X^{k−1} + \ldots + v_2 X^2 + v_1 X + v_0$ (in order to be able to define a \alert{multiplication} on $\{0,1\}^k$)
																				\item \script{144}{Examples}
																				\item \script{145}{Addition and Problem with Multiplication}, addition is complicated version of bitwise Xor:
																			\end{itemize}
																			\item \script{148}{Definition: Residue Class Ring Modulo a Polynomial}: $\overline{F(X)} = F[X] / g(X):=\{\overline{u(X)} \mid u(X) \in F[X]\}$
																			\begin{itemize}
																				\item \alert{residue class of $u(X)$ modulo $g(X)$}: $\overline{u(X)}:=\{v(X) \in F[X] \mid v(X) \bmod g(X)=u(X) \bmod g(X)\}$,\quad$u(X),g(X) \in F[X] \text { with } \operatorname{deg}(g(X))\geq 1$
																				\item exactly $2^k$ different residue classes modulo $g(X)\in GF(2)[X]_{k+1}$, the classes $\overline{r(X)}$ with $deg(r(X)) < k$ in the residue class ring $GF(2)[X]/g(X)$ (can choose $2$ options for every coefficient $v_{k-1}$ to $v_0$)
																				\item \script{148}{Addition and Multiplication in $F[X]/g(x)$}: $\overline{u(X)}+\overline{v(X)}:=\overline{u(X)+v(X)}$ and $\overline{u(X)} \cdot \overline{v(X)}:=\overline{u(X) \cdot v(X)}$, addition of two residue classes is just, one takes an element from the first residue class and an element from the second residue class and then applies operation and then compute the residue class what is computing the remainder modulo $g(X)$
																				\begin{itemize}
																					\item again well-defined, because it doesn't matter what representetives one chooses
																					\item \alert{bijective mapping} $\psi:\{0,1\}^k\rightarrow G F(2)[X] / g(X)$ with $\psi\left(v_{k-1}, \ldots, v_0\right)=\overline{v_{k-1} X^{k-1}+\cdots+v_2 X^2+v_1 X+v_0}$ and $g(X)\in GF(2)[X]_{k+1}$, now $\psi$ from before doesn't map to the polynomial ring, but to the residue class ring over polynomials, bar signifies the residue class ring
																					\item \alert{Addition} in $GF(2)[X]/g(X)$ \enquote{remains} bitwise Xor on $\{0,1\}^k$, because if have polynomial with degree smaller than $k$ and a polynomial with degree smaller than $k$, than the division with $g(X)$ doesn't change anything, becase we already have something smaller than $k$, so the remainder is the element itself
																					\item \script{149}{Working Multiplication in $\{0, 1\}^k$}: $\left(v_{k-1}, \ldots, v_0\right) \cdot\left(w_{k-1}, \ldots, w_0\right):=\psi^{-1}\left(\psi\left(v_{k-1}, \ldots, v_0\right) \cdot \psi\left(w_{k-1}, \ldots, w_0\right)\right)$
																					\begin{itemize}
																						\item multiplication in $\{0,1\}^k$ \enquote{via} multiplication in $GF(2)[X]/g(X)$, works since $GF(2)[X]/g(X)$ is closed under multiplication. So product of two bitvectors defined by using the mapping into residue classes, doing the mulitplication of residue classes and then mapping back
																						\item \script{150}{Example}, polymomial with $X^4$ would be out of the polynomial ring with degree smaller than $k=3$, therefore consider residue classes which means one takes result and reduce by polymomial with degree $k=3$, so remainder has a degree smaller than $3$
																					\end{itemize}
																				\end{itemize}
																				\item \script{152}{Multiplicative inverse for all elements $\ne 0$?}
																				\begin{itemize}
																					\item for $(GF(2)[X]/g(X), +, \cdot)$ to be a field, there must be multiplicative inverse for all elements $\ne 0$
																					\begin{itemize}
																						\item if $(GF(2)[X]/g(X), +, \cdot)$ is a field, then also $(\{0, 1\}^k, +, \cdot)$
																						\item $g(X)$ must be \alert{irreducable}
																					\end{itemize}
																					\item a polynomial $g(X) \in F[X]$ with $deg(g(X)) > 0$ is \alert{irreducible} \textit{iff} the following holds: If $g(X) = u(X) \cdot v(X)$ for $u(X), v(X) \in F[X]$ then $u(X) \in F$ or $v(X) \in F$
																					\begin{itemize}
																						\item \alert{irreducable} intuitively means that one doesn't have a decomposition of $g(X)$ into two non-trivial factors, so if one has $g(x) = u(X)\cdot v(X)$ then either $u(X)$ or $v(X)$ have to be a field element, trivial if one of them is a field element and non-trivial if one can really decompose it into two non-trivial parts, it's similiar to prime numbers, if have integer number, one can also ask whether one can decompose it into the product of two non-trivial numbers or whether for each decomposition into two factors at least one for them has to be the one, if one computes the decomposition into prime factors, one only has one prime factor, one cannot decompose the prime numbers into non-trivial factorisation. Field element $\hat =$ Polynomial of degreee $0$
																					\end{itemize}
																				\end{itemize}
																				\item \underline{Theorem (Polynomial Residue Class Ring):} Let $g(X)\in F[X]$ with $deg(g(X)) > 0$. $g(X)$ is \alert{irreducible} \textit{iff} $(F[X]/g(X), +, \cdot)$ is a field
																				\begin{itemize}
																					\item \script{153}{GCD for Polynomials}
																					\item \underline{Lemma(Polynomial Residue Class Ring):} For each $u(X), v(X) \in F[X] \setminus \{0\}$ with $max(deg(u(X), deg(v(X)) > 0$ there is a gcd $g(X) \in F[X]$ and there is a representation $g(X) = q_u(X) \cdot u(X) + q_v(X) ∙ v(X)$ with $q_u(X), q_v(X) \in F[X]$
																					\begin{itemize}
																						\item \script{164}{Proof and \enquote{Extended} Euklid‘s algorithm (ff.)}, Euklid‘s algorithm working with polynomial division instead of division of numbers, how to to choose this qn this qn you just choose this qn as a n of X time the inverse of f, f to the power of of minus one and then you have here a n minus one of x * F to the power of N - 1 * F and and then the remainder is zero, if you arrive at degree zero you need at most one additional step if it's already zero you don't need an additional step, this part with the representation of G... is just the same and also the prove that Euklids algorithm does the the right thing is also the same
																						\item Theorem (Polynomial Residue Class Ring) can easily proven with this lemma
																					\end{itemize}
																					\item \script{167}{Proof (f.)}, $\Leftarrow$: the proof of the theory is an immediate consequence of the lemma, ...is a field which means that each element in this structure here which is not equal to zero is invertible that is what we want to show, choose in this R of x bar the smallest representative, this R of x has a degree which is strictly smaller than the degree of G of X, therefore the greatest common divisor of R of x and G of X is a divisor of R of x which means that for this here the degree of greatest common divisor of R of x g of x is smaller than the degree of G of X, G of X is the product of this greatest common divisor and some Q of X and the degree of this is smaller than the degree of G of X and therefore the degree of this Q of X has to be greater than zero, because otherwise you come not to the degree of by this multiplication you have not the possibility to come to the degree of G of X which means the degree of Q of x is greater than zero, now we are using un irreducibility of G of x here we have found a product of some of two polynomials namely the greatest common divisor and Q of X and here the degree is greater than zero which means this is a non-trivial decomposition if the degree of the greatest common divisor is not zero then you would have a non-trivial decomposition and this cannot occur because G of X is irreducible and therefore we can conclude that the greatest common divisor has to be a field element. And then you compute the residue classes as in the previous proof which means you make a big bar here and you are using that g of x bar is equal to zero bar which means this second part here can be omitted... now you just use the property that f is a field this capital F is a field which means there is an inverse in the field of this F which is f to the power of minus 1... this means you can can take this and this is the multiplicative inverse of r of x bar... If Q G of X is irreducible you can find a multiplicative inverse to each residue class which is not equal to zero bar. And then you compute the residue classes as in the previous proof which means you make a big bar here and you are using that g of x bar is equal to zero bar which means this second part here can be omitted... now you just use the property that f is a field this capital F is a field which means there is an inverse in the field of this F which is f to the power of minus 1... this means you can can take this and this is the multiplicative inverse of r of x bar... if Q G of X is irreducible you can find a multiplicative inverse to each residue class which is not equal to zero bar, $\Rightarrow$: g of x bar is of course zero bar because if you're are doing a remainder modulo g of x then from g of x you obtain zero bar, this is a property you always have in rings and in this special polinomial ring it's immediately clear because if you multiply a polinomial with a zero polomial you obtain the zero polynomial, contradiction to this assumption that you have a non-trivial decomposition of G of X into two factors this V of X should not be Zero Bar it should be something different so non-trivial factors means that the degree of V of X is greater than zero
																				\end{itemize}
																			\end{itemize}
																		\end{itemize}
																	\end{minipage}
																}
															}
														child {
																node {Analogy in Modular Arithmetic
																		\resizebox{\textwidth}{!}{
																			\begin{minipage}[t]{18cm}
																				\begin{itemize}
																					\item \underline{Theorem (MA):} Let $n \in \mathbb{N}, n > 1$. $n$ is a prime number \textit{iff} $(\mathbb{Z}/n\mathbb{Z}, +, \cdot)$ is a field
																					\begin{itemize}
																						\item $n$ prime number $\hat =$ irreducable polynomial $g(X)$
																						\item \alert{Prime Number}: A number $n \in \mathbb{N} \setminus \{0, 1\}$ is a prime number \textit{iff} the following holds: If $n = u \cdot v$ for $u, v \in N$ then $u = 1$ or $v = 1$
																						\item \script{157}{Proof}, inverse of $\overline{1}$ is $\overline{1}$, for all elements besides $\overline{0}$ there's an inverse element, if $n$ would not be a prime number, then there would be a non-trivial decomposition into two numbers where both numbers are from this set: $\{2, \ldots, n-1\}$ and therefore we can conclude that $n$ must be a prime number
																						\begin{itemize}
																							\item Theorem (MA') implies Theorem(MA)
																						\end{itemize}
																					\end{itemize}
																					\item \underline{Theorem (MA’):} Let $n \in \mathbb{N}, n > 1$. The inverse of $\overline{a} \in (\mathbb{Z}_n, +, \cdot)$ exists \textit{iff} $gcd(a, n) = 1$
																					\begin{itemize}
																						\item this does not only say something about um n equal to a prime number but it also says something about other rings where n is not a prime number and for those rings this theorem exactly characterizes the invertible elements, this are the elements a bar where the greatest common divisor of a and n here is equal to one which means a and n are relatively prime
																						\begin{itemize}
																							\item \underline{reason for $1$:} It is the neutral element of multiplication: $m \cdot m^{-1} = 1$
																						\end{itemize}
																						\item if p is even then the greatest common devisor of 2 to the power of k and p is at least two which is not one of course and therefore if p is even then the element is not invertible. If p is odd then the prime factor decomposition doesn't contain two which means the greatest common divisor of an odd p and 2 to the power of k is one and then we immediately have that this element is invertible which means in this special case with this theorem we can completely characterize the even elements are not invertible the odd elements are invertible
																						\begin{itemize}
																							\item $2^k$ only has $2$ and $1$ as devisors, $2\cdot k$ has a lot of devisors including $1$ and $2$, thus the gcd is $1\cdot 2 = 2$, thus $2^k$ is not invertible
																							\item $2^k$ only has $2$ and $1$ as devisors, $2\cdot k+1$ has a lot of devisors excluding $2$, thus the gcd is $1$, thus $2^k+1$ is invertible
																						\end{itemize}
																						\item \underline{Lemma (MA):} For each $u, v \in \mathbb{Z} \setminus \{0\}$ there is a gcd $g \in \mathbb{N} \setminus \{0\} \subseteq \mathbb{Z} \setminus \{0\}$ and there is a representation $g = q_u \cdot u + q_v \cdot v$ with $q_u, q_v \in \mathbb{Z}$.
																						\begin{itemize}
																							\item \alert{Euklid's algorithm}: euklids algorithm computes the greatest common devisor, always works, the series of the remainders is monotonicaly decreasing and at some point one will arrive at $0$ and then one knows the gcd, because the gcd of some number and $0$ is the number itself, representation example on the right and below how to \script{158}{obtain these factors}
																							\begin{itemize}
																								\item \script{158}{Example: Euklid's algorithm}
																								\item \script{159}{Proof of Euklid's algorithm (f.)}, \underline{common devisor}: We can without loss of generality assume that $u$ and $v$ are natural numbers because some if some of them is negative then you just put in this representation the minus one let's say $u$ is negative then you can consider the absolute value of $u$ and and put the minus to $q_u$, divide $a_0$ by $a_1$ and obtain representation with $q_1$ being the quotient and $+$ the remainder called $a_2$, if one looks at $0\le a_2 < a_1$ and $0\le a_3 < a_2$ the $a_i$ are monotonicaly decreasing, second page prove that result is the gcd, common deviser means $a_n$ devides $a_0$ and it devides $a_2$, now look at the second last equation here you have a combination of a n minus one and a n you know that that a n minus one divides a n minus one and a n also divides a n and therefore a n divides a n minus 2, continue this calculation until you arrive at the first equation,
																								%and then at the first equation you see that a n divides a 0 from the second equation you see that a n divides a 1 by going from bottom to top you know that a n divides a 1 and a n divides a 0 therefore a n is a common divisor of a 0 and a 1
																								\underline{greatest common devisor}: each common divisor d of a one and a zero also divides a n, a 0 and a 1 are the two numbers for which one wanted to compute the gcd
																								\item \script{160}{Proof of representation $g = q_u\cdot u + q_v\cdot v$}, bottom, again just the same as in in the example just by substitution and you can either decide to do this from the second last equation to the first or the other way around let let's say we we start with the first equation, by using the second equation we have a representation of A3 as a linear combination of a0 and a1 and then you can do this until the second last equation% and finally from the second last equation you have a n is equal to a n - 1 - q n - 1 * a n minus one and you're using this for substitution and finally you obtain a n is equal to [representation], 
																								from the first equation we can see that a A2 is a combination of a0 and a1 from the second that A3 is a combination of a0 and A1 and and finally from the second class we conclude that a n is a linear combination of a0 and a1
																								%, we show a n is a common devisor and then we show it's a greatest common devisor
																								\item the approach to compute a representation of $gcd(a_0, a_1)$ as a linear combination of $a_0$ and $a_1$ (proof above) is called \script{161}{\enquote{Extended} Euklid's algorithm}, extended euklids algorithm computes the representation of the greatest common divisor by a linear combination of the original numbers
																							\end{itemize}
																						\end{itemize}
																						\item \script{162}{Proof with help of the Lemma(MA)}, $\Leftarrow$: n bar is equal to zero bar because you're considering the remainder by dividing by n if you divide n by n you obtain one and the remainder is zero therefore this here is zero, this is by the way a method to compute the inverse in such a residue class ring if the greatest common devisor of some element a with n is greater than one then you don't have to look for an inverse because there doesn't exist an inverse, if the greatest common divisor of A and N is one then you apply euklids algorithm you're doing those substitutions of extended euklids algorithm and by those substitutions you obtain representation of one as a combination of A and N and from this you immediately have the multiplicative inverse which is q A bar, so in this extended euklids algorithm you are Computing this QA and a qn you are not interested in the qn but you are interested in the QA because the residue class of this QA is exactly the multiplicative inverse you're looking for, $\Rightarrow$: or in other words the difference of a * B and 1 is a multiple of n or you can say... assume that the greatest common divisor of A and N is some D which means that a is equal to some QA * d and n is equal to some qn * D which is just a definition here we are not using greatest common devisor but we are using common devisor, we insert this here for a in this equation we insert this here for n in this equation, now we just have to have a close look at this equation and use the fact that all numbers here are integers so you have D * and QA is an integer B is an integer Y and qn is an integer you have D * some integer is equal to 1 if D * some integer is equal to one then we can immediately conclude that D is one because otherwise it doesn't work
																						\item used to generate Keys for RSA
																					\end{itemize}
																				\end{itemize}
																			\end{minipage}
																		}
																	}
															}
														child {
																node {Polynomial Division
																		\resizebox{\textwidth}{!}{
																			\begin{minipage}[t]{12cm}
																				\begin{itemize}
																					\item \script{146}{Example and Check}, in normal division on would substract, but since we are doing computations based on $GF(2)$, addition and substraction are the same, doing XOR and therefore each element is it's own inverse, final result as soon as remainder is smaller than divisor. If one divides by a polynomial of degree $k$, one obtains a remainder with degree smaller than $k$
																					\item \script{147}{Lemma: Division with remainder}: For $u(X)$, $g(X) \in F[X]$ with $g(X)\ne 0$ there exist unique polynomials $q(X), r(X) \in F[X]$ with $u(X) = q(X) g(X) + r(X)$ and $deg(r(X)) < deg(g(X))$ or $r(X) = 0$
																					\begin{itemize}
																						\item $u(X) \mod g(X) = r(X)$, $u(X) \operatorname{div} g(X) = q(X)$
																						\item $r(X)$ is a unique remainder
																					\end{itemize}
																				\end{itemize}
																			\end{minipage}
																		}
																	}
															}
													}
											}
									}
							}
					}
				child {
						node {Classification by way to process plaintext}
						child {
								node (streamcipher) {Stream Ciphers
										\resizebox{\textwidth}{!}{
											\begin{minipage}[t]{12cm}
												\begin{itemize}
													\item bitwise Encryption and Decryption
													\item \script{93}{Definition}, $Xor$ ist associative
													\item \script{94}{Perfect encryption system if $\ldots$}
													\item used in e.g. Red Phone between US and SU was implemented in this way
													\item encrypt bits individually
													\item need a method to generate key stream efficiently, starting from some “seed
													\item usually small and fast common in embedded device
												\end{itemize}
											\end{minipage}
										}
									}
								child {
										node {Pseudo-random sequence
												\resizebox{\textwidth}{!}{
													\begin{minipage}[t]{12cm}
														\begin{itemize}
															\item \script{95}{Random function in C-library}, fixed values for $A$, $B$ and $m$, if prefix of plain text is known as many messages start with current date or files of certain type that have a prefix, one can continue the sequence, if $m$ happens to be a prime number, then it's a field and then one can compute inverse elements of each element and then it's just a solution of an equation system, if $m$ is not a prime number, then one does not have a field, but a ring and somtimes one can not compute the inverse or the inverse is not unique and also in this case on can try to solve a equation system to compute $A$ and $B$
															\item pseudo random functions for cryptography it might be a problem
														\end{itemize}
													\end{minipage}
												}
											}
									}
							}
						child {
								node (blockcipher) {Block Ciphers
										\resizebox{\textwidth}{!}{
											\begin{minipage}[t]{12cm}
												\begin{itemize}
													\item always encrypt a full block (several bits)
													\item are common for Internet applications
												\end{itemize}
											\end{minipage}
										}
									}
								child {
										node {More Block Ciphers
												\resizebox{\textwidth}{!}{
													\begin{minipage}[t]{12cm}
														\begin{itemize}
															\item DES (Data Encryption Standard)
															\begin{itemize}
																\item Predecessor of AES
																\item Considered insecure (small key length of 56 bits)
																\item \underline{idea to make DES more secure:} \script{129}{Triple-DES}, decryption and encryption are inverse, also for AES one could use decryption for encryption and encryption for decryption, encrypt, decrypt, encrypt ist becausse of backward compatibility, if have implementation for triple DES and someone sends text with single DES, then K1=K2 and then after 2 rounds again plain text
															\end{itemize}
															\item \script{129}{List of other ciphers}
														\end{itemize}
													\end{minipage}
												}
											}
									}
								child {
										node {Block Cipher Primitives
												\resizebox{\textwidth}{!}{
													\begin{minipage}[t]{12cm}
														\begin{itemize}
															\item \alert{Claude Shannon:} There are two primitive operations with which strong encryption algorithms can be built:
															\begin{enumerate}
																\item \alert{Confusion:} An encryption operation where the \alert{relationship between key and ciphertext is obscured}
																\begin{itemize}
																	\item today, a common element for achieving confusion is substitution, which is found in AES and other ciphers.
																\end{itemize}
																\item \alert{Diffusion:} An encryption operation where the \alert{influence of one plaintext symbol is spread over many ciphertext symbols} with the goal of hiding statistical properties of the plaintext
																\begin{itemize}
																	\item in other context known as Transposition
																	\item a simple diffusion element is the \alert{bit permutation} (in other context known as Tranposition)
																\end{itemize}
															\end{enumerate}
															\begin{itemize}
																\item Both operations by themselves are suboptimal in providing security. A cipher must include confusion and diffusion elements
															\end{itemize}
														\end{itemize}
													\end{minipage}
												}
											}
									}
							}
					}
				child {
						node {Classification by key
								% Crypto System with Keys
								\resizebox{\textwidth}{!}{
									\begin{minipage}[t]{12cm}
										\begin{itemize}
											\item \script{70}{Definition}, $E$ is a set of encryption algorithms and the key $K_E$ selects one special encryption algorithm based on the key, with each key one has a different encryption, same for $K_D$, both algorithms \alert{have to match}, they encryption key should be in a certain relation to the decrytpion key, they can also be just the same
											\item \underline{advantage of crypto systems with keys:}
											\begin{itemize}
												\item keeping the encryption / decryption algorithm secret is not needed
												\item keys can regulary be changed, to increase security
											\end{itemize}
										\end{itemize}
									\end{minipage}
								}
							}
						child {
								node (symmetric) {Symmetric cryptosystems $K_E = K_D$
										\resizebox{\textwidth}{!}{
											\begin{minipage}[t]{12cm}
												\begin{itemize}
													\item encipher and decipher using the same key, \videoeight{1356}{other name secret key encryption}
													\begin{itemize}
														\item or one key is easily derived from the other
													\end{itemize}
													\item \underline{advantage:}
													\begin{itemize}
														\item \videoeight{1536}{Authentication}
													\end{itemize}
													\item \underline{disadvantage:}
													\begin{itemize}
														\item one needs a \alert{key exchange}, the sender and the receiver have to agree on the same key and they key should be secret, one needs a way to transport the secret key via a secure channel from the sender to the receiver
													\end{itemize}
													\item \script{172}{Illustration}
													\item \videoeight{1380}{Reliability and Security relies on key secret}
												\end{itemize}
											\end{minipage}
										}
									}
								child {
										node (aes) {Advanced Encryption Standard (AES)
												\resizebox{\textwidth}{!}{
													\begin{minipage}[t]{12cm}
														\begin{itemize}
															\item \script{101}{Properties and Backstory}, block size always the same, number of rounds depens on key length that can be different, efficiency in \textit{software} and \textit{hardware}, byte-oriented cipher to be appropriate for small microprocessors and -controlers that havea $8$-bit width in the data path
															\item \script{102}{Iterated Cipher types}, with key lengths $128$ ($16$ Bytes), $192$, $256$ and respective numbers of rounds $10, 12, 14$ and always $+1$ subkeys respective
															\item at the moment most widely used symmetric cipher
															\item it is \alert{deterministic}, i.e., with the same key identical plain text blocks are mapped to the same cipher text blocks
														\end{itemize}
													\end{minipage}
												}
											}
										child {
												node {Implementation in Software
														\resizebox{\textwidth}{!}{
															\begin{minipage}[t]{12cm}
																\begin{itemize}
																	\item straightforward implementation is well suited for $8$-bit processors (e.g., smart cards) with small amount of memory, because operations are on bytes, but inefficient on $32$-bit or $64$-bit processors
																	\item \script{121}{More sophisticated approach for $32$-bit processors explained and speed (f.)}, because operations are on bytes that is no efficient for $32$-bit processors, store all precomputed $2^8=256$ possible operation results for one column of the constant matrix in one T-Table (\href{https://crypto.stackexchange.com/questions/19175/efficient-aes-use-of-t-tables}{\inlinebox{Explanation}}), T-Table has $4$ byte entries, because e.g. the result of $T_4[FF] = 01*FF \mid 01*FF \mid 03*FF \mid 02*FF$ ($\mid$ stands for concatenation) for the last column $4$ and input $ff$ has a concatenated width of $4$ bytes, efficient on $32$-bit processor when just Xor $3$ times $4$ $32$-bit words (a T-Table entry has $4$ bytes) that one has to look up in the precomputed T-Table, for $4$ T-Tables need $4KB$%, , $4$ byte entries are depending on the plain text value, for computing one column we need $4$ table lookups and $3$ Xor additions, for complete matrix with 4 columsn one has to do it 4 times, so $4\times 4=16$ accesses to the $4$ T-tables and $4\times 3=12$ Xor operations
																	\item \underline{hardware implementation:} speed up little bit with unrolling if cost doesn't matter, if only implement one round one need some multiplexers, because key schedule generation not exactly the same for each round, multiplexers do different things depending on round number, select right subkey $k_i$
																\end{itemize}
															\end{minipage}
														}
													}
											}
										child {
												node {Security
														\resizebox{\textwidth}{!}{
															\begin{minipage}[t]{12cm}
																\begin{itemize}
																	\item \alert{Brute-force attack:} Due to the key length of $128$, $192$ or $256$ bits, a brute-force attack is not possible
																	\item \alert{Analytical attacks:} There is no analytical attack known that is better than brute-force
																	\item \alert{Side-channel attacks:} Several side-channel and fault-injection attacks have been published attacking the implemntation (not the algorithm)%. Note that side-channel attacks do not attack the underlying algorithm but the implementation of it
																	\item \script{124}{Remark: Importance of nonlinear S-Boxes}, several rounds also wouldn't make a problem, could still solve with linear equation system, because composition of two linears function is a linear function, composition of linear function with an addition is again just an addition
																\end{itemize}
															\end{minipage}
														}
													}
												child {
														node {Encrypting longer plain text
																\resizebox{\textwidth}{!}{
																	\begin{minipage}[t]{12cm}
																		\begin{itemize}
																			\item \script{125}{Example: Image}, because it is deterministic, i.e., with the same key identical plain text blocks are mapped to the same cipher text blocks
																			\item \underline{Countermeasure:} Different modes for block ciphers
																			\begin{itemize}
																				\item \script{126}{ECB, Encryption, Decryption, Advantages, Disadvantages} = \alert{E}lectronic \alert{C}ode \alert{B}ook Mode
																				\begin{itemize}
																					\item Straightforward application of block cipher
																					\item \underline{disadvantage:} \alert{traffic analysis}, if know block with encoding of account number, without knowing how encryption works, one can replace ones own encrypted account number with account number of other account in other transmission (\alert{substituion attack})
																				\end{itemize}
																				\item \script{127}{CBC, Encryption, Decryption, Advantages, Disadvantages} = \alert{C}ipher \alert{B}lock \alert{C}haining Mode
																				\begin{itemize}
																					\item \underline{advantage:} no \alert{traffic analysis} possible
																				\end{itemize}
																			\end{itemize}
																		\end{itemize}
																	\end{minipage}
																}
															}
													}
											}
										child {
												node {Decryption
														\resizebox{\textwidth}{!}{
															\begin{minipage}[t]{14cm}
																\begin{itemize}
																	\item \underline{\script{115}{all layers} must be inverted for decryption:} invert last round (has no MixColumn Layer), then inverse of second last round and finally inverse of first round and then additionaly invert the first key addition
																	\begin{itemize}
																		\item \script{116}{Inv MixColumn layer (f.)}, to reverse the MixColumn operation, each column of the state matrix $C$ must be multiplied with the inverse of the $4\times 4$ matrix, \script{117}{identity matrix}, all arithmetic is done in $GF(2^8)$. Useful that in the encryption one used matrix that is \alert{invertable}
																		\item \script{118}{Inv ShiftRows layer}, all rows of the state matrix B are shifted to the opposite direction (now right)
																		\item \alert{Inv Byte Substitution layer}, since the S-Box is bijective, it is possible to construct an inverse, such that $A_i = S^{–1}(B_i) = S^{–1}(S(A_i))$. The inverse S-Box is used for decryption. It is usually realized as a lookup table. Inverse Lookup Table easy to get from Lookup Table
																		\item \alert{Key Addition layer} is its own inverse (\script{115}{explanation})
																	\end{itemize}
																	\item \alert{Decryption key schedule}, subkeys are needed in reversed order (compared to encryption). \underline{For decryption:} Before starting decryption, first compute all subkeys from the actual key (as done for encryption) and apply in reverse order, same subkey generation as for the encryption
																\end{itemize}
															\end{minipage}
														}
													}
											}
										child {
												node {Round Structure and Internal Structure
														\resizebox{\textwidth}{!}{
															\begin{minipage}[t]{16cm}
																\begin{itemize}
																	\item \script{103}{Round Structure} $X$ instead of $P$, $Y$ instead of $C$. In the last round do the same omitting the MixColumn Layer (\href{https://crypto.stackexchange.com/questions/1346/why-is-mixcolumns-omitted-from-the-last-round-of-aes}{\inlinebox{reason}})
																	\item \script{104}{Internal Structure (f.)}: $16$ bytes block size, arranged in a $4x4$ matrix
																	\begin{itemize}
																		\item \alert{Byte Substitution Layer:} Don't need matrix, is done \textit{Byte by Byte}, S-Box / functions $S$ \alert{identical} for all $16$ bytes, propably to save memory, because a $S$-Boxe is usually represendet in software implementation by a \script{107}{Lookup Table} ($1$-to-$1$ mapping, so no value in the results / cells of table which occurs twice, always the same lookup table for all implementations for AES, it is part of the specification of AES) with input values $0$ to $2^8-1 = 255$ (\script{106}{reference}), S-Boxes are the only \alert{nonlinear} ($ByteSub(A_i) \oplus ByteSub(A_j) \ne ByteSub(A_i \oplus A_j)$) elements of AES and they are \alert{bijective}, i.e. a $1$-to-$1$ mapping of input and output bytes. Can also be described as computing the multiplicative inverse in $GF(2^8)$. Because of the nonlinaer elements, the S-Boxes one can't use that there's a connection between plaint text and cipher text because of linearity
																		\item \alert{Diffusion Layer:}  Provides diffusion over all input state bits. Performs a linear operation on state matrices A and B, i.e., $DIFF(A) \oplus DIFF(B) = DIFF(A \oplus B)$
																		\begin{itemize}
																			\item \script{109}{ShiftRows Layer}: Permutation of the data on a \alert{byte level}. Rows of the state matrix are shifted cyclically
																			\item \script{110}{MixColumn Layer}: Matrix operation which combines (\enquote{mixes}) blocks of $4$ bytes. Can't be decribed by permutation of Bytes or Bits, the input can have more or less bits equal to $1$ then the output. Linear transformation which mixes each column of the state matrix. Each $4$-byte column is considered as a vector and multiplied by a fixed $4\times 4$ matrix. All arithmetic is done in the \alert{Galois field} $GF(2^8)$, where $GF(2^8)$ is $\{0, 1\}^8$ with addition as bitwise $Xor$ and an \enquote{appropriate} multiplication. Boxes are \alert{identical} for all terms, so one does multiplication with the same matrix and this matrix is fixed for AES. This matrix needs to have the property of being \alert{invertable}
																		\end{itemize}
																		\item \script{111}{Key Addition Layer}: Bitwise Xor, in round number $i$ one \enquote{adds} the $16$-byte \alert{state matrix} $C = C_0C_1\ldots C_{15}$ with the $16$-byte \alert{subkey / round key} $k_i$: $C \oplus k_i$. The subkeys are generated in the \alert{key schedule}, they are derived from the original key $k$. Is the only part that depends on the key and everything is else is to hide what one did and make the decryption without knowing the key much more difficult, such that it's never much simpler than a brute-force attack for which the number of keys is too large.
																		\begin{itemize}
																			\item \script{113}{Key Schedule (pr.,f.)}, subkeys are derived recursively from the original 128/192/256-bit input key, $\#subkeys = \#rounds + 1$, each round has $1$ subkey, plus $1$ subkey at the beginning of AES. \alert{Key whitening:} Subkey is used both at the input and output of AES. There are different key schedules for the different key sizes. Called subkeys, because for key-length 192 and 256 they are subkeys, for 128 too even though it doesn't make sense to call them subkeys. The only really random part os the original key. The round key number $0$ is the original key. If one would also do it the same for the first, one would have a cyclic dependency. For round key $2$ its almost the same with the difference, that in the different rounds the $g$ functions are slightly different. \script{114}{Function $g$} rotates its four input bytes and performs a bytewise S-Box substitution (\textit{the same} as before) $\rightarrow$ nonlinearity. The \alert{round coefficient} RC is only added to the leftmost byte and \textit{varies from round to round}. Until $RC[8]$ it is leftshift, then one can't shift anymore and chooses elements of the Galois field again. \underline{Hint:} Multiplication in Galois field has something to do with multiplicatin of polynomials. Only on byte where on adds this $RC[i]$. $x_i$ represents an element in the Galois field $GF(2^8)$
																			% , in each round transformation of the original key, for keywidth 128 one has $10$ rounds and for them one needs $11$ round keys derived from the secret key $k$, 
																		\end{itemize}
																	\end{itemize}
																\end{itemize}
															\end{minipage}
														}
													}
											}
									}
							}
						child {
								node (asymmetric) {Asymmetric cryptosystems $K_E \ne K_D$
										\resizebox{\textwidth}{!}{
											\begin{minipage}[t]{12cm}
												\begin{itemize}
													\item ecncipher and decipher using different keys, \videoeight{1635}{Reason for calling it assymetric}
													\begin{itemize}
														\item computationally infeasible to derive one from other
														% \item it should be computationally infeasible in reasonable time to derive one private key from the public key
													\end{itemize}
													\item in Hybrid Asymmetric-Symmetric System the key exchange is done with the asymmetric algorithm and also \videoeight{2165}{diginal signatures}
													\item \underline{advantages:}
													\begin{itemize}
														\item allows to share public key, secure key exchange not needed (for \alert{confidentiality}, message should be kept secret)
														\item also other applications like authentication (\alert{authenticity}, able to sign message, so it can be sure it can only come from oneself)
													\end{itemize}
													\item \script{173}{Disadvantages and more advantages}, \videoeight{1784}{Disadvantage if done in exactly this way}
													\item \script{173}{Illustration}
													\item \script{175}{Building Public-Key Algorithms}, \alert{one-way function}, overview of main families of mathematically hard problems, \videoeight{2432}{Only proof of evidence}%, \videoeight{2580}{Exponentiation easy}, \videoeight{2674}{Why called discrete logarithm}
													\begin{itemize}
														\item \script{176}{Key Lengths and Security Levels}, \videoeight{2750}{Table rule of thumb for security levels}, \videoeight{2808}{RSA reason why more expensive}, \videoeight{3015}{Confidence comes from...}
													\end{itemize}
												\end{itemize}
											\end{minipage}
										}
									}
								child {
										node {RSA
												\resizebox{\textwidth}{!}{
													\begin{minipage}[t]{12cm}
														\resizebox{\textwidth}{!}{
															\begin{minipage}[t]{16cm}
																\begin{itemize}
																	\item \script{177}{Status and Applications}
																	\item \script{178}{Important Modular Arithmetic (ff.)}, important theorems like \videonine{125}{Lemma Extended Euklids algorithm compute inverse}
																	\begin{itemize}
																		\item \script{179}{Euler's $\phi$-function}, \videoeight{1246}{$gcd(a, n) = 1$ relatively prime}, \videoeight{3372}{Could break RSA if could factor $n$ into $p$ and $q$}, \videoeight{3384}{Computations in ring and not field because...}, \videoeight{3587}{Example Eulers phi function}, \videoeight{3862}{Phi function important key generation and proof of correctness}, \videoeight{3891}{phi p-1 times q-1}, \videoeight{4266}{Special case used in RSA}, \videoeight{4283}{Special case only prime number}, \videoeight{4395}{Definition based on}, Computations done in ring Z mod n Z where n = p times q, but the key generation is done in another ring and that is Z mod phi n Z
																		\item \script{180}{Modular division and modular reduction}, \videonine{357}{Inverse elements don't need to exist in general residue class rings}, \videonine{512}{Complete modular reduction}, \videonine{546}{Doing reduction is choosing other representative}, works because of this
																	\end{itemize}
																	\item \script{181}{Encryption and Decryption}, \videonine{610}{$n$ in which residue class ring do computations}, \videonine{688}{Block based cipher}, block size depends, more than 1000 bits, \videonine{728}{Meaning block and number}, \videonine{843}{Never encryption by iterated multiplication}, \videonine{988}{Complexity}, 																	\begin{itemize}
																		\item \script{182}{Key Generation}, \videonine{1065}{Non-trivial task}, \videonine{1152}{Key Generation: Computations no in Z mod n Z}, \videonine{1228}{How find public exponent}, \videonine{1272}{Take simple number for public exponenet}, small number of ones in the binary representation, \videonine{1342}{Why it has to be invertable}, \videonine{1377}{How can compute this}, extended euclids algorithm to compute inverse, \videonine{1841}{Why could break RSA if could factor n in p and q}, \videonine{1939}{One way function of RSA}
																		\item \script{183}{Example including Key Generation}, \videonine{1574}{Choice of e}, \videonine{1674}{public key is both...}
																	\end{itemize}
																	\item \script{184}{Security}, \videonine{1986}{Use Extented Euklids algorithm to compute inverse}, \videonine{2096}{Number bits of prime numbers fixed}, \videonine{2309}{Prime number table too huge}, density not decreasing that fast
																	\item \script{185}{Correctness}, Euler's Theorem and \videonine{2667}{Explanation}, \videonine{2810}{Follows from Euler's Theorem}, \videonine{2865}{Alternative way to get multiplicative inverse}, under condition that one knows phi of n, \videonine{2898}{If this times a not equal to 1}, then there's no multiplicative inverse, because then gcd is not 1, \videonine{2935}{Alternative way to compute private key d}
																	\begin{itemize}
																		\item \script{187}{Correctness Proof} and \videonine{3201}{Explanation}, \script{186}{Fake proof} and \videonine{2954}{Explanation}
																		\begin{itemize}
																			\item \script{188}{Proof of Eulers Theorem (f.)} and \videonine{4018}{Explanation}, \videonine{4661}{Multiplying with a permutes elements}
																		\end{itemize}
																	\end{itemize}
																	\item \script{190}{Symmetry Property}, \videonine{5414}{In not simple version a hash of message is encrypted}, \videonine{5587}{Also not send as plain text}
																\end{itemize}
															\end{minipage}
														}
													\end{minipage}
												}
											}
										child {
												node {Fast Modular Exponentiation
														\resizebox{\textwidth}{!}{
															\begin{minipage}[t]{12cm}
																\begin{itemize}
																	\item \script{191}{Definition}
																	\item \script{192}{Algorithm: Square and Multiply}, \videoten{395}{Example explanation}, \videoten{692}{Summary of algorithm}, \videoten{740}{Number of square operations and multiplications}, \videoten{761}{Pseudocode explained}, \videoten{888}{Complexity}, simple method number of operations is equal to k, for square and multiply number of operations is linear in the number of bits in the representation of k which is logarithmic, \videoten{948}{Modulo n operation}, \videoten{1021}{Used in encryption and decryption for RSA}
																\end{itemize}
															\end{minipage}
														}
													}
											}
									}
							}
						child {
								node {Hybrid Asymmetric-symm. Systems
										\resizebox{\textwidth}{!}{
											\begin{minipage}[t]{12cm}
												\begin{itemize}
													\item \script{174}{Definition and \alert{Basic key transport protocol}}
												\end{itemize}
											\end{minipage}
										}
									}
							}
					}
				child {
						node {Classification by type of encryption operations}
						% Basic Types of Ciphers
						child {
								node (substitution) {Substitution ciphers
										\resizebox{\textwidth}{!}{
											\begin{minipage}[t]{12cm}
												\begin{itemize}
													\item letters of $P$ replaced with other letters by $E()$
													\item \underline{effects:}
													\begin{itemize}
														\item $C$ hides chars of $P$ (plaintext)
														\item if $> 1$ key alphabet (polyalphabetic), $C$ dissipates high frequency chars
													\end{itemize}
													\item \underline{Sidenote:} One can say $key = 3$ or $key = \enquote{D}$, because $no(D) = 3$
												\end{itemize}
											\end{minipage}
										}
									}
								child {
										node {Monoalphabetic substitution ciphers
												\resizebox{\textwidth}{!}{
													\begin{minipage}[t]{10cm}
														\begin{itemize}
															\item in a general monoalphabetic substitution cipher each letter in $P$ is substituted by a fixed letter using a \alert{substitution table}
															\item the \alert{key} is the substitution table
															\begin{itemize}
																\item there are $26 \times 25 \times \ldots \times 3 \times 2 \times 1 = 26! \approx 2^{88}$ substitution tables ($=$ keys) as depicted \script{78}{here (f.)}, because the mapping has to be injective
															\end{itemize}
															\item \underline{Attacks:}
															\begin{itemize}
																\item \script{79}{Exhaustive search}, Brute-force attack is not feasible
																\item \script{80}{Letter Frequency Analysis (ff.)}: Statistical analysis
																\begin{itemize}
																	\item In practice, not only frequencies of individual letters can be used for an attack, but also the frequency of letter pairs (i.e., \enquote{TH} is very common in English), letter triples, etc
																	\item need better concealing of statistical frequencies and probably also longer keys to avoid exhaustive search $\rightarrow$ Polyalphabetic substitution ciphers
																\end{itemize}
															\end{itemize}
														\end{itemize}
													\end{minipage}
												}
											}
										child {
												node {Caesar Cipher
														\resizebox{\textwidth}{!}{
															\begin{minipage}[t]{12cm}
																\begin{itemize}
																	\item \script{74}{Defintion (f.)}, example key is $3$
																	\item each letter in $P$ is substituted by a fixed letter. In this special case the \enquote{key} is of length $1$, it is the shift amount
																	\item \script{76}{Attack: Exhaustive search}
																\end{itemize}
															\end{minipage}
														}
													}
											}
									}
								child {
										node {Polyalphabetic substitution ciphers
												\resizebox{\textwidth}{!}{
													\begin{minipage}[t]{12cm}
														\begin{itemize}
															\item \script{84}{Definition (f.)}
															\item several key alphabets, flatten (diffuse) somewhat the frequency distribution of letters by combining high and low distributions
															\item \underline{Attack:} If one knows the $n$, then one can break the whole text into $n$ parts and for these a statistical analysis works as for the monialphabetic substitution cipher with the unigram model
															\begin{itemize}
																\item only works if the attacker doesn't know the algorithm, this is not optimal, because one has to hide the algorithm or consider this $n$ also as part of the key, but then $n$ has to be very large to avoid a brute-force attack. Pairs are not hidden, but the unigram attack still works (\script{86}{summary here})
															\end{itemize}
														\end{itemize}
													\end{minipage}
												}
											}
										child {
												node (viginere) {Vigenère Tableaux Method
														\resizebox{\textwidth}{!}{
															\begin{minipage}[t]{12cm}
																\begin{itemize}
																	\item \script{87}{Definition (ff.)}
																	\item special case of polyalphabetic substitution with $n$ key alphabets. For each key alphabet the special case of Caesar cipher is chosen, i.e., each key alphabet can be represented by one letter
																	\item \script{88}{Tableaux}: rows are possible key characters and the columns give the encoding
																	\begin{itemize}
																		\item one can describe a row either by the shift or by saying which letter $c_i$ it mapped to which letter $c_j$ and takes $b_j$ as name of the row, choosing the last choice one can describe $n$ keys by a word of length $n$
																	\end{itemize}
																	\item \script{89}{Example}
																	\item $26^n$ different keys, one has to choose $n$ large enough
																	\item \alert{Attack:}
																	\begin{itemize}
																		\item same problem as for polyalphabetic substitution ciphers
																	\end{itemize}
																	\item \script{97}{Classification}
																\end{itemize}
															\end{minipage}
														}
													}
											}
									}
							}
						child {
								node (transposition) {Transposition (permutation) ciphers
										\resizebox{\textwidth}{!}{
											\begin{minipage}[t]{12cm}
												\begin{itemize}
													\item order of letters in $P$ rearranged by $E()$
													\begin{itemize}
														\item rearrange letters in plaintext to produce ciphertext
													\end{itemize}
													\item \script{90}{Example}
													\item \underline{effects:}
													\begin{itemize}
														\item $C$ scrambles text, hides $n$-grams for $n > 1$ (combinations of $n$ letters, e.g. th)
													\end{itemize}
												\end{itemize}
											\end{minipage}
										}
									}
								child {
										node {Rail-Fence Cipher
												\resizebox{\textwidth}{!}{
													\begin{minipage}[t]{12cm}
														\begin{itemize}
															\item columnar transposition
															\item \script{90}{Example}
															\item Key = Number of columns
															\item \underline{Attack:} Number of columns / the  key space is restricted if the text is short
														\end{itemize}
													\end{minipage}
												}
											}
									}
							}
						child {
								node (product) {Product ciphers
										\resizebox{\textwidth}{!}{
											\begin{minipage}[t]{12cm}
												\begin{itemize}
													\item combine two or more ciphers to enhance the security of the cryptosystem
													\begin{itemize}
														\item $E = E_1 + E_2 + ... + E_n$
														\item built of multiple blocks, either \alert{Substitution} or \alert{Transposition}
													\end{itemize}
													\item \underline{attack}:
													\begin{itemize}
														\item product cipher might not necessarily be stronger than its individual components used separately, \script{91}{Example}
													\end{itemize}
													\item \underline{effects:}
													\begin{itemize}
														\item can do all what Substitution and Transposition ciphers can so, more secure if used well
													\end{itemize}
												\end{itemize}
											\end{minipage}
										}
									}
								child {
										node {Two-block product cipher
												\resizebox{\textwidth}{!}{
													\begin{minipage}[t]{12cm}
														\begin{itemize}
															\item $E2(E1(P, KE1), KE2)$
															\item may be repeated to form several encryption rounds
														\end{itemize}
													\end{minipage}
												}
											}
									}
							}
					}
			}
		child {
				node (test){Basics}
				child {
						node {Security
								\resizebox{\textwidth}{!}{
									\begin{minipage}[t]{12cm}
										\begin{itemize}
											\item \alert{Security:} No complete definition. Could define it as ability to satisfy certain security properties (\script{13}{more precise})
											\begin{itemize}
												\item tampering can be conducted at all levels, hardware is not a root of trust, one only can't change it afterwards
												\item is also a matter of cost, most secure system is a system that doesn't do anything (don't grant access to anybody), but the system also has to be useful, not to expensive etc. (power, area, price and performance). Security processing may add considerable overhead to a resource-constrained embedded system. Only on componentn in a optimzation problem which optimizes the cost of the whole system. Today’s security features can be attacked by tomorrow’s technology
											\end{itemize}
											\item \alert{Safety:} System is designed without any error leading to unintended behavior (\alert{design time})
											\item \alert{Reliability:} A correctly designed system continues to work correctly during its \alert{life-time}
										\end{itemize}
									\end{minipage}
								}
							}
						child {
								node {Possible Actions of Adversaries, Attacks (ff.)
										\resizebox{\textwidth}{!}{
											\begin{minipage}[t]{12cm}
												\begin{itemize}
													\item \alert{side-channel attack:} using sidechannel outputs to extract information, when a chip works it produces more information that occurs on the outputs, measure e.g. energy consumption, electromagnetic field etc. (focuses on \alert{outputs}, i.e. side-channels)
													\item \alert{fault injection:} counterpart to side-channel, focuses on \alert{additional inputs}, e.g. shooting with a laser beam on certain point in a chip which causes a fault and from the effect of this fault one can derive secret information
													\begin{itemize}
														\item \script{23}{Example: Power Analysis}
													\end{itemize}
													% using outputs of chip that are not the usual outputs, 
													\item \script{25}{Example for attacked device: Smart Card}, weak 8-Bit CPU, because of cost, not best encryption methods, easy to get physical access
													\item \script{26}{Example for attacked device: RFID}, even weaker IC, then on Smart Cards
													\item \script{35}{Countermeasures}
												\end{itemize}
											\end{minipage}
										}
									}
							}
						child {
								node {Related Terms
										\resizebox{\textwidth}{!}{
											\begin{minipage}[t]{12cm}
												\begin{itemize}
													\item \alert{Vulnerability:} Weakness in the secure system
													\item \alert{Threat:} Set of circumstances that has the potential to cause loss or harm
													\item \alert{Attack:} The act of a human exploiting the vulnerability in the system
													\begin{itemize}
														\item \script{16}{relationship security vs. safety and reliability}
													\end{itemize}
												\end{itemize}
											\end{minipage}
										}
									}
							}
						child {
								node {Security properties (\script{14}{\enquote{CIAAN}})
										\resizebox{\textwidth}{!}{
											\begin{minipage}[t]{12cm}
												\begin{itemize}
													\item \alert{Confidentiality:} Protecting confidential information from being disclosed to unauthorized parties.
													\begin{itemize}
														\item no unauthorized reads
													\end{itemize}
													\item \alert{Integrity:} Ensuring that information is only modified by authorized parties.
													\begin{itemize}
														\item no unauthorized writes
													\end{itemize}
													\item \alert{Availability:} Making sure that information and systems are accessible to authorized parties when they need them
													\begin{itemize}
														\item e.g. resistance to denial-of-service attacks
													\end{itemize}
													\item \alert{Authenticity:} Ensuring that information and communication come from the source they are supposed to come from.
													\begin{itemize}
														\item receiver of email knows for sure it comes from you, achieved by e.g. digital signatures
													\end{itemize}
													\item \alert{Non-repudiation:} Ensuring that nobody can deny having performed certain actions (like sending / receiving messages, changing data etc.).
													\begin{itemize}
														\item access to system is protected by login and logging when changing file
													\end{itemize}
												\end{itemize}
											\end{minipage}
										}
									}
							}
					}
				child {
						node {Design Cycle and Threats
								\resizebox{\textwidth}{!}{
									\begin{minipage}[t]{12cm}
										\begin{itemize}
											\item \script{41}{Traditional Design Cycle of ICs}
											\item \alert{Problems:}
											\begin{itemize}
												\item \script{42}{Problem 1 (f.)}: \alert{Cost of Manufacturing}, \textit{\enquote{fabless} companies}, \script{44}{untrusted foundry (pr.)}, \script{47}{untrusted assembly (pr.}, can mark corrects IC's as devective and sell them, fab doesn't know exactly fraction of devective to total)
												\item \script{48}{Problem 2}: \alert{Design Complexity}, Company that provides \textit{IP} (intellectual property) block can be untrusted
											\end{itemize}
											\item \script{50}{New Design Cycle of ICs}, Vendors of IP Blocks not trusted, safety (not part of this lecture) and security problem, \alert{Soft IP}: Buying \textit{RTL} (Register transfer level) designs, \alert{Firm IP}: Buying \textit{Gate Level Netlists}, \alert{Hard IP}: Integrating \textit{Layout Data} bought by another company, IP Vendors would not give all information about IP how they came to this layout because it could also not trust, because one could steal it's IP, both sides have to trust, one needs methologies that ensure that IP does the right thing and on the other hand that the person who bought it, is not able to steal it
											\item \script{53}{Vulnerabilities and untrusted parties}, \alert{IC Piracy} is overproduction, selling devective \textit{out-of-spec} IC's etc., \script{58}{Counterfeiting}, \textit{Cloned IC's} when somoeone steals design data or reverse engineering, \script{59}{IC recyling}, Problem because of \alert{Bathtub Curve}
											\item \script{60}{Overview: Supply Chain Vulnerabilities}, \alert{Remark} e.g. weak processor by the name of a much better proecessor, at all points of the supply chain there are threats for the hardware security
										\end{itemize}
									\end{minipage}
								}
							}
					}
			}
	\end{mindmapcontent}
	\begin{edges}
		\edge{viginere}{blockcipher}
		\edge{aes}{product}
		\edge{aes}{blockcipher}
		\edge{finitefields}{polrings}
		\edge{dpa}{des}
		\edge{fault injection}{side channel}
	\end{edges}
	\annotation{test}{annotation}
\end{mindmap}
\end{document}
