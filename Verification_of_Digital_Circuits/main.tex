\documentclass{standalone}

%!Tex Root = ../main.tex

% ┌────────────┐
% │ Formatting │
% └────────────┘
\usepackage[english]{babel}
\usepackage[export]{adjustbox} % use c, l, r for images
\usepackage{csquotes}
\usepackage[parfill]{parskip}
\usepackage{enumitem}

% ┌──────┐
% │ Math │
% └──────┘
\usepackage{amssymb} % for black triangleright, https://tex.stackexchange.com/questions/570303/use-blacktriangleright-as-itemize-label
\usepackage{amsmath}
% \usepackage{mathtools} % for \mathclap and 
% \usepackage{breqn}

% ┌────────┐
% │ Tables │
% └────────┘
\usepackage{tabularray}
 % \UseTblrLibrary{diagbox}

% ┌────────┐
% │ Images │
% └────────┘
\usepackage{graphicx}
% \usepackage{float} % for the letter H
% \graphicspath{figures/}
\usepackage{subcaption}

% ┌────────┐
% │ Graphs │
% └────────┘
\usepackage{tikzit}
\input{tikzit.tikzstyles}

% ┌────────┐
% │ Citing │
% └────────┘
\usepackage[style=numeric]{biblatex}
\addbibresource{./Cyber_Physical_Systems.bib}
% \usepackage{cleveref}

% ┌──────────┐
% │ Diagrams │
% └──────────┘
\usepackage{tikz}
\usetikzlibrary{mindmap, shadows, backgrounds} % , calc

% ┌────────────────────┐
% │ Code hightligthing │
% └────────────────────┘
% \usepackage{minted}

% ┌────────────────────────┐
% │ Latex Programming Help │
% └────────────────────────┘
\usepackage{etoolbox}
\usepackage{xparse}
% https://tex.stackexchange.com/questions/358292/creating-a-subcounter-to-a-counter-i-created
\usepackage{chngcntr}

% ┌───────────────┐
% │ Pretty Boxes  │
% └───────────────┘
\usepackage{xcolor}
\usepackage{tcolorbox}
\tcbuselibrary{theorems}
\tcbuselibrary{skins}

% ┌──────────────┐
% │ Pseudo Code  │
% └──────────────┘
\usepackage{pseudo}

% ┌────────────┐
% │ Misc Tools │
% └────────────┘
% \usepackage{lipsum}

%!Tex Root = ../main.tex

% ┌────────────┐
% │ Formatting │
% └────────────┘
\setlength{\parskip}{0.0cm} % space between paragraphs, https://latexref.xyz/bs-par.html
\setlist[itemize]{itemsep=0cm, topsep=0cm, leftmargin=0.15cm, labelsep=0cm, listparindent=0cm}
\setlength{\parindent}{0.0cm}
\setlength{\parsep}{0.0cm}
\setlength{\partopsep}{0.0cm}

% ┌───────┐
% │ Fonts │
% └───────┘
\usepackage{fontspec}
% \newfontfamily\gyre{DejaVu Math TeX Gyre}
% colored bold
% \newcommand\alert[1]{\textcolor{SwitchColor}{\textbf{#1}}}
\newcommand\alert[1]{\textcolor{SwitchColor}{#1}}

% ┌──────────────┐
% │ Pseudo Code  │
% └──────────────┘
\newcounter{algorithm}
\setcounter{algorithm}{0}
\newtcbtheorem[use counter=algorithm]{algorithm}{\color{SecondaryColor}Algorithm}{pseudo/ruled}{alg}
% \newcommand{\ma}[1]{$\mathcal{#1}$}
% \renewcommand{\tt}[1]{{\small\texttt{#1}}}

% ┌────────┐
% │ Colors │
% └────────┘
\definecolor{PrimaryColor}{HTML}{800080}
\definecolor{PrimaryColorDimmed}{HTML}{D6D6F0}
\definecolor{SecondaryColor}{HTML}{006BB6}
\definecolor{SecondaryColorDimmed}{HTML}{E5F0F8}
\definecolor{SwitchColor}{named}{PrimaryColor}
\colorlet{BoxColor}{gray!10!white}

% ┌───────┐
% │ Links │
% └───────┘
\usepackage[allbordercolors=PrimaryColor, pdfborder={0 0 .2}]{hyperref}

% ┌─────────┐
% │ Mindmap │
% └─────────┘
\renewcommand{\labelitemi}{$\textcolor{SwitchColor}{\bullet}$}
\renewcommand{\labelitemii}{$\textcolor{SwitchColor}{\blacktriangleright}$}
\renewcommand{\labelitemiii}{$\textcolor{SwitchColor}{\blacksquare}$}
\renewcommand{\labelitemiv}{$\textcolor{SwitchColor}{\blacklozenge}$}

%!Tex Root = ../main.tex

% ┌─────────┐
% │ Mindmap │
% └─────────┘
\newlength{\leveldistance}
\setlength{\leveldistance}{25cm}

\newenvironment{edges}{\begin{pgfonlayer}{background}\draw [concept connection]}{;\end{pgfonlayer}}
\newcommand{\edge}[2]{(#1) edge (#2)}
\newcommand{\annotation}[2]{\path (#1) -- node[annotation, above, align=center, pos=0.03] {#2} (middle);}

\newenvironment{resettikz}{\pgfsetlayers{nodelayer,edgelayer}\tikzset{every node/.style={fill opacity=1.0, draw opacity=1.0, minimum size=0cm, inner sep=0pt}}}{}

\newenvironment{mindmap}{
	\begin{tikzpicture}[
			auto,
			huge mindmap,
			fill opacity=0.6,
			draw opacity=0.8,
			concept color = PrimaryColorDimmed,
			every annotation/.style={fill=BoxColor, draw=none, align=center, fill = BoxColor, text width = 2cm},
			grow cyclic,
			level 1/.append style = {
					concept color=SecondaryColorDimmed,
					level distance=\leveldistance,
					sibling angle=360/\the\tikznumberofchildren,
					% https://tex.stackexchange.com/questions/501240/trying-to-use-the-array-environment-inside-a-tikz-node-with-execute-at-begin-no
					execute at begin node=\definecolor{SwitchColor}{named}{SecondaryColor}\definecolor{SwitchColorDimmed}{named}{PrimaryColorDimmed},
				},
			level 2/.append style = {
					concept color=PrimaryColorDimmed,
					level distance=\leveldistance / 2,
					sibling angle=60,
					execute at begin node=\definecolor{SwitchColor}{named}{PrimaryColor}\definecolor{SwitchColorDimmed}{named}{SecondaryColorDimmed},
				},
			level 3/.append style = {
					concept color=SecondaryColorDimmed,
					level distance=\leveldistance / 3,
					execute at begin node=\definecolor{SwitchColor}{named}{SecondaryColor}\definecolor{SwitchColorDimmed}{named}{PrimaryColorDimmed},
				},
			level 4/.append style = {
					concept color=PrimaryColorDimmed,
					level distance=\leveldistance / 4,
					execute at begin node=\definecolor{SwitchColor}{named}{PrimaryColor}\definecolor{SwitchColorDimmed}{named}{SecondaryColorDimmed},
				},
			level 5/.append style = {
					concept color=SecondaryColorDimmed,
					level distance=\leveldistance / 5,
					execute at begin node=\definecolor{SwitchColor}{named}{SecondaryColor}\definecolor{SwitchColorDimmed}{named}{PrimaryColorDimmed},
				},
			level 6/.append style = {
					concept color=PrimaryColorDimmed,
					level distance=\leveldistance / 6,
					execute at begin node=\definecolor{SwitchColor}{named}{PrimaryColor}\definecolor{SwitchColorDimmed}{named}{SecondaryColorDimmed},
				},
			level 7/.append style = {
					concept color=SecondaryColorDimmed,
					level distance=\leveldistance / 7,
					execute at begin node=\definecolor{SwitchColor}{named}{SecondaryColor}\definecolor{SwitchColorDimmed}{named}{PrimaryColorDimmed},
				},
			level 8/.append style = {
					concept color=PrimaryColorDimmed,
					level distance=\leveldistance / 8,
					execute at begin node=\definecolor{SwitchColor}{named}{PrimaryColor}\definecolor{SwitchColorDimmed}{named}{SecondaryColorDimmed},
				},
			level 9/.append style = {
					concept color=SecondaryColorDimmed,
					level distance=\leveldistance / 9,
					execute at begin node=\definecolor{SwitchColor}{named}{PrimaryColor}\definecolor{SwitchColorDimmed}{named}{SecondaryColorDimmed},
				},
			level 10/.append style = {
					concept color=PrimaryColorDimmed,
					level distance=\leveldistance / 10,
					execute at begin node=\definecolor{SwitchColor}{named}{PrimaryColor}\definecolor{SwitchColorDimmed}{named}{SecondaryColorDimmed},
				},
			level 11/.append style = {
					concept color=SecondaryColorDimmed,
					level distance=\leveldistance / 11,
					execute at begin node=\definecolor{SwitchColor}{named}{PrimaryColor}\definecolor{SwitchColorDimmed}{named}{SecondaryColorDimmed},
				},
			level 12/.append style = {
					concept color=PrimaryColorDimmed,
					level distance=\leveldistance / 12,
					execute at begin node=\definecolor{SwitchColor}{named}{PrimaryColor}\definecolor{SwitchColorDimmed}{named}{SecondaryColorDimmed},
				},
			concept connection/.append style = {
					color = BoxColor,
				},
		]
		}{
	\end{tikzpicture}
}

\newenvironment{mindmapcontent}{
	\begin{scope}[
			every node/.style = {concept, circular drop shadow}, % draw=none
			every child/.style={concept},
		]
		}{
		;\end{scope}
}

% ┌───────┐
% │ Boxes │
% └───────┘
\DeclareTotalTCBox{\inlinebox}{ s m }
{standard jigsaw,opacityback=0,colframe=SwitchColor,nobeforeafter,tcbox raise base,top=0mm,bottom=0mm,
	right=0mm,left=0mm,arc=0.1cm,boxsep=0.1cm}
{\IfBooleanTF{#1}%
	{\textcolor{PrimaryColor}{\setBold >\enspace\ignorespaces}#2}%
	{#2}}

\DeclareTotalTCBox{\inlineboxtwo}{ s m }
{standard jigsaw,opacityback=0,colframe=SwitchColorDimmed,nobeforeafter,tcbox raise base,top=0mm,bottom=0mm,
	right=0mm,left=0mm,arc=0.1cm,boxsep=0.1cm}
{\IfBooleanTF{#1}%
	{\textcolor{SwitchColorDimmed}{\setBold >\enspace\ignorespaces}#2}%
	{#2}}

% ┌──────────────────┐
% │ Case distinction │
% └──────────────────┘
% \newtoggle{absolute}
% % \toggletrue{absolute}
% \togglefalse{absolute}
% \newcommand{\lpathgraph}[1]{\iftoggle{absolute}{/home/areo/Documents/Studium/Summaries/x/}{./}#1}

% ┌───────┐
% │ Fixes │
% └───────┘
% https://tex.stackexchange.com/questions/89467/why-does-pdftex-hang-on-this-file
% \newcommand{\colon}{\mathrel{\mathop:}}

% ┌───────┐
% │ Paths │
% └───────┘
\newcommand{\script}[2]{\href{openpdf:/home/areo/Documents/Studium/Semester_1_Master/Verification_of_Digital_Circuits/slides/annotated/Verification_of_Digital_Circuits_all_in_one.pdf:#1}{\inlinebox{#2}}}
\newcommand{\scripttwo}[2]{\href{openpdf:///home/areo/Nextcloud/Verification of Digital Circuits - WS2324/Slides/annotated/bonus/lecture08_sat_aig.pdf:#1}{\inlinebox{#2}}}
\newcommand{\videoseven}[2]{\href{https://youtu.be/ILBLSr7zO70?feature=shared&t=#1}{\inlineboxtwo{#2}}}
\newcommand{\videoeight}[2]{\href{https://youtu.be/DCDNlu0I8GA?feature=shared&t=#1}{\inlineboxtwo{#2}}}
\newcommand{\videonine}[2]{\href{https://youtu.be/DCDNlu0I8GA?feature=shared&t=#1}{\inlineboxtwo{#2}}}
\newcommand{\videoeleven}[2]{\href{https://youtu.be/dKszaFJQF3I?si=v1Qhw6OC-VxS7z7A&t=#1}{\inlineboxtwo{#2}}}
\newcommand{\videotwelve}[2]{\href{https://youtu.be/uVedKmgkKxU?si=Ne2tds-ok_b_k6i2&t=#1}{\inlineboxtwo{#2}}}


\begin{document}
\begin{mindmap}
  \begin{mindmapcontent}
    \node (middle) at (current page.center) {Verification of Digital Circuits
      \resizebox{\textwidth}{!}{
        \begin{minipage}[t]{18cm}
        \end{minipage}
      }
    }
    child {
      node {Basics}
      child {
        node {Complexity classes
          \resizebox{\textwidth}{!}{
            \begin{minipage}[t]{12cm}
              \begin{itemize}
                \item \alert{P}: Problems that can be solved in Polynomial time in the size of the input
                \begin{itemize}
                  \item sorting ($n\cdot log(n)$), deciding if sequence is sortedv, shortest path in graph, matrix mulitplication (naively in $n^3$)
                \end{itemize}
                \item \alert{NP}: $N$ for nondeterministic, can guess a solution (yes answer) and verify efficiently that it is a solution
                \begin{itemize}
                  \item \alert{complete:} It's the hardest in it's class, if can solve SAT, one can solve all the problems in $NP$
                  \begin{itemize}
                    \item traveling salesman, bring in database in certain normal form, knapsack
                    \item noone has ever found an algorithm that solves them efficiently without guessing, all have exponential runtime and also noone was able to proof that there's no efficient algorithm
                  \end{itemize}
                \end{itemize}
                \item \alert{co-NP}: Can't guess the yes answer, one can guess the no
                \begin{itemize}
                  \item noone was able to proof whether NP and co-NP are the same, so one has to disitingiush them
                  \item equivalence checking for combinational circuits is \alert{co-NP complete}, can find an input that deals different outputs, therefore it's NOT equivalent
                \end{itemize}
              \end{itemize}
            \end{minipage}
          }
        }
      }
      child {
        node {Propositional Logic
          \resizebox{\textwidth}{!}{
            \begin{minipage}[t]{12cm}
              \begin{itemize}
                \item \script{128}{Syntax inductively and as context-free grammar}
                \item \script{129}{Semantics}
                \item \script{131}{Literal, Clause, Conjunctive Normal Form (CNF) (f.)}
                \begin{itemize}
                  \item \script{170}{Notation: Set of literals, empty clause}, order of literals does not matter and duplicates do not matter, empty set has role like the neutral element of disjunction, thus the empty clause is \alert{unsatisfiable} by definition \script{172}{Notation: CNF formula set of clauses, empty formula}, empty formula is a set of clauses, if one adds an empty set of clauses nothing changes, the clauses are connected with AND, one has connect $1$ with AND such that nothing changes, thus a empty formula is \alert{satisfiayble} by definition
                  \item \script{171}{Operations on clauses} and \script{172}{Union of two CNF formulas}
                \end{itemize}
                \item \script{144}{Equivalence}
              \end{itemize}
            \end{minipage}
          }
        }
        child {
          node {Resolution
            \resizebox{\textwidth}{!}{
              \begin{minipage}[t]{14cm}
                \begin{itemize}
                  \item \script{174}{Definition, resolvent}: $L\in C_1, \neg L \in C_2, R = (C_1 - \{L\}) \cup (C_2 - \{\neg L\})$, $R = C_1 \otimes_L C_2$
                  \begin{itemize}
                    \item \script{175}{Examples}, can delete all tautology clauses, because they are useless, at the bottom example for empty clause
                  \end{itemize}
                  \item \script{176}{Resolution Lemma}: Let $F$ be a CNF formula and $R$ be the resolvent of two clauses $C_1$ and $C_2$ from $F$. Then $F$ and $F \cup \{R\}$ are \alert{(logically) equivalent}: $F \equiv F \cup \{R\}$
                  \begin{itemize}
                    \item \script{177}{Proof}, \underline{one direction}: trivial, if assignment satisfies all clauses including the resolvent, then all the clauses without the resolvent are satisfied, \underline{other direction}: formula f satisfiable and show that also resolvent is satisfied, A a satisficing assignment of formula f, A is a model of f, if $L$ is true then $C_1$ was satisfied becaus of that, $C_2$ is also satisfied, otherwise the formula would not be satisfied, but it's not satisfied because of $L$, because the $\neg L$ is false, so $C_2$ without the $L$ has to contain another literal that is satisfied and that Literal is also contained in the resolvent and therefore the resolvent is satisfied. The other case wors exactly the same way%, $L$ is assigne to false, then $C_2$ is satisfied because of the $\neg L$, but $C_1$ cannot be satisfied because of the $L$, so $C_1$ has to contain another satisfying literal and that other satisfying literal ends up in the resolvent and satisfies it
                  \end{itemize}
                  \item \script{178}{$Res(F)$ etc.}
                  \begin{itemize}
                    \item $Res(F) = F \cup \{R \mid R \text{ is the resolvent of two clauses in } F\}$
                    \item $Res^0(F) = F$
                    \item $Res^{t+1}(F ) = Res(Res^t(F)) \text{ for } t \ge 0$
                    \item $Res^*(F ) = lim_{t\ge 0}\enspace Res^t(F)$
                    \item do as long as fomrula changes, only finitely many clauses, so it has to terminate
                  \end{itemize}
                  \item \script{179}{Resolution Theorem}: A CNF formula F is \alert{unsatisfiable} \textit{iff} $\square \in Res^*(F)$
                  \begin{itemize}
                    \item \script{180}{Proof}, proof two things: 
                    \begin{itemize}
                      \item \underline{if one can dervive the empty clause the formula is unsatisfied}: Only proof one half, if one can derive the empty clause, the formula is unsatisfied, resolution lemma: can add resolvents to the formula and it's logicaly equivalent, the empty clause is a resolvent, so one can add the empty clause to the formula and every formula that contains the empty clause is unsatisfied, the formula with the resolvent is unsatisfieable, therefore the original formula is unsatisfiable
                      \item \underline{if formula is unsatisfied, one can derive the empty clause:} later eliminate variables using resolution you canot only add resolvents but you can if you do it right get rid of variables using resolution and the result is a equally satisfiable formula. You just apply resolution to eliminate all the variables and you know the result is equisatisfiable to original If one eliminates all the variables there are only two cases of formulas without variables the empty formula satisfiable or the formula that only contains the empty clause that's unsatisfiable so that means when you eliminate all the variables you end up either with empty formula empty clause and you know the outcome is equally satisfible to the original formula that means when the outcome is the empty Formula the original formula was satisfiable if the outcome is the empty close then the original formula was unsatisfiable whenever it's unsatisfied we have to get the empty Clause because it can't get the empty formula 
                    \end{itemize}
                  \end{itemize}
                \end{itemize}
              \end{minipage}
            }
          }
        }
      }
      child {
        node {Important Definitions
          \resizebox{\textwidth}{!}{
            \begin{minipage}[t]{12cm}
              \begin{itemize}
                \item \underline{essential definitions:}
                  \begin{itemize}
                    \item \alert{verification method:} Check equivalence between specification and implementation
                    \item \alert{formal verification:} Mathematical proofs of correctness
                      \begin{itemize}
                        \item one uses \alert{formal methods} to avoid and detect design errors
                        \item because validation by simulation can never cover the complete system behavior
                      \end{itemize}
                    \item \alert{security:} A system should not leak information that should be kept secret
                    \item \alert{safety:} System does what it should do, implementation behaves as said in the specification (in this lecture)
                    \item \alert{canonical:} Exactly one representation
                    \item \alert{topological sort:} A graph traversal in which each node v is visited only after all its dependencies are visited
                    \begin{itemize}
                      \item a topological ordering is possible \alert{iff} the graph is a directed acyclic graph (DAG)
                      \item \underline{algorithm:} The algorithm loops through each node of the graph, in an arbitrary order, initiating a depth-first search that terminates when it hits any node that has already been visited since the beginning of the topological sort or the node has no outgoing edges (i.e. a leaf node)
                    \end{itemize}
                  \end{itemize}
              \end{itemize}
            \end{minipage}
          }
        }
      }
      child {
        node {Design of integrated circuits
          \resizebox{\textwidth}{!}{
            \begin{minipage}[t]{12cm}
              \begin{itemize}
                \item \underline{\script{22}{design of integrated circuits}:} 
                  \begin{itemize}
                    \item start with abstract specification and make it more and more concrete
                    \item implementation of the level before is the specification of the next level
                    \item equivalence checking for single steps easier than between initial specification and final implementation, similiarity is much higher
                    % \item more abstract SystemC Specification can be above Register-Transfer-Level
                    \item \alert{Initial Specification:} Usually in natural language
                      % \begin{itemize}
                      %   \item would need a specification in formal language e.g. specification by properties, on high level called design properties
                      % \end{itemize}
                  \end{itemize}
                  \begin{enumerate}[label=\color{PrimaryColor}\arabic*.]
                    \item \alert{Register-Transfer-Level:} Registers, Operations, Memory, no exact implementation for e.g. multiplier, don't fix the details
                      % \begin{itemize}
                      %   \item[$\textcolor{SwitchColor}{\blacksquare}$] Implementation in Hardware Description Language e.g. VHDL, Verilog
                      % \end{itemize}
                    \item \alert{Gate-Level:} Gates (AND, OR, etc.)
                    \item \alert{Layout-Level:} Placement (fix places of different components / gates) and routing (fix where connections go)
                  \end{enumerate}
                  \begin{itemize}
                    \item \alert{Final implementation:} Design data for producing the chip
                  \end{itemize}
              \end{itemize}
            \end{minipage}
          }
        }
      }
    }
    child {
      node {Basic technologies}
      child {
        node {Binary decision diagrams (BDDs)
          \resizebox{\textwidth}{!}{
            \begin{minipage}[t]{8cm}
              \begin{itemize}
                \item \script{41}{Syntax}
                \item \script{42}{Semantic}
                \item \script{45}{Example}
                \item \script{46}{Drawbacks}
                \item  The \alert{size} is given by the number of non-terminal nodes
                \item \underline{Limitations of BDDs:}
                \begin{itemize}
                  \item Canonical representation has to deal with the available memory.
                  \item Not directly usable with sequential circuits.
                  \item Not directly usable when the specification is a set of properties.
                \end{itemize}
              \end{itemize}
            \end{minipage}
          }
        }
        child {
          node {Construction of ROBDDs}
          child {
            node {Forward construction / symbolic simulation
              \resizebox{\textwidth}{!}{
                \begin{minipage}[t]{12cm}
                  \begin{itemize}
                    \item \script{104}{Algorithm}
                    \item \script{105}{Example}
                  \end{itemize}
                \end{minipage}
              }
            }
            child {
              node {ITE-Operator
                \resizebox{\textwidth}{!}{
                  \begin{minipage}[t]{14cm}
                    \begin{itemize}
                      \item all the binary operation can be reduced to a call to ITE (\enquote{If-Then-Else}-Operator)
                      \item \script{90}{Definition}: $ITE(F, G, H) = (F\wedge G)\vee(\neg F\wedge H)$
                      \begin{itemize}
                        \item the expression derives from the fact that a ITE node can be interpreted as a \alert{multiplexer} with inputs $G$ and $H$ and selector $F$
                        \item $AND(F, G) = ITE(F, G, 0) = (F\wedge G)\vee(\neg F\wedge 0) = F\wedge G$
                        \item $OR(F, G) = ITE(F, 1, G) = (F\wedge 1)\vee(\neg F\wedge G) = F \vee (\neg F \wedge G) = F\vee G$
                        \item $NOT(F) = ITE(F, 0, 1) = (F\wedge 0)\vee(\neg F\wedge 1) = \neg F$
                      \end{itemize}
                    \item \script{91}{Theorem and Proof}: $ITE(F, G, H) = (\neg x_i \wedge ITE(F_{\neg x_i}, G_{\neg x_i}, H_{\neg x_i}))\vee(x_i\wedge ITE(F_{x_i}, G_{x_i}, H_{x_i}))$
                    \item \script{93}{Algorithm to Compute a new ROBDD for ITE(F, G, H) out of ROBDD's F, G and H (ff.)}
                      \begin{itemize}
                        \item \underline{Base cases:} 
                          \begin{itemize}
                            \item $ITE(1, F, G) = ITE(0, G, F) = ITE(F, 1, 0) = ITE(G, F, F) = F$
                            \item $NOT(F) = ITE(F , 0, 1) = \neg F$
                          \end{itemize}
                        \item \script{99}{Runtime}
                      \end{itemize}
                    \end{itemize}
                  \end{minipage}
                }
              }
            }
          }
          child {
            node {Backward construction
              \resizebox{\textwidth}{!}{
                \begin{minipage}[t]{12cm}
                  \begin{itemize}
                    \item \script{116}{Algorithm}
                    \item \script{120}{Example}
                  \end{itemize}
                \end{minipage}
              }
            }
            child {
              node {Substitution operator, compose
                \resizebox{\textwidth}{!}{
                  \begin{minipage}[t]{12cm}
                    \begin{itemize}
                      \item \script{117}{Definition}: $compose(F, G, x_i)(x_1, \ldots , x_n) = F (x_1 , \ldots , x_{i-1}, G (x_1 , \ldots , x_n), x_{i+1}, \ldots , x_n)$
                      \item \script{118}{Theorem}: $compose(F, G, x_i) = ITE(G, F_{x_i}, F_{\neg x_i})$
                      \begin{itemize}
                        \item \script{119}{Proof}
                      \end{itemize}
                    \end{itemize}
                  \end{minipage}
                }
              }
            }
          }
        }
        child {
          node {Reduced Ordered Binary Decision Diagrams (ROBDDs)
            \resizebox{\textwidth}{!}{
              \begin{minipage}[t]{12cm}
                \begin{itemize}
                  \item A BDD G is a Reduced Ordered Binary Decision Diagram iff it is ordered and reduced
                    \begin{itemize}
                      \item In this context, reduced means that none of the reduction rules (\enquote{isomorphism} and \enquote{Shannon}) can be applied (anymore).
                    \end{itemize}
                  \item \script{57}{Example}
                  \item (\script{62}{Remark})
                  \item \script{65}{Proof of the Canonicity of ROBDDs}
                  \begin{itemize}
                    \item \script{64}{Definition Isomorphism}
                  \end{itemize}
                \end{itemize}
              \end{minipage}
            }
          }
          child {
            node {Reduction Rules
              \resizebox{\textwidth}{!}{
                \begin{minipage}[t]{12cm}
                  \begin{itemize}
                    \item OBDDs can be reduced to ROBDDs in layers starting from the terminal nodes
                  \end{itemize}
                \end{minipage}
              }
            }
            child {
              node {\enquote{Isomorphism} reduction
                \resizebox{\textwidth}{!}{
                  \begin{minipage}[t]{12cm}
                    \begin{itemize}
                      \item \script{51}{Definition}
                      \item \script{52}{Example}
                    \end{itemize}
                  \end{minipage}
                }
              }
            }
            child {
              node {\enquote{Shannon} reduction
                \resizebox{\textwidth}{!}{
                  \begin{minipage}[t]{12cm}
                    \begin{itemize}
                      \item \script{53}{Definition and Example}
                    \end{itemize}
                  \end{minipage}
                }
              }
            }
          }
          child {
            node {Ordered BDDs (OBDDs)
              \resizebox{\textwidth}{!}{
                \begin{minipage}[t]{12cm}
                  \begin{itemize}
                    \item A BDD G over the set of variables $X_n = \{x_1, \ldots, x_n\}$ is ordered, iff it is \alert{free} and the \alert{variables on every path} from the root to a terminal node occur in the \alert{same order}
                    \item \script{48}{information about variable order}
                    \item \script{49}{example}
                    \item This is still not enough, there can be redundancy in the OBDD
                  \end{itemize}
                \end{minipage}
              }
            }
            child {
              node {Free BDDs
                \resizebox{\textwidth}{!}{
                  \begin{minipage}[t]{12cm}
                    \begin{itemize}
                      \item A BDD G is free, iff each variable along every path from the root to a terminal node occurs at most once.
                      \item \script{47}{Example}
                    \end{itemize}
                  \end{minipage}
                }
              }
            }
          }
        }
        child {
          node {Shannon theorem
            \resizebox{\textwidth}{!}{
              \begin{minipage}[t]{8cm}
                \begin{itemize}
                  \item \script{43}{Definition}: $F = (\neg x_i\wedge F_{\neg x_i})\vee(x_i\wedge F_{x_i})$
                  \item \script{44}{Proof}
                  \item for BDD's Composition rule and Decomposition rule have the same structure
                \end{itemize}
              \end{minipage}
            }
          }
        }
        child {
          node {Use of ROBDDs in Verification
            \resizebox{\textwidth}{!}{
              \begin{minipage}[t]{12cm}
                \begin{itemize}
                  \item the size of a ROBDD depends strongly on the variable order $\pi$ one chooses
                  \begin{itemize}
                    \item \script{81}{example (f.)}
                  \end{itemize}
                  \item every cofactor regarding the first n variables consitutes a different Boolean function
                  \begin{itemize}
                    \item $2n$ different cofactors
                    \item \script{83}{example}
                  \end{itemize}
                \end{itemize}
              \end{minipage}
            }
          }
          child {
            node {Finding optimal variable order
              \resizebox{\textwidth}{!}{
                \begin{minipage}[t]{8cm}
                  \begin{itemize}
                    \item \alert{Theorem (Bollig, Savicky, Wegener, 1994):} Given a ROBDD with variable order $\pi$, the problem of finding a new variable order $\pi\prime$ with minimal ROBDD-size is NP-Complete
                      \begin{itemize}
                        \item \underline{but there are \alert{Heuristics}:}
                          \begin{itemize}
                            \item define an initial variable order based on the circuit representation
                            \item perform dynamic reordering within an existing variable order to reduce the size of the ROBDD
                          \end{itemize}
                      \end{itemize}
                  \end{itemize}
                \end{minipage}
              }
            }
            child {
              node {Define an initial variable order
                \resizebox{\textwidth}{!}{
                  \begin{minipage}[t]{12cm}
                    \begin{itemize}
                      \item \script{107}{Method of Malik}
                      \item \script{108}{Example}
                    \end{itemize}
                  \end{minipage}
                }
              }
            }
            child {
              node {Dynamic modification of the variable ordering
                \resizebox{\textwidth}{!}{
                  \begin{minipage}[t]{12cm}
                    \begin{itemize}
                      \item \script{110}{General Algorithm}
                      \begin{itemize}
                        \item \script{112}{Sifting}
                        \begin{itemize}
                          \item \script{113}{Example}
                          \item \script{114}{Runtime}
                        \end{itemize}
                      \end{itemize}
                    \end{itemize}
                  \end{minipage}
                }
              }
            }
          }
          child {
            node {Not applicable for all pratical functions
              \resizebox{\textwidth}{!}{
                \begin{minipage}[t]{12cm}
                  \begin{itemize}
                    \item \alert{Theorem (Shannon):} \enquote{Almost every} Boolean function $F\colon\, \mathbb{B}^n\rightarrow\mathbb{B}$ requires more than $(2^n / n) 2$-input gates for an optimal implementation
                    \begin{itemize}
                      \item holds also for ROBDDs, because they can be seen as multiplexer circuits
                      \item \underline{should we care?}: not really, because one does equivalence checking mostly for circuits that are not exponential (the others that are not \enquote{almost every} usually occur, because we're interested in functions with structure)
                    \end{itemize}
                    \item \alert{Lemma (Bryant, 1986):} Independently from the variable order, multiplication is representable with ROBDDs only with exponential complexity in the bit-width
                    \begin{itemize}
                      \item there exist actually polynomial mutlipliers, but there are no ROBDDs with polynomial complexity in the bit-width for them
                    \end{itemize}
                  \end{itemize}
                \end{minipage}
              }
            }
          }
        }
      }
      child {
        node {Satisfiability solvers (SAT / QBF)
          \resizebox{\textwidth}{!}{
            \begin{minipage}[t]{12cm}
              \begin{itemize}
                \item \script{130}{Satisfiable, Model, Unsatisfiable}
                \item \script{133}{Satisfiability-Problem (SAT-Problem)}
                \item \script{164}{Complexity of the SAT Problem}, NP-complete
                % what does NP-Complete mean summary: https://youtu.be/doDYnn9sXWg?feature=shared&t=2286
                % complexity of equivalence checking: https://youtu.be/doDYnn9sXWg?feature=shared&t=2586
                % restrictions ot SAT: https://youtu.be/doDYnn9sXWg?feature=shared&t=2715
                % horn formulas: https://youtu.be/doDYnn9sXWg?feature=shared&t=2761
                \item \script{165}{Practice and Applications}, in worst case all algorithm have exponential runtime, but worst case and pratical case are different
                \item \script{167}{Overview on SAT Algorithms (f.)}
                \begin{itemize}
                  \item \script{167}{complete:} run until either they have found a solution or they can prove there's none, decide all formulas if one has enough time, always terminate and always give right answer
                  \item \script{168}{incomplete methods:} typically can only find satisfying assignments, if one has a formula with many satisfying assignments, if one has a unsatisfiable formula, they will not terminate, they just look for solutions that modify assignment check again etc.
                  \begin{itemize}
                    \item scheduling problem, need to find good order, one knows that there's an order that works 
                    \item in verification they play no role at all, because verification problems either unsatifiying assignments or a few not working corner cases and then incomplete methods are not good at all, either they don't terminate or need very long time to find one of these rare satisfying assignments. Therefore in verification typically only have these complete methods that systematically search for satisfying assignments and can also prove that the formula is unsatisfiable
                    \item typically verification is applied if you cannot find any wrong answers anymore by just doing simulation
                  \end{itemize}
                \end{itemize}
                \item \underline{different kinds of SAT:} $k$-SAT, every clause exactly $k$-literals, $3$-SAT, $2$-SAT (efficiently solvable)
                \begin{itemize}
                  \item \alert{Horn-Formulas:} arbitrary length but contain at most on positive literal (or other way round at most one negative literal), relevance in Prolog, Prolog Statements are Horn clauses, clause with only one literal (if 0 and 1 at the same time, then unsatisfiable), otherwise end up with formula where all clauses have at least length $2$, that means every clause contains a negative literal, if there are no unit clauses, the formula can be satisfied by simply setting all remaining variables negative, can be done efficiently without backtracking
                \end{itemize}
              \end{itemize}
            \end{minipage}
          }
        }
        child {
          node {Naive method
            \resizebox{\textwidth}{!}{
              \begin{minipage}[t]{12cm}
                \begin{itemize}
                  \item \script{181}{Algorithm}
                  \item \script{182}{Complexity}, is limited by the number of different possible clauses (cnf formula is a \alert{set} of \alert{clauses}, not of maxterms, so there are $3$ states and not only $2$)
                  \item \script{183}{Examples (ff.)}, resulting tautology $(x_2, \neg x_2)$ of first and fourth can be ignored
                  \begin{itemize}
                    \item we do not have to resolve two Clauses from F again because we already did that so we always take either two Clauses from the new ones orthird one Clause from F and one new Clause 
                  \end{itemize}
                \end{itemize}
              \end{minipage}
            }
          }
        }
        child {
          node {DP-Algorithm
            \resizebox{\textwidth}{!}{
              \begin{minipage}[t]{14cm}
                \begin{itemize}
                  \item based on \alert{variable elimination} method and uses a couple of optimizations:
                  \begin{itemize}
                    \item \script{189}{Subsumption check}: Let $C_1$ and $C_2$ be two clauses. $C_1$ subsumes $C_2$ iff all literals occurring in $C_1$ also occur in $C_2$: $C_1 \subseteq C_2$
                      \begin{itemize}
                        \item \alert{Idea:} To satisfy a CNF formula $F$, all clauses need to be satisfied, in particular $C_1$. Since all literals of $C_1$ are also contained in $C_2$, every satisfying assignment of $C_1$ also satisfies $C_2$. Therefore $C_2$ does not need to be considered separately and can be deleted. One has to satify $C_1$ in any case for a satifying assigment, if on would satisify the additional literal in $C_2$, still the literals in $C_1$ have to be satisfied that would also satisfy $C_2$
                        % \item the other way does not know if you satisfy C2 then you can satisfy using the additional literal and that does not help for satisfying C1 so you need to set up one more literal but that's not necessary because the one in C1 is sufficient
                        \item \script{190}{Example}
                      \end{itemize}
                    \item \script{191}{Pure literal detection}: Let $F$ be a CNF-formula and $L$ a literal contained in $F$. $L$ is a \alert{pure literal} \textit{iff} $L$ is contained in $F$ either only positive or only negative, but not both $L$ and $\neg L$ appear in $F$. For satsifyiability one just needs one satisfying assignment, for making the literal true and removing clauses one will in every case find a satisfying assignment, but if one would make the literal false, there would be remaning clauses that have to be satisfied and could cause contradictions with the clauses not containing the literals. The first case would only have to satisfy the clauses without this literal, while in the second clauses the clauses without this literal and the ones that contained this literal would have to be satisfied, so one in always on the better side by choosing the literal to be true
                    \begin{itemize}
                      \item \alert{Idea:} Delete from $F$ all clauses which contain a pure literal. They can be satisfied by an according assignment to $L$. This cannot prevent any other clause from being satisfied, because $\neg L$ does not appear in $F$
                      \item \script{192}{Example}
                    \end{itemize}
                    \item these optimizations are not complete in the sense they are sufficient to solve the formula so we need one potentially expensive operation which does the the actual work and that is Variable Elimination
                    \item \script{193}{Variable Elimination}: The DP-algorithm applies \alert{resolution} to eliminate a variable $x_i$ completely from the formula, i. e., all occurrences of $x_i$ and $\neg x_i$. The goal is to reduce the number of variables.
                      \begin{itemize}
                        \item \script{194}{$P$, $N$ and $W$}
                        \item \script{195}{$P \otimes_{x_i} N$}
                        \item \script{196}{Theorem: $F = P \cup N \cup W$ and $F' = (P \otimes_{x_i} N) \wedge W$ satisfiability equivalent}
                          \begin{itemize}
                            \item \script{196}{Proof} and \href{/home/areo/Documents/Studium/Summaries/Verification_of_Digital_Circuits/figures/lecture06_sat_36_05.pdf}{\inlinebox{More detailed proof}}, replace the unions with and. %we still have to apply distributivity for that we take one Clause from the P one Clause from the n and take their Union so and do that with all the combinations what do we get we pick one Clause from that set that's a the same Clause as in p but x i is set to zero so x i is removed and we take one Clause from n the original clause contained the not x i with an x i to one so that's removed because it's zero that's just a resolvent 
                              %get all these ressolvants by applying distributivity to these clauses but we have to pick one element from P one element from n and take the or that's the same as the resolving that you get by picking one clause from P one Clause from n and then resolving it on x i that removes the x i from the p Clause the not x_i from the n clause the same have the W plus all resolvents 
                              Here we see exactly what variable elimination with the resolution does replaces x_i once by 0 once by 1 takes the or because one of the two parts has to be satisfied bring it back to CNF and you get the same as the resolution
                            \item we can reduce checking satisfiability of a formula $F$ to checking satisfiability of $F'$. $F'$ does not contain the variable $x_i$ anymore. If $F'$ is satisfiable, this also holds for $F$ and vice-versa
                            \item will sooner or later terminate because you only have a finite number of variables
                          \end{itemize}
                        \item choose an appropriate variable $x_i$, perform resolution between all pairs of clauses which contain $x_i$ and $\neg x_i$, resp., and replace the original clauses (containing $x_i$ or $\neg x_i$) by the resolvents
                        \item usually the number of resolvents is larger than the number of original clauses which are replaced, i. e., the formula grows in size during variable elimination
                        \item \script{201}{Example}, tautology ignored, at the end one can eliminate X3 if one wants that gives the empty Clause so the formula was unsatisfiable
                      \end{itemize}                             
                    \item \script{202}{Davis-Putnam Algorithm}, unit Clause: we delete all the Clauses that that contain L because that s these are satisfied and from the remaining Clauses that unit resolution step is we resolve all the Clauses that contain not L with the unit Clause that removes the not L from all clauses then we call DP on the simplified formula so delete all satisfied Clauses delete all unsatisfied literals and call it recursively. If all doesn't help one elimantes a variable in the last step so you cannot get an infinite sequence of DP calls
                  \end{itemize}
                  \item the optimizations improve the runtime behavior in practice, but not the worst case complexity of the naı̈ve method, based on resolution plus a few optimizations%, they improve the running time in practice and the memory consumption but not in the worst case
                  \begin{itemize}
                    \item the main problem here is memory consumption so generating all these resolvents is costly
                  \end{itemize}
                \end{itemize}
              \end{minipage}
            }
          }
        }
        child {
          node {DLL/DPLL-Algorithm
            \resizebox{\textwidth}{!}{
              \begin{minipage}[t]{12cm}
                \begin{itemize}
                  \item circumvents the memory problems of resolution-based techniques through a distinction of cases, which leads to \alert{depth-first search}
                  \item \alert{Idea:} If a CNF formula $F$ is satisfiable, then for a variable $x_i$ either $x_i = 1$ or $x_i = 0$ must hold $\Rightarrow$ try both cases one after the other
                  \item \script{206}{DLL-Algorithm}:
                  \begin{itemize}
                    \item Assign $L$ to $1$. Delete all clauses containing $L$. Delete all occurrences of $\neg L$ % At the end we first set the literal to True, try to solve the formula we get by that, if it's satisfiable we are finished otherwise we flip the value to false try again and if that's also unsatisfiable the formula is unsatisfiable 
                    \item \alert{Unit Clause}: A clause consisting of a single literal $L$ is called a unit clause. $L$ is the corresponding unit literal. For $P \otimes_{x_i} N$ with $N =\emptyset$ the result is $\emptyset$, because the condition in the set definition of the $P \otimes_{x_i} N$ is never satisfied, so one gets an empty formula
                    \item \script{208}{Example (ff.)}
                    \item \script{226}{Summary}
                    \item the running time is still exponential because in the worst case we just have to to try all the assignments in N variables and there are 2^n assignments. \underline{Memory Problem:} When we have to do backtracking we need to go back to the old formula so you have to keep the old copy and you create a new formula with the modifications. With many variables the depth of the DFS Tree can be the number of variables
                  \end{itemize}
                \end{itemize}
              \end{minipage}
            }
          }
        }
        child {
          node {Modern SAT-Algorithms
            \resizebox{\textwidth}{!}{
              \begin{minipage}[t]{14cm}
                \begin{itemize}
                  \item \underline{Improvements:} we do we actually modify the formula we can just say oh that clause is satisfied that's one bit of information and we can keep an assignment of the variables then you can just look it up if a variable is unsatisfied or not that's one major Improvement and of course we can carefully select which literal to assign we can learn from the conflicts so we can find out what are the reasons why the formula is unsatisfied not always all decisions are responsible for a unsatisfied formula but only a subset and we want to find out which subset 
                  \item \script{237}{Basic techniques of today’s SAT solvers}
                    \begin{itemize}
                      \item preprocessing
                      \item alternation between ...
                        \begin{itemize}
                          \item choose the next decision variable, decide its value
                          \item boolean Constraint Propagation / Unit Propagation
                          \item if applicable, conflict analysis and backtracking
                        \end{itemize}
                      \item at some fixed points of the search process
                        \begin{itemize}
                          \item Unlearning (some conflict clauses)
                          \item Restarts
                        \end{itemize}
                    \end{itemize}
                  \item \script{238}{Algorithm}
                \end{itemize}
              \end{minipage}
            }
          }
          child {
            node {Preprocessing
              \resizebox{\textwidth}{!}{
                \begin{minipage}[t]{12cm}
                  \begin{itemize}
                    \item \script{240}{in the algorithm}
                    \item \script{241}{goal, practical observations, find good compromise}
                    \item \script{242}{preprocessing techniques}
                  \end{itemize}
                \end{minipage}
              }
            }
            child {
              node {Unit Propagation
                \resizebox{\textwidth}{!}{
                  \begin{minipage}[t]{12cm}
                    \begin{itemize}
                      \item if omitted, then immediately performed during solving process
                      \item \underline{procedure:}
                      \begin{itemize}
                        \item identify unit clauses in the CNF
                        \item assign units, simplify
                        \item ... until the CNF is free from unit clauses
                      \end{itemize}
                    \end{itemize}
                  \end{minipage}
                }
              }
            }
            child {
              node {Unit Propagation Lookahead (UPLA)
                \resizebox{\textwidth}{!}{
                  \begin{minipage}[t]{12cm}
                    \begin{itemize}
                      \item Fix a variable $x_i$ to $0$, check implications; then change its value to $x_i = 1$, check implications. Simplify the formula exploiting the \script{244}{consequences}
                      \item \script{245}{Advantages and Disadvantages}
                    \end{itemize}
                  \end{minipage}
                }
              }
            }
            child {
              node {Self-subsuming resolution
                \resizebox{\textwidth}{!}{
                  \begin{minipage}[t]{12cm}
                    \begin{itemize}
                      \item Appply resolution to two clauses and resolution result subsumes one of the clauses. Saves literals
                      \item \script{246}{Example}
                    \end{itemize}
                  \end{minipage}
                }
              }
            }
            child {
              node {Elimination by Resolution% / clause distribution ?
                \resizebox{\textwidth}{!}{
                  \begin{minipage}[t]{12cm}
                    \begin{itemize}
                      \item \alert{variable elimination} applied to a variable of the CNF formula
                      \item it is applied only if it leads to a simplification of the formula
                      \item \script{247}{Example}
                    \end{itemize}
                  \end{minipage}
                }
              }
            }
            child {
              node {Variable elimination by substitution
                \resizebox{\textwidth}{!}{
                  \begin{minipage}[t]{12cm}
                    \begin{itemize}
                      \item if clauses represent a gate like in the Tseitin transformation, these clauses can be removed and then every occurence of the variable that corresponds to the output of the gate has to be replaced by the other side of the equation for the remaning clauses
                      \item potentially saves variables, literals and clauses, it is applied only if it leads to a simplification of the formula
                      \item \script{248}{Example}
                    \end{itemize}
                  \end{minipage}
                }
              }
            }
            child {
              node {Forward subsumption
                \resizebox{\textwidth}{!}{
                  \begin{minipage}[t]{12cm}
                    \begin{itemize}
                      \item Test if a clause generated during one of the other preprocessing techniques is subsumed by one clause of the initial formula
                      \item Remove all the clauses subsumed
                    \end{itemize}
                  \end{minipage}
                }
              }
            }
            child {
              node {Backward subsumption
                \resizebox{\textwidth}{!}{
                  \begin{minipage}[t]{12cm}
                    \begin{itemize}
                      \item Test if a clause generated during one of the other preprocessing techniques subsumes one (or more) clauses of the initial formula
                      \item Remove all the clauses subsumed
                    \end{itemize}
                  \end{minipage}
                }
              }
            }
            child {
              node {Blocked Clause Elimination
                \resizebox{\textwidth}{!}{
                  \begin{minipage}[t]{12cm}
                    \begin{itemize}
                      \item let $C$ be a clause and $\ell \in C$ a literal, $C$ is blocked by $\ell$ \texttt{iff} for all clauses $D$ with $\neg\ell \in D$ holds: $C \otimes_{\ell} D$ is a tautology
                      \item \alert{Blocked clauses} can be deleted without changing satisfiability
                    \end{itemize}
                  \end{minipage}
                }
              }
            }
            child {
              node {Equivalent Literals
                \resizebox{\textwidth}{!}{
                  \begin{minipage}[t]{12cm}
                    \begin{itemize}
                      \item take only the binary clauses $\{\ell, \kappa\}$
                      \item create the \alert{binary clause graph} $G = (V, E)$ with
                      \begin{itemize}
                        \item $V = \{x, \neg x \mid x \text{ variable}\}$
                        \item $E = \{(\neg \ell, \kappa), (\neg \kappa, \ell) \mid \{\ell, \kappa\} \in \varphi\}$
                      \end{itemize}
                      \item all literals in strongly connected components can be replaced by one representative
                      \item if there is a path from $\neg \ell$ to $\ell$ for some literal $\ell$, we can replace $\ell$ by $1$
                    \end{itemize}
                  \end{minipage}
                }
              }
            }
          }
          child {
            node {Improvements and Differences to DLL
              \resizebox{\textwidth}{!}{
                \begin{minipage}[t]{14cm}
                  \begin{itemize}
                    \item \script{224}{Improvements to DLL}
                    \item \underline{Differences:}
                      \begin{itemize}
                        \item \script{228}{Difference $1$: Recursion and modification of formula (f.)}
                          \begin{itemize}
                            \item \underline{DLL}:
                              \begin{itemize}
                                \item recursive procedure
                                \item for the transition from the recursion level $r$ to the level $r + 1$ the given formula is modified (clauses being satisfied are removed and \enquote{unsatisfied literals} are erased)
                                \item for backtracking from level $r + 1$ to $r$ the original (sub)formula at level $r$ must be restored
                              \end{itemize}
                            \item \underline{Modern SAT}:
                              \begin{itemize}
                                \item non-recursive procedure 
                                \item apart from special cases, during the search process neither satisfied clauses nor resolved literals are removed from the CNF formula no removal from CNF formula
                                \item usually the \alert{pure literal} rule is not applied. \alert{Subsumption check} is applied by modern SAT-algorithms only during preprocessing. \alert{Variable elimination} is applied by modern SAT during preprocessing. Here elimination of a variable is only done if it reduces the formula size (or only slightly increases it)
                              \end{itemize}
                          \end{itemize}
                        \item \script{230}{Difference $2$: Unit clause definition (f.)}
                          \begin{itemize}
                            \item In DLL a clause is made of exactly one literal. In modern procedures also the clauses where all the literals but one are assigned with negated polarity are denoted with the term unit clause. \script{231}{Example of implication}
                            \item \script{232}{Boolean Constraint Propagation (BCP) or Unit Propagation and example}: Determining all the implications forced by the assignment of a variable
                          \end{itemize}
                        \item \script{233}{Difference $3$: Contradiction / conflict (f.)}
                          \begin{itemize}
                            \item empty clause for DLL, Unsatisfied clause for modern SAT algorithms. \script{233}{Example}
                          \end{itemize}
                        \item \script{234}{Difference $4$: Conflict analysis and backtracking: (f.)}
                          \begin{itemize}
                            \item \underline{DLL:} The combination of the previously done decisions will always be considered as the origin of a conflict. Backtracking (recursive back tracing) to the recursion level of the last \enquote{branching} in which one case for a variable assignment has not been explored yet. If none exists the given CNF formula is unsatisfiable
                            \item \underline{Modern SAT:} Complex and deep analysis of the conflict setting, because not all the previously made \enquote{branchings} must be involved in the current conflict Derivation via resolution and learning of a \alert{conflict clause} for the given formula. The conflict clause avoids to run into the identical conflict again by including all the literals that are responsible (because of their assignment) for the current contradiction. Backtrack with the help of the conflict clause or output UNSATISFIABLE
                          \end{itemize}
                      \end{itemize}
                  \end{itemize}
                \end{minipage}
              }
            }
          }
        }
      }
      child {
        node {And-inverter graphs (AIGs)
        }
      }
    }
    child {
      node (test) {Property checking
        \resizebox{\textwidth}{!}{
          \begin{minipage}[t]{12cm}
            \begin{itemize}
              \item Prove that a system specifies a set of properties
            \end{itemize}
          \end{minipage}
        }
      }
      child {
        node {Specification of properties with temporal logics}
      }
      child {
        node {Algorithms for checking properties of circuits}
      }
      child {
        node {Bounded model checking using SAT solvers}
      }
      child {
        node {Unbounded model checking}
        child {
          node {K-Induktion}
        }
        child {
          node {Craig interpolation}
        }
        child {
          node {Property directed reachability (PDR)}
        }
      }
    }
    child {
      node {Equivalence checking
        \resizebox{\textwidth}{!}{
          \begin{minipage}[t]{12cm}
            \begin{itemize}
              \item Prove that two designs have the same (functional) behavior
              \item Given two combinational circuits (i.e., without memory), do they compute the same boolean function?
              \item So one can't do better in all cases, doesn't work for all practical cases, but for many:
              \begin{itemize}
                \item Combinational Equivalence Checking is NP-hard, if there would be polynomial data in a canonical datastructure one would just have to translate both circuits into this datastructure and comparison is easy, can't assume there's a datastructure with polynomial size which is canonical
                \item some function tables are so random, it's not possible to compress them
                \item canonical disjunctive normal form, then one has as many terms as elements in the ON-Set
              \end{itemize}
            \end{itemize}
          \end{minipage}
        }
      }
      child {
        node {Combinational circuits}
        child {
          node {Application of BDDs and SAT for equivalence checking
            \resizebox{\textwidth}{!}{
              \begin{minipage}[t]{12cm}
                \begin{itemize}
                  \item \alert{Equivalence Checking Method:} Convert each circuit into a datastructure that has exactly one representation of each Boolean function. Then compare these representations
                  \item in \alert{BDD-based} equivalence checking the limiting resource is the \alert{available memory}, whereas in the \alert{SAT-based} approach this is the \alert{runtime} (exponential in the size of the formula) of the solving algorithm
                  \begin{itemize}
                    \item size of the formula will be linear in the size of the circuit, so if one can get the circuit into memory, one can also get the formula into memory
                  \end{itemize}
                \end{itemize}
              \end{minipage}
            }
          }
          child {
            node {SAT-based Equivalence Checking
              \resizebox{\textwidth}{!}{
                \begin{minipage}[t]{12cm}
                  \begin{itemize}
                    \item \script{135}{Approach}
                    \item \script{139}{Miter Circuit with multiple outputs (f.)}, convert Miter Circuit into propositional logic formula with the property of being satisfied if the miter circuit outputs a $1$ and the Miter circuit has the property that it's output is $1$ for a certain input assignment if the circuits behave differently
                    \item usually SAT-algorithms take as input only CNF formulas; that means the Boolean function of the miter circuit must be translated into a CNF representation
                    \item \script{162}{Example}
                  \end{itemize}
                \end{minipage}
              }
            }
            child {
              node {Conversion of a Propositional Logic Formula into CNF
                \resizebox{\textwidth}{!}{
                  \begin{minipage}[t]{12cm}
                    \begin{itemize}
                      \item every propositional logic formula $F$ can be translated into an equivalent CNF formula $F'$
                      \begin{itemize}
                        \item \script{144}{Proof}
                      \end{itemize}
                    \end{itemize}
                  \end{minipage}
                }
              }
              child {
                node {Satisfiability equivalent CNF (Tseitin Transformation)
                  \resizebox{\textwidth}{!}{
                    \begin{minipage}[t]{14cm}
                      \begin{itemize}
                        \item it doesn't work good to turn formula into equivalent CNF formula, instead turn miter circuit into formula that is satisfiable \textit{iff} the original two circuits are not equivalent. Don't need the same output for all possible assignments, only need satisfiable \textit{iff} not equivalent. Not logically equivalent to normal CNF, 
                          \item one gives up \alert{logical equivalence} and only requires \alert{equisatisfiability} because because in the formula $X\oplus Y$ you cannot assign the output variable $Z$ in the wrong way, but in the formula $Z \equiv X \oplus Y$ you can, because of the additional variables that would not be assigned in the original formula
                        \item define for $F$ an \alert{equisatisfiable} CNF $F'$ that is satisfiable \textit{iff} $F$ is satisfiable. Every line in the circuit has variable, not only the outputs % ist satisfiable gdw. andere seite....
                        \item \script{152}{Algorithm for conversion into \alert{satisfiability equivalent} CNF}
                        \item \script{153}{Gates}
                        \item \script{155}{Example (ff.)}, satisfying assignments of the formula are exactly the consistents assignments of the circuit, all the output has to be derived using the gate function from the gate, that does not mean one gets a $1$ at the output \script{157}{Example for reason for unit clause in the last line}, all clauses satisfied but output is $0$, can only satisfy this clause if assign output of the circuit to $1$, first 3 lines say that we need a consistent assignment, the output of each gate has to match the gate function applied to the inputs and last line says output must be $1$, this formula is satisfiable \textit{iff} there's a input assignment that makes the output $1$
                        \item as long as for the CNF representation of each single gate only a constant number of clauses is required, the number of clauses in the final CNF will be linear in the number of gates in the circuit (the same holds for the size of the formula)
                        \item \script{159}{Size comparison to equivalent CNF}, linear in the size of the circuit
                      \end{itemize}
                    \end{minipage}
                  }
                }
              }
              child {
                node (convcnf) {Equivalent CNF
                  \resizebox{\textwidth}{!}{
                    \begin{minipage}[t]{14cm}
                      \begin{itemize}
                        \item \script{145}{Algorithm for conversion into \alert{equivalent} CNF}, move nots inside the formula until they only appear directly in front of variables, elimnate double negation, move ands outside with distributivity
                          \begin{itemize}
                            \item \script{148}{Examples (ff.)}, already shortest CNF for the formulas
                          \end{itemize}
                        \item \script{146}{Size of a formula}, in wost case formula increases exponentialy in size
                          \begin{itemize}
                            \item \script{147}{Proof}
                          \end{itemize}
                      \end{itemize}
                    \end{minipage}
                  }
                }
              }
            }
          }
        }
        child {
          node {Exploitation of structural information}
        }
      }
      child {
        node {Sequential circuits}
        child {
          node {Product automata}
        }
        child {
          node {Characteristic functions}
        }
        child {
          node {Image and pre-image computations}
        }
        child {
          node {BDD-based methods for proving equivalence of two sequential circuits}
        }
        child {
          node {Generation of counterexamples}
        }
      }
    }
  \end{mindmapcontent}
  \begin{edges}
    \edge{test}{middle}
  \end{edges}
  \annotation{test}{annotation}
\end{mindmap}
\end{document}
