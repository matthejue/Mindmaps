\documentclass[landscape, a4paper]{article}

%!Tex Root = ../main.tex

% ┌────────────┐
% │ Formatting │
% └────────────┘
\usepackage[english]{babel}
\usepackage[top=0cm,bottom=0cm,left=0cm,right=0cm]{geometry}
\usepackage[export]{adjustbox} % use c, l, r for images
\usepackage{csquotes}
\usepackage[parfill]{parskip}
\usepackage{fontspec}
% \usepackage{anyfontsize}
% \usepackage[]{enumitem}

% ┌──────┐
% │ Math │
% └──────┘
\usepackage{amssymb} % for black triangleright, https://tex.stackexchange.com/questions/570303/use-blacktriangleright-as-itemize-label
\usepackage{amsmath}
\usepackage{mathtools} % for \mathclap and 
\usepackage{breqn}

% ┌────────┐
% │ Tables │
% └────────┘
\usepackage{tabularray}
 % \UseTblrLibrary{diagbox}

% ┌────────┐
% │ Images │
% └────────┘
\usepackage{graphicx}
% \usepackage{float} % for the letter H
% \graphicspath{figures/}
\usepackage{subcaption}

% ┌────────┐
% │ Graphs │
% └────────┘
\usepackage{tikzit}
\usepackage{tikz}
\usetikzlibrary{backgrounds}
\usetikzlibrary{arrows}
\usetikzlibrary{shapes,shapes.geometric,shapes.misc}

% this style is applied by default to any tikzpicture included via \tikzfig
\tikzstyle{tikzfig}=[baseline=-0.25em,scale=0.5]

% these are dummy properties used by TikZiT, but ignored by LaTex
\pgfkeys{/tikz/tikzit fill/.initial=0}
\pgfkeys{/tikz/tikzit draw/.initial=0}
\pgfkeys{/tikz/tikzit shape/.initial=0}
\pgfkeys{/tikz/tikzit category/.initial=0}

% standard layers used in .tikz files
\pgfdeclarelayer{edgelayer}
\pgfdeclarelayer{nodelayer}
\pgfsetlayers{background,edgelayer,nodelayer,main}

% style for blank nodes
\tikzstyle{none}=[inner sep=0mm]

% include a .tikz file
\newcommand{\tikzfig}[1]{%
{\tikzstyle{every picture}=[tikzfig]
\IfFileExists{#1.tikz}
  {\input{#1.tikz}}
  {%
    \IfFileExists{./figures/#1.tikz}
      {\input{./figures/#1.tikz}}
      {\tikz[baseline=-0.5em]{\node[draw=red,font=\color{red},fill=red!10!white] {\textit{#1}};}}%
  }}%
}

% the same as \tikzfig, but in a {center} environment
\newcommand{\ctikzfig}[1]{%
\begin{center}\rm
  \tikzfig{#1}
\end{center}}

% fix strange self-loops, which are PGF/TikZ default
\tikzstyle{every loop}=[]


% ┌────────┐
% │ Citing │
% └────────┘
% \usepackage[style=authortitle]{biblatex}
% \addbibresource{./Graph_Theory.bib}
% \usepackage{cleveref}

% ┌──────────┐
% │ Diagrams │
% └──────────┘
% \usepackage{tikz}
% \usetikzlibrary{shadows, backgrounds} % , calc

% ┌──────────────────┐
% │ Multiple columns │
% └──────────────────┘
% \usepackage{multicol}

% ┌────────────────────┐
% │ Code hightligthing │
% └────────────────────┘
% \usepackage{minted}

% ┌────────────────────────┐
% │ Latex Programming Help │
% └────────────────────────┘
\usepackage{etoolbox}
\usepackage{xparse}
% https://tex.stackexchange.com/questions/358292/creating-a-subcounter-to-a-counter-i-created
\usepackage{chngcntr}

% ┌───────────────┐
% │ Pretty Boxes  │
% └───────────────┘
\usepackage{xcolor}
\usepackage{tcolorbox}
\tcbuselibrary{skins,theorems}

% ┌──────────────┐
% │ Pseudo Code  │
% └──────────────┘
\usepackage{pseudo}

% \usepackage{background}
\usepackage{gradient-text}
\usepackage{rotating}

% \newlength\mylen
% \setlength\mylen{\dimexpr\paperwidth/80\relax}
%
% \SetBgScale{1}
% \SetBgAngle{0}
% \SetBgColor{blue!30}
% \SetBgContents{\tikz{\draw[step=\mylen] (-.5\paperwidth,-.5\paperheight) grid (.5\paperwidth,.5\paperheight);}}

% ┌────────────┐
% │ Misc Tools │
% └────────────┘
\usepackage{lipsum}

%!Tex Root = ../main.tex

% ┌────────────┐
% │ Formatting │
% └────────────┘
% \setlength{\parskip}{0.4cm} % space between paragraphs, https://latexref.xyz/bs-par.html

% ┌───────┐
% │ Fonts │
% └───────┘
\usepackage{fontspec}
\newfontfamily\gyre{DejaVu Math TeX Gyre}
% colored bold
% \newcommand\alert[1]{\textcolor{SwitchColor}{\textbf{#1}}}
\newcommand\alert[1]{\textcolor{SwitchColor}{#1}}

% ┌──────────────┐
% │ Pseudo Code  │
% └──────────────┘
\newcounter{algorithm}
\setcounter{algorithm}{0}
\newtcbtheorem[use counter=algorithm]{algorithm}{\color{SecondaryColor}Algorithm}{pseudo/ruled}{alg}
% \newcommand{\ma}[1]{$\mathcal{#1}$}
% \renewcommand{\tt}[1]{{\small\texttt{#1}}}

% ┌────────┐
% │ Colors │
% └────────┘
\definecolor{PrimaryColor}{HTML}{800080}
\definecolor{PrimaryColorDimmed}{HTML}{D6D6F0}
\definecolor{SecondaryColor}{HTML}{006BB6}
\definecolor{SecondaryColorDimmed}{HTML}{E5F0F8}
\definecolor{SwitchColor}{named}{PrimaryColor}
\colorlet{BoxColor}{gray!10!white}

% ┌───────┐
% │ Links │
% └───────┘
\usepackage[allbordercolors=PrimaryColor, pdfborder={0 0 .2}]{hyperref}

% ┌─────────┐
% │ Mindmap │
% └─────────┘
\renewcommand{\labelitemi}{$\textcolor{SwitchColor}{\bullet}$}
\renewcommand{\labelitemii}{$\textcolor{SwitchColor}{\blacktriangleright}$}
\renewcommand{\labelitemiii}{$\textcolor{SwitchColor}{\blacksquare}$}

%!Tex Root = ../main.tex

% ┌─────────┐
% │ Mindmap │
% └─────────┘
\newlength{\leveldistance}
\setlength{\leveldistance}{25cm}

\newenvironment{edges}{\begin{pgfonlayer}{background}\draw [concept connection]}{;\end{pgfonlayer}}
\newcommand{\edge}[2]{(#1) edge (#2)}
\newcommand{\annotation}[2]{\path (#1) -- node[annotation, above, align=center, pos=0.03] {#2} (middle);}

\newenvironment{resettikz}{\pgfsetlayers{nodelayer,edgelayer}\tikzset{every node/.style={fill opacity=1.0, draw opacity=1.0, minimum size=0cm, inner sep=0pt}}}{}

\newenvironment{mindmap}{
	\begin{tikzpicture}[
			auto,
			huge mindmap,
			fill opacity=0.6,
			draw opacity=0.8,
			concept color = PrimaryColorDimmed,
			every annotation/.style={fill=BoxColor, draw=none, align=center, fill = BoxColor, text width = 2cm},
			grow cyclic,
			level 1/.append style = {
					concept color=SecondaryColorDimmed,
					level distance=\leveldistance,
					sibling angle=360/\the\tikznumberofchildren,
					% https://tex.stackexchange.com/questions/501240/trying-to-use-the-array-environment-inside-a-tikz-node-with-execute-at-begin-no
					execute at begin node=\definecolor{SwitchColor}{named}{SecondaryColor}\definecolor{SwitchColorDimmed}{named}{PrimaryColorDimmed},
				},
			level 2/.append style = {
					concept color=PrimaryColorDimmed,
					level distance=\leveldistance / 2,
					sibling angle=35,
					execute at begin node=\definecolor{SwitchColor}{named}{PrimaryColor}\definecolor{SwitchColorDimmed}{named}{SecondaryColorDimmed},
				},
			level 3/.append style = {
					concept color=SecondaryColorDimmed,
					level distance=\leveldistance / 3,
					execute at begin node=\definecolor{SwitchColor}{named}{SecondaryColor}\definecolor{SwitchColorDimmed}{named}{PrimaryColorDimmed},
				},
			level 4/.append style = {
					concept color=PrimaryColorDimmed,
					level distance=\leveldistance / 4,
					execute at begin node=\definecolor{SwitchColor}{named}{PrimaryColor}\definecolor{SwitchColorDimmed}{named}{SecondaryColorDimmed},
				},
			level 5/.append style = {
					concept color=SecondaryColorDimmed,
					level distance=\leveldistance / 5,
					execute at begin node=\definecolor{SwitchColor}{named}{SecondaryColor}\definecolor{SwitchColorDimmed}{named}{PrimaryColorDimmed},
				},
			level 6/.append style = {
					concept color=PrimaryColorDimmed,
					level distance=\leveldistance / 6,
					execute at begin node=\definecolor{SwitchColor}{named}{PrimaryColor}\definecolor{SwitchColorDimmed}{named}{SecondaryColorDimmed},
				},
			level 7/.append style = {
					concept color=SecondaryColorDimmed,
					level distance=\leveldistance / 7,
					execute at begin node=\definecolor{SwitchColor}{named}{SecondaryColor}\definecolor{SwitchColorDimmed}{named}{PrimaryColorDimmed},
				},
			level 8/.append style = {
					concept color=PrimaryColorDimmed,
					level distance=\leveldistance / 8,
					execute at begin node=\definecolor{SwitchColor}{named}{PrimaryColor}\definecolor{SwitchColorDimmed}{named}{SecondaryColorDimmed},
				},
			level 9/.append style = {
					concept color=SecondaryColorDimmed,
					level distance=\leveldistance / 9,
					execute at begin node=\definecolor{SwitchColor}{named}{SecondaryColor}\definecolor{SwitchColorDimmed}{named}{PrimaryColorDimmed},
				},
			concept connection/.append style = {
					color = BoxColor,
				},
		]
		}{
	\end{tikzpicture}
}

\newenvironment{mindmapcontent}{
	\begin{scope}[
			every node/.style = {concept, circular drop shadow}, % draw=none
			every child/.style={concept},
		]
		}{
		;\end{scope}
}

% ┌───────┐
% │ Boxes │
% └───────┘
\DeclareTotalTCBox{\inlinebox}{ s m }
{standard jigsaw,opacityback=0,colframe=SwitchColor,nobeforeafter,tcbox raise base,top=0mm,bottom=0mm,
	right=0mm,left=0mm,arc=0.1cm,boxsep=0.1cm}
{\IfBooleanTF{#1}%
	{\textcolor{PrimaryColor}{\setBold >\enspace\ignorespaces}#2}%
	{#2}}

\DeclareTotalTCBox{\inlineboxtwo}{ s m }
{standard jigsaw,opacityback=0,colframe=SwitchColorDimmed,nobeforeafter,tcbox raise base,top=0mm,bottom=0mm,
	right=0mm,left=0mm,arc=0.1cm,boxsep=0.1cm}
{\IfBooleanTF{#1}%
	{\textcolor{SwitchColorDimmed}{\setBold >\enspace\ignorespaces}#2}%
	{#2}}

% ┌──────────────────┐
% │ Case distinction │
% └──────────────────┘
% \newtoggle{absolute}
% % \toggletrue{absolute}
% \togglefalse{absolute}
% \newcommand{\lpathgraph}[1]{\iftoggle{absolute}{/home/areo/Documents/Studium/Summaries/x/}{./}#1}

% ┌───────┐
% │ Fixes │
% └───────┘
% https://tex.stackexchange.com/questions/89467/why-does-pdftex-hang-on-this-file
% \newcommand{\colon}{\mathrel{\mathop:}}

% ┌───────┐
% │ Paths │
% └───────┘
% \newcommand{\script}[2]{\href[page=#1]{}{\inlinebox{#2}}}
\newcommand{\script}[2]{\href{openpdf:/home/areo/Documents/Studium/Semester_1_Master/Hardware_Security_and_Trust/slides/Slides annotated/Hardware_Security_and_Trust_all_in_one.pdf:#1}{\inlinebox{#2}}}
\newcommand{\scripttwo}[2]{\href{openpdf:///home/areo/Documents/Studium/Semester_1_Master/Hardware_Security_and_Trust/slides/Slides annotated/bonus/12_Lecture_06Dec.pdf:#1}{\inlinebox{#2}}}
\newcommand{\videoeight}[2]{\href{https://youtu.be/YcHSlFjcndU?feature=shared&t=#1}{\inlineboxtwo{#2}}}
\newcommand{\videonine}[2]{\href{https://youtu.be/3dL-3EOIfJ8?si=l3OakqHOeCpnNayw&t=#1}{\inlineboxtwo{#2}}}
\newcommand{\videoten}[2]{\href{https://youtu.be/6oF737pa510?feature=shared&t=#1}{\inlineboxtwo{#2}}}
\newcommand{\videoeleven}[2]{\href{https://youtu.be/PJTqfzTIYJs?feature=shared&t=#1}{\inlineboxtwo{#2}}}
\newcommand{\videotwelve}[2]{\href{https://youtu.be/oDxAH7aO-Tk?feature=shared&t=#1}{\inlineboxtwo{#2}}}
\newcommand{\videothirteen}[2]{\href{https://youtu.be/3TkSXxe_Ty8?feature=shared&t=#1}{\inlineboxtwo{#2}}}
\newcommand{\videofourteen}[2]{\href{https://youtu.be/1Y3dZuJ0MHg?feature=shared&t=#1}{\inlineboxtwo{#2}}}


\begin{document}

\fontsize{4pt}{5pt}\selectfont

\begin{minipage}[t]{0.19\pagewidth}
	\fbox{General}
	\begin{betterlist}
		\item \alert{Important security properties (\enquote{CIAAN}) are:}
		\begin{betterlist}
			\item \alert{Confidentiality:} Protecting confidential information from being disclosed to unauthorized parties.
			\item \alert{Integrity:} Ensuring that information is only modified by authorized parties.
			\item \alert{Availability:} Making sure that information and systems are accessible to authorized parties when they need them (e.g. resistance to denial-of-service attacks).
			\item \alert{Authenticity:} Ensuring that information and communication come from the source they are supposed to come from.
			\item \alert{Non-repudiation:} Ensuring that nobody can deny having performed certain actions (like sending / receiving messages, changing data etc.).
		\end{betterlist}
		\item \alert{Safety:} System is designed without any error leading to unintended behavior
		\item  \alert{Reliability:} A correctly designed system continues to work correctly during its life-time
	\end{betterlist}
	\fbox{\enquote{Classical} Cryptography}
	\begin{betterlist}
		\item \alert{Cryptology:} cryptography + cryptanalysis
		\item \alert{Cryptography:} art/science of keeping message secure
		\item \alert{Cryptanalysis:} art/science of breaking ciphertext
	\end{betterlist}
	\fbox{Classification by type of encryption operations}
	\begin{betterlist}
		\item \alert{Substitution ciphers}
		\begin{betterlist}
			\item letters of $P$ replaced with other letters by $E()$
			\item \underline{Effects:}
			\begin{betterlist}
				\item $C$ hides chars of $P$ (plaintext)
				\item if $> 1$ key alphabet (polyalphabetic), $C$ dissipates high frequency chars
			\end{betterlist}
			\item \underline{Sidenote:} One can say $key = 3$ or $key = \enquote{D}$, because $no(D) = 3$
		\end{betterlist}
		\begin{betterlist}
			\item \alert{Monoalphabetic substitution ciphers}
			\begin{betterlist}
				\item in a general monoalphabetic substitution cipher each letter in $P$ is substituted by a fixed letter using a \alert{substitution table}
				\item the \alert{key} is the substitution table
				\item \underline{Attacks:}
				\begin{betterlist}
					\item In practice, not only frequencies of individual letters can be used for an attack, but also the frequency of letter pairs (i.e., \enquote{TH} is very common in English), letter triples, etc
					\item need better concealing of statistical frequencies and probably also longer keys to avoid exhaustive search $\rightarrow$ Polyalphabetic substitution ciphers
				\end{betterlist}
			\end{betterlist}
			\begin{betterlist}
				\item \alert{Caesar Cipher}
				\begin{betterlist}
					\item each letter in $P$ is substituted by a fixed letter. In this special case the \enquote{key} is of length $1$, it is the shift amount
				\end{betterlist}
			\end{betterlist}
		\end{betterlist}
		\begin{betterlist}
			\item \alert{Polyalphabetic substitution ciphers}
			\begin{betterlist}
				\item several key alphabets, flatten (diffuse) somewhat the frequency distribution of letters by combining high and low distributions
				\item \underline{Attack:} If one knows the $n$, then one can break the whole text into $n$ parts and for these a statistical analysis works as for the monialphabetic substitution cipher with the unigram model
			\end{betterlist}
			\begin{betterlist}
				\item \alert{Vigenère Tableaux Method}
				\begin{betterlist}
					\item special case of polyalphabetic substitution with $n$ key alphabets. For each key alphabet the special case of Caesar cipher is chosen, i.e., each key alphabet can be represented by one letter
					\begin{betterlist}
						\item one can describe a row either by the shift or by saying which letter $c_i$ it mapped to which letter $c_j$ and takes $b_j$ as name of the row, choosing the last choice one can describe $n$ keys by a word of length $n$
					\end{betterlist}
					\item $26^n$ different keys, one has to choose $n$ large enough
					\item \underline{Attack:}
					\begin{betterlist}
						\item same problem as for polyalphabetic substitution ciphers
					\end{betterlist}
				\end{betterlist}
			\end{betterlist}
		\end{betterlist}
	\end{betterlist}
	\begin{betterlist}
		\item \alert{Transposition (permutation) ciphers}
		\begin{betterlist}
			\item order of letters in $P$ rearranged by $E()$
			\begin{betterlist}
				\item rearrange letters in plaintext to produce ciphertext
			\end{betterlist}
			\item \underline{Effects:}
			\begin{betterlist}
				\item $C$ scrambles text, hides $n$-grams for $n > 1$ (combinations of $n$ letters, e.g. th)
			\end{betterlist}
		\end{betterlist}
		\begin{betterlist}
			\item \alert{Rail-Fence Cipher}
			\begin{betterlist}
				\item columnar transposition
				\item Key = Number of columns
				\item \underline{Attack:} Number of columns / the  key space is restricted if the text is short
			\end{betterlist}
		\end{betterlist}
	\end{betterlist}
	\begin{betterlist}
		\item \alert{Product ciphers}
		\begin{betterlist}
			\item combine two or more ciphers to enhance the security of the cryptosystem
			\begin{betterlist}
				\item $E = E_1 + E_2 + ... + E_n$
				\item built of multiple blocks, either \alert{Substitution} or \alert{Transposition}
			\end{betterlist}
			\item \underline{Attack}:
			\item \underline{Effects:}
			\begin{betterlist}
				\item can do all what Substitution and Transposition ciphers can so, more secure if used well
			\end{betterlist}
		\end{betterlist}
		\begin{betterlist}
			\item \alert{Two-block product cipher}
			\begin{betterlist}
				\item $E2(E1(P, KE1), KE2)$
				\item may be repeated to form several encryption rounds
			\end{betterlist}
		\end{betterlist}
	\end{betterlist}
	% 	\item \underline{Advantage of crypto systems with keys:}
	% 	\begin{betterlist}
	% 		\item  Keeping the encryption / decryption algorithm is not needed.
	% 		\item  Keys can regulary be changed.to increase security
	% 	\end{betterlist}
	% 	\item \alert{Symmetric cryptosystems:} $K_E = K_D$
	% 	\begin{betterlist}
	% 		\item Encipher and decipher using the same key
	% 		\item Or one key is easily derived from the other
	% 	\end{betterlist}
	% 	\item \alert{Asymmetric cryptosystems:} $K_E \ne K_D$
	% 	\begin{betterlist}
	% 		\item Encipher and decipher using different keys
	% 		\item Computationally infeasible to derive one from other
	% 	\end{betterlist}
	% 	\item \underline{Attacks:}
	% 	\begin{betterlist}
	% 		\item \alert{Exhaustive search:} Try all possible keys until you find the right one
	%      \item \alert{Statistical analysis:} The longer the ciphertext C, the more effective it would be, 1-gram (unigram) model of language
	% 	\end{betterlist}
	% 	\item \underline{Types of Ciphers:}
	% 	\begin{betterlist}
	% 		\item \alert{Substitution ciphers}
	% 		\begin{betterlist}
	% 			\item Letters of P replaced with other letters by E()
	% 			\begin{betterlist}
	%          \item \alert{General Monoalphabetic Substitution Cipher}: $26!\approx 2^{88}$
	% 			\end{betterlist}
	% 		\end{betterlist}
	% 		\item \alert{Transposition (permutation) ciphers}
	% 		\begin{betterlist}
	% 			\item Order of letters in P rearranged by E()
	% 		\end{betterlist}
	% 		\item \alert{Product ciphers}
	% 		\begin{betterlist}
	% 			\item Combine two or more ciphers to enhance the security of the cryptosystem
	% 		\end{betterlist}
	% 	\end{betterlist}
	% \end{betterlist}
\end{minipage}
\begin{minipage}[t]{0.19\pagewidth}
	\fbox{Classification by key}
	\begin{betterlist}
		\item \alert{Crypto System with Keys:} $P = D(K_D, E(K_E, P))$
		\begin{betterlist}
			\item \underline{advantage of crypto systems with keys:}
			\begin{betterlist}
				\item keeping the encryption / decryption algorithm secret is not needed
				\item keys can regulary be changed, to increase security
			\end{betterlist}
		\end{betterlist}
	\end{betterlist}
	\begin{betterlist}
		\item \alert{Symmetric cryptosystems} $K_E = K_D$
		\begin{betterlist}
      \item encipher and decipher using the same key or one key is easily derived from the other
			\item \underline{advantage:}
			\item \underline{disadvantage:}
			\begin{betterlist}
				\item one needs a \alert{key exchange}, the sender and the receiver have to agree on the same key and they key should be secret, one needs a way to transport the secret key via a secure channel from the sender to the receiver
			\end{betterlist}
		\end{betterlist}
		\begin{betterlist}
			\item Advanced Encryption Standard (AES)
			\begin{betterlist}
				\item at the moment most widely used symmetric cipher
				\item it is \alert{deterministic}, i.e., with the same key identical plain text blocks are mapped to the same cipher text blocks
			\end{betterlist}
			\begin{betterlist}
				\item Implementation in Software
				\begin{betterlist}
					\item straightforward implementation is well suited for $8$-bit processors (e.g., smart cards) with small amount of memory, because operations are on bytes, but inefficient on $32$-bit or $64$-bit processors
				\end{betterlist}
			\end{betterlist}
			\begin{betterlist}
				\item Security
				\begin{betterlist}
					\item \alert{Brute-force attack:} Due to the key length of $128$, $192$ or $256$ bits, a brute-force attack is not possible
					\item \alert{Analytical attacks:} There is no analytical attack known that is better than brute-force
				\end{betterlist}
				\begin{betterlist}
					\item Encrypting longer plain text
					\begin{betterlist}
						\item \underline{Countermeasure:} Different modes for block ciphers
						\begin{betterlist}
							\item Straightforward application of block cipher
							\item \underline{disadvantage:} \alert{traffic analysis}, if know block with encoding of account number, without knowing how encryption works, one can replace ones own encrypted account number with account number of other account in other transmission (\alert{substituion attack})
						\end{betterlist}
						\begin{betterlist}
							\item \underline{advantage:} no \alert{traffic analysis} possible
						\end{betterlist}
					\end{betterlist}
				\end{betterlist}
			\end{betterlist}
			\begin{betterlist}
				\item Decryption
				\begin{betterlist}
					\item \alert{Inv Byte Substitution layer}, since the S-Box is bijective, it is possible to construct an inverse, such that $A_i = S^{–1}(B_i) = S^{–1}(S(A_i))$. The inverse S-Box is used for decryption. It is usually realized as a lookup table. Inverse Lookup Table easy to get from Lookup Table
					\item \alert{Decryption key schedule}, subkeys are needed in reversed order (compared to encryption). \underline{For decryption:} Before starting decryption, first compute all subkeys from the actual key (as done for encryption) and apply in reverse order, same subkey generation as for the encryption
				\end{betterlist}
			\end{betterlist}
			\begin{betterlist}
				\item Round Structure and Internal Structure
				\begin{betterlist}
					\item \alert{Diffusion Layer:}  Provides diffusion over all input state bits. Performs a linear operation on state matrices A and B, i.e., $DIFF(A) \oplus DIFF(B) = DIFF(A \oplus B)$
				\end{betterlist}
			\end{betterlist}
		\end{betterlist}
	\end{betterlist}
	\begin{betterlist}
		\item \alert{Asymmetric cryptosystems} $K_E \ne K_D$
		\begin{betterlist}
      \item encipher and decipher using different keys, computationally infeasible to derive one from other
			\item \underline{advantages:}
			\begin{betterlist}
				\item allows to share public key, secure key exchange not needed (for \alert{confidentiality}, message should be kept secret)
				\item also other applications like authentication (\alert{authenticity}, able to sign message, so it can be sure it can only come from oneself)
			\end{betterlist}
		\end{betterlist}
	\end{betterlist}
	\begin{betterlist}
		\item RSA
		\begin{betterlist}
			\item Fast Modular Exponentiation
		\end{betterlist}
	\end{betterlist}
\end{minipage}
\begin{minipage}[t]{0.19\pagewidth}
	\fbox{Classification by way to process plaintext}
	\begin{betterlist}
		\item \alert{Stream Ciphers}
		\begin{betterlist}
			\item bitwise Encryption and Decryption
			\item used in e.g. Red Phone between US and SU was implemented in this way
			\item encrypt bits individually
			\item need a method to generate key stream efficiently, starting from some “seed
			\item usually small and fast common in embedded device
		\end{betterlist}
		\begin{betterlist}
			\item Pseudo-random sequence
			\begin{betterlist}
				\item pseudo random functions for cryptography it might be a problem
			\end{betterlist}
		\end{betterlist}
	\end{betterlist}
	\begin{betterlist}
		\item \alert{Block Ciphers}
		\begin{betterlist}
			\item always encrypt a full block (several bits)
			\item are common for Internet applications
		\end{betterlist}
		\begin{betterlist}
			\item More Block Ciphers
			\begin{betterlist}
				\item DES (Data Encryption Standard)
				\begin{betterlist}
					\item Predecessor of AES
					\item Considered insecure (small key length of 56 bits)
				\end{betterlist}
			\end{betterlist}
		\end{betterlist}
		\begin{betterlist}
			\item Block Cipher Primitives
			\begin{betterlist}
				\item \alert{Claude Shannon:} There are two primitive operations with which strong encryption algorithms can be built:
				\begin{enumerate}
					\item \alert{Confusion:} An encryption operation where the \alert{relationship between key and ciphertext is obscured}
					\begin{betterlist}
						\item today, a common element for achieving confusion is substitution, which is found in AES and other ciphers.
					\end{betterlist}
					\item \alert{Diffusion:} An encryption operation where the \alert{influence of one plaintext symbol is spread over many ciphertext symbols} with the goal of hiding statistical properties of the plaintext
					\begin{betterlist}
						\item in other context known as Transposition
						\item a simple diffusion element is the \alert{bit permutation} (in other context known as Tranposition)
					\end{betterlist}
				\end{enumerate}
				\begin{betterlist}
					\item Both operations by themselves are suboptimal in providing security. A cipher must include confusion and diffusion elements
				\end{betterlist}
			\end{betterlist}
		\end{betterlist}
	\end{betterlist}
\end{minipage}
\begin{minipage}[t]{0.19\pagewidth}
	\begin{betterlist}
		\item asdf
	\end{betterlist}
\end{minipage}
\begin{minipage}[t]{0.19\pagewidth}
	\begin{betterlist}
		\item asdf
	\end{betterlist}
\end{minipage}
\end{document}
