%!Tex Root = ../main.tex
% ./Packete.tex
% ./Design.tex
% ./Vorbereitung.tex
% ./Aufgabe1.tex
% ./Aufgabe2.tex
% ./Aufgabe3.tex
% ./Aufgabe4.tex
% ./Appendix.tex

\begin{mindmap}
  \begin{mindmapcontent}
    \node (re) at (current page.center) {Relations
      \resizebox{\textwidth}{!}{
        \begin{minipage}[t]{18cm}
          \begin{itemize}
            \item Seien $A$, $B$ Mengen. Eine Teilmenge $R \subseteq A \times B$ nennt man binäre Relation zwischen $A$ und $B$
            \item $x R y$ is true $\Leftrightarrow (x,y) \in R$ is true $\Leftrightarrow$ an edge from $x$ to $y$ exists in the directed graph representing $R$
          \end{itemize}
        \end{minipage}
      }
    }
    child {
      node {Heterogeneous relations
        \resizebox{\textwidth}{!}{
          \begin{minipage}[t]{12cm}
            \begin{itemize}
              \item a binary relation $R\subseteq A\times B$, where $A$ and $B$ are possibly distinct sets
              \item \alert{left-unique} and \alert{right-total} both describe relationships focusing on the \alert{codomain} $B$ while their correspondings counterparts for functions \alert{right-unique } and \alert{left-total } focus on the \alert{domain} $A$             
            \end{itemize}
          \end{minipage}
        }
      }
      child {
        node {Bijective Function
          \resizebox{\textwidth}{!}{
            \begin{minipage}[t]{8cm}
              \begin{itemize}
                \item 1-to-1 correspondence
              \end{itemize}
              \begin{resettikz}
                \ctikzfig{./figures/bijective_function}
              \end{resettikz}
            \end{minipage}
          }
        }
          child {
            node {Function
              \resizebox{\textwidth}{!}{
                \begin{minipage}[t]{12cm}
                  \begin{itemize}
                    \item left-unique and right-total
                    \item every element from $A$ has \alert{exactly} one partner in $B$: $\forall a \in A \exists ! b \in B:(a, b) \in R$
                    \begin{itemize}
                      \item the set $A$ is called \alert{domain} (germ. Definitionsbereich) and the set $B$ is called \alert{codomain} (germ. Wertebereich)
                      \item \alert{image of subset} M under $f$ (germ. Bildmenge einer Teilmenge $M$ unter $f$):\\ $im(f) = f(M) = \{f(x) \;|\; x\in M\}\subseteq Y, f: X\rightarrow Y, M\subseteq X$ (for the example below $A = X$, $B = Y$) 
                      \begin{itemize}
                        \item \alert{image of a function} (germ. Bildmenge einer Funktion) if $M = X$, it then is the set of all output values that the function can produce
                        \item \alert{image of an element} $x\in X$ under $f$ (germ. Funktionswert) is denoted $f(x)$
                      \end{itemize}
                    \end{itemize}
                    \item \alert{function evaluation:} $y = f(x) \Leftrightarrow (x, y)\in f \Leftrightarrow$ an edge from $x$ to $y$ exists in the directed graph representing $R$
                    \begin{itemize}
                      \item delivers the value of a function $f$ at an element $x$ of its domain, e.g. $f(2)$ delivers $12$ for the function $f = \{(1, 11), (2, 12), (3, 13)\}$
                    \end{itemize}
                    \begin{itemize}
                      \item only works for right unique relations, because $x$ is uniquely mapped to $y$ in right direction when looking at $(x, y)$
                    \end{itemize}
                    \item \alert{notations:}
                    \begin{itemize}
                      \item \alert{functional notation:} $f(x, y) = x + 2y$
                      \item \alert{arrow notation:} $f: X\times Y \rightarrow Z: (x, y)\mapsto x + 2y$% stands for a function in two variables
                      % \begin{itemize}
                      %   \item $x\mapsto f(x, y_0)$ refers to a partially applied function produced by fixing the second argument to the value $y_0$ without introducing a new function name
                      %   \item $f(x_0, y_0)$ refers to the value of the function $f$ at the point $(x_0, y_0)$
                      % \end{itemize}
                      \item \alert{index notation:} $f_t (x) = x + 2t$
                      \begin{itemize}
                        \item for distinguishing some variables called paramters from the \enquote{true variables}, parameters are considered as being fixed during the study of a problem
                        \item for functions whose domain is the set of the natural numbers, $f_i$ is called the $i$th element of the sequence $(f_i)_{i\in\mathbb{N}}$ with $f_i = i + 2$ standing for the function $f(i) = i  + 2$
                      \end{itemize}
                      % \item \alert{dot notation:} $(\cdot) + 2$ stands for $x\rightarrow x + 2$, this may be useful for distinguishing the function $f(\cdot)$ from its value $f(x)$ at $x$
                      \item \alert{as relation:} $f = \{((x, y), x + 2y)\;|\; (x, y)\in X\times Y\} = \{((1, 1), 3), ((1, 2), 5), \ldots\}$
                    \end{itemize}
                  \end{itemize}
                  \begin{resettikz}
                    \ctikzfig{./figures/function}
                  \end{resettikz}
                \end{minipage}
              }
            }
              child {
                node {Right-unique
                  \resizebox{\textwidth}{!}{
                    \begin{minipage}[t]{8cm}
                      \begin{itemize}
                        \item for every element $a$ from the domain $A$ exists \alert{at most} one element $b$ in the codomain $B$: $\forall a \in A\; \forall b, d \in B:(a, b) \in R \wedge(a, d) \in R \Rightarrow b=d$
                        \item with only this property it's a partial function
                      \end{itemize}
                      \begin{resettikz}
                        \ctikzfig{../figures/right_unique}
                      \end{resettikz}
                    \end{minipage}
                  }
                }
              }
              child {
              node {Left-total
                \resizebox{\textwidth}{!}{
                  \begin{minipage}[t]{8cm}
                    \begin{itemize}
                      \item for every element $a$ from the domain $A$ exists \alert{at least} one element $b$ in the codomain $B$: $\forall a \in A\; \exists b \in B:(a, b) \in R$
                      \item with only this property it's a \enquote{Multifunktion} (german)
                    \end{itemize}
                    \begin{resettikz}
                      \ctikzfig{./figures/left_total}
                    \end{resettikz}
                  \end{minipage}
                }
              }
            }
        }
        child {
          node {Bijective
            \resizebox{\textwidth}{!}{
              \begin{minipage}[t]{8cm}
                \begin{itemize}
                  \item injective and surjective
                  \item every element from $B$ has \alert{exactly} one partner in $A$: $\forall b \in B \exists ! a \in A:(a, b) \in R$
                \end{itemize}
                \begin{resettikz}
                  \ctikzfig{./figures/bijective}
                \end{resettikz}
              \end{minipage}
            }
          }
            child {
              node {Injective / Left-unique
                \resizebox{\textwidth}{!}{
                  \begin{minipage}[t]{8cm}
                    \begin{itemize}
                      \item every element from $B$ has \alert{at most} one partner in $A$: $\forall b \in B \forall a, c \in A:(a, b) \in R \wedge(c, b) \in R \Rightarrow a=c$, \textit{mnemonic:} \enquote{a injection shouldn't be made twice at the same spot}
                    \end{itemize}
                    \begin{resettikz}
                      \ctikzfig{./figures/left_unique}
                    \end{resettikz}
                  \end{minipage}
                }
              }
            }
            child {
              node {Surjective / Right-total
                \resizebox{\textwidth}{!}{
                  \begin{minipage}[t]{8cm}
                    \begin{itemize}
                      \item every element from $B$ has \alert{at least} one partner in $A$: $\forall b \in B \exists a \in A:(a, b) \in R$
                    \end{itemize}
                    \begin{resettikz}
                      \ctikzfig{./figures/right_total}
                    \end{resettikz}
                  \end{minipage}
                }
              }
            }
        }
      }
    }
    child {
      node (test) {Homogeneous relations
        \resizebox{\textwidth}{!}{
          \begin{minipage}[t]{8cm}
            \begin{itemize}
              \item a binary relation over $X$ and itself
            \end{itemize}
          \end{minipage}
        }
      }
      child {
        node {Equivalence relations
          \resizebox{\textwidth}{!}{
            \begin{minipage}[t]{8cm}
              \begin{itemize}
                \item reflexive
                \item symmetric
                \item transitive
              \end{itemize}
              \begin{resettikz}
                \ctikzfig{./figures/equivalence_relation}
              \end{resettikz}
            \end{minipage}
          }
        }
      }
      child {
        node {Order relations}
        child {
          node {Partial Order relations
            \resizebox{\textwidth}{!}{
              \begin{minipage}[t]{8cm}
                \begin{itemize}
                  \item reflexive
                  \item antisymmetric
                  \item transitive
                  \item the set of subsets of a given set (its power set) ordered by inclusion is a partial order
                \end{itemize}
                \begin{resettikz}
                  \ctikzfig{./figures/partial_order}
                \end{resettikz}
              \end{minipage}
            }
          }
          child {
            node {Strict Partial Order relations
              \resizebox{\textwidth}{!}{
                \begin{minipage}[t]{8cm}
                  \begin{itemize}
                    \item irreflexive
                    \item antisymmetric
                    \item transitive
                  \end{itemize}
                  \begin{resettikz}
                    \ctikzfig{./figures/strict_partial_order}
                  \end{resettikz}
                \end{minipage}
              }
            }
          }
        }
        child {
          node {Total / Linear Order relations
            \resizebox{\textwidth}{!}{
              \begin{minipage}[t]{8cm}
                \begin{itemize}
                  \item reflexive 
                  \item antisymmetric 
                  \item transitive 
                  \item connected / strongly connected
                  \item Reflexivity already follows from strong connectednes
                \end{itemize}
                \begin{resettikz}
                  \ctikzfig{./figures/total_order}
                \end{resettikz}
              \end{minipage}
            }
          }
          child {
            node {Strict Total Order relations
              \resizebox{\textwidth}{!}{
                \begin{minipage}[t]{8cm}
                  \begin{itemize}
                    \item irreflexive 
                    \item antisymmetric / asymmetric
                    \item transitive 
                    \item connected
                    \item Asymmetry follows from transitivity and irreflexivity;[8] moreover, irreflexivity follows from asymmetry
                  \end{itemize}
                  \begin{resettikz}
                    \ctikzfig{./figures/strict_total_order}
                  \end{resettikz}
                \end{minipage}
              }
            }
          }
        }
      }
      child {
        node {Reflexive
          \resizebox{\textwidth}{!}{
            \begin{minipage}[t]{8cm}
              \begin{itemize}
                \item for all $x\in X$, $xRx$
                \item $\ge$ is a reflexive relation but $>$ is not
              \end{itemize}
              \begin{resettikz}
                \ctikzfig{./figures/reflexive}
              \end{resettikz}
            \end{minipage}
          }
        }
        child {
          node {Irreflexive
            \resizebox{\textwidth}{!}{
              \begin{minipage}[t]{8cm}
                \begin{itemize}
                  \item for all $x\in X$, not $xRx$
                  \item $>$ is a reflexive relation but $\ge$ is not
                \end{itemize}
                \begin{resettikz}
                  \ctikzfig{./figures/irreflexive}
                \end{resettikz}
              \end{minipage}
            }
          }
        }
      }
      child {
        node {Symmetric
          \resizebox{\textwidth}{!}{
            \begin{minipage}[t]{8cm}
              \begin{itemize}
                \item for all $x, y\in X$, if $xRy$ then $yRx$
              \end{itemize}
              \begin{resettikz}
                \ctikzfig{./figures/irreflexive}
              \end{resettikz}
            \end{minipage}
          }
        }
        child {
          node {Antisymmetric
            \resizebox{\textwidth}{!}{
              \begin{minipage}[t]{8cm}
                \begin{itemize}
                  \item for all $x, y\in X$, if $xRy$ and $yRx$ then $x=y$
                  \item $\ge$ is an antisymmetric relation
                \end{itemize}
              \begin{resettikz}
                \ctikzfig{./figures/antisymmetric}
              \end{resettikz}
              \end{minipage}
            }
          }
        }
        child {
          node {Asymmetric
            \resizebox{\textwidth}{!}{
              \begin{minipage}[t]{8cm}
                \begin{itemize}
                  \item for all $x, y\in X$, if $xRy$ then not $yRx$
                  \item a relation is \alert{asymmetric} if and only if it is both \alert{antisymmetric} and \alert{irreflexive}
                  \item $>$ is an asymmetric relation, but $\ge$ is not
                \end{itemize}
              \begin{resettikz}
                \ctikzfig{./figures/asymmetric}
              \end{resettikz}
              \end{minipage}
            }
          }
        }
      }
      child {
        node {Transitive
          \resizebox{\textwidth}{!}{
            \begin{minipage}[t]{8cm}
              \begin{itemize}
                \item for all $x, y, z\in X$ if $xRy$ and $yRz$ then $xRz$
                \item \enquote{is ancestor of} is a transitive relation, while \enquote{is parent of} is not
              \end{itemize}
              \begin{resettikz}
              \ctikzfig{./figures/transitive}
              \end{resettikz}
            \end{minipage}
          }
        }
      }
      child {
        node {Connected
          \resizebox{\textwidth}{!}{
            \begin{minipage}[t]{8cm}
              \begin{itemize}
                \item for all $x, y\in X,$ if $x\ne y$ then $xRy$ or $yRx$
              \end{itemize}
              \begin{resettikz}
              \ctikzfig{./figures/connected}
              \end{resettikz}
            \end{minipage}
          }
        }
        child {
          node {Strongly Connected
            \resizebox{\textwidth}{!}{
              \begin{minipage}[t]{8cm}
                \begin{itemize}
                  \item for all $x, y\in X$, $xRy$ or $yRx$
                  \item formerly called total
                \end{itemize}
              \begin{resettikz}
              \end{resettikz}
              \begin{resettikz}
              \ctikzfig{./figures/strongly_connected}
              \end{resettikz}
              \end{minipage}
            }
          }
        }
      }
      child {
        node {Dense
          \resizebox{\textwidth}{!}{
            \begin{minipage}[t]{8cm}
              \begin{itemize}
                \item for all $x, y\in X$, if $xRy$ then some $z\in X$ exists sucht that $xRz$ and $zRy$
              \end{itemize}
              \begin{resettikz}
              \ctikzfig{./figures/dense}
              \end{resettikz}
            \end{minipage}
          }
        }
      }
    };
  \end{mindmapcontent}
  % \begin{edges}
  %   \edge{test}{re}
  % \end{edges}
  \annotation{re.south}{This mindmap is provided without guarantee of correctness and completeness!};
  \annotation{re.north}{\href{/tmp/current.pdf}{go back}}
\end{mindmap}
