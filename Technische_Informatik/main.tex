\documentclass{standalone}
\usepackage[ngerman]{babel}
% https://tex.stackexchange.com/questions/570303/use-blacktriangleright-as-itemize-label
\usepackage{amssymb} % for black triangleright

\renewcommand{\labelitemi}{$\textcolor{SwitchColor}{\bullet}$}
\renewcommand{\labelitemii}{$\textcolor{SwitchColor}{\blacktriangleright}$}
\renewcommand{\labelitemiii}{$\textcolor{SwitchColor}{\blacksquare}$}
\renewcommand{\labelitemiv}{$\textcolor{SwitchColor}{\blacksquare}$}

% https://tex.stackexchange.com/questions/525959/prevent-latex-from-stretching-math
\setlength{\thinmuskip}{1\thinmuskip}
\setlength{\medmuskip}{1\medmuskip}
\setlength{\thickmuskip}{1\thickmuskip}

\usepackage{array}
\usepackage{booktabs}
\usepackage{boldline}

\usepackage{tabularray}

\usepackage{subcaption}

\usepackage{csquotes}
\usepackage{xcolor}
% \usepackage{anyfontsize}
\usepackage[export]{adjustbox}
% \usepackage[]{enumitem}
\usepackage{nicematrix}
\usepackage{tikz}
\usetikzlibrary{arrows.meta,positioning}
\usetikzlibrary{graphs}
\usetikzlibrary{patterns}
\usetikzlibrary{shadings}
\usetikzlibrary{mindmap, shadows, backgrounds} % , calc

\definecolor{PrimaryColor}{HTML}{0B64A0}
\definecolor{PrimaryColorDimmed}{HTML}{A2D7F3}
\definecolor{SecondaryColor}{HTML}{6FC1EC}
\definecolor{SecondaryColorDimmed}{HTML}{CEEAF9}
\colorlet{BoxColor}{gray!10!white}

\usepackage[allbordercolors=PrimaryColor, pdfborder={0 0 .2}]{hyperref}

% colored bold
% \newcommand\alert[1]{\textcolor{SwitchColor}{\textbf{#1}}}
\newcommand\alert[1]{\textcolor{SwitchColor}{#1}}

\newlength{\leveldistance}
\setlength{\leveldistance}{20cm}

\begin{document}
  \begin{tikzpicture}[
      auto,
      huge mindmap,
      fill opacity=0.6,
      draw opacity=0.8,
      concept color = PrimaryColorDimmed,
      every annotation/.style={fill=BoxColor, draw=none, align=center, fill = BoxColor, text width = 2cm},
      grow cyclic,
      level 1/.append style = {
        concept color=SecondaryColorDimmed,
        level distance=\leveldistance,
        sibling angle=360/\the\tikznumberofchildren,
        % https://tex.stackexchange.com/questions/501240/trying-to-use-the-array-environment-inside-a-tikz-node-with-execute-at-begin-no
        execute at begin node=\definecolor{SwitchColor}{named}{SecondaryColor},
      },
      level 2/.append style = {
        concept color=PrimaryColorDimmed,
        level distance=\leveldistance / 2,
        sibling angle=30,
        execute at begin node=\definecolor{SwitchColor}{named}{PrimaryColor},
      },
      level 3/.append style = {
        concept color=SecondaryColorDimmed,
        level distance=\leveldistance / 3,
        execute at begin node=\definecolor{SwitchColor}{named}{SecondaryColor},
      },
      level 4/.append style = {
        concept color=PrimaryColorDimmed,
        level distance=\leveldistance / 4,
        execute at begin node=\definecolor{SwitchColor}{named}{PrimaryColor},
      },
      level 5/.append style = {
        concept color=SecondaryColorDimmed,
        level distance=\leveldistance / 5,
        execute at begin node=\definecolor{SwitchColor}{named}{SecondaryColor},
      },
      level 6/.append style = {
        concept color=PrimaryColorDimmed,
        level distance=\leveldistance / 6,
        execute at begin node=\definecolor{SwitchColor}{named}{PrimaryColor},
      },
      level 7/.append style = {
        concept color=SecondaryColorDimmed,
        level distance=\leveldistance / 7,
        execute at begin node=\definecolor{SwitchColor}{named}{SecondaryColor},
      },
      level 8/.append style = {
        concept color=PrimaryColorDimmed,
        level distance=\leveldistance / 8,
        execute at begin node=\definecolor{SwitchColor}{named}{PrimaryColor},
      },
      concept connection/.append style = {
        color = BoxColor,
      },
  ]
  % damit Annotationen nicht auch eine Drop Shadow erhalten
  \begin{scope}[
      every node/.style = {concept, circular drop shadow}, % draw=none
      every child/.style={concept},
    ]
  \node (ca) at (current page.center) {Technische Informatik}
    child {
      node {Kodierung von Zeichen}
      child {
        node {Code, Codewörter
          \resizebox{\textwidth}{!}{
            \begin{minipage}[t]{10cm}
              \begin{itemize}
                \item Menge der \alert{Codewörter}: $c(A):=\left\{w\in\{0,1\}^{*}\;\right|\exists a\in A:c(a)=w\}$
                \item Eine Abbildung $c:{A}\rightarrow\{0,1\}^{*}$ oder $c:A\to\{0,1\}^{n}$ heißt \alert{Code}, falls $c$ injektiv ist 
              \end{itemize}
            \end{minipage}
          }
        }
        child {
          node {Alphabete, Wörter, Zeichen
            \resizebox{\textwidth}{!}{
              \begin{minipage}[t]{10cm}
                \begin{itemize}
                  \item nichtleeres Menge $A=\left\{a_{1},\dots,a_{m}\right\}$ heißt (endliches) \alert{Alphabet} der Größe $m$
                  \begin{itemize}
                    \item $a_1,\ldots, a_m$ heißen \alert{Zeichen} des Alphabets
                    \item $A^{*}=\{w\mid w=b_{1}\dots b_{n}\;m i t\;n\in\mathbb{N},\forall i\;m i t\;1\leq i\leq n:b_{i}\in A\}$ ist die \alert{Menge aller endlichen Wörter} über dem Alphabet $A$
                    \item $|b_1\ldots b_n| := n$ heißt Länge des Wortes $b_1\ldots b_n$
                    \item das Wort der Länge $0$ wird mit $\epsilon$
                  \end{itemize}
                \end{itemize}
              \end{minipage}
            }
          }
        }
        child {
          node {Codes fester Länge
            \resizebox{\textwidth}{!}{
              \begin{minipage}[t]{10cm}
                \begin{itemize}
                  \item $c:A\to\{0,1\}^{n}$ heißt \alert{Code fester Länge}
                  \begin{itemize}
                    \item für einen Code $c:A\to\{0,1\}^{n}$ fester Länge gilt: $n \ge \lceil log_2(m)\rceil$
                    \item ist $n=\lceil log_{2}m\rceil+r$ mit $r > 0$, so können die $r$ zusätzlichen Bits zum Test auf \alert{Übertragungsfehler} verwendet werden
                    \item Die Kodierung eines jeden Zeichens besteht aus $n$ Bits
                    \item einfach zu behandeln, unter Umständen wird faber mehr Speicherplatz gebraucht als nötig
                  \end{itemize}
                \end{itemize}
              \end{minipage}
            }
          }
          child {
            node {ASCII
              \resizebox{\textwidth}{!}{
                \begin{minipage}[t]{8cm}
                  \begin{itemize}
                    \item American Standard Code for Information Interchange
                    \item 7 Bits (es gibt Erweiterungen mit 8 Bits)
                  \end{itemize}
                  \includegraphics[width=0.5\textwidth, center]{./figures/ascii.png}
                \end{minipage}
              }
            }
          }
          child {
            node {EBCDIC
              \resizebox{\textwidth}{!}{
                \begin{minipage}[t]{8cm}
                  \begin{itemize}
                    \item Extended Binary Coded Decimal Interchange Code
                    \item 8 Bits
                  \end{itemize}
                \end{minipage}
              }
            }
          }
          child {
            node {Unicode
              \resizebox{\textwidth}{!}{
                \begin{minipage}[t]{8cm}
                  \begin{itemize}
                    \item 16 Bits
                  \end{itemize}
                \end{minipage}
              }
            }
          }
        }
        child {
          node {Häufigkeits-abhängige Codes
            \resizebox{\textwidth}{!}{
              \begin{minipage}[t]{8cm}
                \begin{itemize}
                  \item \alert{Ziel} ist die Reduktion der Länge einer Nachricht durch Wahl \alert{verschieden lange Codewörter} für die verschiedenen Zeichen eines Alphabets
                  \item häufiges Zeichen $\rightarrow$ kurzer Code
                  \item seltenes Zeichen $\rightarrow$ langer Code
                  % \item Häufigkeitsverteilung ist bekannt $\rightarrow$ \alert{statische Kompression}
                  % \item Häufigkeitsverteilung ist nicht bekannt $\rightarrow$ \alert{dynamische Kompression}
                \end{itemize}
              \end{minipage}
            }
          }
          child {
            node {Präfixcodes
              \resizebox{\textwidth}{!}{
                \begin{minipage}[t]{8cm}
                  \begin{itemize}
                    \item $a_{1}\ldots a_{p}\in A^{*}$ über einem Alphabet $A$ der Größe $m$ heißt \alert{Präfix} von $b_{1}\ldots b_{l}\in A^{*}$, falls $p\le l$ und $\forall i: a_i=b_i$, $1\le i\le p$
                    \item Ein Code $c:A\rightarrow\{0,1\}^{*}$ heißt \alert{Präfixcode}, falls es kein Paar $i,j\in\{\,1,\ldots,m\}$ gibt, so dass $c(a_i)$ Präfix von $c(a_j)$
                    \item Bei Präfixcodes können Wörter über $\{0, 1\}$ eindeutig dekodiert werden. (Sie entsprechen Binärbäumen mit codierten Zeichen an den Blättern.)
                  \end{itemize}
                \end{minipage}
              }
            }
            child {
              node {Huffman
                \resizebox{\textwidth}{!}{
                  \begin{minipage}[t]{10cm}
                    \begin{itemize}
                      \item \alert{Vorgehen:}
                      \begin{enumerate}
                        \item Baue binären Baum, indem die beiden kleinsten Häufigkeiten jeweils zu einem neuen Knoten addiert werden
                        \item Markiere die linken Kanten mit $0$ und die rechten Kanten mit $1$
                      \end{enumerate}
                      \item Huffman-Code ist ein bzgl. mittlerer Codelänge optimaler \alert{Präfixcode} (unter Voraussetzung einer bekannten Häufigkeitsverteilung)
                      \item Kommt als Teilschritt z.B. in MP3 oder JPEG vor.
                    \end{itemize}
                    \includegraphics[width=0.8\textwidth, center]{./figures/huffman_codierung.png}
                  \end{minipage}
                }
              }
            }
          }
        }
      }
    }
    child {
      node {Kodierung von Zahlen}
      child {
        node {Zahlensystem
          \resizebox{\textwidth}{!}{
            \begin{minipage}[t]{12cm}
              \begin{itemize}
                \item $S = (b, Z, \delta)$, 
                \begin{itemize}
                  \item \alert{Basis} $b\in\mathbb{N}: b\ge 1$ des Stellenwertsystems mit welcher für jede \alert{Stelle} $i$ der \alert{Stellenwert} $b^i$ berechnet wird
                  \begin{itemize}
                    \item bei der $n$-stelligen Binärdarstellung einer Zahl werden dem \alert{LSB} und \alert{MSB} jeweils die Stellenwerte $b^0$ und $b^{n-1}$ zugeordnet
                  \end{itemize}
                  \item $b$-elementige \alert{Ziffernmenge} von Symbolen $Z$, die \alert{Ziffern}
                  \item \alert{Ziffernwertigkeit} $\delta: Z\rightarrow \{0, 1, \ldots, b-1\}$ ordnet jeder \alert{Ziffer} bzw. Symbol ihre \alert{Wertigkeit} zu
                  \begin{itemize}
                    \item im z.B. \alert{Hexadezimalsystem} werden den sechzehn Ziffern $0, 1, 2, 3, 4, 5, 6, 7, 8, 9, A, B, C, D, E$ und $F$ jeweils die Werte der Dezimalzahlen von $0$ bis $15$ zugeordnet.
                  \end{itemize}
                \end{itemize}
                \item verwendetes Zahlensystem wird durch Anhängen der \alert{Basis als Index} an die Ziffernfolge ${d_{n-1}\ldots d_{0}}_b$ vermittelt
                \item \alert{Eselsbrücken fürs Hexadezimalsystem:} \alert{C}wölf, \alert{D}reizehn, \alert{F}ünfzehn
                  % \begin{itemize}
                  %   \begin{itemize}
                  %   \end{itemize}
                  %     % $(z_i)_{i=0,\ldots,n}$
                  %     % $z\ =\ z_{n-1}\ldots z_{0}$
                  % \end{itemize}
                  % \begin{itemize}
                  %     \begin{itemize}
                  %     \end{itemize}
                  % \end{itemize}
              \end{itemize}
            \end{minipage}
          }
        }
        child {
          node {Warum sich manche Zahlensystem einfacher ineinander unformen lassen
            \resizebox{\textwidth}{!}{
              \begin{minipage}[t]{8cm}
                \begin{align*}
                  ba_{16} &= b_{16} \cdot 16^1 + a_{16} \cdot 16^0\\
                          &= 13_{8} \cdot 16^1 + 12_{8} \cdot 16^0\\
                          &= (1_{8} \cdot 8^1 + 3_{8}) \cdot 16^1 + (1_{8} \cdot 8^1+  2_{8}) \cdot 16^0\\
                          &= (1_{8} \cdot 8^1 + 3_{8}) \cdot (8 \cdot 2)^1 + (1_{8} \cdot 8^1+  2_{8}) \cdot (8 \cdot 2)^0\\
                          &= 2_{8} \cdot 8^2 + 6_{8} \cdot 8^1 + 1_{8} \cdot 8^1 \cdot (8 \cdot 2)^0 + 2_{8} \cdot (8 \cdot 2)^0\\
                          &= 2_{8} \cdot 8^2 + 7_{8} \cdot 8^1 + 2_{8} \cdot 1 = 272_{8} \\
                  ba_{16} &= b_{16} \cdot 16^1 + a_{16} \cdot 16^0\\
                          &= 1011_{2} \cdot (2^4)^1 + 1010_{2} \cdot (2^4)^0 = 10111010_{2}
                \end{align*}
              \end{minipage}
            }
          }
        }
        child {
          node {Natürliche Zahl
            \resizebox{\textwidth}{!}{
              \begin{minipage}[t]{10cm}
                \begin{itemize}
                  \item \alert{positiver Wert} $\displaystyle \langle d\rangle\ =\ \sum_{i=0}^{n-1} \delta(d_{i})\cdot b^{i}$ einer \alert{nicht-negativen Natürlichen Zahl}, wobei $d=d_{n-1}\ldots d_0$ mit $d_i\in Z$ eine \alert{Folge} von $n$ \alert{Ziffern} bzw. Symbolen ist
                \end{itemize}
              \end{minipage}
            }
          }
        }
        child {
          node {Festkommazahl
            \resizebox{\textwidth}{!}{
              \begin{minipage}[t]{12cm}
                \begin{itemize}
                  \item eine endliche Folge von Ziffern aus einem Zahlensystem zur Basis $b$ mit Ziffernmenge $Z$
                  \item \alert{positiver Wert} $\displaystyle \langle d\rangle=\sum_{i=-k}^{n-1}\delta(d_{i})\cdot b^{i}$ einer \alert{nicht-negativen Festkommazahl}, wobei $d=d_{n-1}\ldots d_0\ldots d_{-k}$ mit $d_i\in Z$
                    \begin{itemize}
                      \item \alert{die Anzahl der Nachkommastellen ist fest:} $\langle d_{n-1}\ldots d_0 d_{-1}\ldots d_{-k}\rangle \cdot 2^{-k} = \langle d_{n-1}\ldots d_0\textcolor{PrimaryColor}{.}d_{-1}\ldots d_{-k}\rangle$
                      \item Beispiel 3-Bit Festkommazahlen mit $n=1$ und $k=2$:
                        \begin{table}
                          \raggedright
                          \begin{tblr}{
                              cells = {c, BoxColor},
                              column{1} = {PrimaryColor,fg=white},
                            }
                            $d$                & 0.00 & 0.01 & 0.10 & 0.11 & 1.00 & 1.01 & 1.10 & 1.11 \\
                            $\langle d\rangle$ & 0.0  & 0.25 & 0.5  & 0.75 & 1.0  & 1.25 & 1.5  & 1.75 \\
                          \end{tblr}
                        \end{table}
                    \end{itemize}
                    \item Probleme:
                    \begin{itemize}
                      \item keine ganz großen bzw. kleinen Zahlen darstellbar
                      \item weitere Probleme \href[page=152]{./Technische_Informatik_all_in_one_with_go_back.pdf}{hier} erklärt
                    \end{itemize}
                \end{itemize}
                \begin{figure}
                  \begin{subfigure}{0.4\textwidth}
                    \centering
                    \includegraphics[width=0.8\linewidth]{figures/binary_fraction}
                    \caption{Binärbrüche}
                    \label{fig:binaryfraction}
                  \end{subfigure}
                  \begin{subfigure}{0.4\textwidth}
                    \centering
                    \includegraphics[width=0.6\linewidth]{figures/decimal_fraction}
                    \caption{Dezimalbrüche}
                    \label{fig:decimalfraction}
                  \end{subfigure}
                \end{figure}
              \end{minipage}
            }
          }
          child {
            node {Negative Festkommazahl
              \resizebox{\textwidth}{!}{
                \begin{minipage}[t]{8cm}
                  \begin{itemize}
                    \item mehrere mögliche Darstellungen für \alert{negative Festkommazahlen}, wobei $d=d_{n}d_{n-1}\ldots d_0\ldots d_{-k}$ mit $\forall i\langle n:d_i\in Z$ und $d_n\in\{0, 1\}$:
                    \begin{itemize}
                      \item $n+1$ \alert{Vorkommastellen}, wovon allderings ein Bit ein \alert{Vorzeichenbit} ist und $n$ Bits \alert{Nicht-Vorzichenbits} sind
                      \item $k$ \alert{Nachkommastellen}
                    \end{itemize}
                    \item ein Vorteil von Einerkomplement- und Zweierkomplement-Darstellung ist, dass sie \alert{zyklisch} sind:
                      \begin{align*}
                        [01.1]_2 + 0.5 &= [10.0]_2 &\text{(nach } 1.5 \text{ geht es mit } {-}2 \text{ weiter)}\\
                        [01.1]_1 + 0.5 &= [10.0]_1 &\text{(nach } 1.5 \text{ geht es mit } {-}1.5 \text{ weiter)}\\
                      \end{align*}
                  \end{itemize}
                \end{minipage}
              }
            }
            child {
              node {Betrag und Vorzeichen Darstellung
                \resizebox{\textwidth}{!}{
                  \begin{minipage}[t]{12cm}
                    \begin{itemize}
                      \item \alert{potentiell negativer Wert} $\displaystyle[d]_{BV} = (-1)^{d_n}\sum_{i=0}^{n-1}\delta(d_i)2^i$ in \alert{Darstellung durch Betrag und Vorzeichen}
                        \begin{itemize}
                          \item Beispiel 3-Bit Festkommazahlen mit Vorzeichenbit, $n=1$ und $k=1$:
                            \begin{table}
                              \raggedright
                              \begin{tblr}{
                                  cells = {c, BoxColor},
                                  column{1} = {PrimaryColor,fg=white},
                                }
                                $d$        &  11.1  & 11.0 & 10.1 & 10.0 / 00.0 & 00.1 & 01.0 & 01.1 \\
                                $[d]_{BV}$ & -1.5 & -1.0 & -0.5 & 0.0 & 0.5 & 1.0  & 1.5 \\
                              \end{tblr}
                            \end{table}
                        \end{itemize}
                    \end{itemize}
                  \end{minipage}
                }
              }
            }
            child {
              node {Einer-Komplement Darstellung
                \resizebox{\textwidth}{!}{
                  \begin{minipage}[t]{12cm}
                    \begin{itemize}
                      \item \alert{potentiell negativer Wert} $\displaystyle[d]_{1} = \sum_{i=0}^{n-1}\delta(d_i) 2^i - \delta(d_n)(2^n-2^{-k})$ in \alert{Einerkomplement-Darstellung}
                        \begin{itemize}
                          \item Beispiel 3-Bit Festkommazahlen mit Vorzeichenbit, $n=1$ und $k=1$:
                            \begin{table}
                              \raggedright
                              \begin{tblr}{
                                  cells = {c, BoxColor},
                                  column{1} = {PrimaryColor,fg=white},
                                }
                                $d$     & 10.0 & 10.1 & 11.0 & 11.1 / 00.0 & 00.1 & 01.0 & 01.1 \\
                                $[d]_1$ & -1.5  & -1.0 & -0.5  & 0.0 & 0.5 & 1.0  & 1.5 \\
                              \end{tblr}
                            \end{table}
                        \end{itemize}
                    \end{itemize}
                    \begin{itemize}
                      \item im Negativen hat eine Folge mit mehr $1$en anders als im Positiven betragsmäßig eine kleineren Wert, denn umso mehr $1$en da sind, umso größer wird die Summe $\sum_{i=0}^{n-1}\delta(d_i)2^i$ und umso mehr kann von der subtrahierten größten positiven Zahl $2^n-1$ ausgeglichen werden
                      \item um den Wert einer negativen Dezimalzahl binär darzustellen gibt es 2 Methoden:
                        \begin{itemize}
                          \item mit größtmöglichen positven Zahl anfangen $-(2^n-1)$ und überlegen, welche $2$er Potenzen (Stellenwerte) man draufaddieren muss, um die gewünschte Zahl zu erhalten und an den entsprechenden Bits, die diesen $2$er Potenzen entsprechen $1$en setzen
                          \item die passende Ziffernfolge wie gewohnt erstellen, aber mit vertauschten $1$en und $0$en und das Vorzeichenbit ist eine $1$
                        \end{itemize}
                    \end{itemize}
                  \end{minipage}
                }
              }
              child {
                node {Inversion
                  \resizebox{\textwidth}{!}{
                    \begin{minipage}[t]{8cm}
                      \begin{itemize}
                        \item $[a']_1 = -[a]_1$
                      \end{itemize}
                    \end{minipage}
                  }
                }
              }
            }
            child {
              node {Zweier-Komplement Darstellung
                \resizebox{\textwidth}{!}{
                  \begin{minipage}[t]{12cm}
                    \begin{itemize}
                      \item man springt einen kleisten Zahlenabstand weiter wenn das Vorzeichenbit $1$ ist als beim Einerkomplement
                      \item \alert{potentiell negativer Wert} $\displaystyle[d]_{2} = \sum_{i=0}^{n-1}\delta(d_i)2^i - \delta(d_n)2^n$ in \alert{Zweierkomplement-Darstellung}
                        \begin{itemize}
                          \item Beispiel 3-Bit Festkommazahlen mit Vorzeichenbit, $n=1$ und $k=1$:
                            \begin{table}
                              \raggedright
                              \begin{tblr}{
                                  cells = {c, BoxColor},
                                  column{1} = {PrimaryColor,fg=white},
                                }
                                $d$                & 10.0 & 10.1 & 11.0 & 11.1 & 00.0 & 00.1 & 01.0 & 01.1 \\
                                $[d]_2$ & -2  & -1.5 & -1.0  & -0.5 & 0.0  & 0.5 & 1.0  & 1.5 \\
                              \end{tblr}
                            \end{table}
                        \end{itemize}
                    \end{itemize}
                    \begin{itemize}
                      \item genauso wie beim Einerkomplment bedeuten mehr $1$en im Negativen einen betragsmäßig kleineren Wert
                      \item um den Wert einer negativen Dezimalzahl binär darzustellen gibt es 2 Methoden
                        \begin{itemize}
                          \item mit der größtmöglichen positven Zahl + kleinmöglichen positven Zahl ungleich $0$ anfangen $-(2^n)$ und überlegen, welche $2$er Potenzen (Stellenwerte) man draufaddieren muss, um die gewünschte Zahl zu erhalten und an den entsprechenden Bits, die diesen $2$er Potenzen entsprechen $1$en setzen
                          \item die passende Ziffernfolge wie gewohnt erstellen, aber die kleinstmögliche positve Zahl ungleich $0$ wird als Startwert genommen und die $1$en und $0$en sind vertauscht, sowie das Vorzeichenbit ist eine $1$
                        \end{itemize}
                      \item wenn man die kleinste negative Zahl komplementiert: $[10.0]_2' = [01.1]_1 + 0.5 = [10.0]_2$, dann erhält man erneut die kleinste negative Zahl. Das passt auch ganz gut, da die kleinste negative Zahl im Zweierkomplement keine komplementäre positive Zahl hat. Folglich auch hier: $[a]_2 + [a']_2 + 2^{-k} = [1.00]_2 + [0.11]_2 + 0.5 = -2 + 1.5 + 0.5 = 0$ für $a$ kleinste Zweierkomplement-Zahl
                    \end{itemize}
                  \end{minipage}
                }
              }
              child {
                node {Inversion
                  \resizebox{\textwidth}{!}{
                    \begin{minipage}[t]{8cm}
                      \begin{itemize}
                        \item $[a']_2 + 1 = -[a]_2$
                      \end{itemize}
                    \end{minipage}
                  }
                }
              }
            }
          }
        }
        child {
          node {Gleitkommazahl
            \resizebox{\textwidth}{!}{
              \begin{minipage}[t]{16cm}
                \begin{itemize}
                  \item Darstellung \alert{negativer Gleitkommazahlen}, wobei $\underbrace{d_n}_{sign} \underbrace{d_{n-1} \ldots d_0}_{exponent} \underbrace{d_{-1}\ldots d_{-k}}_{fraction / mantissa}$ mit exponent, fraction bits und sign bit aus $\{0, 1\}$:
                    \begin{itemize}
                      \item \alert{varierende Anzahl von Nachkommastellen} im Gegensatz zu Festkommazahlen
                      \item \alert{Normalisierte Zahlen:} $(-1)^{d_n} \cdot \langle 1 d_{-1}\ldots d_{-k}\rangle \cdot 2^{-k+\langle d_{n-1}\ldots d_{0}\rangle-(2^{n-1}-1)} = (-1)^{sign} \times (1 + fraction) \times 2^{exponent - bias}$
                        \begin{itemize}
                          \item \alert{Bias:} $2^{n-1} - 1 = \dfrac{2^2}{2} - 1 = 2 - 1 = \boxed{01_{2}}$, \alert{größter, kleinster Exponent:} $11_2 - 01_2 = \boxed{10_2}$, $01_2 - 01_2 = \boxed{00_2}$
                          \item \alert{Betragsmäßig kleinste, größte Zahl:} $\pm 1.0\times 2^{1-1} =\pm 1.0$ ($\boxed{\frac{0}{1}\mid 01\mid 0}$), $\pm 1.5 \times 2^{2-1} =\pm 3$ ($\boxed{\frac{0}{1}\mid 10\mid 1}$)
                        \end{itemize}
                      \item \alert{Denormalisierte Zahlen:} $(-1)^{d_n} \cdot \langle 0 d_{-1}\ldots d_{-k}\rangle \cdot 2^{-k+\langle d_{n-1}\ldots d_{0}\rangle-(2^{n-1}-2)} = (-1)^{sign} \times (0 + fraction) \times 2^{-bias}$
                        \begin{itemize}
                          \item \alert{Bias:} $2^{n-1} - 2 = \dfrac{2^2}{2} - 2 = 2 - 2 = \boxed{00_{2}}$, \alert{einziger Exponent:} $00_2 - 00_2 = \boxed{00_2}$
                          \item \alert{Betragsmäßig kleinste, größte Zahl:} $0.0 \times 2^{0-0} = 0$ ($\boxed{\frac{0}{1}\mid 00\mid 0}$), $\pm 0.5 \times 2^{0-0} =\pm 0.5$ ($\boxed{\frac{0}{1}\mid 00\mid 1}$)
                        \end{itemize}
                        \begin{itemize}
                          \item $1 \textcolor{PrimaryColor}{.}d_{-1}\ldots d_{-k}$ wird als \alert{normalized significand} bezeichnet
                          \item $0 \textcolor{PrimaryColor}{.}d_{-1}\ldots d_{-k}$ wird als \alert{normed significand} bezeichnet
                        \end{itemize}
                      \item der \alert{Exponent} ist immer als \alert{nicht-negative} Natürliche Zahl zu interpretieren, die \alert{Mantissa} ist immer als \alert{nicht-negativer Bruch} zu interpretieren, also mit $2^{-k}$ zu multiplizieren
                      \item der \alert{Bias} macht den $encoded\_exponent$ immer \alert{positiv}
                        \begin{itemize}
                          \item $encoded\_exponent = real\_exponent + bias$
                          \item $real\_exponent = encoded\_exponent - bias$
                        \end{itemize}
                      \item \alert{Zero:} $\boxed{\frac{0}{1}\mid 00\mid 0}$, \alert{Infinity:} $\boxed{\frac{0}{1}\mid 11\mid 0}$, \alert{NaN:} $\boxed{\frac{0}{1}\mid 11\mid 1}$
                      \item Beispiel 4-Bit Gleitkommazahl mit Vorzeichenbit, $2$ Exponentbits und Mantissabit:
                        {\tiny
                          \begin{table}
                            \raggedright
                            \begin{tblr}{
                                cells = {c, BoxColor},
                                column{1} = {PrimaryColor,fg=white},
                              }
                              $d$      & $1\mid00\mid0$ & $1\mid00\mid1$ & $1\mid01\mid0$  & $1\mid01\mid1$ & $1\mid10\mid0$ & $1\mid10\mid1$ & $1\mid11\mid0$ & $1\mid11\mid1$ \\
                              $[d]_{GK}$  & 0.0  & -0.5 & -1.0 & -1.5 & -2.0  & -3.0 & $-\infty$  & NaN \\
                              $d$ als BV  & 0.0            & -0.1           & -1.0           & -1.1           & -10.0          & -11.0          & -              & -              \\
                            \end{tblr}
                          \end{table}
                          \begin{table}
                            \raggedright
                            \begin{tblr}{
                                cells = {c, BoxColor},
                                column{1} = {PrimaryColor,fg=white},
                              }
                              $d$      & $0\mid00\mid0$ & $0\mid00\mid1$ & $0\mid01\mid0$ & $0\mid01\mid1$ & $0\mid10\mid0$ & $0\mid10\mid1$ & $0\mid11\mid0$ & $0\mid11\mid1$ \\
                              $[d]_{GK}$  & 0.0  & 0.5 & 1.0 & 1.5 & 2.0  & 3.0 & $\infty$  & NaN \\
                              $d$ als BV  & 0.0            &  0.1           &  1.0           & 1.1            & 10.0           & 11.0           & -              & -              \\
                              % 0.0, 0.5, 1.0, 1.5, 2.0, 3.0, 4.0, 6.0
                            \end{tblr}
                          \end{table}
                        }
                      \item Beispiel \alert{ohne angepassten Bias} ($-2$) für \alert{Denormalisierte Zahlen} mit 4-Bit Gleitkommazahl mit Vorzeichenbit, $2$ Exponentbits und Mantissabit:
                        {
                          \tiny
                          % \begin{table}
                          %   \raggedright
                          %   \begin{tblr}{
                          %       cells = {c, BoxColor},
                          %       column{1} = {PrimaryColor,fg=white},
                          %     }
                          %     $d$      & $1\mid00\mid0$ & $1\mid00\mid1$ & $1\mid01\mid0$  & $1\mid01\mid1$ & $1\mid10\mid0$ & $1\mid10\mid1$ & $1\mid11\mid0$ & $1\mid11\mid1$ \\
                          %     $[d]_{GK}$  & 0.0  & -0.25& -1.0 & -1.5 & -2.0  & -3.0 & $\infty$  & NaN \\
                          %     $d$ als BV  & 0.0            & -0.01           & -1.0           & -1.1           & -10.0          & -11.0          & -              & -              \\
                          %   \end{tblr}
                          % \end{table}
                          \begin{table}
                            \raggedright
                            \begin{tblr}{
                                cells = {c, BoxColor},
                                column{1} = {PrimaryColor,fg=white},
                              }
                              $d$      & $0\mid00\mid0$ & $0\mid00\mid1$ & $0\mid01\mid0$ & $0\mid01\mid1$ & $0\mid10\mid0$ & $0\mid10\mid1$ & $0\mid11\mid0$ & $0\mid11\mid1$ \\
                              $[d]_{GK}$  & 0.0  & 0.25& 1.0 & 1.5 & 2.0  & 3.0 & $\infty$  & NaN \\
                              $d$ als BV  & 0.0            &  0.01           &  1.0           & 1.1            & 10.0           & 11.0           & -              & -              \\
                            \end{tblr}
                          \end{table}
                        }
                      \item Beispiel \alert{ohne um $-1$ geshifteten Bias} mit 4-Bit Gleitkommazahl mit Vorzeichenbit, $2$ Exponentbits und Mantissabit:
                        {
                          \tiny
                          %     \begin{table}
                          %       \raggedright
                          %       \begin{tblr}{
                          %           cells = {c, BoxColor},
                          %           column{1} = {PrimaryColor,fg=white},
                          %         }
                          %         $d$      & $1\mid00\mid0$ & $1\mid00\mid1$ & $1\mid01\mid0$  & $1\mid01\mid1$ & $1\mid10\mid0$ & $1\mid10\mid1$ & $1\mid11\mid0$ & $1\mid11\mid1$ \\
                          %           $[d]_{GK}$  & 0.0  & -0.25& -0.5 & -0.75 & -1.0  & -2.0 & $\infty$  & NaN \\
                          % % 0.0, 0.25, 0.5, 0.75, 1.0, 1.5, 2.0, 3.0
                          %           $d$ als BV  & 0.0            &  -0.01           &  -0.1           & -0.11            & -1.0           & -10.0           & -              & -              \\
                          %       \end{tblr}
                          %     \end{table}
                          \begin{table}
                            \raggedright
                            \begin{tblr}{
                                cells = {c, BoxColor},
                                column{1} = {PrimaryColor,fg=white},
                              }
                              $d$      & $0\mid00\mid0$ & $0\mid00\mid1$ & $0\mid01\mid0$ & $0\mid01\mid1$ & $0\mid10\mid0$ & $0\mid10\mid1$ & $0\mid11\mid0$ & $0\mid11\mid1$ \\
                              $[d]_{GK}$  & 0.0  & 0.25& 0.5 & 0.75 & 1.0  & 1.5 & $\infty$  & NaN \\
                              % 0.0, 0.25, 0.5, 0.75, 1.0, 1.5, 2.0, 3.0
                              $d$ als BV  & 0.0            &  0.01           &  0.1           & 0.11            & 1.0           & 1.1           & -              & -              \\
                            \end{tblr}
                          \end{table}
                        }
                        \begin{itemize}
                          \item es gibt mehr positive Exponenten als negative Exponenten 
                          \item die Zahl $1.0$ und damit viele Festkommazahlen mit so vielen Mantissa-Bits, wie für die Gleitkommazahl zu Verfügung stehen sind sehr einfach zu kodieren, da $(1+fraction) \cdot 2^{\sum_{i=0}^{n-2} 1 \cdot 2^i - (2^{n-1}-1)} = (1+fraction) \cdot 2^{2^{n-1}-1 - 2^{n-1}+1} = (1+fraction) \cdot 2^0 = (1+fraction)$
                        \end{itemize}
                      \item es gibt \alert{verschiedene Standards}, wie \alert{Bfloat16} ($1$ sign bit, $8$ exponent bits, $7$ fraction bits), \alert{Single-precision} ($1$ sign bit, $8$ exponent bits, $23$ fraction bits) und \alert{Double-precision} ($1$ sign bit, $11$ exponent bits, $52$ fraction bits), die sich in der \alert{Anzahl der Bits} für \alert{Exponent} und \alert{Mantissa} unterscheiden
                      \item \alert{Runden mit GRS-Bits:}
                        \begin{itemize}
                          \item mit GRS braucht man nur $3$ zusätzliche Bits und braucht so keinen großen und langsamen Addierer mit dem man die Berechnungen erstellt. Ohne GRS müsste man die Berechnungen mit allen Bits machen und am Ende runden
                          \item \alert{Guard und Round bit:} Zwei zusätzliche Bits die bei Berechnungen mit Gleitkommazahlen rechts angehängt werden, um die Rundungsgenauigkeit zu erhöhen
                          \item \alert{Sticky bit:} Zusätzliches Bits rechts der Bits Guard und Round, dass gesetzt wird, wann immer es Bits rechts davon gibt, die nicht $0$ sind
                          \item \alert{Ties to even:} Numbers exactly in the middle between two integer numbers (\enquote{ties}) are rounded towards the even number\\
                              \begin{minipage}{0.5\linewidth}
                                % \vspace{-0.25cm}
                                \begin{align*}
                                  0.5 \to 0,\\
                                  1.5 \to 2,\\
                                  2.5 \to 2
                                \end{align*}
                              \end{minipage}
                              \begin{minipage}{0.5\linewidth}
                                % \vspace{-0.75cm}
                                \begin{table}
                                  \centering
                                  \begin{tblr}{
                                      cells = {c, BoxColor},
                                      row{1} = {PrimaryColor,fg=white},
                                    }
                                    G & R & S & Ergebnis \\
                                      &     & 1/8 & 0.125 \\
                                      & 1/4 &     & 0.25  \\
                                      & 1/4 & 1/8 & 0.375 \\
                                    1/2 &     &     & 0.5   \\
                                    1/2 &     & 1/8 & 0.625 \\
                                    1/2 & 1/4 &     & 0.75  \\
                                    1/2 & 1/4 & 1/8 & 0.875
                                  \end{tblr}
                                \end{table}
                              \end{minipage}
                        \end{itemize}
  \end{itemize}
      \begin{itemize}
        \item the Anordnung, dass der \alert{Exponent} immer vor der \alert{Mantissa} steht und der Fakt, dass der kodierte Exponent positiv ist, ist so gewählt, damit man Gleitkommazahlen möglichst einfach vergleichen kann
        \item \alert{Festkommazahlen} haben eine \alert{kleinere Repräsentationsspanne}, da sie nur eine \alert{feste Anzahl an Nachkommastellen} haben. Gleitkommazahlen sind \alert{nicht gleichmäßig verteilt}, man hat eine sehr \alert{hohe Dichte kleiner Zahlen} nahe der $0$, neben einem schmallen Bereich nahe der $0$ mit gar keinen Zahlen.\\[0.25cm]
        \begin{figure}
          \includegraphics[width=\linewidth]{./figures/scaling.png}
          \caption{Verteilung von Gleitkommazahlen}
        \end{figure}
        \vspace{-0.5cm}
        \tiny
        \begin{table}
          \centering
          \begin{tblr}{
              cells = {c, white},
              column{1} = {PrimaryColor,fg=white},
            }
            $d[i]$      & 0.101 & 1.01 & 10.1 & 101 \\
            $\langle d[i]\rangle$  & 0.625  & 1.25 & 2.5 & 5 & \\
            $\langle d[i]\rangle - \langle d[i-1]\rangle$  & 0.3125 & 0.625 & 1.25 & 2.5 & \\
          \end{tblr}
          \caption{Abstände erhöhen sich exponentiel}
        \end{table}
      \end{itemize}
                \end{itemize}
              \end{minipage}
            }
          }
        }
      }
    }
    child {
      node {Mathematische Grundlagen
        \resizebox{\textwidth}{!}{
          \begin{minipage}[t]{12cm}
          \end{minipage}
        }
      }
      child {
        node {Groß-O-Notation
          \resizebox{\textwidth}{!}{
            \begin{minipage}[t]{12cm}
              \begin{itemize}
                \item Man will Funktionen irendwie vergleichen und das einzige sinnvolle was man vergleichen kann ist die \alert{Wachstumsrate}, denn für zwei Funktionen kann gleichzeitig gelten: $g = O(f)$, also $g$ wächst nicht stärker als $f$ und $g > f$, also $g$ ist überall echt größer als $f$
                \item Konstate Faktoren und Summanden $c \cdot x$ und $c + x$ sollen keine Rolle, aber es geht um \alert{Wachstumsrate}, Konstante Faktoren und Summanden spielen keine Rolle
                \item man interessiert sich meist nur für die kleinste/größtere untere/obere Grenze, z.B. wenn $f\in O(n)$ dann gilt automatisch auch $f\in O(n^2)$, aber man interessiert sich nur für die kleinste untere Grenze $f\in \Omega(n)$
                \begin{itemize}
                  \item man sagt die \enquote{Laufzeit eines Algorihmus ist: $\Omega(n\cdot log(n))$}, man sagt \alert{NICHT}, dass ein \enquote{Algorithmus Laufzeit mindestens $O(n\cdot log(n))$ hat}
                \end{itemize}
                \item Die Operatoren $O, \Omega, \Theta, o, \omega$ sind auf Funktionen, was die Operatoren $\le, \ge, =, <, >$ auf Zahlen sind
                \item $f= O(g)$ wird gesprochen als \enquote{$f$ ist Groß-O von $g$}, wobei eigentlich $f\in O(g)$ mathematisch korrekt ist
              \end{itemize}
            \end{minipage}
          }
        }
        child {
          node {Landau-Symbole
            \resizebox{\textwidth}{!}{
              \begin{minipage}[t]{16cm}
            \begin{itemize}
              \item \alert{formal:}
              \begin{itemize}
                \item $O(g)= \{h: \mathbb{N}\to\mathbb{R} \;|\; \exists n_0\in\mathbb{N}, \exists c > 0, \forall n\ge n_0, h(n)\le c \cdot g(n)\}$
                  \begin{itemize}
                    \item $f\in O(g)$ bedeutet \enquote{$f$ wächst \alert{höchstens} so stark wie $g$}
                  \end{itemize}
                \item $\Omega(g)= \{h: \mathbb{N}\to\mathbb{R} \;|\; \exists n_0\in\mathbb{N}, \exists c > 0, \forall n\ge n_0, h(n)\ge c \cdot g(n)\}$
                  \begin{itemize}
                    \item $f\in \Omega(g)$ bedeutet \enquote{$f$ wächst \alert{mindestens} so stark wie $g$}
                  \end{itemize}
                \item $\Theta(g)= \{h: \mathbb{N}\to\mathbb{R} \;|\; \exists n_0\in\mathbb{N}, \exists c_1 > 0, \exists c_2 > 0, \forall n\ge n_0, c_1 \cdot g(n) \le h(n)\le c_2 \cdot g(n)\}$
                  \begin{itemize}
                    \item $f\in \Theta(g)$ bedeutet \enquote{$f$ wächst \alert{genauso} stark wie $g$}
                  \end{itemize}
                \item $o(g)= \{h: \mathbb{N}\to\mathbb{R} \;|\; \forall c > 0, \exists n_0\in\mathbb{N}, \forall n\ge n_0, h(n)\le c \cdot g(n)\}$
                  \begin{itemize}
                    \item $f\in o(g)$ bedeutet \enquote{$f$ wächst \alert{(strikt) langsamer} als $g$}
                  \end{itemize}
                \item $\omega(g)= \{h: \mathbb{N}\to\mathbb{R} \;|\; \forall c > 0, \exists n_0\in\mathbb{N}, \forall n\ge n_0, h(n)\ge c \cdot g(n)\}$
                  \begin{itemize}
                    \item $f\in \omega(g)$ bedeutet \enquote{$f$ wächst \alert{(strikt) schneller} als $g$}
                  \end{itemize}
              \end{itemize}
            \end{itemize}
            \begin{itemize}
              \item $o(g)\cap\omega(g)=\emptyset$,
              \item $\Theta(g) = O(g)\cap\Omega(g)$
              \item $f(n)\in\Theta(g(n))$ $\Leftrightarrow$ $f(n)\in O(g(n))$ and $f(n)\in \Omega(g(n))$
            \end{itemize}
            \begin{itemize}
              \item \alert{Transpose symmetry:}
              \begin{itemize}
                \item $f(n)=O(g(n))$ $\Leftrightarrow$ $g(n)=\Omega(f(n))$
                \item $f(n)=o(g(n))$ $\Leftrightarrow$ $g(n)=\omega(f(n))$
              \end{itemize}
              \item \alert{Bestimmung über Grenzwerte:}
              \begin{enumerate}
                \item $\displaystyle f=O(g) \Leftrightarrow \operatorname{lim}_{n\to\infty}\frac{f(n)}{g(n)} < \infty$
                \item $\displaystyle f=\Omega(g) \Leftrightarrow \operatorname{lim}_{n\to\infty}\frac{f(n)}{g(n)} > 0$
                \item $\displaystyle f=\Theta(g) \Leftrightarrow \operatorname{lim}_{n\to\infty}\frac{f(n)}{g(n)} > 0$ und $\displaystyle\operatorname{lim}_{n\to\infty}\frac{f(n)}{g(n)} < \infty$
                \item $\displaystyle f=o(g) \Leftrightarrow \operatorname{lim}_{n\to\infty}\frac{f(n)}{g(n)} = 0$
                \item $\displaystyle f=\omega(g) \Leftrightarrow \operatorname{lim}_{n\to\infty}\frac{f(n)}{g(n)} = \infty$
              \end{enumerate}
            \end{itemize}
              \end{minipage}
            }
          }
        }
        child {
          node {Regel von L'Hopital
            \resizebox{\textwidth}{!}{
              \begin{minipage}[t]{8cm}
                \begin{itemize}
                  \item $\displaystyle \operatorname*{lim}_{x\to c}{\frac{f(x)}{g(x)}}=\operatorname*{lim}_{x\to c}{\frac{f^{\prime}(x)}{g^{\prime}(x)}}$
                  \begin{itemize}
                    \item wird verwendet bei Grenzwertbetrachtung bei der man bei Zähler und Nenner entweder den Fall $\frac{0}{0}$ oder $\frac{\pm\infty}{\pm\infty}$ hat und beide Ableitungen differenzierbar % und $g(x)\ne 0$ für $x\ne y$ für den Fall $\frac{0}{0}$ und $g(x)'\ne 0$ für den Fall $\frac{\infty}{\infty}$ 
                    \item dann Zähler und Nenner des Bruches getrennt voneinander ableiten und dann nochmal Grenwertbetrachtung
                    \item wenn nochmal unbestimmter Ausdruck rauskommt und wieder entweder der Fall $\frac{0}{0}$ oder $\frac{\infty}{\infty}$ und beide Ableitungen differenzierbar, dann nochmal anwenden oder sonst Pech gehabt
                  \end{itemize}
                \end{itemize}
              \end{minipage}
            }
          }
        }
      }
      child {
        node {Aussagenlogik}
      }
      child {
        node {Mengenlehre
          \resizebox{\textwidth}{!}{
            \begin{minipage}[t]{12cm}
              \begin{itemize}
                \item \alert{Menge:} Zusammenfassung von paarweise verschiedenen Objekten zu einem Ganzen
                \begin{itemize}
                  \item die Objekte nennt man \alert{Elemente} $a_i\in M$ der Menge
                \end{itemize}
                \item \underline{\alert{Spezifikation} einer Menge, Beispiele:}
                \begin{itemize}
                  \item z.B. $\mathbb{Z} = \{z, -z \;|\; z\in\mathbb{N}\}$
                  \item z.B. $\mathbb{Q} = \{p/q \;|\; p\in\mathbb{Z}, q\in\mathbb{N}, q\ne 0, p, q\text{ teilerfremd}\}$
                \end{itemize}
                \item \alert{Potenzmenge:} $\mathcal{P}(M)=\{m\;|\;m\subseteq M\}$
                \item \alert{Mächtigkeit / Kardinalität:} Anzahl $|M|$ der Elemente einer Mengen $M$
                \item \alert{Operationen:}
                \begin{itemize}
                  \item \alert{Mengendifferenz:}\\ 
                    $M_1 \setminus M_2 = \{m \;|\; m\in M_1 \text{ und } m\not\in M_2\}$
                  \item \alert{Mengenschnitt:}\\
                    $M_1 \cap M_2 = \{m \;|\; m\in M_1 \text{ und } m\in M_2\}$
                  \item \alert{Mengenvereinigung:}\\
                    $M_1 \cup M_2 = \{m \;|\; m\in M_1 \text{ oder } m\in M_2\}$
                  \item \alert{Kartesisches Produkt:}\\
                    $M_1 \times M_2 = \{(m_1, m_2) \;|\; m_1\in M_1 \text{ und } m_2\in M_2\}$
                  \begin{itemize}
                    \item \alert{Notation:} $M^n = \overbrace{M\times \ldots \times M}^{n\text{ mal}}$
                    \item bei einem Tupel $(m_1, m_2)$ ist im Gegensatz zu einer Menge $\{m_1, m_2\}$ die Reihenfolge wichtig
                    \item \alert{Relation:} $R\subseteq A\times B$
                    \begin{itemize}
                      \item statt $(x, y)\in \mathbb{R}$ schreibt man $xRy$
                      \item \alert{Spezifikation} gleich wie bei Mengen, da eine Relation eine Menge von Tupeln ist, z.B.:\\ $R = \{(a, b)\;|\; a,b\in \mathbb{N}, a + b\text{ ungerade}\}$
                    \end{itemize}
                  \end{itemize}
                \end{itemize}
              \end{itemize}
            \end{minipage}
          }
        }
        child {
          node {}
        }
      }
      child {
        node {Beweise}
      }
    }
    child {
      node {RETI-Maschine}
    };
  \end{scope}
  % ┌───────────────────┐
  % │ Verbindungslinien │
  % └───────────────────┘
  \begin{pgfonlayer}{background}
  % \draw [concept connection]
  %     (datahazardsforbranches) edge (forwarding);
  \end{pgfonlayer}
  % ┌──────────────┐
  % │ Annotationen │
  % └──────────────┘
  % https://tex.stackexchange.com/questions/302976/node-positioning-middle-point-mind-map-connection-bar
  \node [annotation, below] at (ca.south) {This mindmap is provided without guarantee of correctness and completeness!};
  \end{tikzpicture}
\end{document}
