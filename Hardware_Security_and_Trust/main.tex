\documentclass{standalone}

%!Tex Root = ../main.tex

% ┌────────────┐
% │ Formatting │
% └────────────┘
\usepackage[english]{babel}
\usepackage[top=0cm,bottom=0cm,left=0cm,right=0cm]{geometry}
\usepackage[export]{adjustbox} % use c, l, r for images
\usepackage{csquotes}
\usepackage[parfill]{parskip}
\usepackage{fontspec}
% \usepackage{anyfontsize}
% \usepackage[]{enumitem}

% ┌──────┐
% │ Math │
% └──────┘
\usepackage{amssymb} % for black triangleright, https://tex.stackexchange.com/questions/570303/use-blacktriangleright-as-itemize-label
\usepackage{amsmath}
\usepackage{mathtools} % for \mathclap and 
\usepackage{breqn}

% ┌────────┐
% │ Tables │
% └────────┘
\usepackage{tabularray}
 % \UseTblrLibrary{diagbox}

% ┌────────┐
% │ Images │
% └────────┘
\usepackage{graphicx}
% \usepackage{float} % for the letter H
% \graphicspath{figures/}
\usepackage{subcaption}

% ┌────────┐
% │ Graphs │
% └────────┘
\usepackage{tikzit}
\usepackage{tikz}
\usetikzlibrary{backgrounds}
\usetikzlibrary{arrows}
\usetikzlibrary{shapes,shapes.geometric,shapes.misc}

% this style is applied by default to any tikzpicture included via \tikzfig
\tikzstyle{tikzfig}=[baseline=-0.25em,scale=0.5]

% these are dummy properties used by TikZiT, but ignored by LaTex
\pgfkeys{/tikz/tikzit fill/.initial=0}
\pgfkeys{/tikz/tikzit draw/.initial=0}
\pgfkeys{/tikz/tikzit shape/.initial=0}
\pgfkeys{/tikz/tikzit category/.initial=0}

% standard layers used in .tikz files
\pgfdeclarelayer{edgelayer}
\pgfdeclarelayer{nodelayer}
\pgfsetlayers{background,edgelayer,nodelayer,main}

% style for blank nodes
\tikzstyle{none}=[inner sep=0mm]

% include a .tikz file
\newcommand{\tikzfig}[1]{%
{\tikzstyle{every picture}=[tikzfig]
\IfFileExists{#1.tikz}
  {\input{#1.tikz}}
  {%
    \IfFileExists{./figures/#1.tikz}
      {\input{./figures/#1.tikz}}
      {\tikz[baseline=-0.5em]{\node[draw=red,font=\color{red},fill=red!10!white] {\textit{#1}};}}%
  }}%
}

% the same as \tikzfig, but in a {center} environment
\newcommand{\ctikzfig}[1]{%
\begin{center}\rm
  \tikzfig{#1}
\end{center}}

% fix strange self-loops, which are PGF/TikZ default
\tikzstyle{every loop}=[]


% ┌────────┐
% │ Citing │
% └────────┘
% \usepackage[style=authortitle]{biblatex}
% \addbibresource{./Graph_Theory.bib}
% \usepackage{cleveref}

% ┌──────────┐
% │ Diagrams │
% └──────────┘
% \usepackage{tikz}
% \usetikzlibrary{shadows, backgrounds} % , calc

% ┌──────────────────┐
% │ Multiple columns │
% └──────────────────┘
% \usepackage{multicol}

% ┌────────────────────┐
% │ Code hightligthing │
% └────────────────────┘
% \usepackage{minted}

% ┌────────────────────────┐
% │ Latex Programming Help │
% └────────────────────────┘
\usepackage{etoolbox}
\usepackage{xparse}
% https://tex.stackexchange.com/questions/358292/creating-a-subcounter-to-a-counter-i-created
\usepackage{chngcntr}

% ┌───────────────┐
% │ Pretty Boxes  │
% └───────────────┘
\usepackage{xcolor}
\usepackage{tcolorbox}
\tcbuselibrary{skins,theorems}

% ┌──────────────┐
% │ Pseudo Code  │
% └──────────────┘
\usepackage{pseudo}

% \usepackage{background}
\usepackage{gradient-text}
\usepackage{rotating}

% \newlength\mylen
% \setlength\mylen{\dimexpr\paperwidth/80\relax}
%
% \SetBgScale{1}
% \SetBgAngle{0}
% \SetBgColor{blue!30}
% \SetBgContents{\tikz{\draw[step=\mylen] (-.5\paperwidth,-.5\paperheight) grid (.5\paperwidth,.5\paperheight);}}

% ┌────────────┐
% │ Misc Tools │
% └────────────┘
\usepackage{lipsum}

%!Tex Root = ../main.tex

% ┌────────────┐
% │ Formatting │
% └────────────┘
% \setlength{\parskip}{0.4cm} % space between paragraphs, https://latexref.xyz/bs-par.html

% ┌───────┐
% │ Fonts │
% └───────┘
\usepackage{fontspec}
\newfontfamily\gyre{DejaVu Math TeX Gyre}
% colored bold
% \newcommand\alert[1]{\textcolor{SwitchColor}{\textbf{#1}}}
\newcommand\alert[1]{\textcolor{SwitchColor}{#1}}

% ┌──────────────┐
% │ Pseudo Code  │
% └──────────────┘
\newcounter{algorithm}
\setcounter{algorithm}{0}
\newtcbtheorem[use counter=algorithm]{algorithm}{\color{SecondaryColor}Algorithm}{pseudo/ruled}{alg}
% \newcommand{\ma}[1]{$\mathcal{#1}$}
% \renewcommand{\tt}[1]{{\small\texttt{#1}}}

% ┌────────┐
% │ Colors │
% └────────┘
\definecolor{PrimaryColor}{HTML}{800080}
\definecolor{PrimaryColorDimmed}{HTML}{D6D6F0}
\definecolor{SecondaryColor}{HTML}{006BB6}
\definecolor{SecondaryColorDimmed}{HTML}{E5F0F8}
\definecolor{SwitchColor}{named}{PrimaryColor}
\colorlet{BoxColor}{gray!10!white}

% ┌───────┐
% │ Links │
% └───────┘
\usepackage[allbordercolors=PrimaryColor, pdfborder={0 0 .2}]{hyperref}

% ┌─────────┐
% │ Mindmap │
% └─────────┘
\renewcommand{\labelitemi}{$\textcolor{SwitchColor}{\bullet}$}
\renewcommand{\labelitemii}{$\textcolor{SwitchColor}{\blacktriangleright}$}
\renewcommand{\labelitemiii}{$\textcolor{SwitchColor}{\blacksquare}$}

%!Tex Root = ../main.tex

% ┌─────────┐
% │ Mindmap │
% └─────────┘
\newlength{\leveldistance}
\setlength{\leveldistance}{25cm}

\newenvironment{edges}{\begin{pgfonlayer}{background}\draw [concept connection]}{;\end{pgfonlayer}}
\newcommand{\edge}[2]{(#1) edge (#2)}
\newcommand{\annotation}[2]{\path (#1) -- node[annotation, above, align=center, pos=0.03] {#2} (middle);}

\newenvironment{resettikz}{\pgfsetlayers{nodelayer,edgelayer}\tikzset{every node/.style={fill opacity=1.0, draw opacity=1.0, minimum size=0cm, inner sep=0pt}}}{}

\newenvironment{mindmap}{
	\begin{tikzpicture}[
			auto,
			huge mindmap,
			fill opacity=0.6,
			draw opacity=0.8,
			concept color = PrimaryColorDimmed,
			every annotation/.style={fill=BoxColor, draw=none, align=center, fill = BoxColor, text width = 2cm},
			grow cyclic,
			level 1/.append style = {
					concept color=SecondaryColorDimmed,
					level distance=\leveldistance,
					sibling angle=360/\the\tikznumberofchildren,
					% https://tex.stackexchange.com/questions/501240/trying-to-use-the-array-environment-inside-a-tikz-node-with-execute-at-begin-no
					execute at begin node=\definecolor{SwitchColor}{named}{SecondaryColor}\definecolor{SwitchColorDimmed}{named}{PrimaryColorDimmed},
				},
			level 2/.append style = {
					concept color=PrimaryColorDimmed,
					level distance=\leveldistance / 2,
					sibling angle=35,
					execute at begin node=\definecolor{SwitchColor}{named}{PrimaryColor}\definecolor{SwitchColorDimmed}{named}{SecondaryColorDimmed},
				},
			level 3/.append style = {
					concept color=SecondaryColorDimmed,
					level distance=\leveldistance / 3,
					execute at begin node=\definecolor{SwitchColor}{named}{SecondaryColor}\definecolor{SwitchColorDimmed}{named}{PrimaryColorDimmed},
				},
			level 4/.append style = {
					concept color=PrimaryColorDimmed,
					level distance=\leveldistance / 4,
					execute at begin node=\definecolor{SwitchColor}{named}{PrimaryColor}\definecolor{SwitchColorDimmed}{named}{SecondaryColorDimmed},
				},
			level 5/.append style = {
					concept color=SecondaryColorDimmed,
					level distance=\leveldistance / 5,
					execute at begin node=\definecolor{SwitchColor}{named}{SecondaryColor}\definecolor{SwitchColorDimmed}{named}{PrimaryColorDimmed},
				},
			level 6/.append style = {
					concept color=PrimaryColorDimmed,
					level distance=\leveldistance / 6,
					execute at begin node=\definecolor{SwitchColor}{named}{PrimaryColor}\definecolor{SwitchColorDimmed}{named}{SecondaryColorDimmed},
				},
			level 7/.append style = {
					concept color=SecondaryColorDimmed,
					level distance=\leveldistance / 7,
					execute at begin node=\definecolor{SwitchColor}{named}{SecondaryColor}\definecolor{SwitchColorDimmed}{named}{PrimaryColorDimmed},
				},
			level 8/.append style = {
					concept color=PrimaryColorDimmed,
					level distance=\leveldistance / 8,
					execute at begin node=\definecolor{SwitchColor}{named}{PrimaryColor}\definecolor{SwitchColorDimmed}{named}{SecondaryColorDimmed},
				},
			level 9/.append style = {
					concept color=SecondaryColorDimmed,
					level distance=\leveldistance / 9,
					execute at begin node=\definecolor{SwitchColor}{named}{SecondaryColor}\definecolor{SwitchColorDimmed}{named}{PrimaryColorDimmed},
				},
			concept connection/.append style = {
					color = BoxColor,
				},
		]
		}{
	\end{tikzpicture}
}

\newenvironment{mindmapcontent}{
	\begin{scope}[
			every node/.style = {concept, circular drop shadow}, % draw=none
			every child/.style={concept},
		]
		}{
		;\end{scope}
}

% ┌───────┐
% │ Boxes │
% └───────┘
\DeclareTotalTCBox{\inlinebox}{ s m }
{standard jigsaw,opacityback=0,colframe=SwitchColor,nobeforeafter,tcbox raise base,top=0mm,bottom=0mm,
	right=0mm,left=0mm,arc=0.1cm,boxsep=0.1cm}
{\IfBooleanTF{#1}%
	{\textcolor{PrimaryColor}{\setBold >\enspace\ignorespaces}#2}%
	{#2}}

\DeclareTotalTCBox{\inlineboxtwo}{ s m }
{standard jigsaw,opacityback=0,colframe=SwitchColorDimmed,nobeforeafter,tcbox raise base,top=0mm,bottom=0mm,
	right=0mm,left=0mm,arc=0.1cm,boxsep=0.1cm}
{\IfBooleanTF{#1}%
	{\textcolor{SwitchColorDimmed}{\setBold >\enspace\ignorespaces}#2}%
	{#2}}

% ┌──────────────────┐
% │ Case distinction │
% └──────────────────┘
% \newtoggle{absolute}
% % \toggletrue{absolute}
% \togglefalse{absolute}
% \newcommand{\lpathgraph}[1]{\iftoggle{absolute}{/home/areo/Documents/Studium/Summaries/x/}{./}#1}

% ┌───────┐
% │ Fixes │
% └───────┘
% https://tex.stackexchange.com/questions/89467/why-does-pdftex-hang-on-this-file
% \newcommand{\colon}{\mathrel{\mathop:}}

% ┌───────┐
% │ Paths │
% └───────┘
% \newcommand{\script}[2]{\href[page=#1]{}{\inlinebox{#2}}}
\newcommand{\script}[2]{\href{openpdf:/home/areo/Documents/Studium/Semester_1_Master/Hardware_Security_and_Trust/slides/Slides annotated/Hardware_Security_and_Trust_all_in_one.pdf:#1}{\inlinebox{#2}}}
\newcommand{\scripttwo}[2]{\href{openpdf:///home/areo/Documents/Studium/Semester_1_Master/Hardware_Security_and_Trust/slides/Slides annotated/bonus/12_Lecture_06Dec.pdf:#1}{\inlinebox{#2}}}
\newcommand{\videoeight}[2]{\href{https://youtu.be/YcHSlFjcndU?feature=shared&t=#1}{\inlineboxtwo{#2}}}
\newcommand{\videonine}[2]{\href{https://youtu.be/3dL-3EOIfJ8?si=l3OakqHOeCpnNayw&t=#1}{\inlineboxtwo{#2}}}
\newcommand{\videoten}[2]{\href{https://youtu.be/6oF737pa510?feature=shared&t=#1}{\inlineboxtwo{#2}}}
\newcommand{\videoeleven}[2]{\href{https://youtu.be/PJTqfzTIYJs?feature=shared&t=#1}{\inlineboxtwo{#2}}}
\newcommand{\videotwelve}[2]{\href{https://youtu.be/oDxAH7aO-Tk?feature=shared&t=#1}{\inlineboxtwo{#2}}}
\newcommand{\videothirteen}[2]{\href{https://youtu.be/3TkSXxe_Ty8?feature=shared&t=#1}{\inlineboxtwo{#2}}}
\newcommand{\videofourteen}[2]{\href{https://youtu.be/1Y3dZuJ0MHg?feature=shared&t=#1}{\inlineboxtwo{#2}}}


\begin{document}
\begin{mindmap}
  \begin{mindmapcontent}
    \node (middle) at (current page.center) {Hardware Security and Trust
      \resizebox{\textwidth}{!}{
        \begin{minipage}[t]{16cm}
          \begin{itemize}
            \item \alert{Hardware Security} goes beyond classical cryptography, does not only consider the algorithm itself, but it considers the implementation of the algorithms (e.g. AES is believed to be secure, but attacks against the implementation like the side-channel attack: power analysis are pretty successful). % It protects the implementations of cryptographic algorithms against physical attacks, side-channel attacks etc.
              % - Avoids tampering with devices
          \end{itemize}
        \end{minipage}
      }
    }
    child {
      node {Classical Cryptography
        \resizebox{\textwidth}{!}{
          \begin{minipage}[t]{12cm}
            \begin{itemize}
                \item \alert{Cryptology:} Cryptography + cryptanalysis
                \begin{itemize}
                  \item \alert{Cryptography:} Art/science of keeping message secure
                  \begin{itemize}
                    \item is about algorithms protecting secret information
                    \item \script{67}{Basic Cryptographic Scheme}
                    \item \script{68}{Formal Notation}, $E$ \alert{injective} and both setes have \alert{same cardinality}, so \alert{one-to-one}, \alert{bijective}, so it is \alert{reversable}, $D$ is the \alert{inverse function} of $E$ and by this also bijective
                    \item \script{69}{Practice: Sending and Receiving secure messages}
                  \end{itemize}
                  \item \alert{Cryptanalysis:} Art/science of breaking ciphertext
                \end{itemize}
            \end{itemize}
          \end{minipage}
        }
      }
      child {
        node {Classification by way to process plaintext}
        child {
          node (streamcipher) {Stream Ciphers
            \resizebox{\textwidth}{!}{
              \begin{minipage}[t]{12cm}
                \begin{itemize}
                  \item bitwise Encryption and Decryption
                  \item \script{93}{Definition}, $Xor$ ist associative
                  \item \script{94}{Perfect encryption system if $\ldots$}
                  \item used in e.g. Red Phone between US and SU was implemented in this way
                  \item encrypt bits individually
                  \item need a method to generate key stream efficiently, starting from some “seed
                  \item usually small and fast common in embedded device
                \end{itemize}
              \end{minipage}
            }
          }
          child {
            node {Pseudo-random sequence
              \resizebox{\textwidth}{!}{
                \begin{minipage}[t]{12cm}
                  \begin{itemize}
                    \item \script{95}{Random function in C-library}, fixed values for $A$, $B$ and $m$, if prefix of plain text is known as many messages start with current date or files of certain type that have a prefix, one can continue the sequence, if $m$ happens to be a prime number, then it's a field and then one can compute inverse elements of each element and then it's just a solution of an equation system, if $m$ is not a prime number, then one does not have a field, but a ring and somtimes one can not compute the inverse or the inverse is not unique and also in this case on can try to solve a equation system to compute $A$ and $B$
                    \item pseudo random functions for cryptography it might be a problem
                  \end{itemize}
                \end{minipage}
              }
            }
          }
        }
        child {
          node (blockcipher) {Block Ciphers
            \resizebox{\textwidth}{!}{
              \begin{minipage}[t]{12cm}
                \begin{itemize}
                  \item always encrypt a full block (several bits)
                  \item are common for Internet applications
                \end{itemize}
              \end{minipage}
            }
          }
          child {
            node {Block Cipher Primitives
              \resizebox{\textwidth}{!}{
                \begin{minipage}[t]{12cm}
                  \begin{itemize}
                    \item \alert{Claude Shannon:} There are two primitive operations with which strong encryption algorithms can be built:
                      \begin{enumerate}
                        \item \alert{Confusion:} An encryption operation where the \alert{relationship between key and ciphertext is obscured}
                          \begin{itemize}
                            \item today, a common element for achieving confusion is substitution, which is found in AES and other ciphers.
                          \end{itemize}
                        \item \alert{Diffusion:} An encryption operation where the \alert{influence of one plaintext symbol is spread over many ciphertext symbols} with the goal of hiding statistical properties of the plaintext
                          \begin{itemize}
                            \item in other context known as Transposition
                            \item a simple diffusion element is the \alert{bit permutation} (in other context known as Tranposition)
                          \end{itemize}
                      \end{enumerate}
                      \begin{itemize}
                        \item Both operations by themselves are suboptimal in providing security. A cipher must include confusion and diffusion elements
                      \end{itemize}
                  \end{itemize}
                \end{minipage}
              }
            }
          }
        }
      }
      child {
        node {Classification by key
          % Crypto System with Keys
          \resizebox{\textwidth}{!}{
            \begin{minipage}[t]{12cm}
              \begin{itemize}
                \item \script{70}{Definition}, $E$ is a set of encryption algorithms and the key $K_E$ selects one special encryption algorithm based on the key, with each key one has a different encryption, same for $K_D$, both algorithms \alert{have to match}, they encryption key should be in a certain relation to the decrytpion key, they can also be just the same
                \item \underline{advantage of crypto systems with keys:}
                \begin{itemize}
                  \item keeping the encryption / decryption algorithm secret is not needed
                  \item keys can regulary be changed, to increase security
                \end{itemize}
              \end{itemize}
            \end{minipage}
          }
        }
        child {
          node (symmetric) {Symmetric cryptosystems $K_E = K_D$
            \resizebox{\textwidth}{!}{
              \begin{minipage}[t]{12cm}
                \begin{itemize}
                  \item encipher and decipher using the same key
                  \begin{itemize}
                    \item or one key is easily derived from the other
                  \end{itemize}
                  \item \underline{disadvantage:}
                  \begin{itemize}
                    \item one needs a \alert{key exchange}, the sender and the receiver have to agree on the same key and they key should be secret, one needs a way to transport the secret key via a secure channel from the sender to the receiver
                  \end{itemize}
                \end{itemize}
              \end{minipage}
            }
          }
          child {
            node (aes) {Advanced Encryption Standard (AES)
              \resizebox{\textwidth}{!}{
                \begin{minipage}[t]{12cm}
                  \begin{itemize}
                    \item \script{101}{Properties and Backstory}, block size always the same, number of rounds depens on key length that can be different, efficiency in \textit{software} and \textit{hardware}, byte-oriented cipher to be appropriate for small microprocessors and -controlers that havea $8$-bit width in the data path
                    \item \script{102}{Iterated Cipher types, key length, number rounds}
                    \item \script{103}{Round Structure}, in each round transformation of the original key, for keywidth 128 one has $10$ rounds and for them one needs $11$ round keys derived from the secret key $k$, X instead of P, Y instead of C
                    \item \script{104}{Internal Structure (f.)}: $16$ Bytes block size, arranged in a $4x4$ matrix
                      \begin{itemize}
                        \item \script{106}{Byte Substitution Layer (f.)}: Don't need matrix, is done Byte by Byte, S-Box / functions $S$ identical for all $16$ Bytes, propably to save memory, because $S$-Boxes are usually represendet in software by lookup tables with input values $0$ to $2^8-1 = 255$ , S-Boxes are the only \alert{nonlinear} elements of AES
                        \item \script{108}{Diffusion Layer}: asdf
                          \begin{itemize}
                            \item \script{109}{ShiftRows Layer}: Computes Permutation on the Bytes
                            \item \script{110}{MixColumn Layer}: Takes in this case $4$ Bytes and somehow does computations with these $4$ Bytes to obtain Diffusion. In the last round do the same omitting the MixColumn Layer (\href{https://crypto.stackexchange.com/questions/1346/why-is-mixcolumns-omitted-from-the-last-round-of-aes}{\inlinebox{reason}})
                          \end{itemize}
                        \item \script{111}{Key Addition Layer}: Bitwise Xor, in round number $i$ one add the round key $k_i$
                      \end{itemize}
                    \item at the moment most widely used symmetric cipher
                  \end{itemize}
                \end{minipage}
              }
            }
          }
        }
        child {
          node (asymmetric) {Asymmetric cryptosystems $K_E \ne K_D$
            \resizebox{\textwidth}{!}{
              \begin{minipage}[t]{12cm}
                \begin{itemize}
                  \item ecncipher and decipher using different keys
                  \begin{itemize}
                    \item computationally infeasible to derive one from other
                    % \item it should be computationally infeasible in reasonable time to derive one private key from the public key
                  \end{itemize}
                  \item \underline{advantages:}
                    \begin{itemize}
                      \item allows to share public key, secure key exchange not needed (for \alert{confidentiality}, message should be kept secret)
                      \item also other applications like authentication (\alert{authenticity}, able to sign message, so it can be sure it can only come from oneself)
                    \end{itemize}
                \end{itemize}
              \end{minipage}
            }
          }
        }
      }
      child {
        node {Classification by type of encryption operations}
        % Basic Types of Ciphers
        child {
          node (substitution) {Substitution ciphers
            \resizebox{\textwidth}{!}{
              \begin{minipage}[t]{12cm}
                \begin{itemize}
                  \item letters of $P$ replaced with other letters by $E()$
                  \item \underline{effects:}
                  \begin{itemize}
                    \item $C$ hides chars of $P$ (plaintext)
                    \item if $> 1$ key alphabet (polyalphabetic), $C$ dissipates high frequency chars
                  \end{itemize}
                  \item \underline{Sidenote:} One can say $key = 3$ or $key = \enquote{D}$, because $no(D) = 3$
                \end{itemize}
              \end{minipage}
            }
          }
          child {
            node {Monoalphabetic substitution ciphers
              \resizebox{\textwidth}{!}{
                \begin{minipage}[t]{10cm}
                  \begin{itemize}
                    \item in a general monoalphabetic substitution cipher each letter in $P$ is substituted by a fixed letter using a \alert{substitution table}
                    \item the \alert{key} is the substitution table
                    \begin{itemize}
                      \item there are $26 \times 25 \times \ldots \times 3 \times 2 \times 1 = 26! \approx 2^{88}$ substitution tables ($=$ keys) as depicted \script{78}{here (f.)}, because the mapping has to be injective
                    \end{itemize}
                    \item \underline{Attacks:}
                    \begin{itemize}
                      \item \script{79}{Exhaustive search}, Brute-force attack is not feasible
                      \item \script{80}{Letter Frequency Analysis (ff.)}: Statistical analysis
                      \begin{itemize}
                        \item In practice, not only frequencies of individual letters can be used for an attack, but also the frequency of letter pairs (i.e., \enquote{TH} is very common in English), letter triples, etc
                        \item need better concealing of statistical frequencies and probably also longer keys to avoid exhaustive search $\rightarrow$ Polyalphabetic substitution ciphers
                      \end{itemize}
                    \end{itemize}
                  \end{itemize}
                \end{minipage}
              }
            }
            child {
              node {Caesar Cipher
                \resizebox{\textwidth}{!}{
                  \begin{minipage}[t]{12cm}
                    \begin{itemize}
                      \item \script{74}{Defintion (f.)}, example key is $3$
                      \item each letter in $P$ is substituted by a fixed letter. In this special case the \enquote{key} is of length $1$, it is the shift amount
                      \item \script{76}{Attack: Exhaustive search}
                    \end{itemize}
                  \end{minipage}
                }
              }
            }
          }
          child {
            node {Polyalphabetic substitution ciphers
              \resizebox{\textwidth}{!}{
                \begin{minipage}[t]{12cm}
                  \begin{itemize}
                    \item \script{84}{Definition (f.)}
                    \item several key alphabets, flatten (diffuse) somewhat the frequency distribution of letters by combining high and low distributions
                    \item \underline{Attack:} If one knows the $n$, then one can break the whole text into $n$ parts and for these a statistical analysis works as for the monialphabetic substitution cipher with the unigram model
                    \begin{itemize}
                      \item only works if the attacker doesn't know the algorithm, this is not optimal, because one has to hide the algorithm or consider this $n$ also as part of the key, but then $n$ has to be very large to avoid a brute-force attack. Pairs are not hidden, but the unigram attack still works (\script{86}{summary here})
                    \end{itemize}
                  \end{itemize}
                \end{minipage}
              }
            }
            child {
              node (viginere) {Vigenère Tableaux Method
                \resizebox{\textwidth}{!}{
                  \begin{minipage}[t]{12cm}
                    \begin{itemize}
                      \item \script{87}{Definition (ff.)}
                      \item special case of polyalphabetic substitution with $n$ key alphabets. For each key alphabet the special case of Caesar cipher is chosen, i.e., each key alphabet can be represented by one letter
                      \item \script{88}{Tableaux}: rows are possible key characters and the columns give the encoding
                      \begin{itemize}
                        \item one can describe a row either by the shift or by saying which letter $c_i$ it mapped to which letter $c_j$ and takes $b_j$ as name of the row, choosing the last choice one can describe $n$ keys by a word of length $n$ 
                      \end{itemize}
                      \item \script{89}{Example}
                      \item $26^n$ different keys, one has to choose $n$ large enough
                      \item \alert{Attack:}
                      \begin{itemize}
                        \item same problem as for polyalphabetic substitution ciphers
                      \end{itemize}
                      \item \script{97}{Classification}
                    \end{itemize}
                  \end{minipage}
                }
              }
            }
          }
        }
        child {
          node (transposition) {Transposition (permutation) ciphers
            \resizebox{\textwidth}{!}{
              \begin{minipage}[t]{12cm}
                \begin{itemize}
                  \item order of letters in $P$ rearranged by $E()$
                  \begin{itemize}
                    \item rearrange letters in plaintext to produce ciphertext
                  \end{itemize}
                  \item \underline{effects:}
                  \begin{itemize}
                    \item $C$ scrambles text, hides $n$-grams for $n > 1$ (combinations of $n$ letters, e.g. th)
                  \end{itemize}
                \end{itemize}
              \end{minipage}
            }
          }
          child {
            node {Rail-Fence Cipher
              \resizebox{\textwidth}{!}{
                \begin{minipage}[t]{12cm}
                  \begin{itemize}
                    \item columnar transposition
                    \item \script{90}{Example}
                    \item Key = Number of columns
                    \item \underline{Attack:} Number of columns / the  key space is restricted if the text is short
                  \end{itemize}
                \end{minipage}
              }
            }
          }
        }
        child {
          node (product) {Product ciphers
            \resizebox{\textwidth}{!}{
              \begin{minipage}[t]{12cm}
                \begin{itemize}
                  \item combine two or more ciphers to enhance the security of the cryptosystem
                  \begin{itemize}
                    \item $E = E_1 + E_2 + ... + E_n$
                    \item built of multiple blocks, either \alert{Substitution} or \alert{Transposition}
                  \end{itemize}
                  \item \underline{attack}:
                  \begin{itemize}
                    \item product cipher might not necessarily be stronger than its individual components used separately, \script{91}{Example}
                  \end{itemize}
                  \item \underline{effects:}
                  \begin{itemize}
                    \item can do all what Substitution and Transposition ciphers can so, more secure if used well
                  \end{itemize}
                \end{itemize}
              \end{minipage}
            }
          }
          child {
            node {Two-block product cipher
              \resizebox{\textwidth}{!}{
                \begin{minipage}[t]{12cm}
                  \begin{itemize}
                    \item $E2(E1(P, KE1), KE2)$
                    \item may be repeated to form several encryption rounds
                  \end{itemize}
                \end{minipage}
              }
            }
          }
        }
      }
    }
    child {
      node (test){Basics}
      child {
        node {Security
          \resizebox{\textwidth}{!}{
            \begin{minipage}[t]{12cm}
              \begin{itemize}
                \item \alert{Security:} No complete definition. Could define it as ability to satisfy certain security properties (\script{13}{more precise})
                  \begin{itemize}
                    \item tampering can be conducted at all levels, hardware is not a root of trust, one only can't change it afterwards
                    \item is also a matter of cost, most secure system is a system that doesn't do anything (don't grant access to anybody), but the system also has to be useful, not to expensive etc. (power, area, price and performance). Security processing may add considerable overhead to a resource-constrained embedded system. Only on componentn in a optimzation problem which optimizes the cost of the whole system. Today’s security features can be attacked by tomorrow’s technology
                  \end{itemize}
                \item \alert{Safety:} System is designed without any error leading to unintended behavior (\alert{design time})
                \item \alert{Reliability:} A correctly designed system continues to work correctly during its \alert{life-time}
              \end{itemize}
            \end{minipage}
          }
        }
        child {
          node {Possible Actions of Adversaries, Attacks (ff.)
            \resizebox{\textwidth}{!}{
              \begin{minipage}[t]{12cm}
                \begin{itemize}
                  \item \alert{side-channel attack:} using sidechannel outputs to extract information, when a chip works it produces more information that occurs on the outputs, measure e.g. energy consumption, electromagnetic field etc. (focuses on \alert{outputs}, i.e. side-channels)
                  \item \alert{fault injection:} counterpart to side-channel, focuses on \alert{additional inputs}, e.g. shooting with a laser beam on certain point in a chip which causes a fault and from the effect of this fault one can derive secret information
                    \begin{itemize}
                      \item \script{23}{Example: Power Analysis}
                    \end{itemize}
                    % using outputs of chip that are not the usual outputs, 
                  \item \script{25}{Example for attacked device: Smart Card}, weak 8-Bit CPU, because of cost, not best encryption methods, easy to get physical access
                  \item \script{26}{Example for attacked device: RFID}, even weaker IC, then on Smart Cards
                  \item \script{35}{Countermeasures}
                \end{itemize}
              \end{minipage}
            }
          }
        }
        child {
          node {Related Terms
            \resizebox{\textwidth}{!}{
              \begin{minipage}[t]{12cm}
                \begin{itemize}
                  \item \alert{Vulnerability:} Weakness in the secure system
                  \item \alert{Threat:} Set of circumstances that has the potential to cause loss or harm
                  \item \alert{Attack:} The act of a human exploiting the vulnerability in the system
                    \begin{itemize}
                      \item \script{16}{relationship security vs. safety and reliability}
                    \end{itemize}
                \end{itemize}
              \end{minipage}
            }
          }
        }
        child {
          node {Security properties (\script{14}{\enquote{CIAAN}})
            \resizebox{\textwidth}{!}{
              \begin{minipage}[t]{12cm}
                \begin{itemize}
                  \item \alert{Confidentiality:} Protecting confidential information from being disclosed to unauthorized parties.
                    \begin{itemize}
                      \item no unauthorized reads
                    \end{itemize}
                  \item \alert{Integrity:} Ensuring that information is only modified by authorized parties.
                    \begin{itemize}
                      \item no unauthorized writes
                    \end{itemize}
                  \item \alert{Availability:} Making sure that information and systems are accessible to authorized parties when they need them
                    \begin{itemize}
                      \item e.g. resistance to denial-of-service attacks
                    \end{itemize}
                  \item \alert{Authenticity:} Ensuring that information and communication come from the source they are supposed to come from.
                    \begin{itemize}
                      \item receiver of email knows for sure it comes from you, achieved by e.g. digital signatures
                    \end{itemize}
                  \item \alert{Non-repudiation:} Ensuring that nobody can deny having performed certain actions (like sending / receiving messages, changing data etc.).
                    \begin{itemize}
                      \item access to system is protected by login and logging when changing file
                    \end{itemize}
                \end{itemize}
              \end{minipage}
            }
          }
        }
      }
      child {
        node {Design Cycle and Threats
          \resizebox{\textwidth}{!}{
            \begin{minipage}[t]{12cm}
              \begin{itemize}
                \item \script{41}{Traditional Design Cycle of ICs}
                \item \alert{Problems:}
                \begin{itemize}
                  \item \script{42}{Problem 1 (f.)}: \alert{Cost of Manufacturing}, \textit{\enquote{fabless} companies}, \script{44}{untrusted foundry (pr.)}, \script{47}{untrusted assembly (pr.}, can mark corrects IC's as devective and sell them, fab doesn't know exactly fraction of devective to total)
                  \item \script{48}{Problem 2}: \alert{Design Complexity}, Company that provides \textit{IP} (intellectual property) block can be untrusted
                \end{itemize}
                \item \script{50}{New Design Cycle of ICs}, Vendors of IP Blocks not trusted, safety (not part of this lecture) and security problem, \alert{Soft IP}: Buying \textit{RTL} (Register transfer level) designs, \alert{Firm IP}: Buying \textit{Gate Level Netlists}, \alert{Hard IP}: Integrating \textit{Layout Data} bought by another company, IP Vendors would not give all information about IP how they came to this layout because it could also not trust, because one could steal it's IP, both sides have to trust, one needs methologies that ensure that IP does the right thing and on the other hand that the person who bought it, is not able to steal it
                \item \script{53}{Vulnerabilities and untrusted parties}, \alert{IC Piracy} is overproduction, selling devective \textit{out-of-spec} IC's etc., \script{58}{Counterfeiting}, \textit{Cloned IC's} when somoeone steals design data or reverse engineering, \script{59}{IC recyling}, Problem because of \alert{Bathtub Curve}
                \item \script{60}{Overview: Supply Chain Vulnerabilities}, \alert{Remark} e.g. weak processor by the name of a much better proecessor, at all points of the supply chain there are threats for the hardware security
              \end{itemize}
            \end{minipage}
          }
        }
      }
    }
  \end{mindmapcontent}
  \begin{edges}
    \edge{viginere}{blockcipher}
    \edge{aes}{product}
    \edge{aes}{blockcipher}
  \end{edges}
  \annotation{test}{annotation}
\end{mindmap}
\end{document}
